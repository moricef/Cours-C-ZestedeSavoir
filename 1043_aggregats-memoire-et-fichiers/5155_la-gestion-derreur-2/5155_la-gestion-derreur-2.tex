\chapter{La gestion d'erreur (2)}
\label{la-gestion-d-erreur-(2)}

Jusqu'à présent, nous vous avons présenté les bases de la gestion 
d'erreurs en vous parlant des retours de fonctions et de la variable 
globale \mybox{errno}. Dans ce chapitre, nous allons approfondir 
cette notion afin de réaliser des programmes plus robustes.

\section{Gestion de ressources}
\label{gestion-de-ressources}

\subsection{Allocation de ressources}
\label{allocation-de-ressources}

Dans cette deuxième partie, nous avons découvert l'allocation dynamique
de mémoire et la gestion de fichiers. Ces deux sujets ont un point en
commun : il est nécessaire de passer par une fonction d'allocation et
par une fonction de libération lors de leur utilisation. Il s'agit d'un
processus commun en programmation appellé la \textbf{gestion de
ressources}. Or, celle-ci pose un problème particulier dans le cas de la
gestion d'erreurs. En effet, regardez de plus près ce code simplifié
permettant l'allocation dynamique d'un tableau multidimensionnel de
trois fois trois \mybox{int}.

\begin{C}
#include <stdio.h>
#include <stdlib.h>


int main(void)
{
    int **p;
    unsigned i;

    p = malloc(3 * sizeof(int *));

    if (p == NULL)
    {
        fprintf(stderr, "Échec de l'allocation\n");
        return EXIT_FAILURE;
    }

    for (i = 0; i < 3; ++i)
    {
        p[i] = malloc(3 * sizeof(int));

        if (p[i] == NULL)
        {
            fprintf(stderr, "Échec de l'allocation\n");
            return EXIT_FAILURE;
        }
    }

    /* ... */
    return 0;
}
\end{C}

Vous ne voyez rien d'anormal dans le cas de la seconde boucle ? Celle-ci
quitte le programme en cas d'échec de la fonction \mybox{malloc()}.
Oui, mais nous avons \emph{déjà} fait appel à la fonction
\mybox{malloc()} auparavant, ce qui signifie que si nous quittons le
programme à ce stade, nous le ferons \emph{sans avoir libéré certaines
ressources} (ici de la mémoire).

Seulement voilà, comment faire cela sans rendre le programme bien plus
compliqué et bien moins lisible ? C'est ici que l'utilisation de
l'instruction \mybox{goto} devient pertinente et d'une aide précieuse.
La technique consiste à placer la libération de ressources en fin de
fonction et de se rendre au bon endroit de cette zone de libération à
l'aide de l'instruction \mybox{goto}. Autrement dit, voici ce que cela
donne avec le code précédent.

\begin{C}
#include <stdio.h>
#include <stdlib.h>


int main(void)
{
    int **p;
    unsigned i;
    int status = EXIT_FAILURE;

    p = malloc(3 * sizeof(int *));

    if (p == NULL)
    {
        fprintf(stderr, "Échec de l'allocation\n");
        goto fin;
    }

    for (i = 0; i < 3; ++i)
    {
        p[i] = malloc(3 * sizeof(int));

        if (p[i] == NULL)
        {
            fprintf(stderr, "Échec de l'allocation\n");
            goto liberer_p;
        }
    }

    status = 0;

    /* Zone de libération */
liberer_p:
    while (i > 0)
    {
        --i;
        free(p[i]);
    }

    free(p);
fin:
    return status;
}
\end{C}

Comme vous le voyez, nous avons placé deux étiquettes : une référençant
l'instruction \mybox{return} et une autre désignant la première
instruction de la zone de libération. Ainsi, nous quittons la fonction
en cas d'échec de la première allocation, mais nous libérons les
ressources auparavant dans le cas des allocations suivantes. Notez que
nous avons ajouté une variable \mybox{status} afin de pouvoir retourner
la bonne valeur suivant qu'une erreur est survenue ou non.

\subsection{Utilisation de ressources}
\label{utilisation-de-ressources}

Si le problème se pose dans le cas de l'allocation de ressources, il se
pose également dans le cas de leur utilisation.

\begin{C}
#include <stdio.h>
#include <stdlib.h>


int main(void)
{
    FILE *fp;

    fp = fopen("texte.txt", "w");

    if (fp == NULL)
    {
        fprintf(stderr, "Le fichier texte.txt n'a pas pu être ouvert\n");
        return EXIT_FAILURE;
    }
    if (fputs("Au revoir !\n", fp) == EOF)
    {
        fprintf(stderr, "Erreur lors de l'écriture d'une ligne\n");
        return EXIT_FAILURE;
    }
    if (fclose(fp) == EOF)
    {
        fprintf(stderr, "Erreur lors de la fermeture du flux\n");
        return EXIT_FAILURE;        
    }

    return 0;
}
\end{C}

Dans le code ci-dessus, l'exécution du programme est stoppée en cas
d'erreur de la fonction \mybox{fputs()}. Cependant, l'arrêt s'effectue
alors qu'une ressource n'a pas été libérée (en l'occurrence le flux
\mybox{fp}). Aussi, il est nécessaire d'appliquer la solution présentée
juste avant pour rendre ce code correct.

\begin{C}
#include <stdio.h>
#include <stdlib.h>


int main(void)
{
    FILE *fp;
    int status = EXIT_FAILURE;

    fp = fopen("texte.txt", "w");

    if (fp == NULL)
    {
        fprintf(stderr, "Le fichier texte.txt n'a pas pu être ouvert\n");
        goto fin;
    }
    if (fputs("Au revoir !\n", fp) == EOF)
    {
        fprintf(stderr, "Erreur lors de l'écriture d'une ligne\n");
        goto fermer_flux;
    }
    if (fclose(fp) == EOF)
    {
        fprintf(stderr, "Erreur lors de la fermeture du flux\n");
        goto fin;   
    }

    status = 0;

fermer_flux:
    fclose(fp);
fin:
    return status;
}
\end{C}

Nous attirons votre attention sur deux choses :

\begin{itemize}
\item
  En cas d'échec de la fonction \mybox{fclose()}, le programme est
  arrêté immédiatement. Étant donné que c'est la fonction
  \mybox{fclose()} qui pose problème, il n'y a pas lieu de la rappeler
  (notez que cela n'est toutefois pas une erreur de l'appeler une
  seconde fois).
\item
  Le retour de la fonction \mybox{fclose()} n'est pas vérifié en cas
  d'échec de la fonction \mybox{fputs()} étant donné que nous sommes
  \emph{déjà} dans un cas d'erreur.
\end{itemize}
  
\section{Fin d'un programme}
\label{fin-d-un-programme}
  
Vous le savez, l'exécution de votre programme se termine lorsque celui-ci 
quitte la fonction \mybox{main()}. Toutefois, que se passe-t-il exactement 
lorsque celui-ci s'arrête ? En fait, un petit bout de code appelé   
\textbf{épilogue} est exécuté afin de réaliser quelques tâches. Parmis
celles-ci figure la vidange et la fermeture de tous les flux encore
ouvert. Une fois celui-ci exécuté, la main est rendue au système
d'exploitation qui, le plus souvent, se chargera de libérer toutes les
ressources qui étaient encore allouées.

\begin{questionbox} 
 Mais ?! Vous venez de nous dire qu'il
était nécessaire de libérer les ressources avant l'arrêt du programme.
Du coup, pourquoi s'amuser à appeler les fonctions \mybox{fclose()} ou
\mybox{free()} alors que l'épilogue ou le système d'exploitation s'en
charge ?
\end{questionbox}


Pour quatre raisons principales :

\begin{enumerate}
\def
\labelenumi{\arabic{enumi}.}
\item
  Afin de libérer des ressources pour les autres programmes. En effet,
  si vous ne libérerez les ressources allouées qu'à la fin de votre
  programme alors que celui-ci n'en a plus besoin, vous gaspillez des
  ressources qui pourraient être utilisées par d'autres programmes.
\item
  Pour être à même de détecter des erreurs, de prévenir l'utilisateur en
  conséquence et d'éventuellement lui proposer des solutions.
\item
  En vue de rendre votre code plus facilement modifiable par après et
  d'éviter les oublis.
\item
  Parce qu'il n'est pas garanti que les ressources seront effectivement
  libérées par le système d'exploitaion.
\end{enumerate}

\begin{attentionbox}
 Aussi, de manière générale :
 \begin{itemize}
  \item 
considérez que l'objectif de l'épilogue est de fermer \emph{uniquement} 
les flux \mybox{stdin}, \mybox{stdout} et \mybox{stderr} ; 
  \item 
ne comptez \emph{pas} sur votre système d'exploitation pour la libération
de quelques ressources que ce soit (et notamment la mémoire) ;
  \item 
libérez vos ressources le plus tôt possible, autrement dit dès que votre 
programme n'en a plus l'utilité.
  \end{itemize}
\end{attentionbox}


\subsection{Terminaison normale}
\label{terminaison-normale}

Le passage par l'épilogue est appelée la \textbf{terminaison normale} du
programme. Il est possible de s'y rendre directement en appelant la
fonction \mybox{exit()} qui est déclarée dans l'en-tête
\mybox{\textless{}stdlib.h\textgreater{}}.

\begin{C}
void exit(int status);
\end{C}

Appeler cette fonction revient au même que de quitter la fonction
\mybox{main()} à l'aide de l'instruction \mybox{return}. L'argument
attendu est une expression entière identique à celle fournie comme
opérande de l'instruction \mybox{return}. Ainsi, il vous est possible
de mettre fin directement à l'exécution de votre programme \emph{sans}
retourner jusqu'à la fonction \mybox{main()}.

\begin{C}
#include <stdio.h>
#include <stdlib.h>


void fin(void)
{
    printf("C'est la fin du programme\n");
    exit(0);
}


int main(void)
{
    fin();
    printf("Retour à la fonction main\n");
    return 0;
}
\end{C}

\begin{C}
C'est la fin du programme
\end{C}

Comme vous le voyez l'exécution du programme se termine une fois que la
fonction \mybox{exit()} est appelée. La fin de la fonction
\mybox{main()} n'est donc \emph{jamais} exécutée.

\subsection{Terminaison anormale}
\label{terminaison-anormale}

Il est possible de terminer l'exécution d'un programme \emph{sans passer
par l'épilogue} à l'aide de la fonction \mybox{abort()} (définie
également dans l'en-tête \mybox{\textless{}stdlib.h\textgreater{}}).
Dans un tel cas, il s'agit d'une \textbf{terminaison anormale} du
programme.

\begin{C}
void abort(void);
\end{C}

Une terminaison anormale signifie qu'une condition \emph{non prévue} par
le programmeur est survenue. Elle est donc à distinguer d'une
terminaison normale survenue suite à une erreur qui, elle, était prévue
(comme une erreur lors d'une saisie de l'utilisateur). De manière
générale, la terminaison anormale est utilisée lors de la phase de
développement d'un logiciel afin de faciliter la détection d'erreurs de
programmation, celle-ci entraînant le plus souvent la production d'une
\textbf{image mémoire} (\emph{core dump} en anglais) qui pourra être
analysée à l'aide d'un \textbf{débogueur} (\emph{debugger} en anglais).

\begin{infobox} 
 Une image mémoire est en fait un
fichier contenant l'état des registres et de la mémoire d'un pogramme
lors de la survenance d'un problème.
\end{infobox}  


\section{Les assertions}  
\label{les-assertions}

\subsection{La macrofonction assert}
\label{la-macrofonction-assert}

Une des utilisation majeure de la fonction \mybox{abort()} est réalisée
via la macrofonction \mybox{assert()} (définie dans l'en-tête
\mybox{\textless{}assert.h\textgreater{}}). Cette dernière est utilisée
pour placer des tests à certains points d'un programme. Dans le cas où
un de ces tests s'avère faux, un message d'erreur est affiché (ce
dernier comprend la condition dont l'évaluation est fausse, le nom du
fichier et la ligne courante) après quoi la fonction \mybox{abort()}
est appelée afin de produire une image mémoire.

Cette technique est très utile pour détecter rapidement des erreurs au
sein d'un programme lors de sa phase de développement. Généralement, les
assertions sont placées en début de fonction afin de vérifier un
certains nombres de conditions. Par exemple, si nous reprenons la
fonction \mybox{long\_colonne()} du TP sur le Puissance 4.

\begin{C}
static unsigned long_colonne(unsigned joueur, unsigned col, unsigned ligne, unsigned char (*grille)[6])
{
    unsigned i;
    unsigned n = 1;

    for (i = ligne + 1; i < 6; ++i)
    {
        if (grille[col][i] == joueur)
            ++n;
        else
            break;
    }

    return n;
}
\end{C}

Nous pourrions ajouter quatre assertions vérifiant si :

\begin{itemize}
\item
  le numéro du joueur est un ou deux ;
\item
  le numéro de la colonne est compris entre zéro et six inclus ;
\item
  le numéro de la ligne est compris entre zéro et cinq ;
\item
  le pointeur \mybox{grille} n'est pas nul.
\end{itemize}

Ce qui nous donne le code suivant.

\begin{C}
static unsigned long_colonne(unsigned joueur, unsigned col, unsigned ligne, unsigned char (*grille)[6])
{
    unsigned i;
    unsigned n = 1;

    assert(joueur == 1 || joueur == 2);
    assert(col < 7);
    assert(ligne < 6);
    assert(grille != NULL);

    for (i = ligne + 1; i < 6; ++i)
    {
        if (grille[col][i] == joueur)
            ++n;
        else
            break;
    }

    return n;
}
\end{C}

Ainsi, si nous ne respectons pas une de ces conditions pour une raison
ou pour une autre, l'exécution du programme s'arrêtera et nous aurons
droit à un message du type (dans le cas où la première assertion
échoue).

\begin{C}
a.out: main.c:71: long_colonne: Assertion `joueur == 1 || joueur == 2' failed.
Aborted
\end{C}

Comme vous le voyez, le nom du programme est indiqué, suivi du nom du
fichier, du numéro de ligne, du nom de la fonction et de la condition
qui a échoué. Intéressant, non ? :)

\subsection{Suppression des assertions}
\label{suppression-des-assertions}

Une fois votre programme développé et dûment testé, les assertions ne
vous sont plus vraiment utiles étant donné que celui-ci fonctionne. Les
conserver alourdirait l'exécution de votre programme en ajoutant des
vérifications. Toutefois, heureusement, il est possible de supprimer ces
assertions \emph{sans modifier votre code} en ajoutant simplement
l'option \mybox{-DNDEBUG} lors de la compilation.

\begin{C}
 $ zcc -DNDEBUG main.c
\end{C}


\section{Les fonctions strerror et perror}
\label{les-fonctions-strerror-et-perror}

Vous le savez, certaines (beaucoup) de fonctions standards peuvent 
échouer. Toutefois, en plus de signaler à l'utilisateur qu'une de 
ces fonctions a échoué, cela serait bien de lui spécifier \emph{pourquoi}. 
C'est ici qu'entre en jeu les fonctions \mybox{sterror()} et \mybox{perror()}.

\begin{C}
char *strerror(int num);
\end{C}

La fonction \mybox{strerror()} (déclarée dans l'en-tête
\mybox{\textless{}string.h\textgreater{}}) retourne une chaîne de
caractères correspondant à la valeur entière fournie en argument. Cette
valeur sera en fait \emph{toujours} celle de la variable \mybox{errno}.
Ainsi, il vous est possible d'obtenir plus de détails quand une fonction
standard rencontre un problème. L'exemple suivant illustre l'utilisation
de cette fonction en faisant appel à la fonction \mybox{fopen()} afin
d'ouvrir un fichier qui n'existe pas (ce qui provoque une erreur).

\begin{C}
#include <errno.h>
#include <stdio.h>
#include <stdlib.h>
#include <string.h>


int main(void)
{
    FILE *fp;

    fp = fopen("nawak.txt", "r");

    if (fp == NULL)
    {
        fprintf(stderr, "fopen: %s\n", strerror(errno));
        return EXIT_FAILURE;
    }

    fclose(fp);
    return 0;
}
\end{C}

\begin{C}
fopen: No such file or directory
\end{C}

Comme vous le voyez, nous avons obtenu des informations supplémentaires
: la fonction a échoué parce que le fichier « nawak.txt » n'existe pas.

\begin{infobox} 
 Notez que les messages d'erreur
retournés par la fonction \mybox{strerror()} sont le plus souvent en
anglais.
\end{infobox}


\begin{C}
void perror(char *message);
\end{C}

La fonction \mybox{perror()} (déclarée dans l'en-tête
\mybox{\textless{}stdio.h\textgreater{}}) écrit sur le flux d'erreur
standard (\mybox{stderr}, donc) la chaîne de caractères fournie en
argument, suivie du caractère \mybox{:}, d'un espace et du retour de la
fonction \mybox{strerror()} avec comme argument la valeur de la
variable \mybox{errno}. Autrement dit, cette fonction revient au même
que l'appel suivant.

\begin{C}
fprintf(stderr, "message: %s\n", strerror(errno));
\end{C}

L'exemple précédent peut donc également s'écrire comme suit.

\begin{C}
#include <stdio.h>
#include <stdlib.h>


int main(void)
{
    FILE *fp;

    fp = fopen("nawak.txt", "r");

    if (fp == NULL)
    {
        perror("fopen");
        return EXIT_FAILURE;
    }

    fclose(fp);
    return 0;

\end{C}

\begin{C}
 fopen: No such file or directory
\end{C}


\hrulefill

Voici qui clôture la deuxième partie de ce cours qui aura été 
riche en nouvelles notions. N'hésitez pas à reprendre certains 
passages avant de commencer la troisième partie qui vous fera 
plonger sous le capot.