\chapter{Les fichiers (1)}
\label{les-fichiers-(1)}

Jusqu'à présent, nous n'avons manipulé que des données en
mémoire, ce qui nous empêchait de les stocker de manière permanente.
Dans ces deux chapitres, nous allons voir comment conserver des
informations de manière permanente à l'aide des fichiers.

\section{Les fichiers}
\label{les-fichiers}

En informatique, un fichier est un ensemble d'informations stockées 
sur un support, réuni sous un même nom et manipulé comme une unité.

\subsection{Extension de fichier}
\label{extension-de-fichier}

Le contenu de chaque fichier est lui-même organisé suivant un format qui
dépend des données qu'il contient. Il en existe une pléthore pour chaque
type de données :

\begin{itemize}
\item
  audio : Ogg, MP3, MP4, FLAC, Wave, etc. ;
\item
  vidéo : Ogg, WebM, MP4, AVI, etc. ;
\item
  documents : ODT, DOC et DOCX, XLS et XLSX, PDF, Postscript, etc.
\end{itemize}

Afin d'aider l'utilisateur, ces formats sont le plus souvent indiqués à
l'aide d'une \textbf{extension} qui se traduit par un suffixe ajouter au
nom de fichier. Toutefois, cette extension est purement
\emph{indicative} et \emph{facultative}, elle n'influence \emph{en rien}
le contenu du fichier. Son seul objectif est d'aider à déterminer le
type de contenu d'un fichier.

\subsection{Système de fichiers}
\label{systeme-de-fichiers}

Afin de faciliter leur localisation et leur gestion, les fichiers sont
classés et organisés sur leur support suivant un \textbf{système de
fichiers}. C'est lui qui permet à l'utilisateur de répartir ses fichiers
dans une arborescence de dossiers et de localiser ces derniers à partir
d'un chemin d'accès.

\subsubsection{Le chemin d'accès}
\label{le-chemin-dacces}

Le chemin d'accès d'un fichier est une suite de caractères décrivant la
position de celui-ci dans le système de fichiers. Un chemin d'accès se
compose au minimum du nom du fichier visé et au maximum de la suite de
tous les dossiers qu'il est nécessaire de traverser pour l'atteindre
depuis le \textbf{répertoire racine}. Ces éventuels dossiers à traverser
sont séparés par un caractère spécifique qui est \mybox{/} sous
Unixoïde et \mybox{\textbackslash{}} sous Windows.

La \textbf{racine} d'un système de fichier est le point de départ de
l'arborescence des fichiers et dossiers. Sous Unixoïdes, il s'agit du
répertoire \mybox{/} tandis que sous Windows, chaque lecteur est une
racine (comme \mybox{C:} par exemple). Si un chemin d'accès commence
par la racine, alors celui-ci est dit \textbf{absolu} car il permet
d'atteindre le fichier depuis n'importe quelle position dans
l'arborescence. Si le chemin d'accès ne commence pas par la racine, il
est dit \textbf{relatif} et ne permet de parvenir au fichier que depuis
un point précis dans la hiérarchie de fichiers.

Ainsi, si nous souhaitons accéder à un fichier nommé « texte.txt » situé
dans le dossier « documents » lui-même situé dans le dossier «
utilisateur » qui est à la racine, alors le chemin d'accès absolu vers
ce fichier serait \mybox{/utilisateur/documents/texte.txt} sous
Unixoïde et
\mybox{C:\textbackslash{}utilisateur\textbackslash{}documents\textbackslash{}texte.txt}
sous Windows (à supposer qu'il soit sur le lecteur \mybox{C:}).
Toutefois, si nous sommes déjà dans le dossier « utilisateur », alors
nous pouvons y accéder à l'aide du chemin relatif
\mybox{documents/texte.txt} ou
\mybox{documents\textbackslash{}texte.txt}. Néanmoins, ce chemin
relatif n'est utilisable que si nous sommes dans le dossier «
utilisateur ».

\subsubsection{Métadonnées}
\label{metadonnees}

Également, le système de fichier se charge le plus souvent de conserver
un certain nombre de données concernant chaque fichier comme :

\begin{itemize}
\item
  sa taille ;
\item
  ses droits d'accès (les utilisateurs autorisés à le manipuler) ;
\item
  la date de dernier accès ;
\item
  la date de dernière modification ;
\item
  etc.
\end{itemize}

\section{Les flux : un peu de théorie}
\label{les-flux-:-un-peu-de-theorie}

La bibliothèque standard vous fourni différentes fonctions pour manipuler
les fichiers, toutes déclarées dans l'en-tête
\mybox{\textless{}stdio.h\textgreater{}}. Toutefois, celles-ci
manipulent non pas des fichiers, mais des \textbf{flux} de données en
provenance ou à destination de fichiers. Ces flux peuvent être de deux
types :

\begin{itemize}
\item
  des flux de textes qui sont des suites de caractères terminées par un
  caractère de fin de ligne (\mybox{\textbackslash{}n}) et formant
  ainsi des lignes ;
\item
  des flux binaires qui sont des suites de multiplets.
\end{itemize}

\subsection{Pourquoi utiliser des flux ?}
\label{pourquoi-utiliser-des-flux-?}

Pourquoi recourir à des flux plutôt que de manipuler directement des
fichiers nous demanderez-vous ? Pour deux raisons : les disparités entre
système d'exploitation quant à la représentation des lignes et la
lourdeur des opérations de lecture et d'écriture.

\subsubsection{Disparités entre systèmes d'exploitation}
\label{disparites-entre-systemes-dexploitation}

Les différents systèmes d'exploitation ne représentent pas les lignes de
la même manière :

\begin{itemize}
\item
  sous Mac OS (avant Mac OS X), la fin d'une ligne était indiquée par le
  caractère \mybox{\textbackslash{}r} ;
\item
  sous Windows, la fin de ligne est indiquée par la suite de caractères
  \mybox{\textbackslash{}r\textbackslash{}n} ;
\item
  sous Unixoïdes (GNU/Linux, *BSD, Mac OS X, Solaris, etc.), la fin de
  ligne est indiquée par le caractère \mybox{\textbackslash{}n}.
\end{itemize}

Aussi, si nous manipulions directement des fichiers, nous devrions
prendre en compte ces disparités, ce qui rendrait notre code nettement
plus pénible à rédiger. En passant par les fonctions de la bibliothèque
standard, nous évitons ce casse-tête car celle-ci remplace le ou les
caractères de fin de ligne par un \mybox{\textbackslash{}n} (ce que
nous avons pu expérimenter avec les fonctions \mybox{printf()} et
\mybox{scanf()}) et inversément.

\subsubsection{Lourdeur des opérations de lecture et d'écriture}
\label{lourdeur-des-operations-de-lecture-et-decriture}

Lire depuis un fichier ou écrire dans un fichier signifie le plus
souvent accéder au disque dur. Or, rappelez-vous, il s'agit de la
mémoire la plus lente d'un ordinateur. Dès lors, si pour lire une ligne
il était nécessaire de récupérer les caractères un à un depuis le disque
dur, cela prendrait un temps fou.

Pour éviter ce problème, les flux de la bibliothèque standard recourent
à un mécanisme appelé la \textbf{temporisation} ou \textbf{mémorisation}
(\emph{buffering} en anglais). La bibliothèque standard fournit deux
types de temporisation :

\begin{itemize}
\item
  la temporisation par blocs ;
\item
  et la temporisation par lignes.
\end{itemize}

Avec la \textbf{temporisation par blocs}, les données sont récupérées
depuis le fichier et écrites dans le fichier sous forme de blocs d'une
taille déterminée. L'idée est la suivante : plutôt que de lire les
caractères un à un, nous allons demander un bloc de données d'une taille
déterminée que nous conserverons en mémoire vive pour les accès
suivants. Cette technique est également utilisée lors des opérations
d'écritures : les données sont stockées en mémoire jusqu'à ce qu'elles
atteignent la taille d'un bloc. Ainsi, le nombre d'accès au disque dur
sera limité à la taille totale du fichier à lire (ou des données à
écrire) divisée par la taille d'un bloc.

La \textbf{temporisation par lignes}, comme son nom l'indique, se
contente de mémoriser une ligne. Techniquement, celle-ci est utilisée
lorsque le flux est associé à un périphérique interactif comme un
terminal. En effet, si nous appliquions la temporisation par blocs pour
les entrées de l'utilisateur, ce dernier devrait entrer du texte jusqu'à
atteindre la taille d'un bloc. Pareillement, nous devrions attendre
d'avoir écrit une quantité de données égale à la taille d'un bloc pour
que du texte soit affiché à l'écran. Cela serait assez gênant et peu
interactif\ldots{} À la place, c'est la temporisation par lignes qui est
utilisée : les données sont stockées jusqu'à ce qu'un caractère de fin
de ligne soit rencontré (\mybox{\textbackslash{}n}, donc) ou jusqu'à ce
qu'une taille maximale soit atteinte.

La bibliothèque standard vous permet également de ne pas temporiser les
données en provenance ou à destination d'un flux.

\subsection{stdin, stdout et stderr}
\label{stdin-stdout-et-stderr}

Par défaut, trois flux \emph{de texte} sont ouverts lors du démarrage
d'un programme et sont déclarés dans l'en-tête
\mybox{\textless{}stdio.h\textgreater{}} : \mybox{stdin},
\mybox{stdout} et \mybox{stderr}.

\begin{itemize}
\item
  \mybox{stdin} correspond à l'\textbf{entrée standard}, c'est-à-dire
  le flux depuis lequel vous pouvez récupérer les informations fournies
  par l'utilisateur.
\item
  \mybox{stdout} correspond à la \textbf{sortie standard}, il s'agit du
  flux vous permettant de transmettre des informations à l'utilisateur
  qui seront le plus souvent affichées dans le terminal.
\item
  \mybox{stderr} correspond à la \textbf{sortie d'erreur standard},
  c'est ce flux que vous devez privilégier lorsque vous souhaitez
  transmettre des messages d'erreurs ou des avertissements à l'attention
  de l'utilisateur (nous verrons comment l'utiliser un peu plus tard
  dans ce chapitre). Comme pour \mybox{stdout}, ces données sont le
  plus souvent affichées dans le terminal.
\end{itemize}

Les flux \mybox{stdin} et \mybox{stdout} sont temporisés par lignes
(sauf s'ils sont associés à des fichiers au lieu de périphériques
interactifs, auxquels cas ils seront temporisés par blocs) tandis que le
flux \mybox{stderr} est \emph{au plus} temporisé par lignes (ceci afin
que les informations soient transmises le plus rapidement possible à
l'utilisateur).

\section{Ouverture et fermeture d'un flux}
\label{ouverture-et-fermeture-flux}

\section{La fonction fopen}
\label{la-fonction-fopen}

\begin{C}
FILE *fopen(char *chemin, char *mode);
\end{C}

La fonction \mybox{fopen()} permet d'ouvrir un flux. Celle-ci attend
deux arguments : un \textbf{chemin d'accès} vers un fichier qui sera
associé au flux et un \textbf{mode} qui détermine le type de flux (texte
ou binaire) et la nature des opérations qui seront réalisées sur le
fichier via le flux (lecture, écriture ou les deux). Elle retourne un
pointeur vers un flux en cas de succès et un pointeur nul en cas
d'échec.

\subsubsection{Le mode}
\label{le-mode}

Le mode est une chaîne de caractères composée d'une ou plusieurs lettres
qui décrit le type du flux et la nature des opérations qu'il doit
réaliser.

Cette chaîne commence \emph{obligatoirement} par la seconde de ces
informations. Il existe six possibilités reprises dans le tableau
ci-dessous.

\begin{table}[ht!]
\centering
\rowcolors{1}{}{gris-clair-tab}
\begin{tabular}{|l|p{5cm}|p{7cm}|}\hline
%\begin{tabular}{|l|l|l|}\hline
\rowcolor{gris-tab-entete}\textbf{
Mode}&\textbf{
Type(s) d'opération(s)}&\textbf{
Effets}
\tabularnewline\hline
\mybox{r}&
Lecture&
Néant
\tabularnewline\hline
\mybox{r+}&
Lecture et écriture&
Néant
\tabularnewline\hline
\mybox{w}&
Écriture&
Si le fichier n'existe pas, il est créé.

Si le fichier existe, son contenu est effacé.
\tabularnewline\hline
\mybox{w+}&
Lecture et écriture&
\emph{Idem}
\tabularnewline\hline
\mybox{a}&
Écriture&
Si le fichier n'existe pas, il est créé.

Place les données à la fin du fichier
\tabularnewline\hline
\mybox{a+}&
Lecture et écriture&
\emph{Idem}
\tabularnewline\hline
\end{tabular}
\caption{Les modes d'accès à un fichier}
\end{table}

Par défaut, les flux sont des flux de textes. Pour obtenir un flux
binaire, il suffit d'ajouter la lettre \mybox{b} à la fin de la chaîne
décrivant le mode.

\subsubsection{Exemple}
\label{exemple-21}

Le code ci-dessous tente d'ouvrir un fichier nommé « texte.txt » en
lecture seule dans le dossier courant. Notez que dans le cas où il
n'existe pas, la fonction \mybox{fopen()} retournera un pointeur nul
(seul le mode \mybox{r} permet de produire ce comportement).

\begin{C}
#include <stdio.h>
#include <stdlib.h>


int main(void)
{
    FILE *fp;

    fp = fopen("texte.txt", "r");

    if (fp == NULL)
    {
        printf("Le fichier texte.txt n'a pas pu être ouvert\n");
        return EXIT_FAILURE;
    }

    printf("Le fichier texte.txt existe\n");
    return 0;
}
\end{C}

\subsection{La fonction fclose}
\label{la-fonction-fclose}

\begin{C}
int fclose(FILE *flux);
\end{C}

La fonction \mybox{fclose()} termine l'association entre un flux et un
fichier. S'il reste des données temporisées, celles-ci sont écrites. La
fonction retourne zéro en cas de succès et \mybox{EOF} en cas d'erreur.

\begin{infobox} 
 \mybox{EOF} est une constante
définie dans l'en-tête \mybox{\textless{}stdio.h\textgreater{}} et est
utilisée par les fonctions déclarées dans ce dernier pour indiquer soit
l'arrivée à la fin d'un fichier (nous allons y venir) soit la survenance
d'une erreur. La valeur de cette constante est \emph{toujours} un entier
\emph{négatif}.
\end{infobox}

Nous pouvons désormais compléter l'exemple précédent comme suit.

\begin{C}
#include <stdio.h>
#include <stdlib.h>


int main(void)
{
    FILE *fp;

    fp = fopen("texte.txt", "r");

    if (fp == NULL)
    {
        printf("Le fichier texte.txt n'a pas pu être ouvert\n");
        return EXIT_FAILURE;
    }

    printf("Le fichier texte.txt existe\n");

    if (fclose(fp) == EOF)
    {
        printf("Erreur lors de la fermeture du flux\n");
        return EXIT_FAILURE;        
    }

    return 0;
}
\end{C}

\begin{attentionbox} 
 Veillez qu'à chaque appel à la fonction
\mybox{fopen()} corresponde un appel à la fonction \mybox{fclose()}.
\end{attentionbox}

\section{Écriture vers un flux de texte }
\label{ecriture-vers-un-flux-de-texte }

\subsection{Écrire un caractère}
\label{ecrire-un-caractère}

\begin{C}
int putc(int ch, FILE *flux);
int fputc(int ch, FILE *flux);
int putchar(int ch);
\end{C}

Les fonctions \mybox{putc()} et \mybox{fputc()} écrivent un caractère
dans un flux. Il s'agit de l'opération d'écriture la plus basique sur
laquelle reposent toutes les autres fonctions d'écriture. Ces deux
fonctions retournent soit le caractère écrit, soit \mybox{EOF} si une
erreur est rencontrée. La fonction \mybox{putchar()}, quant à elle, est
identique aux fonctions \mybox{putc()} et \mybox{fputc()} si ce n'est
qu'elle écrit dans le flux \mybox{stdout}.

\begin{infobox} 
 Techniquement, \mybox{putc()} et
\mybox{fputc()} sont identiques, si ce n'est que \mybox{putc()} est en
fait le plus souvent une macrofonction. Étant donné que nous n'avons pas
encore vu de quoi il s'agit, préférez utiliser la fonction
\mybox{fputc()} pour l'instant.
\end{infobox}


L'exemple ci-dessous écrit le caractère « C » dans le fichier «
texte.txt ». Étant donné que nous utilisons le mode \mybox{w}, le
fichier est soit créé s'il n'existe pas, soit vidé de son contenu s'il
existe (revoyer le tableau des modes à la section précédente si vous
êtes perdus).

\begin{C}
#include <stdio.h>
#include <stdlib.h>


int main(void)
{
    FILE *fp;

    fp = fopen("texte.txt", "w");

    if (fp == NULL)
    {
        printf("Le fichier texte.txt n'a pas pu être ouvert\n");
        return EXIT_FAILURE;
    }
    if (fputc('C', fp) == EOF)
    {
        printf("Erreur lors de l'écriture d'un caractère\n");
        return EXIT_FAILURE;
    }
    if (fclose(fp) == EOF)
    {
        printf("Erreur lors de la fermeture du flux\n");
        return EXIT_FAILURE;        
    }

    return 0;

\end{C}

\subsection{Écrire une ligne}
\label{ecrire-une-ligne}

\begin{C}
int fputs(char *ligne, FILE *flux);
int puts(char *ligne);
\end{C}

La fonction \mybox{fputs()} écrit une ligne dans le flux \mybox{flux}.
La fonction retourne un nombre positif ou nul en cas de succès et
\mybox{EOF} en cas d'erreurs. La fonction \mybox{puts()} est identique
si ce n'est qu'elle ajoute automatiquement un caractère de fin de ligne
et qu'elle écrit sur le flux \mybox{stdout}.

\begin{infobox}
  Maintenant que nous savons comment
écrire une ligne dans un flux précis, nous allons pouvoir diriger nos
messages d'erreurs vers le flux \mybox{stderr} afin que ceux-ci soient
affichés le plus rapidement possible.
\end{infobox}


L'exemple suivant écrit le mot « Bonjour » suivi d'un caractère de fin
de ligne au sein du fichier « texte.txt ».

\begin{C}
#include <stdio.h>
#include <stdlib.h>


int main(void)
{
    FILE *fp;

    fp = fopen("texte.txt", "w");

    if (fp == NULL)
    {
        fputs("Le fichier texte.txt n'a pas pu être ouvert\n", stderr);
        return EXIT_FAILURE;
    }
    if (fputs("Bonjour\n", fp) == EOF)
    {
        fputs("Erreur lors de l'écriture d'une ligne\n", stderr);
        return EXIT_FAILURE;
    }
    if (fclose(fp) == EOF)
    {
        fputs("Erreur lors de la fermeture du flux\n", stderr);
        return EXIT_FAILURE;        
    }

    return 0;
}
\end{C}

\begin{attentionbox} 
 La norme\footnote{Programming Language
  C, X3J11/88-090, § 4.9.2, Streams, al. 4} vous garanti qu'une ligne peut
contenir jusqu'à 254 caractères (caractère de fin de ligne inclus).
Aussi, veillez à ne pas écrire de ligne d'une taille supérieure à cette
limite.
\end{attentionbox}


\subsection{La fonction fprintf}
\label{la-fonction-fprintf}

\begin{C}
int fprintf(FILE *flux, char *format, ...);
\end{C}

La fonction \mybox{fprintf()} est la même que la fonction
\mybox{printf()} si ce n'est qu'il est possible de lui spécifier sur
quel flux écrire (au lieu de \mybox{stdout} pour \mybox{printf()}).
Elle retourne le nombre de caractères écrits ou une valeur négative en
cas d'échec.

\section{Lecture depuis un flux de texte }
\label{lecture-depuis-un-flux-de-texte }

\subsection{Récupérer un caractère}
\label{recuperer-un-caractere}

\begin{C}
int getc(FILE *flux);
int fgetc(FILE *flux);
int getchar(void);
\end{C}

Les fonctions \mybox{getc()} et \mybox{fgetc()} sont les exacts
miroirs des fonctions \mybox{putc()} et \mybox{fputc()} : elles
récupèrent un caractère depuis le flux fourni en argument. Il s'agit de
l'opération de lecture la plus basique sur laquelle reposent toutes les
autres fonctions de lecture. Ces deux fonctions retournent soit le
caractère lu, soit \mybox{EOF} si la fin de fichier est rencontrée
\emph{ou} si une erreur est rencontrée. La fonction \mybox{getchar()},
quant à elle, est identique à ces deux fonctions si ce n'est qu'elle
récupère un caractère depuis le flux \mybox{stdin}.

\begin{infobox}
  Comme \mybox{putc()}, la fonction
\mybox{getc()} est le plus souvent une macrofonction. Utilisez donc
plutôt la fonction \mybox{fgetc()} pour le moment.
\end{infobox}

L'exemple ci-dessous lit un caractère provenant du fichier
\mybox{texte.txt}.

\begin{C}
#include <stdio.h>
#include <stdlib.h>


int main(void)
{
    FILE *fp;
    int ch;

    fp = fopen("texte.txt", "r");

    if (fp == NULL)
    {
        fprintf(stderr, "Le fichier texte.txt n'a pas pu être ouvert\n");
        return EXIT_FAILURE;
    }
    if ((ch = fgetc(fp)) != EOF)
        printf("%c\n", ch);
    if (fclose(fp) == EOF)
    {
        fprintf(stderr, "Erreur lors de la fermeture du flux\n");
        return EXIT_FAILURE;        
    }

    return 0;
}
\end{C}

\begin{infobox}
  Notez que nous utilisons une
affectation comme premier opérande de l'opérateur \mybox{!=}. Une
affectation étant une expression en C, ce genre d'écriture est tout à
fait valide. Vous en rencontrerez fréquemment comme expression de
contrôle de boucles.
\end{infobox}


\subsection{Récupérer une ligne}
\label{recuperer-une-ligne}

\begin{C}
char *fgets(char *tampon, int taille, FILE *flux);
\end{C}

La fonction \mybox{fgets()} lit une ligne depuis le flux \mybox{flux}
et la stocke dans le tableau \mybox{tampon}. Cette dernière lit au plus
un nombre de caractères égal à \mybox{taille} diminué de un afin de
laisser la place pour le caractère nul, qui est automatiquement ajouté.
Dans le cas où elle rencontre un caractère de fin de ligne :
\emph{celui-ci est conservé au sein du tableau}, un caractère nul est
ajouté et la lecture s'arrête.

La fonction retourne l'adresse du tableau \mybox{tampon} en cas de
succès et un pointeur nul si la fin du fichier est atteinte \emph{ou si
une erreur est survenue}.

L'exemple ci-dessous réalise donc la même opération que le code
précédent, mais en utilisant la fonction \mybox{fgets()}.

\begin{infobox} 
 Étant donné que la norme nous garanti
qu'une ligne peut contenir jusqu'à 254 caractères (caractère de fin de
ligne inclus), nous utilisons un tableau de 255 caractères pour les
contenir (puisqu'il est nécessaire de prévoir un espace pour le
caractère nul).
\end{infobox}


\begin{C}
#include <stdio.h>

int main(void)
{
    char buf[255];
    FILE *fp;

    fp = fopen("texte.txt", "r");

    if (fp == NULL)
    {
        fprintf(stderr, "Le fichier texte.txt n'a pas pu être ouvert\n");
        return EXIT_FAILURE;
    }
    if (fgets(buf, sizeof buf, fp) != NULL)
        printf("%s\n", buf);
    if (fclose(fp) == EOF)
    {
        fprintf(stderr, "Erreur lors de la fermeture du flux\n");
        return EXIT_FAILURE;        
    }

    return 0;
}
\end{C}

Toutefois, il y a un petit problème : la fonction \mybox{fgets()}
conserve le caractère de fin de ligne qu'elle rencontre. Dès lors, nous
affichons deux retours à la ligne : celui contenu dans la chaîne
\mybox{buf} et celui affiché par \mybox{printf()}. Aussi, il serait
préférable d'en supprimer un, de préférence celui de la chaîne de
caractères. Pour ce faire, nous pouvons faire appel à une petite
fonction (que nous appellerons \mybox{chomp()} en référence à la
fonction éponyme du langage Perl) qui se chargera de remplacer le
caractère de fin de ligne par un caractère nul.

\begin{C}
void chomp(char *s)
{
    while (*s != '\n' && *s != '\0')
        ++s;

    if (*s == '\n')
        *s = '\0';
}
\end{C}

\subsection{La fonction fscanf}
\label{la-fonction-fscanf}

\begin{C}
int fscanf(FILE *flux, char *format, ...);
\end{C}

La fonction \mybox{fscanf()} est identique à la fonction
\mybox{scanf()} si ce n'est qu'elle récupère les données depuis le flux
fourni en argument (au leu de \mybox{stdin} pour \mybox{scanf()}).

\begin{infobox} 
 Le flux \mybox{stdin} étant le plus
souvent mémorisé par lignes, ceci vous explique pourquoi nous lisions
les caractères restant après un appel à \mybox{scanf()} jusqu'à
rencontrer un caractère de fin de ligne : pour vider le tampon du flux
\mybox{stdin}.
\end{infobox}


La fonction \mybox{fscanf()} retourne le nombre de conversions réussies
(voire zéro, si aucune n'est demandée ou n'a pu être réalisée) ou
\mybox{EOF} si une erreur survient \emph{avant} qu'une conversion n'ait
eu lieu.

\section{Écriture vers un flux binaire}
\label{ecriture-vers-un-flux-binaire}

\subsection{Écrire un multiplet}
\label{ecrire-un-multiplet}

\begin{C}
int putc(int ch, FILE *flux);
int fputc(int ch, FILE *flux);
\end{C}

Comme pour les flux de texte, il vous est possible de recourir aux
fonctions \mybox{putc()} et \mybox{fputc()}. Dans le cas d'un flux
binaire, ces fonctions écrivent un multiplet (sous la forme d'un
\mybox{int} converti en \mybox{unsigned\ char}) dans le flux spécifié.

\subsection{Écrire une suite de multiplets}
\label{ecrire-une-suite-de-multiplets}

\begin{C}
size_t fwrite(void *ptr, size_t taille, size_t nombre, FILE *flux);
\end{C}

La fonction \mybox{fwrite()} écrit le tableau référencé par
\mybox{ptr} composé de \mybox{nombre} éléments de \mybox{taille}
multiplets dans le flux \mybox{flux}. Elle retourne une valeur égale à
\mybox{nombre} en cas de succès et une valeur inférieure en cas
d'échec.

L'exemple suivant écrit le contenu du tableau \mybox{tab} dans le
fichier \mybox{binaire.bin}. Dans le cas où un \mybox{int} fait 4
octets, 20 octets seront donc écrits.

\begin{C}
#include <stddef.h>
#include <stdio.h>
#include <stdlib.h>


int main(void)
{
    int tab[5] = { 1, 2, 3, 4, 5 };
    const size_t n = sizeof tab / sizeof tab[0];
    FILE *fp;

    fp = fopen("binaire.bin", "wb");

    if (fp == NULL)
    {
        fprintf(stderr, "Le fichier binaire.bin n'a pas pu être ouvert\n");
        return EXIT_FAILURE;
    }
    if (fwrite(&tab, sizeof tab[0], n, fp) != n)
    {
            fprintf(stderr, "Erreur lors de l'écriture du tableau\n");
            return EXIT_FAILURE;
    }
    if (fclose(fp) == EOF)
    {
        fprintf(stderr, "Erreur lors de la fermeture du flux\n");
        return EXIT_FAILURE;        
    }

    return 0;
}
\end{C}
 
\section{Lecture depuis un flux binaire}
\label{lecture-depuis-un-flux-binaire}
  
\subsection{Lire un multiplet}
\label{lire-un-multiplet}

\begin{C}
int getc(FILE *flux);
int fgetc(FILE *flux);
\end{C}

Lors de la lecture depuis un flux binaire, les fonctions
\mybox{fgetc()} et \mybox{getc()} permettent de récupérer un multiplet
(sous la forme d'un \mybox{unsigned\ char} converti en \mybox{int})
depuis un flux.

\subsection{Lire une suite de multiplets}
\label{lire-une-suite-de-multiplets}

\begin{C}
size_t fread(void *ptr, size_t taille, size_t nombre, FILE *flux);
\end{C}

La fonction \mybox{fread()} est l'inverse de la fonction
\mybox{fwrite()} : elle lit \mybox{nombre} éléments de \mybox{taille}
multiplets depuis le flux \mybox{flux} et les stocke dans l'objet
référencé par \mybox{ptr}. Elle retourne une valeur égale à
\mybox{nombre} en cas de succès ou une valeur inférieure en cas
d'échec.

Dans le cas où nous disposons du fichier \mybox{binaire.bin} produit
par l'exemple de la section précédente, nous pouvons reconstituer le
tableau \mybox{tab}.

\begin{C}
#include <stddef.h>
#include <stdio.h>
#include <stdlib.h>


int main(void)
{
    int tab[5] = { 1, 2, 3, 4, 5 };
    const size_t n = sizeof tab / sizeof tab[0];
    FILE *fp;
    unsigned i;

    fp = fopen("binaire.bin", "rb");

    if (fp == NULL)
    {
        fprintf(stderr, "Le fichier binaire.bin n'a pas pu être ouvert\n");
        return EXIT_FAILURE;
    }
    if (fread(&tab, sizeof tab[0], n, fp) != n)
    {
            fprintf(stderr, "Erreur lors de la lecture du tableau\n");
            return EXIT_FAILURE;
    }
    if (fclose(fp) == EOF)
    {
        fprintf(stderr, "Erreur lors de la fermeture du flux\n");
        return EXIT_FAILURE;        
    }

    for (i = 0; i < n; ++i)
        printf("tab[%u] = %d\n", i, tab[i]);

    return 0;
}
\end{C}

\hrulefill

Dans le chapitre suivant, nous continuerons notre découverte des
fichiers en attaquant quelques notions plus avancées : la gestion
d'erreur et de fin de fichier, le déplacement au sein d'un flux, la
temporisation et les subtilités liées aux flux ouverts en lecture et
écriture.