\chapter{L'allocation dynamique}
\label{lallocation-dynamique}

Il est à présent temps d'aborder la troisième et dernière utilisation
majeure des pointeurs : l'allocation dynamique de mémoire.

Comme nous vous l'avons dit dans le chapitre sur les pointeurs, il n'est
pas toujours possible de savoir quelle quantité de mémoire sera utilisée
par un programme. Par exemple, si vous demandez à l'utilisateur de vous
fournir un tableau, vous devrez lui fixer une limite, ce qui pose deux
problèmes :

\begin{itemize}
\item
  la limite en elle-même, qui ne convient peut-être pas à votre
  utilisateur ;
\item
  l'utilisation excessive de mémoire du fait que vous réservez un
  tableau d'une taille fixée à l'avance.
\end{itemize}

De plus, si vous utilisez un tableau de classe de stockage statique,
alors cette quantitée de mémoire superflue sera inutilisable jusqu'à la
fin de votre programme.

Or, un ordinateur ne dispose que d'une quantité limitée de mémoire vive,
il est donc important de ne pas en réserver abusivement. L'allocation
dynamique permet de réserver une partie de la mémoire vive inutilisée
pour stocker des données et de libérer cette même partie une fois
qu'elle n'est plus nécessaire.

\section{La notion d'objet}
\label{la-notion-dobjet}

Jusqu'à présent, nous avons toujours recouru au système de types et de
variables du langage C pour stocker nos données sans jamais vraiment
nous soucier de ce qui se passait « en-dessous ».

Il est à présent temps de lever une partie de ce voile en abordant la
notion d'\textbf{objet}.

En C, un objet est une zone mémoire pouvant contenir des données et est
composé d'une suite contiguë d'un ou plusieurs multiplets. En fait, tous
les types du langage C manipulent des objets. La différence entre les
types tient simplement en la manière dont ils répartissent les données
au sein de ces objets, ce qui est appelé leur \textbf{représentation}
(celle-ci sera abordée dans la troisième partie du cours). Ainsi, la
valeur 1 n'est pas représentée de la même manière dans un objet de type
\mybox{int} que dans un objet de type \mybox{double}.

Un objet étant une suite contiguë de multiplets, il est possible d'en
examiner le contenu en lisant ses multiplets un à un. Ceci peut se
réaliser en C à l'aide de l'adresse de l'objet et d'un pointeur sur
\mybox{unsigned\ char}, le type \mybox{char} du C ayant
\emph{toujours} la taille d'un multiplet. Notez qu'il est
\emph{impératif} d'utiliser la version non signée du type \mybox{char}
afin d'éviter des problèmes de conversions.

L'exemple ci-dessous affiche les multiplets composant un objet de type
\mybox{int} et un objet de type \mybox{double} en hexadécimal.

\begin{C}
#include <stdio.h>


int main(void)
{
    int n = 1;
    double f = 1.;
    unsigned char *byte;
    unsigned i;

    byte = (unsigned char *)&n;

    for (i = 0; i < sizeof n; ++i)
        printf("%x ", byte[i]);

    printf("\n");
    byte = (unsigned char *)&f;

    for (i = 0; i < sizeof f; ++i)
        printf("%x ", byte[i]);

    printf("\n");
    return 0;
}
\end{C}

\begin{C}
1 0 0 0 
0 0 0 0 0 0 f0 3f 
\end{C}

\begin{infobox}
Il se peut que vous n'obteniez pas le
même résultat que nous, ce dernier dépends de votre machine.
\end{infobox}

Comme vous le voyez, la représentation de la valeur 1 n'est pas du tout
la même entre le type \mybox{int} et le type \mybox{double}.

\section{Malloc et consoeurs}
\label{malloc-et-consoeurs}

La bibliothèque standard fourni trois fonctions vous permettant d'allouer 
de la mémoire : \mybox{malloc()}, \mybox{calloc()} et \mybox{realloc()} et 
une vous permettant de la libérer : \mybox{free()}. Ces quatres fonctions 
sont déclarées dans l'en-tête \mybox{\textless{}stdlib.h\textgreater{}}.

\subsection{malloc}
\label{malloc}

\begin{C}
void *malloc(size_t taille);
\end{C}

La fonction \mybox{malloc()} vous permet d'allouer un objet de la
taille fournie en argument (qui représente un nombre de multiplets) et
retourne l'adresse de cet objet sous la forme d'un pointeur générique.
En cas d'échec de l'allocation, elle retourne un pointeur nul.

\begin{attentionbox} 
 Vous devez \emph{toujours} vérifier le
retour d'une fonction d'allocation afin de vous assurer que vous
manipulez bien un pointeur valide.
\end{attentionbox}


\subsubsection{Allocation d'un objet}
\label{allocation-dun-objet}

Dans l'exemple ci-dessous, nous réservons un objet de la taille d'un
\mybox{int}, nous y stockons ensuite la valeur dix et l'affichons. Pour
cela, nous utilisons un pointeur sur \mybox{int} qui va se voir
affecter l'adresse de l'objet ainsi alloué et qui va nous permettre de
le manipuler comme nous le ferions s'il référençait une variable de type
\mybox{int}.

\begin{C}
#include <stdio.h>
#include <stdlib.h>


int main(void)
{
    int *p;

    p = malloc(sizeof(int));

    if (p == NULL)
    {
        printf("Échec de l'allocation\n");
        return EXIT_FAILURE;
    }

    *p = 10;
    printf("%d\n", *p);
    return 0;
}
\end{C}

\begin{C}
10
\end{C}

\subsubsection{Allocation d'un tableau}
\label{allocation-dun-tableau}

Pour allouer un tableau, vous devez réserver un bloc mémoire de la
taille d'un élément multiplié par le nombre d'éléments composant le
tableau. L'exemple suivant alloue un tableau de dix \mybox{int},
l'initialise et affiche son contenu.

\begin{C}
#include <stdio.h>
#include <stdlib.h>


int main(void)
{
    int *p;
    unsigned i;

    p = malloc(sizeof(int) * 10);

    if (p == NULL)
    {
        printf("Échec de l'allocation\n");
        return EXIT_FAILURE;
    }

    for (i = 0; i < 10; ++i)
    {
        p[i] = i * 10;
        printf("p[%u] = %d\n", i, p[i]);
    }

    return 0;
}
\end{C}

\begin{C}
p[0] = 0
p[1] = 10
p[2] = 20
p[3] = 30
p[4] = 40
p[5] = 50
p[6] = 60
p[7] = 70
p[8] = 80
p[9] = 90
\end{C}

\begin{infobox} 
 Autrement dit et de manière plus
générale : pour allouer dynamiquement un objet de type \textbf{T}, il
vous faut créer un pointeur sur le type \textbf{T} qui conservera son
adresse.
\end{infobox}


\begin{erreurbox} 
 La fonction \mybox{malloc()} n'effectue
\emph{aucune} initialisation, le contenu du bloc alloué est donc
indéterminé.
\end{erreurbox}


\subsection{free}
\label{free}

\begin{C}
void free(void *ptr);
\end{C}

La fonction \mybox{free()} libère le bloc précédemment alloué par une
fonction d'allocation dont l'adresse est fournie en argument. Dans le
cas où un pointeur nul lui est fourni, elle n'effectue aucune opération.

\begin{attentionbox}
Retenez bien la règle suivante : à chaque appel à une fonction d'allocation 
doit correspondre un appel à la fonction \mybox{free()}.
\end{attentionbox}


Dès lors, nous pouvons compléter les exemples précédents comme suit.

\begin{C}
#include <stdio.h>
#include <stdlib.h>


int main(void)
{
    int *p;

    p = malloc(sizeof(int));

    if (p == NULL)
    {
        printf("Échec de l'allocation\n");
        return EXIT_FAILURE;
    }

    *p = 10;
    printf("%d\n", *p);
    free(p);
    return 0;
}
\end{C}

\begin{C}
#include <stdio.h>
#include <stdlib.h>


int main(void)
{
    int *p;
    unsigned i;

    p = malloc(sizeof(int) * 10);

    if (p == NULL)
    {
        printf("Échec de l'allocation\n");
        return EXIT_FAILURE;
    }

    for (i = 0; i < 10; ++i)
    {
        p[i] = i * 10;
        printf("p[%u] = %d\n", i, p[i]);
    }

    free(p);
    return 0;
}
\end{C}

\begin{infobox}
Remarquez que même si le deuxième exemple alloue un tableau, il 
n'y a bien eu qu'\emph{une seule} allocation. Un seul appel à la 
fonction \mybox{free()} est donc nécessaire.
\end{infobox}


\subsection{calloc}
\label{calloc}

\begin{C}
void *calloc(size_t nombre, size_t taille);
\end{C}

La fonction \mybox{calloc()} attends deux arguments : le nombre
d'éléments à allouer et la taille de chacun de ces éléments.
Techniquement, elle revient au même que d'appelé \mybox{malloc()} comme
suit.

\begin{C}
malloc(nombre * taille);
\end{C}

à un détail près : \emph{le bloc de mémoire ainsi alloué est initialisé
à zéro}.

\begin{erreurbox}
Faites attention : cette initialisation n'est pas similaire à celle des 
variables de classe de stockage statique ! À l'inverse de ces dernières 
qui seront soit initialisées à zéro, soit seront des pointeurs nuls, 
l'initialisation réalisée par la fonction \mybox{calloc()} ne s'applique 
qu'aux entiers ou aux chaînes de caractères (nous verrons cela plus en détails 
dans la troisième partie du cours).
\end{erreurbox}

\subsection{realloc}
\label{realloc}

\begin{C}
void *realloc(void *p, size_t taille);
\end{C}

La fonction \mybox{realloc()} libère un bloc de mémoire précédemment
alloué, en réserve un nouveau de la taille demandée et copie le contenu
de l'ancien objet dans le nouveau. Dans le cas où la taille demandée est
inférieure à celle du bloc d'origine, le contenu de celui-ci sera copié
à hauteur de la nouvelle taille. À l'inverse, si la nouvelle taille est
supérieure à l'ancienne, l'excédant n'est pas initialisé.

La fonction attends deux arguments : l'adresse d'un bloc précédemment
alloué à l'aide d'une fonction d'allocation et la taille du nouveau bloc
à allouer. Elle retourne l'adresse du nouveau bloc ou un pointeur nul en
cas d'erreur.

L'exemple ci-dessous alloue un tableau de dix \mybox{int} et utilise
\mybox{realloc()} pour agrandir celui-ci à vingt \mybox{int}.

\begin{C}
#include <stdio.h>
#include <stdlib.h>


int main(void)
{
    int *p;
    int *tmp;
    unsigned i;

    p = malloc(sizeof(int) * 10);

    if (p == NULL)
    {
        printf("Échec de l'allocation\n");
        return EXIT_FAILURE;
    }

    for (i = 0; i < 10; ++i)
        p[i] = i * 10;

    tmp = realloc(p, sizeof(int) * 20);

    if (tmp == NULL)
    {
        printf("Échec de l'allocation\n");
        return EXIT_FAILURE;
    }

    p = tmp;

    for (i = 10; i < 20; ++i)
        p[i] = i * 10;

    for (i = 0; i < 20; ++i)
        printf("p[%u] = %d\n", i, p[i]);

    free(p);
    return 0;
}
\end{C}

\begin{C}
p[0] = 0
p[1] = 10
p[2] = 20
p[3] = 30
p[4] = 40
p[5] = 50
p[6] = 60
p[7] = 70
p[8] = 80
p[9] = 90
p[10] = 100
p[11] = 110
p[12] = 120
p[13] = 130
p[14] = 140
p[15] = 150
p[16] = 160
p[17] = 170
p[18] = 180
p[19] = 190
\end{C}

Remarquez que nous avons utilisé une autre variable, \mybox{tmp}, pour
vérifier le retour de la fonction \mybox{realloc()}. En effet, si nous
avions procéder comme ceci.

\begin{C}
p = realloc(p, sizeof(int) * 20);
\end{C}

il nous aurait été impossible de libérer le bloc mémoire référencé par
\mybox{p} en cas d'erreur puisque celui-ci serait devenu un pointeur
nul. Il est donc \emph{impératif} d'utiliser une seconde variable afin
d'éviter des \MYhref{http://fr.wikipedia.org/wiki/Fuite_de_mémoire}{fuites
de mémoire}.

\section{Les tableaux multidimensionnels}
\label{les-tableaux-multidimensionnels-2}

L'allocation de tableaux multidimensionnels est un petit peu
plus complexe que celles des autres objets. Techniquement, il existe
deux solutions : l'allocation d'un seul bloc de mémoire (comme pour les
tableaux simples) et l'allocation de plusieurs tableaux eux-mêmes
référencés par les éléments d'un autre tableau.

\subsection{Allocation en un bloc}
\label{allocation-en-un-bloc}

Comme pour un tableau simple, il vous est possible d'allouer un bloc de
mémoire dont la taille correspond à la multiplication des longueurs de
chaque dimension, elle-même multipliée par la taille d'un élément.
Toutefois, cette solution vous contraint à effectuer une partie du
calcul d'adresse vous-même puisque vous allouez en fait un seul tableau.

L'exemple ci-dessous illustre ce qui vient d'être dit en allouant un
tableau à deux dimensions de trois fois trois \mybox{int}.

\begin{C}
#include <stdio.h>
#include <stdlib.h>


int main(void)
{
    int *p;
    unsigned i;
    unsigned j;

    p = malloc(3 * 3 * sizeof(int));

    if (p == NULL)
    {
        printf("Échec de l'allocation\n");
        return EXIT_FAILURE;
    }

    for (i = 0; i < 3; ++i)
        for (j = 0; j < 3; ++j)
        {
            p[(i * 3) + j] = (i * 3) + j;
            printf("p[%u][%u] = %d\n", i, j, p[(i * 3) + j]);
        }

    free(p);
    return 0;
}
\end{C}

\begin{C}
p[0][0] = 0
p[0][1] = 1
p[0][2] = 2
p[1][0] = 3
p[1][1] = 4
p[1][2] = 5
p[2][0] = 6
p[2][1] = 7
p[2][2] = 8
\end{C}

Comme vous le voyez, une partie du calcul d'adresse doit être effectué
en multipliant le premièr indice par la longueur de la première
dimension, ce qui permet d'arriver à la bonne « ligne ». Ensuite, il ne
reste plus qu'à sélectionner le bon élément de la ligne à l'aide du
second indice.

Bien qu'un petit peu plus complexe quant à l'accès aux éléments, cette
solution à l'avantage de n'effectuer qu'une seule allocation de mémoire.

\subsection{Allocation de plusieurs tableaux}
\label{allocation-de-plusieurs-tableaux}

La seconde solution consiste à allouer plusieurs tableaux plus un autre
qui les référencera. Dans le cas d'un tableau à deux dimensions, cela
signifie allouer un tableau de pointeurs dont chaque élément se verra
affecter l'adresse d'un tableau également alloué dynamiquement. Cette
technique nous permet d'accéder aux éléments des différents tableaux de
la même manière que pour un tableau multidimensionnel puisque nous
utilisons cette fois plusieurs tableaux.

L'exemple ci-dessous revient au même que le précédent, mais utilise le
procédé qui vient d'être décrit. Notez que puisque nous réservons un
tableau de pointeurs sur \mybox{int}, l'adresse de celui-ci doit être
stockée dans un pointeur de pointeur sur \mybox{int} (rappelez-vous la
règle générale lors de la présentation de la fonction
\mybox{malloc()}).

\begin{C}
#include <stdio.h>
#include <stdlib.h>


int main(void)
{
    int **p;
    unsigned i;
    unsigned j;

    p = malloc(3 * sizeof(int *));

    if (p == NULL)
    {
        printf("Échec de l'allocation\n");
        return EXIT_FAILURE;
    }

    for (i = 0; i < 3; ++i)
    {
        p[i] = malloc(3 * sizeof(int));

        if (p[i] == NULL)
        {
            printf("Échec de l'allocation\n");
            return EXIT_FAILURE;
        }
    }

    for (i = 0; i < 3; ++i)
        for (j = 0; j < 3; ++j)
        {
            p[i][j] = (i * 3) + j;
            printf("p[%u][%u] = %d\n", i, j, p[i][j]);
        }

    for (i = 0; i < 3; ++i)
        free(p[i]);

    free(p);
    return 0;
}
\end{C}

Si cette solution permet de faciliter l'accès aux différents éléments,
elle présente toutefois l'inconvénient de réaliser plusieurs allocations
et donc de nécessiter plusieurs appels à la fonction \mybox{free()}.

\hrulefill

Ce chapitre nous aura permis de découvrir la notion d'objet ainsi que le
mécanisme de l'allocation dynamique. Dans le chapitre suivant, nous
verrons comment manipuler les \textbf{fichiers}.