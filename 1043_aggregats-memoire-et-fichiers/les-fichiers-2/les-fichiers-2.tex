\chapter{Les fichiers (2)}
\label{les-fichiers-(2)}

Dans ce chapitre, nous allons poursuivre notre lancée et vous présenter
plusieurs points un peu plus avancés en rapport avec les fichiers et les
flux.

\section{Détection d'erreurs et fin de fichier}
\label{detection-derreurs-et-fin-de-fichier}

Lors de la présentation des fonctions
\mybox{fgetc()}, \mybox{getc()} et \mybox{fgets()} vous aurez
peut-être remarqué que ces fonctions utilisent un même retour pour
indiqué soit la rencontre de la fin du fichier, soit la survenance d'une
erreur. Du coup, comment faire pour distinguer l'un et l'autre cas ?

Il existe deux fonctions pour clarifier une telle situation :
\mybox{feof()} et \mybox{ferror()}.

\subsection{La fonction feof}
\label{la-fonction-feof}

\begin{C}
int feof(FILE *flux);
\end{C}

La fonction \mybox{feof()} retourne une valeur non nulle dans le cas où
la fin du fichier associé au flux spécifié est atteinte.

\subsection{La fonction ferror}
\label{la-fonction-ferror}

\begin{C}
int ferror(FILE *flux);
\end{C}

La fonction \mybox{ferror()} retourne une valeur non nulle dans le cas
où une erreur s'est produite lors d'une opération sur le flux visé.

\subsection{Exemple}
\label{exemple-13}

L'exemple ci-dessous utilise ces deux fonctions pour déterminer le type
de problème rencontré par la fonction \mybox{fgets()}.

\begin{C}
#include <stdio.h>

int main(void)
{
    char buf[255];
    FILE *fp;

    fp = fopen("texte.txt", "r");

    if (fp == NULL)
    {
        fprintf(stderr, "Le fichier texte.txt n'a pas pu être ouvert\n");
        return EXIT_FAILURE;
    }
    if (fgets(buf, sizeof buf, fp) != NULL)
        printf("%s\n", buf);
    else if (ferror(fp))
    {
        fprintf(stderr, "Erreur lors de la lecture\n");
        return EXIT_FAILURE;
    }
    else
    {
        fprintf(stderr, "Fin de fichier rencontrée\n");
        return EXIT_FAILURE;
    }
    if (fclose(fp) == EOF)
    {
        fprintf(stderr, "Erreur lors de la fermeture du flux\n");
        return EXIT_FAILURE;        
    }

    return 0;
}
\end{C}

De manière imagée, un fichier peut être vu comme une longue bande
magnétique depuis laquelle des données sont lues ou sur laquelle des
informations sont écrites. Ainsi, au fur et à mesure que les données
sont lues ou écrites, nous avançons le long de cette bande.

Toutefois, il est parfois nécessaire de se rendre directement à un point
précis de cette bande (par exemple à la fin) afin d'éviter des lectures
inutiles pour parvenir à un endroit souhaité. À cet effet, la
bibliothèque standard fournit plusieurs fonctions.

\subsection{La fonction ftell}
\label{la-fonction-ftell}

\begin{C}
long ftell(FILE *flux);
\end{C}

La fonction \mybox{ftell()} retourne la position actuelle au sein du
fichier sous forme d'un \mybox{long}. Elle retourne un nombre négatif
en cas d'erreur. Dans le cas des flux \emph{binaires}, cette valeur
correspond au nombre de multiplets séparant la position actuelle du
début du fichier.

\begin{attentionbox}
  Lors de l'ouverture d'un flux, la
position courante correspond au début du fichier, \emph{sauf} pour les
modes \mybox{a} et \mybox{a+} pour lesquels la position initiale est
soit le début, soit la fin du fichier.
\end{attentionbox}


\subsection{La fonction fseek}
\label{la-fonction-fseek}

\begin{C}
int fseek(FILE *flux, long distance, int repere);
\end{C}

La fonction \mybox{fseek()} permet d'effectuer un déplacement d'une
distance fournie en argument depuis un repère donné. Elle retourne zéro
en cas de succès et une autre valeur en cas d'échec. Il existe trois
repères possibles :

\begin{itemize}
\item
  \mybox{SEEK\_SET} qui correspond au début du fichier ;
\item
  \mybox{SEEK\_CUR} qui correspond à la position courante ;
\item
  \mybox{SEEK\_END} qui correspond à la fin du fichier.
\end{itemize}

Cette fonction s'utilise différemment suivant qu'elle opère sur un flux
de texte ou sur un flux binaire.

\subsubsection{Les flux de texte}
\label{les-flux-de-texte}

Dans le cas d'un flux de texte, il y a deux possibilités :

\begin{itemize}
\item
  la distance fournie est nulle ;
\item
  la distance est une valeur fournie par un précédent appel à la
  fonction \mybox{ftell()} et le repère est \mybox{SEEK\_SET}.
\end{itemize}

\subsubsection{Les flux binaires}
\label{les-flux-binaires}

Dans le cas d'un flux binaire, seuls les repères \mybox{SEEK\_SET} et
\mybox{SEEK\_CUR} peuvent être utilisés. La distance correspond à un
nombre de multiplets à passer depuis le repère fourni.

\begin{attentionbox}
Dans la cas des modes \mybox{a} et \mybox{a+}, un déplacement a lieu 
à la fin du fichier \emph{avant chaque opération d'écriture} et ce, 
\emph{qu'il y ait eu déplacement auparavant ou non}.
\end{attentionbox}


\subsection{La fonction rewind}
\label{la-fonction-rewind}

\begin{C}
int fseek(FILE *flux, long distance, int repere);
\end{C}

La fonction \mybox{rewind()} vous ramène au début du fichier (autrement
dit, elle rembobine la bande).

\begin{erreurbox}
  Si vous utilisez un flux ouvert en lecture
\emph{et} écriture vous \emph{devez} appeler une fonction de déplacement
entre deux opérations de nature différentes ou utiliser la fonction
\mybox{fflush()} (présentée dans l'extrait suivant) entre une opération
d'écriture et de lecture. Si vous ne souhaitez pas vous déplacer, vous
pouvez utilisez l'appel \mybox{fseek(flux,\ 0,\ SEEK\_CUR)} afin de
respecter cette condition sans réellement effectuer un déplacement.
\end{erreurbox}

\section{La temporisation}
\label{la-temporisation}

Dans le chapitre précédent, nous vous avons précisé que les flux
étaient le plus souvent temporisés afin d'optimiser les opérations de
lecture et d'écriture sous-jacentes. Dans cette section, nous allons
nous pencher un peu plus sur cette notion.

\subsection{Introduction}
\label{introduction-4}

Nous vous avons dit auparavant que deux types de temporisations
existaient : la temporisation par lignes et celle par blocs. Une des
conséquences logiques de cette temporisation est que les fonctions de
lecture/écriture récupèrent les données et les inscrivent dans ces
tampons. Ceci peut paraître évident, mais cela peut avoir des
conséquences parfois surprenantes si ce fait est oublié ou inconnu.

\begin{C}
#include <stdio.h>
#include <stdlib.h>


int main(void)
{
    char nom[64];
    char prenom[64];

    printf("Quel est votre nom ? ");

    if (scanf("%63s", nom) != 1)
    {
        fprintf(stderr, "Erreur lors de la saisie\n");
        return EXIT_FAILURE;
    }

    printf("Quel est votre prénom ? ");

    if (scanf("%63s", prenom) != 1)
    {
        fprintf(stderr, "Erreur lors de la saisie\n");
        return EXIT_FAILURE;
    }

    printf("Votre nom est %s\n", nom);
    printf("Votre prénom est %s\n", prenom);
    return 0;
}
\end{C}

\begin{C}
Quel est votre nom ? Charles Henri
Quel est votre prénom ? Votre nom est Charles
Votre prénom est Henri
\end{C}

Comme vous le voyez, le programme ci-dessus réalise deux saisies, mais
si l'utilisateur entre par exemple « Charles Henri », il n'aura
l'occasion d'entrer des données qu'une seule fois. Ceci est dû au fait
que l'indicateur \mybox{s} récupère une suite de caractères exempte
d'espaces (ce qui fait qu'il s'arrête à « Charles ») et que la suite «
Henri » \emph{demeure dans le tampon du flux} \mybox{stdin}. Ainsi,
lors de la deuxième saisie, il n'est pas nécessaire de récupérer de
nouvelles données depuis le terminal puisqu'il y en a déjà en attente
dans le tampon, d'où le résultat obtenu.

Le même problème peut se poser si par exemple les données fournies ont
une taille supérieure par rapport à l'objet qui doit les accueillir.

\begin{C}
#include <stdio.h>
#include <stdlib.h>


int main(void)
{
    char chaine1[16];
    char chaine2[16];

    printf("Un morceau : ");

    if (fgets(chaine1, sizeof chaine1, stdin) == NULL)
    {
        printf("Erreur lors de la saisie\n");
        return EXIT_FAILURE;
    }

    printf("Un autre morceau : ");

    if (fgets(chaine2, sizeof chaine2, stdin) == NULL)
    {
        printf("Erreur lors de la saisie\n");
        return EXIT_FAILURE;
    }

    printf("%s ; %s\n", chaine1, chaine2);
    return 0;

\end{C}

\begin{C}
Un morceau : Une chaîne de caractères, vraiment, mais alors vraiment trop longue
Un autre morceau : Une chaîne de  ; caractères, vr
\end{C}

Ici, la chaîne entrée est trop importante pour être contenue dans
\mybox{chaine1}, les données non lues sont alors conservées dans le
tampon du flux \mybox{stdin} et lues lors de la seconde saisie (qui ne
lit par l'entièreté non plus).

\subsection{Intérargir avec la temporisation}
\label{interargir-avec-la-temporisation}

\subsubsection{Vider un tampon}
\label{vider-un-tampon}

Si la temporisation nous évite des coûts en terme d'opérations de
lecture/écriture, il nous est parfois nécessaire de passer outre cette
mécanique pour vider manuellement le tampon d'un flux.

\subsubsection{Opération de lecture}
\label{operation-de-lecture}

Si le tampon contient des données qui proviennent d'une opération de
lecture, celles-ci peuvent être abandonnées soit en appelant une
fonction de positionnement soit en lisant les données, tout simplement.
Il y a toutefois un bémol avec la première solution : les fonctions de
positionnement ne fonctionnent pas dans le cas où le flux ciblé est lié
à un périphérique « interactif », c'est-à-dire le plus souvent un
terminal.

Autrement dit, pour que l'exemple précédent recoure bien à deux saisies,
il nous est nécessaire de vérifier que la fonction \mybox{fgets()} a
bien lu un caractère \mybox{\textbackslash{}n} (qui signifie que la fin
de ligne est atteinte et donc celle du tampon s'il s'agit du flux
\mybox{stdin}). Si ce n'est pas le cas, alors il nous faut lire les
caractères restants jusqu'au \mybox{\textbackslash{}n} final (ou la fin
du fichier).

\begin{C}
#include <stdio.h>
#include <stdlib.h>
#include <string.h>


int vider_tampon(FILE *fp)
{
    int c;

    do
        c = fgetc(fp);
    while (c != '\n' && c != EOF);

    return ferror(fp) ? 0 : 1;
}


int main(void)
{
    char chaine1[16];
    char chaine2[16];

    printf("Un morceau : ");

    if (fgets(chaine1, sizeof chaine1, stdin) == NULL)
    {
        printf("Erreur lors de la saisie\n");
        return EXIT_FAILURE;
    }
    if (strchr(chaine1, '\n') == NULL)
        if (!vider_tampon(stdin))
        {
            fprintf(stderr, "Erreur lors de la vidange du tampon.\n");
            return EXIT_FAILURE;
        }

    printf("Un autre morceau : ");

    if (fgets(chaine2, sizeof chaine2, stdin) == NULL)
    {
        printf("Erreur lors de la saisie\n");
        return EXIT_FAILURE;
    }
    if (strchr(chaine2, '\n') == NULL)
        if (!vider_tampon(stdin))
        {
            fprintf(stderr, "Erreur lors de la vidange du tampon.\n");
            return EXIT_FAILURE;
        }

    printf("%s ; %s\n", chaine1, chaine2);
    return 0;
}
\end{C}

\begin{C}
Un morceau : Une chaîne de caractères vraiment, mais alors vraiment longue
Un autre morceau : Une autre chaîne de caractères
Une chaîne de  ; Une autre chaî
\end{C}

\subsubsection{Opération d'écriture}
\label{operation-decriture}

Si, en revanche, le tampon comprend des données en attente d'écriture,
il est possible de forcer celle-ci soit à l'aide d'une fonction de
positionnement, soit à l'aide de la fonction \mybox{fflush()}.

\begin{C}
int fflush(FILE *flux);
\end{C}

Celle-ci vide le tampon du flux spécifié et retourne zéro en cas de
succès ou \mybox{EOF} en cas d'erreur.

\begin{attentionbox}
  Notez bien que cette fonction ne peut
être employée que pour vider un tampon comprenant des données en attente
d'\emph{écriture}.
\end{attentionbox}


\subsubsection{Modifier un tampon}
\label{modifier-un-tampon}

Techniquement, il ne vous est pas possible de modifier directement le
contenu d'un tampon, ceci est réalisé par les fonctions de la
bibliothèque standard au gré de vos opérations de lecture et/ou
d'écriture. Il y a toutefois une exception à cette règle : la fonction
\mybox{ungetc()}.

\begin{C}
int ungetc(int ch, FILE *flux);
\end{C}

Cette fonction est un peu particulière : elle place le caractère
\mybox{ch} dans le tampon du flux \mybox{flux}. Ce caractère pourra
être lu lors d'un appel ultérieur à une fonction de \emph{lecture}. Elle
retourne le caractère ajouté en cas de succès et \mybox{EOF} en cas
d'échec.

Cette fonction est très utile dans le cas où les actions d'un programme
dépendent du contenu d'un flux. Imaginez par exemple que votre programme
doit déterminer si l'entrée standard contient une suite de caractères ou
un nombre et doit ensuite afficher celui-ci. Vous pourriez utiliser
\mybox{getchar()} pour récupérer le premier caractère et déterminer
s'il s'agit d'un chiffre. Toutefois, le premier caractère du flux est
alors lu et cela complique votre tâche pour la suite\ldots{} La fonction
\mybox{ungetc()} vous permet de résoudre ce problème en replaçant ce
caractère dans le tampon du flux \mybox{stdin}.

\begin{C}
#include <stdio.h>
#include <stdlib.h>

static void afficher_nombre(void);
static void afficher_chaine(void);


static void afficher_nombre(void)
{
    double f;

    if (scanf("%lf", &f) == 1)
        printf("Vous avez entré le nombre : %f\n", f);
    else
        fprintf(stderr, "Erreur lors de la saisie\n");
}


static void afficher_chaine(void)
{
    char chaine[255];

    if (scanf("%254s", chaine) == 1)
        printf("Vous avez entré la chaine : %s\n", chaine);
    else
        fprintf(stderr, "Erreur lors de la saisie\n");
}


int main(void)
{
    int ch;

    ch = getchar();

    if (ungetc(ch, stdin) == EOF)
    {
        fprintf(stderr, "Impossible de replacer un caractère\n");
        return EXIT_FAILURE;
    }

    switch (ch)
    {
    case '0':
    case '1':
    case '2':
    case '3':
    case '4':
    case '5':
    case '6':
    case '7':
    case '8':
    case '9':
        afficher_nombre();
        break;

    default:
        afficher_chaine();
        break;
    }

    return 0;
}
\end{C}

\begin{attentionbox}
  La fonction \mybox{ungetc()} ne vous
permet de replacer qu'\emph{un seul caractère} avant une opération de
\emph{lecture}.
\end{attentionbox}

\section{Flux ouverts en lecture et écriture}
\label{flux-ouverts-en-lecture-et-ecriture}

Pour clore ce chapitre, un petit mot sur le cas des flux ouverts en
lecture \emph{et} en écriture (soit à l'aide des modes \mybox{r+},
\mybox{w+}, \mybox{a+} et leurs équivalents binaires).

Si vous utilisez un tel flux, vous \emph{devez} appeler une fonction de
déplacement entre deux opérations de nature différentes ou utiliser la
fonction \mybox{fflush()} entre une opération d'écriture et de lecture.
Si vous ne souhaitez pas vous déplacer, vous pouvez utiliser l'appel
\mybox{fseek(flux,\ 0L,\ SEEK\_CUR)} afin de respecter cette condition
sans réellement effectuer un déplacement.

\begin{infobox}
Vous l'aurez sans doute compris : cette règle impose en fait de vider le 
tampon du flux entre deux opérations de natures différentes.
\end{infobox}

\begin{C}
#include <stdio.h>
#include <stdlib.h>


int main(void)
{
    char chaine[255];
    FILE *fp;

    fp = fopen("texte.txt", "w+");

    if (fp == NULL)
    {
        fprintf(stderr, "Le fichier texte.txt n'a pas pu être ouvert.\n");
        return EXIT_FAILURE;
    }
    if (fputs("Une phrase.\n", fp) == EOF)
    {
        fprintf(stderr, "Erreur lors de l'écriture d'une ligne.\n");
        return EXIT_FAILURE;
    }

    /* Retour au début : la condition est donc remplie. */

    if (fseek(fp, 0L, SEEK_SET) != 0)
    {
        fprintf(stderr, "Impossible de revenir au début du fichier.\n");
        return EXIT_FAILURE;
    }
    if (fgets(chaine, sizeof chaine, fp) == NULL)
    {
        fprintf(stderr, "Erreur lors de la lecture.\n");
        return EXIT_FAILURE;
    }

    printf("%s\n", chaine);

    if (fclose(fp) == EOF)
    {
        fputs("Erreur lors de la fermeture du flux\n", stderr);
        return EXIT_FAILURE;        
    }

    return 0;

\end{C}

\begin{C}
 Une phrase.
\end{C}

\hrulefill

Dans le chapitre suivant, nous ferons une petite pause
dans notre apprentissage afin de réaliser un jeu de \emph{Puissance 4}.