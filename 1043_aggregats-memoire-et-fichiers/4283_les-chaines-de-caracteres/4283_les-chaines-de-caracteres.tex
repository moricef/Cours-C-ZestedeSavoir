\chapter{Les chaînes de caractères}
\label{Les chaînes de caractères}

Dans ce chapitre, nous allons apprendre à manipuler du texte ou,
en langage C, des \textbf{chaînes de caractères}

\section{Qu'est-ce qu'une chaîne de caractères ?}
\label{qu-est-ce-qu-une-chaine-de-caractères ?}

Dans le chapitre sur les variables, nous avions mentionné le type \mybox{char}.
Pour rappel, nous vous avions dit que le type \mybox{char} servait surtout
au stockage de caractères, mais que comme ces derniers étaient stockés
dans l'ordinateur sous forme de nombres, il était également possible
d'utiliser ce type pour mémoriser des nombres.

Le seul problème, c'est qu'une variable de type \mybox{char} ne peut
stocker qu'une seule lettre, ce qui est insuffisant pour stocker une
phrase ou même un mot. Si nous voulons mémoriser un texte, nous avons
besoin d'un outil pour rassembler plusieurs lettres dans un seul objet,
manipulable dans notre langage. Cela tombe bien, nous en avons justement
découvert un au chapitre précédent : les tableaux.

C'est ainsi que le texte est géré en C : sous forme de tableaux de
\mybox{char} appelés \textbf{chaînes de caractères} (\emph{strings} en
anglais).

\subsection{Représentation en mémoire}
\label{representation-en-memoire-2}

Néanmoins, il y a une petite subtilité. Une chaîne de caractères est un
plus qu'un tableau : c'est un objet à part entière qui doit être
manipulable directement. Or, ceci n'est possible que si nous connaissons
sa taille.

\subsubsection{Avec une taille intégrée}
\label{avec-une-taille-integree}

Dans certains langages de programmation, les chaines de caractères sont
stockées sous la forme d'un tableau de \mybox{char} auquel est adjoint
un entier pour indiquer sa longueur. Plus précisément, lors de
l'allocation du tableau, le compilateur réserve un élément
supplémentaire pour conserver la taille de la chaîne. Ainsi, il est aisé
de parcourir la chaîne et de savoir quand la fin de celle-ci est
atteinte. De telles chaines de caractères sont souvent appelées des
\emph{Pascal strings}, s'agissant d'une convention apparue avec le
langage de programmation Pascal.

\subsubsection{Avec une sentinelle}
\label{avec-une-sentinelle}

Toutefois, une telle technique limite la taille des chaînes de
caractères à la capacité du type entier utilisé pour mémoriser la
longueur de la chaine. Dans la majorité des cas, il s'agit d'un
\mybox{unsigned\ char}, ce qui donne une limite de 255 caractères
maximum sur la plupart des machines. Pour ne pas subir cette limitation,
les concepteurs du langage C ont adopté une autre solution : la fin de
la chaîne de caractères sera indiquée par un caractère spécial, en
l'occurrence zéro (noté
\mybox{\textquotesingle{}\textbackslash{}0\textquotesingle{}}). Les
chaines de caractères qui fonctionnent sur ce principe sont appelées
\emph{null terminated strings}, ou encore \emph{C strings}.

Cette solution a toutefois deux inconvénients :

\begin{itemize}
\item
  la taille de chaque chaîne doit être calculée en la parcourant
  jusqu'au caractère nul ;
\item
  le programmeur doit s'assurer que chaque chaîne qu'il construit se
  termine bien par un caractère nul.
\end{itemize}

\section{Définition, initialisation et utilisation }
\label{definition,-initialisation-et-utilisation }

\subsection{Définition}
\label{definition-5}

Définir une chaine de caractères, c'est avant tout définir un tableau de
\mybox{char}, ce que nous avons vu au chapitre précédent. L'exemple
ci-dessous définit un tableau de vingt-cinq \mybox{char}.

\begin{C}
char tab[25];
\end{C}

\subsection{Initialisation}
\label{initialisation-6}

Il existe deux méthodes pour initialiser une chaîne de caractères :

\begin{itemize}
\item
  de la même manière qu'un tableau ;
\item
  à l'aide d'une chaîne de caractères littérale.
\end{itemize}

\subsubsection{Avec une liste d'initialisation}
\label{avec-une-liste-dinitialisation}

\subsubsection{Initialisation avec une longueur explicite}
\label{initialisation-avec-une-longueur-explicite-3}

Comme pour n'importe quel tableau, l'initialisation se réalise à l'aide
d'une liste d'initialisation. L'exemple ci-dessous définit donc un
tableau de vingt-cinq \mybox{char} et initialise les sept premiers avec
la suite de lettres « Bonjour ».

\begin{C}
char chaine[25] = { 'B', 'o', 'n', 'j', 'o', 'u', 'r' };
\end{C}

Étant donné que seule une partie des éléments sont initialisés, les
autres sont implicitement mis à zéro, ce qui nous donne une chaîne de
caractères valides puisqu'elle est bien terminée par un caractère nul.
Faites cependant attention à ce qu'il y ait \emph{toujours} de la place
pour un caractère nul.

\subsubsection{Initialisation avec une longueur implicite}
\label{initialisation-avec-une-longueur-implicite-3}

Dans le cas où vous ne spécifiez pas de taille lors de la définition, il
vous faudra ajouter le caractère nul à la fin de la liste
d'initialisation pour obtenir une chaîne valide.

\begin{C}
char chaine[] = { 'B', 'o', 'n', 'j', 'o', 'u', 'r', '\0' };
\end{C}

\subsubsection{Avec une chaîne littérale}
\label{avec-une-chaine-litterale}

Bien que tout à fait valide, cette première solution est toutefois assez
fastidieuse. Aussi, il en existe une seconde : recourir à une
\textbf{chaîne de caractères littérales} pour initialiser un tableau.
Une chaîne de caractères littérales est une suite de caractères entourée
par le symbole \mybox{"}. Nous en avons déjà utilisé auparavant comme
argument des fonctions \mybox{printf()} et \mybox{scanf()}.

Techniquement, une chaîne littérale est un tableau de \mybox{char}
terminé par un caractère nul. Elles peuvent donc s'utiliser comme
n'importe quel autre tableau. Si vous exécutez le code ci-dessous, vous
remarquerez que l'opérateur \mybox{sizeof} retourne bien le nombre de
caractères composant la chaîne littérale (n'oubliez pas de compter le
caractère nul) et que l'opérateur \mybox{{[}{]}} peut effectivement
leur être appliqué.

\begin{C}
#include <stdio.h>


int main(void)
{
    printf("%u\n", (unsigned)sizeof "Bonjour");
    printf("%c\n", "Bonjour"[3]);
    return 0;
}
\end{C}

\begin{C}
8
j
\end{C}

Ces chaînes de caractères littérales peuvent également être utilisées à
la place des listes d'initialisation. En fait, il s'agit de la troisième
et dernière exception à la règle de conversion implicite des tableaux.

\subsubsection{Initialisation avec une longueur explicite}
\label{initialisation-avec-une-longueur-explicite-4}

Dans le cas où vous spécifiez la taille de votre tableau, faites bien
attention à ce que celui-ci dispose de suffisamment de place pour
accueillir la chaîne entière, c'est-à-dire les caractères qui la
composent \emph{et} le caractère nul.

\begin{C}
char chaine[25] = "Bonjour";
\end{C}

\subsubsection{Initialisation avec une longueur implicite}
\label{initialisation-avec-une-longueur-implicite-4}

L'utilisation d'une chaîne littérale pour initialiser un tableau dont la
taille n'est pas spécifiée vous évite de vous soucier du caractère nul
puisque celui-ci fait partie de la chaîne littérale.

\begin{C}
char chaine[] = "Bonjour";
\end{C}

\subsubsection{Utilisation de pointeurs}
\label{utilisation-de-pointeurs}

Nous vous avons dit que les chaînes littérales n'étaient rien d'autre
que des tableaux de \mybox{char} terminés par un caractère nul. Dès
lors, comme pour n'importe quel tableau, il vous est loisible de les
référencer à l'aide de pointeurs.

\begin{C}
char *ptr = "Bonjour";
\end{C}

Néanmoins, les chaînes littérales sont des \emph{constantes}, il vous
est donc impossible de les modifier. L'exemple ci-dessous est donc
incorrect.

\begin{C}
int main(void)
{
    char *ptr = "bonjour";

    ptr[0] = 'B'; /* Incorrect. */
    return 0;
}
\end{C}

\begin{attentionbox}
  Notez bien la différence entre les
exemples précédents qui initialisent un tableau avec le contenu d'une
chaîne littérale (il y a donc copie de la chaîne littérale) et cet
exemple qui initialise un pointeur avec l'adresse du premier élément
d'une chaîne littérale.
\end{attentionbox}


\subsection{Utilisation}
\label{utilisation-1}

Pour le reste, une chaîne de caractères s'utilise comme n'importe quel
autre tableau. Aussi, pour modifier son contenu, il vous faudra accéder
à ses éléments un à un.

\begin{C}
#include <stdio.h>


int main(void)
{
    char chaine[25] = "Bonjour";

    printf("%s\n", chaine);
    chaine[0] = 'A';
    chaine[1] = 'u';
    chaine[2] = ' ';
    chaine[3] = 'r';
    chaine[4] = 'e';
    chaine[5] = 'v';
    chaine[6] = 'o';
    chaine[7] = 'i';
    chaine[8] = 'r';
    chaine[9] = '\0'; /* N'oubliez pas le caractère nul ! */
    printf("%s\n", chaine);
    return 0;
}
\end{C}

\begin{C}
Bonjour
Au revoir
\end{C}


\section{Afficher et récupérer une chaîne de caractères}
\label{afficher-etrecuperer-une-chaine-de-caracteres}

Les chaînes littérales n'étant rien
d'autre que des tableaux de \mybox{char}, il vous est possible
d'utiliser des chaînes de caractères là où vous employiez des chaînes
littérales. Ainsi, les deux exemples ci-dessous afficheront la même
chose.

\begin{C}
#include <stdio.h>


int main(void)
{
    char chaine[] = "Bonjour\n";

    printf(chaine);
    return 0;
}
\end{C}

\begin{C}
#include <stdio.h>


int main(void)
{
    printf("Bonjour\n");
    return 0;
}
\end{C}

\begin{C}
Bonjour
\end{C}

Toutefois, les fonctions \mybox{printf()} et \mybox{scanf()} disposent
d'un indicateur de conversion vous permettant d'afficher ou de demander
une chaîne de caractères : \mybox{s}.

\subsection{Printf}
\label{printf}

L'exemple suivant illustre l'utilisation de cet indicateur de conversion
avec \mybox{printf()} et affiche la même chose que les deux codes
précédents.

\begin{C}
#include <stdio.h>


int main(void)
{
    char chaine[] = "Bonjour";

    printf("%s\n", chaine);
    return 0;
}
\end{C}

\begin{C}
Bonjour
\end{C}

\subsection{Scanf}
\label{scanf-2}

Le même indicateur de conversion peut être utiliser avec
\mybox{scanf()} pour demander à l'utilisateur d'entrer une chaîne de
caractères. Cependant, un problème se pose : étant donné que nous devons
créer un tableau de taille finie pour accueillir la saisie de
l'utilisateur, nous devons impérativement limiter la longueur des
données que nous fournit l'utilisateur.

Pour éviter ce problème, il est possible de spécifier une taille
maximale à la fonction \mybox{scanf()}. Pour ce faire, il vous suffit
de placer un nombre entre le symbole \mybox{\%} et l'indicateur de
conversion \mybox{s}. L'exemple ci-dessous demande à l'utilisateur
d'entrer son prénom (limité à 254 caractères) et affiche ensuite
celui-ci.

\begin{erreurbox}
La fonction \mybox{scanf()} ne décompte
pas le caractère nul final de la limite fournie ! Il vous est donc
nécessaire de lui indiquer la taille de votre tableau diminuée de un.
\end{erreurbox}


\begin{C}
#include <stdio.h>
#include <stdlib.h>


int main(void)
{
    char chaine[255];

    printf("Quel est votre prénom ? ");

    if (scanf("%254s", chaine) != 1)
    {
        printf("Erreur lors de la saisie\n");
        return EXIT_FAILURE;
    }

    printf("Bien le bonjour %s !\n", chaine);
    return 0;
}
\end{C}

\begin{C}
Quel est votre prénom ? Albert
Bien le bonjour Albert !
\end{C}

\subsubsection{Chaîne de caractères avec des espaces}
\label{chaine-de-caracteres-avec-des-espaces}

Sauf qu'en fait, l'indicateur \mybox{s} signifie : « la plus longue
suite de caractère ne comprenant \emph{pas} d'espaces » (les espaces
étant ici entendu comme une suite d'un ou plusieurs des caractères
suivant : \mybox{\textquotesingle{}\ \textquotesingle{}},
\mybox{\textquotesingle{}\textbackslash{}f\textquotesingle{}},
\mybox{\textquotesingle{}\textbackslash{}n\textquotesingle{}},
\mybox{\textquotesingle{}\textbackslash{}r\textquotesingle{}},
\mybox{\textquotesingle{}\textbackslash{}t\textquotesingle{}},
\mybox{\textquotesingle{}\textbackslash{}v\textquotesingle{}})\ldots{}

Autrement dit, si vous entrez « Bonjour tout le monde », la fonction
\mybox{scanf()} va s'arrêter au mot « Bonjour », ce dernier étant suivi
par un espace.

\begin{questionbox}
Comment peut-on récupérer une chaîne complète alors ?
\end{questionbox}


\emph{Eh} bien, il va falloir nous débrouiller avec l'indicateur
\mybox{c} de la fonction \mybox{scanf()} et réaliser nous même une
fonction employant ce dernier au sein d'une boucle. Ainsi, nous pouvons
par exemple créer une fonction recevant un pointeur sur \mybox{char} et
la taille du tableau référencé qui lit un caractère jusqu'à être arrivé
à la fin de la ligne ou à la limite du tableau.

\begin{questionbox}
Et comment sait-on que la lecture est arrivée à la fin d'une ligne ?
\end{questionbox}


La fin de ligne est indiquée par le caractère
\mybox{\textbackslash{}n}.\\
Avec ceci, vous devriez pouvoir construire une fonction adéquate.\\
Allez, \emph{hop}, au boulot et faites gaffe aux retours d'erreur !


\begin{C}
#include <stdio.h>


int lire_ligne(char *chaine, size_t max)
{
    size_t i;
    char c;

    for (i = 0; i < max - 1; ++i)
    {
        if (scanf("%c", &c) != 1)
            return 0;
        else if (c == '\n')
            break;

        chaine[i] = c;
    }

    chaine[i] = '\0';
    return 1;
}


int main(void)
{
    char chaine[255];

    printf("Quel est votre prénom ? ");

    if (lire_ligne(chaine, sizeof chaine))
        printf("Bien le bonjour %s !\n", chaine);

    return 0;
}

\end{C}

\begin{C}
 Quel est votre prénom ? Charles Henri
Bien le bonjour Charles Henri !
\end{C}

\begin{infobox}
  Gardez bien cette fonction sous le
coude, nous allons en avoir besoin pour la suite.
\end{infobox}

\section{Lire et écrire depuis et dans une chaîne de caractères}
\label{lire-et-ecrire-depuis-et-dans-une-chaine-de-caractères}

S'il est possible d'afficher et récupérer une chaîne
de caractères, il est également possible de lire depuis une chaîne et
d'écrire dans une chaîne. À cette fin, deux fonctions qui devraient vous
sembler familières existent : \mybox{sprintf()} et \mybox{sscanf()}.

\subsection{La fonction sprintf}
\label{la-fonction-sprintf}

\begin{C}
int sprintf(char *chaine, char *format, ...);
\end{C}

\begin{infobox}
  Les trois petits points à la fin du
prototype de la fonction signifient que celle-ci attend un nombre
variable d'arguments. Nous verrons ce mécanisme plus en détail dans la
troisième partie du cours.
\end{infobox}


La fonction \mybox{sprintf()} est identique à la fonction
\mybox{printf()} mise à part que celle-ci écrit les données produites
dans une chaîne de caractères au lieu de les afficher à l'écran.
Celle-ci retourne le nombre de caractères écrit (sans compter le
caractère nul final !) ou bien un nombre négatif en cas d'erreur.

Cette fonction peut vous permettre, entre autres, d'écrire un nombre
dans une chaîne de caractères.

\begin{C}
#include <stdio.h>

int main(void)
{
    char chaine[16];
    int n = 64;

    sprintf(chaine, "%d", n);
    printf("%s\n", chaine);
    return 0;
}

\end{C}

\begin{C}
64
\end{C}

\begin{attentionbox}
 La fonction \mybox{sprintf()}
n'effectue \emph{aucune} vérification quant à la taille de la chaîne de
destination, vous devez donc vous assurer qu'elle dispose de
suffisamment de place pour accueillir la chaîne finale (caractère nul
compris !).
\end{attentionbox}


Comment dès lors s'assurer qu'il n'y aura aucun débordement ?
Malheureusement, il n'est pas possible de spécifier une taille comme
avec l'indicateur \mybox{s} de la fonction \mybox{scanf()}, aussi,
deux solutions s'offrent à vous :

\begin{itemize}
\item
  vérifier que le nombre en question ne dépasse pas un certain seuil ;
\item
  compter la quantité de chiffres composant le nombre avant d'appeler
  \mybox{sprintf()}.
\end{itemize}

Ainsi, l'exemple ci-dessous ne pose pas de problèmes puisque nous savons
que si le nombre est inférieur ou égal à 999 999 999, il n'excèdera pas
neuf caractères (\emph{n'oubliez pas de compter le caractère nul final !}).

\begin{C}
#include <stdio.h>
#include <stdlib.h>


int main(void)
{
    char chaine[10];
    long n;

    printf("Entrez un nombre : ");

    if (scanf("%ld", &n) != 1 || n > 999999999L)
    {
        printf("Erreur lors de la saisie\n");
        return EXIT_FAILURE;
    }

    sprintf(chaine, "%ld", n);
    printf("%s\n", chaine);
    return 0;
}
\end{C}

\begin{C}
Entrez un nombre : 890765456789
Erreur lors de la saisie

Entrez un nombre : 5678
5678
\end{C}

\subsection{La fonction sscanf}
\label{la-fonction-sscanf}

\begin{C}
int sscanf(char *chaine, char *format, ...);
\end{C}

La fonction \mybox{sscanf()} est identique à la fonction
\mybox{scanf()} si ce n'est que celle-ci extrait les données depuis une
chaîne de caractères plutôt qu'en provenance d'une saisie de
l'utilisateur. Cette dernière retourne le nombre de conversions réussies
\emph{ou} un nombre inférieur si elles n'ont pas toutes été réalisées
\emph{ou} enfin un nombre négatif en cas d'erreur.

Voici un exemple d'utilisation.

\begin{C}
#include <stdio.h>
#include <stdlib.h>


int main(void)
{
    char chaine[10];
    int n;

    if (sscanf("5 abcd", "%d %9s", &n, chaine) != 2)
    {
        printf("Erreur lors de l'examen de la chaîne\n");
        return EXIT_FAILURE;
    }

    printf("%d %s\n", n, chaine);
    return 0;
}
\end{C}

\begin{C}
5 abcd
\end{C}

\begin{infobox} 
Notez que la fonction \mybox{sscanf()} ne souffre pas du même problème
que \mybox{scanf()} en ce qui concerne de potentiels caractères non lus,
nous y reviendrons un peu plus tard.
\end{infobox}

\section{Les classes de caractères}
\label{les-classes-de-caracteres}

Pour terminer cette partie théorique sur une note un
peu plus légère, sachez que la bibliothèque standard 
fournie unen-tête \mybox{\textless{}ctype.h\textgreater{}}
qui permet de classifier les caractères. Onze fonctions sont
ainsi définies

\begin{C}
int isalnum(int c);
int isalpha(int c);
int iscntrl(int c);
int isdigit(int c);
int isgraph(int c);
int islower(int c);
int isprint(int c);
int ispunct(int c);
int isspace(int c);
int isupper(int c);
int isxdigit(int c);
\end{C}

Chacune d'entre elles attend en argument un caractère et retourne un
nombre positif ou zéro suivant que le caractère fourni appartienne ou
non à la catégorie déterminée par la fonction.

\begin{table}[ht!]
\tiny
\centering
\rowcolors{1}{}{gris-clair-tab}
%\begin{tabularx}{\textwidth}{|l|l|X|X|}\hline
\begin{tabular}{|p{1.5cm}|p{2cm}|p{3.5cm}|p{5.5cm}|}\hline
\rowcolor{gris-tab-entete}\textbf{Fonction} & \textbf{Catégorie} & \textbf{Description} & \textbf{Par défaut}\tabularnewline\hline
%\noindent\parbox[c]{\hsize}isupper &
isupper &
Majuscule & 
Détermine si le caractère entré est une lettre majuscule &
\mybox{\textquotesingle{}A\textquotesingle{}},
\mybox{\textquotesingle{}B\textquotesingle{}},
\mybox{\textquotesingle{}C\textquotesingle{}},
\mybox{\textquotesingle{}D\textquotesingle{}},
\mybox{\textquotesingle{}E\textquotesingle{}},
\mybox{\textquotesingle{}F\textquotesingle{}},
\mybox{\textquotesingle{}G\textquotesingle{}},
\mybox{\textquotesingle{}H\textquotesingle{}},
\mybox{\textquotesingle{}I\textquotesingle{}},
\mybox{\textquotesingle{}J\textquotesingle{}},
\mybox{\textquotesingle{}K\textquotesingle{}},
\mybox{\textquotesingle{}L\textquotesingle{}},
\mybox{\textquotesingle{}M\textquotesingle{}},
\mybox{\textquotesingle{}N\textquotesingle{}},
\mybox{\textquotesingle{}O\textquotesingle{}},
\mybox{\textquotesingle{}P\textquotesingle{}},
\mybox{\textquotesingle{}Q\textquotesingle{}},
\mybox{\textquotesingle{}R\textquotesingle{}},
\mybox{\textquotesingle{}S\textquotesingle{}},
\mybox{\textquotesingle{}T\textquotesingle{}},
\mybox{\textquotesingle{}U\textquotesingle{}},
\mybox{\textquotesingle{}V\textquotesingle{}},
\mybox{\textquotesingle{}W\textquotesingle{}},
\mybox{\textquotesingle{}X\textquotesingle{}},
\mybox{\textquotesingle{}Y\textquotesingle{}} ou
\mybox{\textquotesingle{}Z\textquotesingle{}}\tabularnewline\hline
islower&
Minuscule&
Détermine si le caractère entré est une lettre minuscule&
\mybox{\textquotesingle{}a\textquotesingle{}},
\mybox{\textquotesingle{}b\textquotesingle{}},
\mybox{\textquotesingle{}c\textquotesingle{}},
\mybox{\textquotesingle{}d\textquotesingle{}},
\mybox{\textquotesingle{}e\textquotesingle{}},
\mybox{\textquotesingle{}f\textquotesingle{}},
\mybox{\textquotesingle{}g\textquotesingle{}},
\mybox{\textquotesingle{}h\textquotesingle{}},
\mybox{\textquotesingle{}i\textquotesingle{}},
\mybox{\textquotesingle{}j\textquotesingle{}},
\mybox{\textquotesingle{}k\textquotesingle{}},
\mybox{\textquotesingle{}l\textquotesingle{}},
\mybox{\textquotesingle{}m\textquotesingle{}},
\mybox{\textquotesingle{}o\textquotesingle{}},
\mybox{\textquotesingle{}n\textquotesingle{}},
\mybox{\textquotesingle{}p\textquotesingle{}},
\mybox{\textquotesingle{}q\textquotesingle{}},
\mybox{\textquotesingle{}r\textquotesingle{}},
\mybox{\textquotesingle{}s\textquotesingle{}},
\mybox{\textquotesingle{}t\textquotesingle{}},
\mybox{\textquotesingle{}u\textquotesingle{}},
\mybox{\textquotesingle{}v\textquotesingle{}},
\mybox{\textquotesingle{}w\textquotesingle{}},
\mybox{\textquotesingle{}x\textquotesingle{}},
\mybox{\textquotesingle{}y\textquotesingle{}} ou
\mybox{\textquotesingle{}z\textquotesingle{}}\tabularnewline\hline
isdigit&
Chiffre décimal&
Détermine si le caractère entré est un chiffre décimal&
\mybox{\textquotesingle{}0\textquotesingle{}},
\mybox{\textquotesingle{}1\textquotesingle{}},
\mybox{\textquotesingle{}2\textquotesingle{}},
\mybox{\textquotesingle{}3\textquotesingle{}},
\mybox{\textquotesingle{}4\textquotesingle{}},
\mybox{\textquotesingle{}5\textquotesingle{}},
\mybox{\textquotesingle{}6\textquotesingle{}},
\mybox{\textquotesingle{}7\textquotesingle{}},
\mybox{\textquotesingle{}8\textquotesingle{}} ou
\mybox{\textquotesingle{}9\textquotesingle{}}\tabularnewline\hline
isxdigit&
Chiffre hexadécimal&
Détermine si le caractère entré est un chiffre hexadécimal&
\mybox{\textquotesingle{}0\textquotesingle{}},
\mybox{\textquotesingle{}1\textquotesingle{}},
\mybox{\textquotesingle{}2\textquotesingle{}},
\mybox{\textquotesingle{}3\textquotesingle{}},
\mybox{\textquotesingle{}4\textquotesingle{}},
\mybox{\textquotesingle{}5\textquotesingle{}},
\mybox{\textquotesingle{}6\textquotesingle{}},
\mybox{\textquotesingle{}7\textquotesingle{}},
\mybox{\textquotesingle{}8\textquotesingle{}},
\mybox{\textquotesingle{}9\textquotesingle{}},
\mybox{\textquotesingle{}A\textquotesingle{}},
\mybox{\textquotesingle{}B\textquotesingle{}},
\mybox{\textquotesingle{}C\textquotesingle{}},
\mybox{\textquotesingle{}D\textquotesingle{}},
\mybox{\textquotesingle{}E\textquotesingle{}},
\mybox{\textquotesingle{}F\textquotesingle{}},
\mybox{\textquotesingle{}a\textquotesingle{}},
\mybox{\textquotesingle{}b\textquotesingle{}},
\mybox{\textquotesingle{}c\textquotesingle{}},
\mybox{\textquotesingle{}d\textquotesingle{}},
\mybox{\textquotesingle{}e\textquotesingle{}} ou
\mybox{\textquotesingle{}f\textquotesingle{}}\tabularnewline\hline
isspace&
Espace&
Détermine si le caractère entré est un espace&
\mybox{\textquotesingle{}\ \textquotesingle{}},
\mybox{\textquotesingle{}\textbackslash{}f\textquotesingle{}},
\mybox{\textquotesingle{}\textbackslash{}n\textquotesingle{}},
\mybox{\textquotesingle{}\textbackslash{}r\textquotesingle{}},
\mybox{\textquotesingle{}\textbackslash{}t\textquotesingle{}} ou
\mybox{\textquotesingle{}\textbackslash{}v\textquotesingle{}}\tabularnewline\hline
iscntrl&
Contrôle&
Détermine si le caractère est un caractère dit « de contrôle »&
\mybox{\textquotesingle{}\textbackslash{}0\textquotesingle{}},
\mybox{\textquotesingle{}\textbackslash{}a\textquotesingle{}},
\mybox{\textquotesingle{}\textbackslash{}b\textquotesingle{}},
\mybox{\textquotesingle{}\textbackslash{}f\textquotesingle{}},
\mybox{\textquotesingle{}\textbackslash{}n\textquotesingle{}},
\mybox{\textquotesingle{}\textbackslash{}r\textquotesingle{}},
\mybox{\textquotesingle{}\textbackslash{}t\textquotesingle{}} ou
\mybox{\textquotesingle{}\textbackslash{}v\textquotesingle{}}\tabularnewline\hline
ispunct&
Ponctuation&
Détermine si le caractère entré est un signe de ponctuation&
\mybox{\textquotesingle{}!\textquotesingle{}},
\mybox{\textquotesingle{}"\textquotesingle{}},
\mybox{\textquotesingle{}\#\textquotesingle{}},
\mybox{\textquotesingle{}\%\textquotesingle{}},
\mybox{\textquotesingle{}\&\textquotesingle{}},
\mybox{\textquotesingle{}\textquotesingle{}\textquotesingle{}},
\mybox{\textquotesingle{}(\textquotesingle{}},
\mybox{\textquotesingle{})\textquotesingle{}},
\mybox{\textquotesingle{}*\textquotesingle{}},
\mybox{\textquotesingle{}+\textquotesingle{}},
\mybox{\textquotesingle{},\textquotesingle{}},
\mybox{\textquotesingle{}-\textquotesingle{}},
\mybox{\textquotesingle{}.\textquotesingle{}},
\mybox{\textquotesingle{}/\textquotesingle{}},
\mybox{\textquotesingle{}:\textquotesingle{}},
\mybox{\textquotesingle{};\textquotesingle{}},
\mybox{\textquotesingle{}\textless{}\textquotesingle{}},
\mybox{\textquotesingle{}=\textquotesingle{}},
\mybox{\textquotesingle{}\textgreater{}\textquotesingle{}},
\mybox{\textquotesingle{}?\textquotesingle{}},
\mybox{\textquotesingle{}{[}\textquotesingle{}},
\mybox{\textquotesingle{}\textbackslash{}\textquotesingle{}},
\mybox{\textquotesingle{}{]}\textquotesingle{}},
\mybox{\textquotesingle{}\^{}\textquotesingle{}},
\mybox{\textquotesingle{}\_\textquotesingle{}},
\mybox{\textquotesingle{}\{\textquotesingle{}},
\mybox{\textquotesingle{}\textless{}barre\ droite\textgreater{}\textquotesingle{}},
\mybox{\textquotesingle{}\}\textquotesingle{}} ou
\mybox{\textquotesingle{}\textasciitilde{}\textquotesingle{}}\tabularnewline\hline
isalpha&
Alphabétique&
Détermine si le caractère entré est une lettre alphabétique&
Les deux ensembles de caractères de \mybox{islower()} et
\mybox{isupper()}\tabularnewline\hline
isalnum&
Alphanumérique&
Détermine si le caractère entré est une lettre alphabétique ou un
chiffre décimal&
Les trois ensembles de caractères de \mybox{islower()},
\mybox{isupper()} et \mybox{isdigit()}\tabularnewline\hline
isgraph&
Graphique&
Détermine si le caractère est représentable graphiquement&
Tout sauf l'ensemble de \mybox{iscntrl()} et l'espace
(\mybox{\textquotesingle{}\ \textquotesingle{}})\tabularnewline\hline
isprint&
Affichable&
Détermine si le caractère est « affichable »&
Tout sauf l'ensemble de \mybox{iscntrl()}\tabularnewline\hline
%\end{tabularx}
\end{tabular}
\end{table}

\begin{infobox}
La suite \mybox{\textless{}barre\_droite\textgreater{}} symbolise 
le caractère \mybox{\textbar{}}.
\end{infobox}

Le tableau ci-dessous vous présente chacune de ces onze fonctions ainsi
que les ensembles de caractères qu'elles décrivent. La colonne « par
défaut » vous détail leur comportement en cas d'utilisation de la
\textbf{\emph{locale}} \mybox{C}. Nous reviendrons sur les
\emph{locales} dans la troisième partie de ce cours ; pour l'heure,
considérez que ces fonctions ne retournent un nombre positif que si un
des caractères de leur ensemble « par défaut » leur est fourni en
argument.

L'exemple ci-dessous utilise la fonction \mybox{isdigit()} pour
déterminer si l'utilisateur a bien entré une suite de chiffres.

\begin{C}
#include <stddef.h>
#include <stdio.h>
#include <stdlib.h>

/* lire_ligne() */


int main(void)
{
    char suite[255];
    unsigned i;

    if (!lire_ligne(suite, sizeof suite))
    {
        printf("Erreur lors de la saisie.\n");
        return EXIT_FAILURE;
    }

    for (i = 0; suite[i] != '\0'; ++i)
        if (!isdigit(suite[i]))
        {
            printf("Veuillez entrer une suite de chiffres.\n");
            return EXIT_FAILURE;
        }

    printf("C'est bien une suite de chiffres.\n");
    return 0;
}

\end{C}

\begin{C}
122334
C'est bien une suite de chiffres.

5678a
Veuillez entre une suite de chiffres.
\end{C}

\begin{infobox}
Notez que cet en-tête fourni également deux fonctions : \mybox{tolower()}
et \mybox{toupper()} retournant respectivement la version minuscule ou
majuscule de la lettre entrée. Dans le cas où un caractère autre qu'une
lettre est entré (ou que celle-ci est déjà en minuscule ou en majuscule),
la fonction retourne celui-ci.
\end{infobox}

\section{Exercices }
\label{exercices-5}

\subsection{Palindrome}
\label{palindrome}

Un palindrome est un texte identique lorsqu'il est lu de gauche à droite
et de droite à gauche. Ainsi, le mot \emph{radar} est un palindrome, de
même que les phrases \emph{engage le jeu que je le gagne} et \emph{élu
par cette crapule}. Normalement, il n'est pas tenu compte des accents,
trémas, cédilles ou des espaces. Toutefois, pour cet exercice, nous nous
contenterons de vérifier si un mot donné est un palindrome.

\begin{C}
int palindrome(char *s)
{
    size_t len = 0;
    size_t i;

    while (s[len] != '\0')
        ++len;

    for (i = 0; i < len; ++i)
        if (s[i] != s[len - 1 - i])
            return 0;

    return 1;
}
\end{C}

\subsection{Compter les parenthèses}
\label{compter-les-parentheses}

Écrivez un programme qui lit une ligne et vérifie que chaque parenthèse
ouvrante est bien refermée par la suite.

\begin{C}
Entrez une ligne : printf("%u\n", (unsigned)sizeof(int);
Il manque des parenthèses.
\end{C}

\begin{C}
#include <stdio.h>
#include <stdlib.h>

/* lire_ligne() */


int main(void)
{
    char s[255];
    char *t;
    long n = 0;

    printf("Entrez une ligne : ");

    if (!lire_ligne(s, sizeof s))
    {
        printf("Erreur lors de la saisie.\n");
        return EXIT_FAILURE;
    }

    for (t = s; *t != '\0'; ++t)
        if (*t == '(')
            ++n;
        else if (*t == ')')
            --n;

    if (n == 0)
        printf("Le compte est bon.\n");
    else
        printf("Il manque des parenthèses.\n");

    return 0;
}
\end{C}

\hrulefill

Le chapitre suivant sera l'occasion de mettre en pratique ce que nous
avons vu dans les chapitres précédents à l'aide d'un TP.