\chapter{Le préprocesseur}
\label{le-preprocesseur}

Le \textbf{préprocesseur} est un programme qui réalise
des traitements sur le code source \emph{avant} que ce dernier ne soit
réellement compilé. Globalement, il a trois grands rôles :

\begin{itemize}
\item
  réaliser des \textbf{inclusions} (la fameuse directive
  \mybox{\#include}) ;
\item
  définir des \textbf{macros} qui sont des substituts à des morceaux de
  code. Après le passage du préprocesseur, tous les appels à ces macros
  seront remplacés par le code associé ;
\item
  permettre la \textbf{compilation conditionnelle}, c'est-à-dire de
  moduler le contenu d'un fichier source suivant certaines
  conditions.
\end{itemize}
  
\section{Les inclusions}
\label{les-inclusions}

Nous avons vu dès le début du cours comment inclure des fichiers d'en-tête
avec la directive \mybox{\#include}, sans toutefois véritablement
expliquer son rôle. Son but est très simple : inclure le contenu d'un
fichier dans un autre. Ainsi, si nous nous retrouvons par exemple avec
deux fichiers comme ceux-ci avant la compilation.

\begin{C}
/* Fichier d'en-tête fichier.h */

#ifndef FICHIER_H
#define FICHIER_H

extern int glob_var;

extern void f1(int);
extern long f2(double, char);

#endif
\end{C}

\begin{C}
/* Fichier source fichier.c */

#include "fichier.h"

void f1(int arg)
{
    /* du code */
}

long f2(double arg, char c)
{
    /* du code */
}
\end{C}

Après le passage du préprocesseur et avant la compilation à proprement
parler, le code obtenu sera le suivant :

\begin{C}
/* Fichier source fichier.c */

extern int glob_var;

extern void f1(int arg);
extern long f2(double arg, char c);

void f1(int arg)
{
    /* du code */
}

long f2(double arg, char c)
{
    /* du code */
}
\end{C}

Nous pouvons voir que les déclarations contenues dans le fichier «
fichier.h » ont été incluses dans le fichier « fichier.c » et que toutes
les directives du préprocesseur (les lignes commençant par le symbole
\mybox{\#}) ont disparu.

\begin{infobox}
  Vous pouvez utiliser l'option
\mybox{-E} lors de la compilation pour requérir uniquement
l'utilisation du préprocesseur. Ainsi, vous pouvez voir ce que donne
votre code \emph{après} son passage. 
\begin{minted}
 $ zcc -E fichier.c
\end{C}
\end{minted}
\end{infobox}

\section{Les macroconstantes}
\label{les-macroconstantes}

Comme nous l'avons dit dans l'introduction, le
  préprocesseur permet la définition de macros, c'est-à-dire de
  substituts à des morceaux de code. Une macro est constituée des
  éléments suivants.
  
\begin{itemize}
\item
  La directive \mybox{\#define}.
\item
  Le nom de la macro qui, par convention, est souvent écrit en
  majuscules. Notez que vous pouvez choisir le nom que vous voulez, à
  condition de respecter les mêmes règles que pour les noms de variable
  ou de fonction.
\item
  Une liste \emph{optionnelle} de paramètres.
\item
  La définition, c'est-à-dire le code par lequel la macro sera
  remplacée.
\end{itemize}

Dans le cas particulier où la macro n'a pas de paramètre, on parle de
\textbf{macroconstante} (ou constante de préprocesseur). Leur
substitution donne toujours le même résultat (d'où l'adjectif «
constante »).

\subsection{Substitutions de constantes}
\label{substitutions-de-constantes}

Par exemple, pour définir une macroconstante \mybox{TAILLE} avec pour
valeur \mybox{100}, nous utiliserons le code suivant.

\begin{C}
#define TAILLE 100
\end{C}

L'exemple qui suit utilise cette macroconstante afin de ne pas
multiplier l'usage de constantes entières. À chaque fois que la
macroconstante \mybox{TAILLE} est utilisée, le préprocesseur remplacera
celle-ci par sa définition, à savoir \mybox{100}.

\begin{C}
#include <stdio.h>
#define TAILLE 100

int main(void)
{
    int variable = 5;

    /* On multiplie par TAILLE */
    variable *= TAILLE;
    printf("Variable vaut : %d\n", variable);

    /* On additionne TAILLE */
    variable += TAILLE;
    printf("Variable vaut : %d\n", variable);
    return 0;
}
\end{C}

Ce code sera remplacé, après le passage du préprocesseur, par celui
ci-dessous.

\begin{C}
int main(void)
{
    int variable = 5;

    variable *= 100;
    printf("Variable vaut : %d\n", variable);
    variable += 100;
    printf("Variable vaut : %d\n", variable);
    return 0;
}
\end{C}

Nous n'avons pas inclus le contenu de
\mybox{\textless{}stdio.h\textgreater{}} car celui-ci est trop long et
trop compliqué pour le moment. Néanmoins, l'exemple permet d'illustrer
le principe des macros et surtout de leur avantage : il suffit de
changer la définition de la macroconstante \mybox{TAILLE} pour que le
reste du code s'adapte.

\begin{questionbox} 
 D'accord, mais je peux aussi très bien
utiliser une variable constante, non ?
\end{questionbox}


Dans certains cas (comme dans l'exemple ci-dessus), utiliser une
variable constante donnera le même résultat. Toutefois, une variable
nécessitera le plus souvent de réserver de la mémoire et, si celle-ci
doit être partagée par différents fichiers, de jouer avec les
déclarations et les définitions.

De plus, les macroconstantes peuvent être employées pour définir la
longueur d'un tableau, alors que ceci est impossible avec une variable,
même constante.

\begin{infobox}
  Notez qu'il vous est possible de
définir une macroconstante lors de la compilation à l'aide de l'option
\mybox{-D}.
\begin{C}
 zcc -DTAILLE=100
\end{C}
Ceci recient à définir une macroconstante
\mybox{TAILLE} au début de chaque fichier qui sera substituée par
\mybox{100}.
\end{infobox}


\subsection{Substitutions d'instructions}
\label{substitutions-dinstructions}

Si les macroconstantes sont souvent utilisées en vue de substituer des
constantes, il est toutefois aussi possible que leur définition
comprenne des suites d'instructions. L'exemple ci-dessous défini une
macroconstante \mybox{BONJOUR} qui sera remplacée par
\mybox{puts("Bonjour\ !")}.

\begin{C}
#include <stdio.h>

#define BONJOUR  puts("Bonjour !")

int main(void)
{
    BONJOUR;
    return 0;
}
\end{C}

Notez que nous n'avons pas placé de point-virgule dans la définitions de
la macroconstante, tout d'abord afin de pouvoir en placer un lors de son
utilisation, ce qui est plus naturel et, d'autre part, pour éviter des
doublons qui peuvent s'avérer facheux.

En effet, si nous ajoutons un point virgule à la fin de la définition,
il ne faut pas en rajouter un lors de l'appel, sous peine de risquer des
erreurs de syntaxe.

\begin{C}
#include <stdio.h>

#define BONJOUR puts("Bonjour !");
#define AUREVOIR puts("Au revoir !");

int main(void)
{
    if (1)
        BONJOUR; /* erreur ! */
    else
        AUREVOIR;

    return 0;
}
\end{C}

Le code donnant en réalité ceci.

\begin{C}
int main(void)
{
    if (1)
        puts("Bonjour !");;
    else
        puts("Au revoir !");;

    return 0;
}
\end{C}

Ce qui est incorrect. Pour faire simple : le corps d'une condition ne
peut comprendre qu'une instruction, or \mybox{puts("Bonjour\ !");} en
est une, le point-virgule seul (\mybox{;}) en est une autre. Dès lors,
il y a une instruction entre le \mybox{if} et le \mybox{else}, ce qui
n'est pas permis. L'exemple ci-dessous pose le même problème en étant un
peu plus limpide.

\begin{C}
#include <stdio.h>


int main(void)
{
    if (1)
        puts("Bonjour !");

    puts("Je suis une instruction mal placee !");

    else
        puts("Au revoir !");

    return 0;
}
\end{C}

\subsection{Macros dans d'autres macros}
\label{macros-dans-dautres-macros}

La définition d'une macro peut parfaitement comprendre une ou plusieurs
autres macros (qui, à leur tour, peuvent également comprendre des macros
dans leur définition et ainsi de suite). L'exemple suivant illustre ceci
en définissant une macroconstante \mybox{NB\_PIXELS} correspondant à la
multiplication entre les deux macrococonstantes \mybox{LONGUEUR} et
\mybox{HAUTEUR}.

\begin{C}
#define LONGUEUR 1024
#define HAUTEUR 768
#define NB_PIXELS (LONGUEUR * LARGEUR)
\end{C}

La règle suivante permet d'éviter de répéter les opérations de
remplacement « à l'infini » :

\begin{erreurbox}
  Une macro n'est pas substituée si elle est
utilisée dans sa propre définition.
\end{erreurbox}


Ainsi, dans le code ci-dessous, \mybox{LONGUEUR} donnera
\mybox{LONGUEUR\ *\ 2\ /\ 2} et \mybox{HAUTEUR} sera substituée par
\mybox{HAUTEUR\ /\ 2\ *\ 2}. En effet, lors de la substitution de la
macroconstante \mybox{LONGUEUR}, la macroconstante \mybox{HAUTEUR} est
remplacée par sa définition : \mybox{LONGUEUR\ *\ 2}. Or, comme nous
sommes en train de substituer la macroconstante \mybox{LONGUEUR},
\mybox{LONGUEUR} sera laissé tel quel. Le même raisonnement peut être
appliqué pour la macroconstante \mybox{HAUTEUR}.

\begin{C}
#define LONGUEUR (HAUTEUR / 2)
#define HAUTEUR (LONGUEUR * 2)
\end{C}

\subsection{Définition sur plusieurs lignes}
\label{definition-sur-plusieurs-lignes}

La définition d'une macro doit normalement tenir sur une seule ligne.
C'est-à-dire que dès que la fin de ligne est atteinte, la définition est
considérée comme terminée. Ainsi, dans le code suivant, la définition de
la macroconstante \mybox{BONJOUR\_AUREVOIR} ne comprend en fait que
\mybox{puts("Bonjour");}.

\begin{C}
#define BONJOUR_AUREVOIR puts("Bonjour");
puts("Au revoir")
\end{C}

Pour que cela fonctionne, nous sommes donc contraints de la rédiger
comme ceci.

\begin{C}
#define BONJOUR_AUREVOIR puts("Bonjour"); puts("Au revoir")
\end{C}

Ce qui est assez peu lisible et peu pratique si notre définition doit
comporter une multitude d'instructions ou, pire, des blocs
d'instructions\ldots{} Heureusement, il est possible d'indiquer au
préprocesseur que plusieurs lignes doivent être fusionnées. Cela se fait
à l'aide du symbole \mybox{\textbackslash{}}, que nous placons en fin
de ligne.

\begin{C}
#define BONJOUR_AUREVOIR puts("Bonjour");\
puts("Au revoir")
\end{C}

Lorsque le préprocesseur rencontre une barre oblique inverse
(\mybox{\textbackslash{}}) suivie d'une fin de ligne, celles-ci sont
supprimées et la ligne qui suit est fusionnée avec la ligne courante.

\begin{attentionbox}
  Notez bien qu'\emph{aucun espace} n'est
ajouté durant l'opération ! 
\begin{C}
#define BONJOUR_AUREVOIR\
puts("Bonjour");\
puts("Au revoir")                         
\end{C}
Le code ci-dessus est donc incorrect car il donne
en vérité ceci.
\begin{C}
 #define BONJOUR_AUREVOIRputs("Bonjour");puts("Au revoir")
\end{C}
\end{attentionbox}

\begin{infobox}
  L'utilisation de la fusion ne se
limite pas aux définitions de macros, il est possible de l'employer
n'importe où dans le code source. Bien que cela ne soit pas obligatoire
(une instruction pouvant être répartie sur plusieurs lignes), il est
possible d'y recourir pour couper une ligne un peu trop longue tout en
indiquant cette césure de manière claire. 
\begin{C}
 printf("Une phrase composée de plusieurs résultats à présenter : %d, %f, %f, %d, %d\n",\
a, b, c, d, e, f);
\end{C}
\end{infobox}

Si vous souhaitez inclure un bloc d'instructions au sein d'une
définition, ne perdez pas de vue que celui-ci constitue une instruction
et donc que vous retomberez sur le problème exposé plus haut.

\begin{C}
#include <stdio.h>

#define BONJOUR\
    {\
        puts("Bonjour !");\
    }
#define AUREVOIR\
    {\
        puts("Au revoir !");\
    }

int main(void)
{
    if (1)
        BONJOUR; /* erreur ! */
    else
        AUREVOIR;

    return 0;
}
\end{C}

Une solution fréquente à ce problème consiste à recourir à une boucle
\mybox{do\ \{\}\ while} avec une condition nulle (de sorte qu'elle ne
soit exécutée qu'une seule fois), celle-ci ne constituant qu'une seule
instruction \emph{avec} le point-virgule final.

\begin{C}
#include <stdio.h>

#define BONJOUR\
    do {\
        puts("Bonjour !");\
    } while(0)
#define AUREVOIR\
    do {\
        puts("Au revoir !");\
    } while(0)

int main(void)
{
    if (1)
        BONJOUR; /* ok */
    else
        AUREVOIR;

    return 0;
}
\end{C}

\subsubsection{Définition nulle}
\label{definition-nulle}

Sachez que le corps d'une macro peut parfaitement être vide.

\begin{C}
#define MACRO
\end{C}

Dans un tel cas, la macro ne sera tout simplement pas substituée par
quoi que ce soit. Bien que cela puisse paraître étrange, cette technique
est souvent utilisée, notamment pour la compilation conditionnelle que
nous verrons bientôt.

\subsection{Annuler une définition}
\label{annuler-une-definition}

Enfin, une définition peut être annulée à l'aide de la directive
\mybox{\#undef} en lui spécifiant le nom de la macro qui doit être
détruite.

\begin{C}
#define TAILLE 100
#undef TAILLE

/* TAILLE n'est à présent plus utilisable. */
\end{C}

\section{Les macrofonctions}
\label{les-macrofonctions}

Pour l'instant, nous n'avons manipulé que des macroconstantes,
c'est-à-dire des macros n'employant pas de paramètres. Comme vous vous
en doutez, une \textbf{macrofonction} est une macro qui accepte des
paramètres et les emploie dans sa définition.

\begin{C}

\end{C}

Pour ce faire, le nom de la macro est suivi de parenthèses comprenant le
nom des paramètres séparés par une virgule. Chacun d'eux peut ensuite
être utilisé dans la définition de la macrofonction et sera remplacé par
la suite fournie en argument lors de l'appel à la macrofonction.

\begin{erreurbox} 
 Notez bien que nous n'avons pas parlé de «
valeur » pour les arguments. En effet, n'importe quelle suite de
symboles peut-être passée en argument d'une macrofonction, y compris du
code. C'est d'ailleurs ce qui fait la puissance du préprocesseur.
\end{erreurbox}


Illustrons ce nouveau concept avec un exemple : nous allons écrire deux
macros : \mybox{EUR} qui convertira une somme en euro en francs
(français) et \mybox{FRF} qui fera l'inverse. Pour rappel, un euro
équivaut à 6,55957 francs français.

\begin{C}
#include <stdio.h>

#define EUR(x) ((x) / 6.55957)
#define FRF(x) ((x) * 6.55957)

int main(void)
{
   printf("Dix francs français valent %f euros.\n", EUR(10));
   printf("Dix euros valent %f francs français.\n", FRF(10));
   return 0;
}
\end{C}

\begin{C}
Dix francs français valent 1.524490 euros.
Dix euros valent 65.595700 francs français.
\end{C}


Appliquons encore ce concept avec un deuxième exercice : essayez de
créer la macro \mybox{MIN} qui renvoie le minimum entre deux nombres.

\begin{C}
 #include <stdio.h>

#define MIN(a, b)  ((a) < (b) ? (a) : (b))

int main(void)
{
   printf("Le minimum entre 16 et 32 est %d.\n", MIN(16, 32));
   printf("Le minimum entre 2+9+7 et 3*8 est %d.\n", MIN(2+9+7, 3*8));
   return 0;
}
\end{C}

\begin{C}
Le minimum entre 16 et 32 est 16.
Le minimum entre 2+9+7 et 3*8 est 18.
\end{C}

\begin{infobox}
 Remarquez que nous avons utilisés des expressions
composées lors de la deuxième utilisation de la macrofonction
\mybox{MIN}.
\end{infobox}


\subsection{Priorité des opérations}
\label{priorite-des-operations}

Quelque chose vous a peut-être frappé dans les corrections : pourquoi
écrire \mybox{(x)} et pas simplement \mybox{x} ?

En fait, il s'agit d'une protection en vue d'éviter certaines
ambiguïtés. En effet, si l'on n'y prend pas garde, on peut par exemple
avoir des surprises dues à la priorité des opérateurs. Prenons l'exemple
d'une macro \mybox{MUL} qui effectue une multiplication.

\begin{C}
#define MUL(a, b)  (a * b)
\end{C}

Tel quel, le code peut poser des problèmes. En effet, si nous appellons
la macrofonction comme ceci.

\begin{C}
MUL(2+3, 4+5)
\end{C}

Nous obtenons comme résultat 19 (la macro sera remplacée par
\mybox{2\ +\ 3\ *\ 4\ +\ 5}) et non 45, qui est le résultat attendu.
Pour garantir la bonne marche de la macrofonction, nous devons rajouter
des parenthèses.

\begin{C}
#define MUL(a, b)  ((a) * (b))
\end{C}

Dans ce cas, nous obtenons bien le résultat souhaité, c'est-à-dire 45
(\mybox{(2\ +\ 3)\ *\ (4\ +\ 5)}).

\begin{infobox}
  Nous vous conseillons de rajouter des
parenthèses en cas de doute pour éviter toute erreur.
\end{infobox}


\subsection{Les effets de bords}
\label{les-effets-de-bords}

Pour finir, une petite mise en garde : évitez d'utiliser plus d'une fois
un paramètre dans la définition d'une macro en vue d'éviter de
multiplier d'éventuels \textbf{effets de bord}.

\begin{questionbox}
  Des effets de quoi ?
\end{questionbox}


Un effet de bord est une modification du contexte d'exécution. Vous
voilà bien avancé nous direz-vous\ldots{} En fait, vous en avez déjà
rencontré, l'exemple le plus typique étant une affectation.

\begin{C}
a = 10;
\end{C}

Dans cet exemple, le contexte d'exécution du programme (qui comprends
ses variables) est modifié puisque la valeur d'une variable est changée.
Ainsi, imaginez que la macro \mybox{MUL} soit appelée comme suit.

\begin{C}
MUL(a = 10, a = 20)
\end{C}

Après remplacement, celle-ci donnerait l'expression suivante :
\mybox{((a\ =\ 10)\ *\ (a\ =\ 20))} qui est assez problématique\ldots{}
En effet, quelle sera la valeur de \mybox{a}, finalement ? 10 ou 20 ?
Ceci est impossible à dire sans fixer une règle d'évaluation et\ldots{}
la norme n'en prévoit aucune dans ce cas ci.

Aussi, pour éviter ce genre de problèmes tordus, veillez à n'utiliser
chaque paramètre qu'\emph{une seule fois}.

\section{Les directives conditionnelles}
\label{les-directives-conditionnelles}

Le préprocesseur dispose de cinq directives conditionnelles :\mybox{\#if},
\mybox{\#ifdef}, \mybox{\#ifndef}, \mybox{\#elif},et \mybox{\#else}. Ces 
dernières permettent de conserver ou non une potion de code en fonction
de la validité d'une condition (si la condition est vraie, le code est gardé 
sinon il est passé). Chacune de ces directives (ou suite de directives)
doit être terminée par une directive \mybox{\#endif}.

\subsection{Les directives \#if, \#elif et \#else}
\label{les-directives-if-elif-et-else}

\subsubsection{Les directives \#if et \#elif}
\label{les-directives-if-et-elif}

Les directives \mybox{\#if} et \mybox{\#elif}, comme l'instruction
\mybox{if}, attendent une expression conditionelle ; toutefois,
celle-ci sont plus restreintes étant donné que nous sommes dans le cadre
du préprocesseur. Ce dernier se contentant d'effectuer des
substitutions, il n'a par exemple aucune connaissance des mots-clés du
langage C ou des variables qui sont employées.

Ainsi, les conditions ne peuvent comporter que des expressions
\emph{entières} (ce qui exclut les nombres flottants et les pointeurs)
et \emph{constantes} (ce qui exclut l'utilisation d'opérateurs à effets
de bord comme \mybox{=}). Aussi, les mots-clés et autres
identificateurs (hormis les macros) présent dans la définition
\emph{sont ignorés} (plus précisément, ils sont remplacés par
\mybox{0}).

L'exemple ci-dessous explicite ceci.

\begin{C}
#if 1.89 > 1.88 /* Incorrect, ce ne sont pas des entiers */
#if sizeof(int) == 4 /* Équivalent à 0(0) == 4, `sizeof' et `int' étant des mots-clés */
#if (a = 20) == 20 /* Équivalent à (0 = 20) == 4, `a' étant un identificateur */
\end{C}

Techniquement, seuls les opérateurs suivant peuvent être utilisés :
\mybox{+} (binaire ou unaire), \mybox{-} (binaire ou unaire),
\mybox{*}, \mybox{/}, \mybox{\%}, \mybox{==}, \mybox{!=},
\mybox{\textless{}=}, \mybox{\textless{}}, \mybox{\textgreater{}},
\mybox{\textgreater{}=}, \mybox{!}, \mybox{\&\&},
\mybox{\textbar{}\textbar{}}, \mybox{,}, l'opérateur ternaire et les
opérateurs de manipulations des \emph{bits}, que nous verrons au
chapitre suivant.

\begin{infobox}
  Bien entendu, vous pouvez également
utiliser des parenthèses afin de régler la priorité des opérations.
\end{infobox}


\subsubsection{L'opérateur defined}
\label{loperateur-defined}

Le préprocesseur fourni un opérateur supplémentaire utilisable dans les
condtions : \mybox{defined}. Celui-ci prend comme opérande un nom de
macro et retourne 1 ou 0 suivant que ce nom correspond ou non à une
macro définie.

\begin{C}
#if defined TAILLE
\end{C}

Celui-ci est fréquemment utilisé pour produire des programmes portables.
En effet, chaque système et chaque compilateur définissent généralement
une ou des macroconstantes qui lui sont propres propre. En vérifiant si
une de ces constantes existe, nous pouvons déterminer sur quelle
plate-forme et avec quel compilateur nous compilons et adapter le code
en conséquence.

\begin{attentionbox}
  Notez que ces constantes sont propres à
un système d'exploitation et/ou compilateur donné, elles ne sont donc
pas spécifiées par la norme du langage C.
\end{attentionbox}


\subsubsection{La directive \#else}
\label{la-directive-else}

La directive \mybox{\#else} quant à elle, se comporte comme
l'instruction éponyme.

\subsubsection{Exemple}
\label{exemple-14}

Voici un exemple d'utilisation.

\begin{C}
#include <stdio.h>

#define A 2

int main(void)
{
#if A < 0
    puts("A < 0");
#elif A > 0
    puts("A > 0");
#else
    puts("A == 0");
#endif
    return 0;
}
\end{C}

\begin{C}
A > 0
\end{C}

\begin{infobox} 
 Notez que les directives
conditionnelles peuvent être utilisées à la place des commentaires en
vue d'empêcher la compilation d'un morceau de code. L'avantage de cette
technique par rapport aux commentaires est qu'elle vous évite de
produire des commentaires imbriqués.
\begin{C}
 #if 0
/* On multiplie par TAILLE */
variable *= TAILLE;
printf("Variable vaut : %d\n", variable);
#endif
\end{C}
\end{infobox}


\subsection{Les directives \#ifdef et \#ifndef}
\label{les-directives-ifdef-et-ifndef}

Ces deux directives sont en fait la contraction, respectivement, de la
directive \mybox{\#if\ defined} et \mybox{\#if\ !defined}. Si vous
n'avez qu'une seule constante à tester, il est plus rapide d'utiliser
ces deux directives à la place de leur version longue.

\subsubsection{Protection des fichiers d'en-tête}
\label{protection-des-fichiers-den-tuxeate}

Ceci étant dit, vous devriez à présent être en mesure de comprendre le
fonctionnement des directives jusqu'à présent utilisées dans les
fichiers en-têtes.

\begin{C}
#ifndef CONSTANTE_H
#define CONSTANTE_H

/* Les déclarations */

#endif
\end{C}

La première directive vérifie que la macroconstante CONSTANTE\_H n'a pas
encore été définie. Si ce n'est pas le cas, elle est définie (ici avec
une valeur nulle) et le bloc de la condition (comprenant le contenu du
fichier d'en-tête) est inclus. Si elle a déjà été définie, le bloc est
sauté et le contenu du fichier n'est ainsi pas à nouveau inclus.

\begin{infobox}
  Pour rappel, ceci est nécessaire pour
éviter des problèmes d'inclusions multiples, par exemple lorsqu'un
fichier A inclut un fichier B, qui lui-même inclut le fichier A.
\end{infobox}

\hrulefill

Avec ce chapitre, vous devriez pouvoir utiliser le préprocesseur de manière 
basique et éviter plusieurs de ses pièges fréquents. Toutefois, si vous 
souhaitez aller plus loin et appronfondir son utilisation, nous vous conseillons 
la lecture \MYhref{https://openclassrooms.com/courses/le-preprocesseur-3}
{du cours de Pouet\_forever}.