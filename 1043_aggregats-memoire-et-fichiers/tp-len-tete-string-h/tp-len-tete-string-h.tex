\chapter{TP : l'en-tête <string.h>}
\label{tp-:-len-tete-<string.h>}

Les quatre derniers cours ayant été riche en nouveautés, 
posons nous un instant pour mettre en pratique tout cela.

\section{Préparation }
\label{preparation }

Dans le chapitre précédent, nous vous avons dit qu'une chaîne de caractères
se manipulait comme un tableau, à savoir en parcourant ses éléments un à
un. Cependant, si cela s'arrêtait là, cela serait assez gênant. En
effet, la moindre opération sur une chaîne nécessiterait d'accéder à ses
différents éléments, que ce soit une modification, une copie, une
comparaison, etc. Heureusement pour nous, la bibliothèque standard
fourni une suite de fonction nous permettant de réaliser plusieurs
opérations de base sur des chaînes de caractères. Ces fonctions sont
déclarées dans l'en-tête \mybox{\textless{}string.h\textgreater{}}.

L'objectif de ce TP sera de réaliser une version pour chacune des
fonctions de cet en-tête qui vont vous êtes présentées. :)

\subsection{strlen}
\label{strlen-1}

\begin{C}
size_t strlen(char *chaine);
\end{C}

La fonction \mybox{strlen()} vous permet de connaître la taille d'une
chaîne fournie en argument. Celle-ci retourne une valeur de type
\mybox{size\_t}. Notez bien que la longueur retournée ne comprend
\emph{pas} le caractère nul. L'exemple ci-dessous affiche la taille de
la chaîne « Bonjour ».

\begin{C}
#include <stdio.h>
#include <string.h>


int main(void)
{
    printf("Longueur : %u\n", (unsigned)strlen("Bonjour"));
    return 0;
}
\end{C}

\begin{C}
Longueur : 7
\end{C}

\subsection{strcpy}
\label{strcpy-1}

\begin{C}
char *strcpy(char *destination, char *source);
\end{C}

La fonction \mybox{strcpy()} copie le contenu de la chaîne
\mybox{source} dans la chaîne \mybox{destination}, caractère nul
compris. La fonction retourne l'adresse de \mybox{destination}.
L'exemple ci-dessous copie la chaîne « Au revoir » dans la chaîne
\mybox{chaine}.

\begin{C}
#include <stdio.h>


int main(void)
{
    char chaine[25] = "Bonjour\n";

    strcpy(chaine, "Au revoir");
    printf("%s\n", chaine);
    return 0;
}
\end{C}

\begin{C}
Au revoir
\end{C}

\begin{erreurbox}
  La fonction \mybox{strcpy()} n'effectue
\emph{aucune} vérifications. Vous devez donc vous assurer que la chaîne
de destination dispose de suffisamment d'espace pour accueillir la
chaîne qui doit être copiée (caractère nul compris !).
\end{erreurbox}


\subsection{strcat}
\label{strcat-1}

\begin{C}
char *strcat(char *destination, char *source);
\end{C}

La fonction \mybox{strcat()} ajoute le contenu de la chaine
\mybox{source} à celui de la chaîne \mybox{destination}, caractère nul
compris. La fonction retourne l'adresse de \mybox{destination}.
L'exemple ci-dessous ajoute la chaîne « tout le monde » au contenu de la
chaîne \mybox{chaine}.

\begin{C}
#include <stdio.h>
#include <string.h>


int main(void)
{
    char chaine[25] = "Bonjour";

    strcat(chaine, " tout le monde");
    printf("%s\n", chaine);
    return 0;
}
\end{C}

\begin{C}
Bonjour tout le monde
\end{C}

\begin{erreurbox}
  Comme \mybox{strcpy()}, la fonction
\mybox{strcat()} n'effectue \emph{aucune} vérifications. Vous devez
donc vous assurer que la chaîne de destination dispose de suffisamment
d'espace pour accueillir la chaîne qui doit être ajoutée (caractère nul
compris !).
\end{erreurbox}


\subsection{strcmp}
\label{strcmp-1}

\begin{C}
int strcmp(char *chaine1, char *chaine2);
\end{C}

La fonction \mybox{strcmp()} compare deux chaines de caractères. Cette
fonction retourne :

\begin{itemize}
\item
  une valeur positive si la première chaîne est « plus grande » que la
  seconde ;
\item
  zéro si elles sont égales ;
\item
  une valeur négative si la seconde chaîne est « plus grande » que la
  première.
\end{itemize}

\begin{infobox}
  Ne vous inquiétez pas au sujet des
valeurs positives et négatives, nous y reviendront en temps voulu
lorsque nous aborderons les notions de jeux de caractères et d'encodages
dans la troisième partie du cours. En attendant, effectuez simplement
une comparaison entre les deux caractères qui diffèrent.
\end{infobox}


\begin{C}
#include <stdio.h>
#include <string.h>


int main(void)
{
    char chaine1[] = "Bonjour";
    char chaine2[] = "Au revoir";

    if (strcmp(chaine1, chaine2) == 0)
        printf("Les deux chaînes sont identiques\n");
    else
        printf("Les deux chaînes sont différentes\n");

    return 0;
}
\end{C}

\begin{C}
Les deux chaînes sont différentes
\end{C}

\subsection{strchr}
\label{strchr-1}

\begin{C}
char *strchr(char *chaine, int ch);
\end{C}

La fonction \mybox{strchr()} recherche la présence du caractère
\mybox{ch} dans la chaîne \mybox{chaine}. Si celui-ci est rencontré,
la fonction retourne l'adresse de la première occurrence de celui-ci au
sein de la chaîne. Dans le cas contraire, la fonction renvoie un
pointeur nul.

\begin{C}
#include <stddef.h>
#include <stdio.h>
#include <string.h>


int main(void)
{
    char chaine[] = "Bonjour";
    char *p;

    p = strchr(chaine, 'o');

    if (p != NULL)
        printf("La chaîne `%s' contient la lettre %c\n", chaine, *p);

    return 0;
}
\end{C}

\begin{C}
La chaîne `Bonjour' contient la lettre o
\end{C}

\subsection{strpbrk}
\label{strpbrk-1}

\begin{C}
char *strpbrk(char *chaine, char *liste);
\end{C}

La fonction \mybox{strpbrk()} recherche la présence \emph{d'un} des
caractères de la chaîne \mybox{liste} dans la chaîne \mybox{chaine}.
Si un de ceux-ci est rencontré, la fonction retourne l'adresse de la
première occurrence au sein de la chaîne. Dans le cas contraire, la
fonction renvoie un pointeur nul.

\begin{C}
#include <stddef.h>
#include <stdio.h>
#include <string.h>


int main(void)
{
    char chaine[] = "Bonjour";
    char *p;

    p = strpbrk(chaine, "aeiouy");

    if (p != NULL)
        printf("La première voyelle de la chaîne `%s' est : %c\n", chaine, *p);

    return 0;
}
\end{C}

\begin{C}
La première voyelle de la chaîne `Bonjour' est : o
\end{C}

\subsection{strstr}
\label{strstr-1}

\begin{C}
char *strstr(char *chaine1, char *chaine2);
\end{C}

La fonction \mybox{strstr()} recherche la présence de la chaîne
\mybox{chaine2} dans la chaîne \mybox{chaine1}. Si celle-ci est
rencontrée, la fonction retourne l'adresse de la première occurrence de
celle-ci au sein de la chaîne. Dans le cas contraire, la fonction
renvoie un pointeur nul.

\begin{C}
#include <stddef.h>
#include <stdio.h>
#include <string.h>


int main(void)
{
    char chaine[] = "Bonjour";
    char *p;

    p = strstr(chaine, "jour");

    if (p != NULL)
        printf("La chaîne `%s' contient la chaîne `%s'\n", chaine, p);

    return 0;
}
\end{C}

\begin{C}
La chaîne `Bonjour' contient la chaîne `jour'
\end{C}

Ceci étant dit, à vous de jouer. ;)

\begin{infobox}
Par convention, nous commencerons le nom de nos fonctions par la 
lettre « x » afin d'éviter des collisions avec ceux de l'en-tête 
\mybox{\textless{}string.h\textgreater{}}.
\end{infobox}

\subsection{strlen}
\label{strlen}

\begin{C}
size_t xstrlen(char *chaine)
{
    size_t i;

    for (i = 0; chaine[i] != '\0'; ++i)
        ;

    return i;
}
\end{C}

\subsection{strcpy}
\label{strcpy}

\begin{C}
char *xstrcpy(char *destination, char *source)
{
    size_t i;

    for (i = 0; source[i] != '\0'; ++i)
        destination[i] = source[i];

    destination[i] = '\0' /* N'oubliez pas le caractère nul final ! */
    return destination;
}
\end{C}

\subsection{strcat}
\label{strcat}

\begin{C}
char *xstrcat(char *destination, char *source)
{
    size_t i;

    while (*destination != '\0')
        ++destination;

    for (i = 0; source[i] != '\0'; ++i)
        destination[i] = source[i];

    destination[i] = '\0'; /* N'oubliez pas le caractère nul final ! */
    return destination;
}
\end{C}

\subsection{strcmp}
\label{strcmp}

\begin{C}
int xstrcmp(char *chaine1, char *chaine2)
{
    while (*chaine1 == *chaine2)
    {
        if (*chaine1 == '\0')
            return 0;

        ++chaine1;
        ++chaine2;
    }

    return (*chaine1 < *chaine2) ? -1 : 1;
}
\end{C}

\subsection{strchr}
\label{strchr}

\begin{C}
char *xstrchr(char *chaine, int ch)
{
    while (*chaine != '\0')
        if (*chaine == ch)
            return chaine;
        else
            ++chaine;

    return NULL;
}
\end{C}

\subsection{strpbrk}
\label{strpbrk}

\begin{C}
char *xstrpbrk(char *chaine, char *liste)
{
    while (*chaine != '\0')
    {
        char *p = liste;

        while (*p != '\0')
        {
            if (*chaine == *p)
                return chaine;

            ++p;
        }

        ++chaine;
    }

    return NULL;
}
\end{C}

\subsection{strstr}
\label{strstr}

\begin{C}
char *xstrstr(char *chaine1, char *chaine2)
{
    while (*chaine1 != '\0')
    {
        char *s = chaine1;
        char *t = chaine2;

        while (*s != '\0' && *t != '\0')
        {
            if (*s != *t)
                break;
      
            ++s;
            ++t;
        }

        if (*t == '\0')
            return chaine1;

        ++chaine1;
    }

    return NULL;
}
\end{C}

\section{Pour allez plus loin : strtok}
\label{pour-allez-plus-loin-:-strtok}


Pour les plus aventureux d'entre-vous, nous vous
proposons de réaliser en plus la mise en œuvre de la fonction
\mybox{strtok()} qui est un peu plus complexe que ses congénères. :)

\begin{C}
char *strtok(char *chaine, char *liste);
\end{C}

La fonction \mybox{strtok()} est un peu particulière : elle divise la
chaîne \mybox{chaine} en une suite de sous-chaînes délimitée par
\emph{un} des caractères présent dans la chaîne \mybox{liste}. En fait,
cette dernière remplace les caractères de la chaîne \mybox{liste} par
des caractères nuls dans la chaîne \mybox{chaine} (elle modifie donc la
chaîne \mybox{chaine} !) et renvoie les différentes sous-chaînes ainsi
créées au fur et à mesure de ses appels.

Lors des appels subséquents, la chaîne \mybox{chaine} doit être un
pointeur nul afin de signaler à \mybox{strtok()} que nous souhaitons la
sous-chaîne suivante. S'il n'y a plus de sous-chaîne ou qu'aucun
caractère de la chaîne \mybox{liste} n'est trouvé, celle-ci retourne un
pointeur nul.

L'exemple ci-dessous tente de scinder la chaîne « un, deux, trois » en
trois sous-chaînes.

\begin{C}
#include <stddef.h>
#include <stdio.h>
#include <string.h>


int main(void)
{
    char chaine[] = "un, deux, trois";
    char *p;
    unsigned i;

    p = strtok(chaine, ", ");

    for (i = 0; p != NULL; ++i)
    {
        printf("%u : %s\n", i, p);
        p = strtok(NULL, ", ");
    }

    return 0;
}
\end{C}

\begin{C}
0 : un
1 : deux
2 : trois
\end{C}

La fonction \mybox{strtok()} étant un peu plus complexe que les autres,
voici une petite marche à suivre pour réaliser cette fonction.

\begin{itemize}
\item
  regarder si la chaîne fournie (ou la sous-chaîne suivante) contient,
  \emph{à son début}, des caractères présent dans la seconde chaîne. Si
  oui, ceux-ci doivent être passés ;
\item
  si la fin de la (sous-)chaîne est atteinte, retourner un pointeur nul
  ;
\item
  parcourir la chaîne fournie (ou la sous-chaîne courante) jusqu'à
  rencontrer un des caractères composant la seconde chaîne. Si un
  caractère est rencontré, le remplacer par un caractère nul, conserver
  la position actuelle dans la chaîne et retourner la sous-chaîne ainsi
  créée (\emph{sans} les éventuels caractères passés au début) ;
\item
  si la fin de la (sous-)chaîne est atteinte, retourner la (sous-)chaîne
  (\emph{sans} les éventuels caractères passés au début).
\end{itemize}

\begin{infobox}
  Vous aurez besoin d'une variable de
classe de stockage statique pour réaliser cette fonction. Également,
l'instruction \mybox{goto} pourra vous être utile.
\end{infobox}


\subsection{Correction}
\label{correction-17}

\begin{C}
char *xstrtok(char *chaine, char *liste)
{
    static char *dernier;
    char *base = (chaine != NULL) ? chaine : dernier;
    char *s;
    char *t;

    if (base == NULL)
        return NULL;

separateur_au_debut:
    for (t = liste; *t != '\0'; ++t)
        if (*base == *t)
        {
            ++base;
            goto separateur_au_debut;
        }

    if (*base == '\0')
    {
        dernier = NULL;
        return NULL;
    }

    for (s = base; *s != '\0'; ++s)
        for (t = liste; *t != '\0'; ++t)
            if (*s == *t)
            {
                *s = '\0';
                dernier = s + 1;
                return base;
            }

    dernier = NULL;
    return base;
}
\end{C}

\hrulefill

le chapitre suivant, nous aborderons le mécanisme de
l'\textbf{allocation dynamique de mémoire}.