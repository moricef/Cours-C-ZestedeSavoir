\usepackage{fontspec}
\usepackage{shorttoc}%pour la réalisation d'un sommaire.

%\usepackage{hyperref}%rend actif les liens, références croisée, toc, ...
%		\hypersetup{colorlinks,%%
%		citecolor=black,%
%		filecolor=black,%
%		linkcolor=black,%
%		urlcolor=black} 

\usepackage[colorlinks = true,
	    linkcolor = blue,
            urlcolor  = blue,
            citecolor = blue,
            anchorcolor = blue]{hyperref}
\newcommand{\MYhref}[3][blue]{\href{#2}{\color{#1}{#3}}}%

\usepackage{textcomp}
\usepackage{animate}
\usepackage{tabularx}
\usepackage{colortbl}
\usepackage{makecell}
\renewcommand\theadfont{}
\usepackage[usenames,dvipsnames,table]{xcolor}
\definecolor{lightgray}{RGB}{218, 218, 218}
\definecolor{light-gray}{gray}{0.65}
\definecolor{cardinal}{RGB}{196, 30, 38}
\definecolor{bleu-info}{RGB}{218, 234, 238}
\definecolor{ocre-attention}{RGB}{238, 231, 218}
\definecolor{violet-question}{RGB}{226, 218, 238}
\definecolor{rouge-erreur}{RGB}{238, 218, 218}
\definecolor{gris-tab-entete}{RGB}{221, 221, 221}
\definecolor{gris-clair-tab}{RGB}{247, 247, 247}
\definecolor{gris-secret}{RGB}{238, 238, 238}

\usepackage[most]{tcolorbox}
\tcbuselibrary{minted, skins}

\newtcbox{mybox}{on line, fontupper=\scriptsize, arc=0pt,outer arc=0pt, colback=blue!5!white,
colframe=black!30!white, colupper=cardinal, boxsep=0pt,left=1pt,right=1pt,top=1pt,bottom=1pt,
boxrule=0.5pt,bottomrule=0.5pt,toprule=0.5pt}

\newtcolorbox{questionbox}{breakable, enhanced, arc=0mm, colback=violet-question, colframe=violet-question, leftrule=6mm,%
overlay={\node[anchor= west,outer sep=2pt] at (frame.west) {\includegraphics[width=5mm]{images/question.png}}; }}

\newtcolorbox{attentionbox}{breakable, enhanced, arc=0mm, colback=ocre-attention, colframe=ocre-attention, leftrule=6mm,%
overlay={\node[anchor= west,outer sep=2pt] at (frame.west) {\includegraphics[width=5mm]{images/attention.png}}; }}


\newtcolorbox{infobox}{breakable, enhanced, arc=0mm, colback=bleu-info, colframe=bleu-info,  leftrule=6mm,%
overlay={\node[anchor= west,outer sep=2pt] at (frame.west) {\includegraphics[width=5mm]{images/info.png}}; }}

\newtcolorbox{erreurbox}{breakable, enhanced, arc=0mm, colback=rouge-erreur, colframe=rouge-erreur,  leftrule=6mm,%
overlay={\node[anchor= west,outer sep=2pt] at (frame.west) {\includegraphics[width=5mm]{images/erreur.png}}; }}

\newtcolorbox{secretbox}{breakable, enhanced, arc=0mm, colback=gris-secret, colframe=gris-secret,  leftrule=6mm}


\usepackage{minted}
\newminted[bash]{bash}{
style = xcode,
autogobble,
frame=single,%
linenos,%
fontsize=\footnotesize%
%breaklines=true%
}
\newminted[C]{C}{
style=default,
autogobble,
frame=single,%
linenos,%
fontsize=\scriptsize%
%breaklines=true%
}


\usepackage[english,french]{babel}%pour un document en français
\usepackage[babel=true]{csquotes} % csquotes va utiliser la langue définie dans babel
\usepackage{graphicx}%pour insérer images et pdf entre autres
	\graphicspath{{images/}}%pour spécifier le chemin d'accès aux images
\usepackage{wrapfig}
\usepackage[left=2.5cm,right=2.5cm,top=2.5cm,bottom=2.5cm]{geometry}%réglages des marges du document selon vos préférences ou celles de votre établissement
\usepackage{array}
\usepackage{multirow}
\usepackage{lastpage}
\usepackage[Bjornstrup]{fncychap}%pour de jolis titres de chapitres voir la doc pour d'autres styles.
\usepackage{fancyhdr}%pour les entêtes et pieds de pages
	\setlength{\headheight}{14.2pt}% hauteur de l'entête
	
%%%%%%%%%%%%%%%%%%%style front%%%%%%%%%%%%%%%%%%%%%%%%%%%%%%%%%%%%%%%%%	
	\fancypagestyle{front}{%
  		\fancyhf{}%on vide les entêtes
  		\fancyfoot[C]{page \thepage}%
  		\renewcommand{\headrulewidth}{0pt}%trait horizontal pour l'entête
  		\renewcommand{\footrulewidth}{0.4pt}%trait horizontal pour les pieds de pages
		}


%%%%%%%%%%%%%%%%%%%style main%%%%%%%%%%%%%%%%%%%%%%%%%%%%%%%%%%%%
	\fancypagestyle{main}{%
		\fancyhf{}
  		\renewcommand{\chaptermark}[1]{\markboth{\chaptername\ \thechapter.\ ##1}{}}% redéfintion pour avoir ici les titres des chapitres des sections en minuscules
  		\renewcommand{\sectionmark}[1]{\markright{\thesection\ ##1}}
		\fancyhead[c]{}
		\fancyhead[RO,LE]{\rightmark}%
  		\fancyhead[LO,RE]{\leftmark}
		\fancyfoot[C]{}
		\fancyfoot[RO,LE]{\thepage}%
  		\fancyfoot[LO,RE]{https://zestedesavoir.com}
  		}

%%%%%%%%%%%%%%%%%%%style back%%%%%%%%%%%%%%%%%%%%%%%%%%%%%%%%%%%%%%%%%	
	\fancypagestyle{back}{%
  		\fancyhf{}%on vide les entêtes
  		\fancyfoot[C]{ \thepage}%
  		\renewcommand{\headrulewidth}{0pt}%trait horizontal pour l'entête
  		\renewcommand{\footrulewidth}{0.4pt}%trait horizontal pour les pieds de pages
		}


%%%%%%%%%%%%%%%%%%%%%%%%%%%%index%%%%%%%%%%%%%%%%%%%%%%%%%%%%%%%%%%%%%%%
\usepackage{makeidx}
\makeindex

%%%%%%%%%%%%%%%%%%%%%%%%%%%%%glossaire%%%%%%%%%%%%%%%%%%%%%%%%%%%%%%%%%%%
%\usepackage{glossaries}
%\makeglossaries		

%%%%%%%%%%%%%%%%%%%%%%%%%%%%liste des abbréviations%%%%%%%%%%%%%%		
%\usepackage[french]{nomencl}
%\makenomenclature
%\renewcommand{\nomname}{Liste des abréviation, des sigles et des symboles}

