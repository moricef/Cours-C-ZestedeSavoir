\PassOptionsToPackage{unicode=true}{hyperref} % options for packages loaded elsewhere
\PassOptionsToPackage{hyphens}{url}
%
\documentclass[]{article}
\usepackage{lmodern}
\usepackage{amssymb,amsmath}
\usepackage{ifxetex,ifluatex}
\usepackage{fixltx2e} % provides \textsubscript
\ifnum 0\ifxetex 1\fi\ifluatex 1\fi=0 % if pdftex
  \usepackage[T1]{fontenc}
  \usepackage[utf8]{inputenc}
  \usepackage{textcomp} % provides euro and other symbols
\else % if luatex or xelatex
  \usepackage{unicode-math}
  \defaultfontfeatures{Ligatures=TeX,Scale=MatchLowercase}
\fi
% use upquote if available, for straight quotes in verbatim environments
\IfFileExists{upquote.sty}{\usepackage{upquote}}{}
% use microtype if available
\IfFileExists{microtype.sty}{%
\usepackage[]{microtype}
\UseMicrotypeSet[protrusion]{basicmath} % disable protrusion for tt fonts
}{}
\IfFileExists{parskip.sty}{%
\usepackage{parskip}
}{% else
\setlength{\parindent}{0pt}
\setlength{\parskip}{6pt plus 2pt minus 1pt}
}
\usepackage{hyperref}
\hypersetup{
            pdfborder={0 0 0},
            breaklinks=true}
\urlstyle{same}  % don't use monospace font for urls
\usepackage{color}
\usepackage{fancyvrb}
\newcommand{\VerbBar}{|}
\newcommand{\VERB}{\Verb[commandchars=\\\{\}]}
\DefineVerbatimEnvironment{Highlighting}{Verbatim}{commandchars=\\\{\}}
% Add ',fontsize=\small' for more characters per line
\newenvironment{Shaded}{}{}
\newcommand{\KeywordTok}[1]{\textcolor[rgb]{0.00,0.44,0.13}{\textbf{{#1}}}}
\newcommand{\DataTypeTok}[1]{\textcolor[rgb]{0.56,0.13,0.00}{{#1}}}
\newcommand{\DecValTok}[1]{\textcolor[rgb]{0.25,0.63,0.44}{{#1}}}
\newcommand{\BaseNTok}[1]{\textcolor[rgb]{0.25,0.63,0.44}{{#1}}}
\newcommand{\FloatTok}[1]{\textcolor[rgb]{0.25,0.63,0.44}{{#1}}}
\newcommand{\ConstantTok}[1]{\textcolor[rgb]{0.53,0.00,0.00}{{#1}}}
\newcommand{\CharTok}[1]{\textcolor[rgb]{0.25,0.44,0.63}{{#1}}}
\newcommand{\SpecialCharTok}[1]{\textcolor[rgb]{0.25,0.44,0.63}{{#1}}}
\newcommand{\StringTok}[1]{\textcolor[rgb]{0.25,0.44,0.63}{{#1}}}
\newcommand{\VerbatimStringTok}[1]{\textcolor[rgb]{0.25,0.44,0.63}{{#1}}}
\newcommand{\SpecialStringTok}[1]{\textcolor[rgb]{0.73,0.40,0.53}{{#1}}}
\newcommand{\ImportTok}[1]{{#1}}
\newcommand{\CommentTok}[1]{\textcolor[rgb]{0.38,0.63,0.69}{\textit{{#1}}}}
\newcommand{\DocumentationTok}[1]{\textcolor[rgb]{0.73,0.13,0.13}{\textit{{#1}}}}
\newcommand{\AnnotationTok}[1]{\textcolor[rgb]{0.38,0.63,0.69}{\textbf{\textit{{#1}}}}}
\newcommand{\CommentVarTok}[1]{\textcolor[rgb]{0.38,0.63,0.69}{\textbf{\textit{{#1}}}}}
\newcommand{\OtherTok}[1]{\textcolor[rgb]{0.00,0.44,0.13}{{#1}}}
\newcommand{\FunctionTok}[1]{\textcolor[rgb]{0.02,0.16,0.49}{{#1}}}
\newcommand{\VariableTok}[1]{\textcolor[rgb]{0.10,0.09,0.49}{{#1}}}
\newcommand{\ControlFlowTok}[1]{\textcolor[rgb]{0.00,0.44,0.13}{\textbf{{#1}}}}
\newcommand{\OperatorTok}[1]{\textcolor[rgb]{0.40,0.40,0.40}{{#1}}}
\newcommand{\BuiltInTok}[1]{{#1}}
\newcommand{\ExtensionTok}[1]{{#1}}
\newcommand{\PreprocessorTok}[1]{\textcolor[rgb]{0.74,0.48,0.00}{{#1}}}
\newcommand{\AttributeTok}[1]{\textcolor[rgb]{0.49,0.56,0.16}{{#1}}}
\newcommand{\RegionMarkerTok}[1]{{#1}}
\newcommand{\InformationTok}[1]{\textcolor[rgb]{0.38,0.63,0.69}{\textbf{\textit{{#1}}}}}
\newcommand{\WarningTok}[1]{\textcolor[rgb]{0.38,0.63,0.69}{\textbf{\textit{{#1}}}}}
\newcommand{\AlertTok}[1]{\textcolor[rgb]{1.00,0.00,0.00}{\textbf{{#1}}}}
\newcommand{\ErrorTok}[1]{\textcolor[rgb]{1.00,0.00,0.00}{\textbf{{#1}}}}
\newcommand{\NormalTok}[1]{{#1}}
\setlength{\emergencystretch}{3em}  % prevent overfull lines
\providecommand{\tightlist}{%
  \setlength{\itemsep}{0pt}\setlength{\parskip}{0pt}}
\setcounter{secnumdepth}{5}
% Redefines (sub)paragraphs to behave more like sections
\ifx\paragraph\undefined\else
\let\oldparagraph\paragraph
\renewcommand{\paragraph}[1]{\oldparagraph{#1}\mbox{}}
\fi
\ifx\subparagraph\undefined\else
\let\oldsubparagraph\subparagraph
\renewcommand{\subparagraph}[1]{\oldsubparagraph{#1}\mbox{}}
\fi

% set default figure placement to htbp
\makeatletter
\def\fps@figure{htbp}
\makeatother


\date{}

\begin{document}

{
\setcounter{tocdepth}{3}
\tableofcontents
}
\subsubsection{En résumé}\label{en-ruxe9sumuxe9}

\begin{enumerate}
\def\labelenumi{\arabic{enumi}.}
\tightlist
\item
  Sauf si le nom de l'énumération n'est pas renseignée, une définition
  d'énumération créer un type énuméré et des contantes énumérées ;
\item
  Sauf si une valeur leur est attribuée, la valeur de chaque constantes
  énumérées est celle de la précédente augmentée de un et celle de la
  première est zéro.
\item
  Le type entier sous-jacent à un type énuméré est indéterminé ; les
  constantes énumérées sont de type \texttt{int}.Jusqu'à présent, nous
  avons toujours employé le préprocesseur pour définir des constantes au
  sein de nos codes. Toutefois, une solution un peu plus commonde existe
  pour les constantes entières : les \textbf{énumérations}.Dans la
  pratique, les énumérations servent essentiellement à fournir des
  informations supplémentaires via le typage, par exemple pour les
  retours d'erreurs. En effet, le plus souvent, les fonctions retournent
  un entier pour préciser si leur exécution s'est bien déroulée.
  Toutefois, indiquer un retour de type \texttt{int} ne fourni pas
  énormément d'information. Un type énuméré prend alors tout son sens.
\end{enumerate}

La fonction \texttt{vider\_tampon()} du dernier T.P. s'y prêterait par
exemple bien.

\begin{Shaded}
\begin{Highlighting}[]
\KeywordTok{enum} \NormalTok{erreur \{ E_OK, E_ERR \};}


\DataTypeTok{static} \KeywordTok{enum} \NormalTok{erreur vider_tampon(FILE *fp)}
\NormalTok{\{}
    \DataTypeTok{int} \NormalTok{c;}

    \ControlFlowTok{do}
        \NormalTok{c = fgetc(fp);}
    \ControlFlowTok{while} \NormalTok{(c != }\CharTok{'\textbackslash{}n'} \NormalTok{&& c != EOF);}

    \ControlFlowTok{return} \NormalTok{ferror(fp) ? E_ERR : E_OK;}
\NormalTok{\}}
\end{Highlighting}
\end{Shaded}

De cette manière, il est plus clair à la lecture que la fonction
retourne le statut de son exécution.

Dans la même idée, il est possible d'utiliser un type énuméré pour la
fonction \texttt{statut\_jeu()} (également employée dans la correction
du dernier T.P.) afin de décrire plus amplement son type de retour.

\begin{Shaded}
\begin{Highlighting}[]
\KeywordTok{enum} \NormalTok{statut \{ STATUT_OK, STATUT_GAGNE, STATUT_EGALITE \};}


\DataTypeTok{static} \KeywordTok{enum} \NormalTok{statut statut_jeu(}\KeywordTok{struct} \NormalTok{position *pos, }\DataTypeTok{char} \NormalTok{jeton)}
\NormalTok{\{}
    \ControlFlowTok{if} \NormalTok{(grille_complete())}
        \ControlFlowTok{return} \NormalTok{STATUT_EGALITE;}
    \ControlFlowTok{else} \ControlFlowTok{if} \NormalTok{(calcule_nb_jetons_depuis(pos, jeton) >= }\DecValTok{4}\NormalTok{)}
        \ControlFlowTok{return} \NormalTok{STATUT_GAGNE;}

    \ControlFlowTok{return} \NormalTok{STATUT_OK;}
\NormalTok{\}}
\end{Highlighting}
\end{Shaded}

Dans un autre registre, un type enuméré peut être utilisé pour contenir
des drapeaux. Par exemple, la fonction \texttt{traitement()} présentée
dans le chapitre relatif aux opérateurs de manipulation des \emph{bits}
peut être réecrite comme suit.

```c enum drapeau \{ PAIR = 0x00, PUISSANCE = 0x01, PREMIER = 0x02 \};

void traitement(int nombre, enum drapeau drapeaux) \{ if (drapeaux \&
PAIR) /* Si le nombre est pair \emph{/ \{ /} \ldots{} \emph{/ \} if
(drapeaux \& PUISSANCE) /} Si le nombre est une puissance de deux
\emph{/ \{ /}\ldots{} \emph{/ \} if (drapeaux \& PREMIER) /} Si le
nombre est premier \emph{/ \{ /}\ldots{} */ \} \}
``\texttt{Une\ énumération\ se\ défini\ à\ l\textquotesingle{}aide\ du\ mot-clé}enum`
suivi du nom de l'énumération et de ses membres.

\begin{Shaded}
\begin{Highlighting}[]
\KeywordTok{enum} \NormalTok{naturel \{ ZERO, UN, DEUX, TROIS, QUATRE, CINQ \};}
\end{Highlighting}
\end{Shaded}

La particularité de cette définition est qu'elle crée en vérité deux
choses : un type dit « énuméré » \texttt{enum\ naturel} et des
constantes dites « énumérées » \texttt{ZERO}, \texttt{UN},
\texttt{DEUX}, etc. Le type énuméré ainsi produit peut être utilisé de
la même manière que n'importe quel autre type. Quant aux constantes
énumérées, il s'agit de constantes entières.

Certes me direz-vous, mais que valent ces constantes ? \emph{Eh} bien, à
défaut de préciser leur valeur, chaque constante énumérée se voit
attribuer la valeur de celle qui la précède augmentée de un, sachant que
la première constante est mise à zéro . Dans notre cas donc, la
constante \texttt{ZERO} vaut zéro, la constante \texttt{UN} un et ainsi
de suite jusque cinq.

L'exemple suivant illustre ce qui vient d'être dit.

\begin{Shaded}
\begin{Highlighting}[]
\PreprocessorTok{#include }\ImportTok{<stdio.h>}

\KeywordTok{enum} \NormalTok{naturel \{ ZERO, UN, DEUX, TROIS, QUATRE, CINQ \};}


\DataTypeTok{int} \NormalTok{main(}\DataTypeTok{void}\NormalTok{)}
\NormalTok{\{}
    \KeywordTok{enum} \NormalTok{naturel n = ZERO;}

    \NormalTok{printf(}\StringTok{"n = %d.}\SpecialCharTok{\textbackslash{}n}\StringTok{"}\NormalTok{, (}\DataTypeTok{int}\NormalTok{)n);}
    \NormalTok{printf(}\StringTok{"UN = %d.}\SpecialCharTok{\textbackslash{}n}\StringTok{"}\NormalTok{, UN);}
    \ControlFlowTok{return} \DecValTok{0}\NormalTok{;}
\NormalTok{\}}
\end{Highlighting}
\end{Shaded}

\begin{verbatim}
n = 0.
UN = 1.
\end{verbatim}

{[}{[}information{]}{]} \textbar{} Notez qu'il n'est pas obligatoire de
préciser un nom lors de la définition d'une énumération. Dans un tel
cas, seules les constantes énumérées sont produites. \textbar{}
\textbar{}\texttt{c\ \textbar{}\ enum\ \{\ ZERO,\ UN,\ DEUX,\ TROIS,\ QUATRE,\ CINQ\ \};\ \textbar{}}

Toutefois, il est possible de préciser la valeur de certaines constantes
(voire de toutes les constantes) à l'aide d'une affectation.

\begin{Shaded}
\begin{Highlighting}[]
\KeywordTok{enum} \NormalTok{naturel \{ DIX = }\DecValTok{10}\NormalTok{, ONZE, DOUZE, TREIZE, QUATORZE, QUINZE \};}
\end{Highlighting}
\end{Shaded}

Dans un tel cas, la règle habituelle s'applique : les constantes sans
valeur se voit attribuer celle de la constante précédente augmentée de
un et celle dont la valeur est spécifiée sont initialisées avec
celle-ci. Dans le cas ci-dessus, la constante \texttt{DIX} vaut donc
dix, la constante \texttt{ONZE} onze et ainsi de suite jusque quinze.
Notez que le code ci-dessous est parfaitement équivalent.

\begin{Shaded}
\begin{Highlighting}[]
\KeywordTok{enum} \NormalTok{naturel \{ DIX = }\DecValTok{10}\NormalTok{, ONZE = }\DecValTok{11}\NormalTok{, DOUZE = }\DecValTok{12}\NormalTok{, TREIZE = }\DecValTok{13}\NormalTok{, QUATORZE = }\DecValTok{14}\NormalTok{, QUINZE = }\DecValTok{15} \NormalTok{\};}
\end{Highlighting}
\end{Shaded}

\subsection{Types entiers
sous-jacents}\label{types-entiers-sous-jacents}

Vous aurez sans doute remarqué que, dans notre exemple, nous avons
converti la variable \texttt{n} vers le type \texttt{int}. Cela tient au
fait qu'un type enuméré est un type entier (ce qui est logique puisqu'il
est censé stocker des constantes entières), mais que le type sous-jacent
n'est pas déterminé (cela peut donc être \texttt{char}, \texttt{short},
\texttt{int} ou \texttt{long}) et dépend entre autre des valeurs devant
être contenues. Ainsi, une conversion s'impose afin de pouvoir utiliser
un format d'affichage correct.

Pour ce qui est des constantes énumérées, c'est plus simple : elles sont
toujours de type \texttt{int}.

\end{document}
