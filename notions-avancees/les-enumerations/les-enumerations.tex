\chapter{Les énumérations}
\label{les-enumerations}

  Jusqu'à présent, nous   avons toujours employé le préprocesseur pour 
  définir des constantes au sein de nos codes. Toutefois, une solution 
  un peu plus commonde existe pour les constantes entières : les 
  \textbf{énumérations}.

\section{Définition}
\label{definition-4}

Une énumération se défini à l'aide du mot-clé \mybox{enum}
suivi du nom de l'énumération et de ses membres.

\begin{C}
enum naturel { ZERO, UN, DEUX, TROIS, QUATRE, CINQ };
\end{C}

La particularité de cette définition est qu'elle crée en vérité deux
choses : un type dit « énuméré » \mybox{enum\ naturel} et des
constantes dites « énumérées » \mybox{ZERO}, \mybox{UN},
\mybox{DEUX}, etc. Le type énuméré ainsi produit peut être utilisé de
la même manière que n'importe quel autre type. Quant aux constantes
énumérées, il s'agit de constantes entières.

Certes me direz-vous, mais que valent ces constantes ? \emph{Eh} bien, à
défaut de préciser leur valeur, chaque constante énumérée se voit
attribuer la valeur de celle qui la précède augmentée de un, sachant que
la première constante est mise à zéro . Dans notre cas donc, la
constante \mybox{ZERO} vaut zéro, la constante \mybox{UN} un et ainsi
de suite jusque cinq.

L'exemple suivant illustre ce qui vient d'être dit.

\begin{C}
#include <stdio.h>

enum naturel { ZERO, UN, DEUX, TROIS, QUATRE, CINQ };


int main(void)
{
    enum naturel n = ZERO;

    printf("n = %d.\n", (int)n);
    printf("UN = %d.\n", UN);
    return 0;
}
\end{C}

\begin{C}
n = 0.
UN = 1.
\end{C}

\begin{infobox}
  Notez qu'il n'est pas obligatoire de
préciser un nom lors de la définition d'une énumération. Dans un tel
cas, seules les constantes énumérées sont produites.
\begin{C}
enum { ZERO, UN, DEUX, TROIS, QUATRE, CINQ };
\end{C}
\end{infobox}

Toutefois, il est possible de préciser la valeur de certaines constantes
(voire de toutes les constantes) à l'aide d'une affectation.

\begin{C}
enum naturel { DIX = 10, ONZE, DOUZE, TREIZE, QUATORZE, QUINZE };
\end{C}

Dans un tel cas, la règle habituelle s'applique : les constantes sans
valeur se voit attribuer celle de la constante précédente augmentée de
un et celle dont la valeur est spécifiée sont initialisées avec
celle-ci. Dans le cas ci-dessus, la constante \mybox{DIX} vaut donc
dix, la constante \mybox{ONZE} onze et ainsi de suite jusque quinze.
Notez que le code ci-dessous est parfaitement équivalent.

\begin{C}
enum naturel { DIX = 10, ONZE = 11, DOUZE = 12, TREIZE = 13, QUATORZE = 14, QUINZE = 15 };
\end{C}

\subsubsection{Types entiers sous-jacents}
\label{types-entiers-sous-jacents}

Vous aurez sans doute remarqué que, dans notre exemple, nous avons
converti la variable \mybox{n} vers le type \mybox{int}. Cela tient au
fait qu'un type enuméré est un type entier (ce qui est logique puisqu'il
est censé stocker des constantes entières), mais que le type sous-jacent
n'est pas déterminé (cela peut donc être \mybox{char}, \mybox{short},
\mybox{int} ou \mybox{long}) et dépend entre autre des valeurs devant
être contenues. Ainsi, une conversion s'impose afin de pouvoir utiliser
un format d'affichage correct.

Pour ce qui est des constantes énumérées, c'est plus simple : elles sont
toujours de type \mybox{int}.

\section{Utilisation }
\label{utilisation-4}

Dans la pratique, les énumérations servent 
  essentiellement à fournir des informations supplémentaires via le 
  typage, par exemple pour les retours d'erreurs. En effet, le plus 
  souvent, les fonctions retournent un entier pour préciser si leur 
  exécution s'est bien déroulée. Toutefois, indiquer un retour de 
  type \mybox{int} ne fourni pas énormément d'information. Un type 
  énuméré prend alors tout son sens.

La fonction \mybox{vider\_tampon()} du dernier T.P. s'y prêterait par
exemple bien.

\begin{C}
enum erreur { E_OK, E_ERR };


static enum erreur vider_tampon(FILE *fp)
{
    int c;

    do
        c = fgetc(fp);
    while (c != '\n' && c != EOF);

    return ferror(fp) ? E_ERR : E_OK;
}
\end{C}

De cette manière, il est plus clair à la lecture que la fonction
retourne le statut de son exécution.

Dans la même idée, il est possible d'utiliser un type énuméré pour la
fonction \mybox{statut\_jeu()} (également employée dans la correction
du dernier T.P.) afin de décrire plus amplement son type de retour.

\begin{C}
enum statut { STATUT_OK, STATUT_GAGNE, STATUT_EGALITE };


static enum statut statut_jeu(struct position *pos, char jeton)
{
    if (grille_complete())
        return STATUT_EGALITE;
    else if (calcule_nb_jetons_depuis(pos, jeton) >= 4)
        return STATUT_GAGNE;

    return STATUT_OK;
}
\end{C}

Dans un autre registre, un type enuméré peut être utilisé pour contenir
des drapeaux. Par exemple, la fonction \mybox{traitement()} présentée
dans le chapitre relatif aux opérateurs de manipulation des \emph{bits}
peut être réecrite comme suit.

\begin{C}
enum drapeau {
    PAIR = 0x00,
    PUISSANCE = 0x01,
    PREMIER = 0x02
};


void traitement(int nombre, enum drapeau drapeaux)
{
    if (drapeaux & PAIR) /* Si le nombre est pair */
    {
        /* ... */
    }
    if (drapeaux & PUISSANCE) /* Si le nombre est une puissance de deux */
    {
        /*... */
    }
    if (drapeaux & PREMIER) /* Si le nombre est premier */
    {
        /*... */
    }
}
\end{C}

\hrulefill

\subsubsection*{\small{En résumé}}
\label{en-resume-3}

\begin{enumerate}
\def
\labelenumi{\arabic{enumi}.}
\item
  Sauf si le nom de l'énumération n'est pas renseignée, une définition
  d'énumération créer un type énuméré et des contantes énumérées ;
\item
  Sauf si une valeur leur est attribuée, la valeur de chaque constantes
  énumérées est celle de la précédente augmentée de un et celle de la
  première est zéro.
\item
  Le type entier sous-jacent à un type énuméré est indéterminé ; les
  constantes énumérées sont de type \mybox{int}.
  \end{enumerate}