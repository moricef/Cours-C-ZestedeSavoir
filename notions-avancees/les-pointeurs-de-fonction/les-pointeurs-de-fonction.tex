\PassOptionsToPackage{unicode=true}{hyperref} % options for packages loaded elsewhere
\PassOptionsToPackage{hyphens}{url}
%
\documentclass[]{article}
\usepackage{lmodern}
\usepackage{amssymb,amsmath}
\usepackage{ifxetex,ifluatex}
\usepackage{fixltx2e} % provides \textsubscript
\ifnum 0\ifxetex 1\fi\ifluatex 1\fi=0 % if pdftex
  \usepackage[T1]{fontenc}
  \usepackage[utf8]{inputenc}
  \usepackage{textcomp} % provides euro and other symbols
\else % if luatex or xelatex
  \usepackage{unicode-math}
  \defaultfontfeatures{Ligatures=TeX,Scale=MatchLowercase}
\fi
% use upquote if available, for straight quotes in verbatim environments
\IfFileExists{upquote.sty}{\usepackage{upquote}}{}
% use microtype if available
\IfFileExists{microtype.sty}{%
\usepackage[]{microtype}
\UseMicrotypeSet[protrusion]{basicmath} % disable protrusion for tt fonts
}{}
\IfFileExists{parskip.sty}{%
\usepackage{parskip}
}{% else
\setlength{\parindent}{0pt}
\setlength{\parskip}{6pt plus 2pt minus 1pt}
}
\usepackage{hyperref}
\hypersetup{
            pdfborder={0 0 0},
            breaklinks=true}
\urlstyle{same}  % don't use monospace font for urls
\usepackage{color}
\usepackage{fancyvrb}
\newcommand{\VerbBar}{|}
\newcommand{\VERB}{\Verb[commandchars=\\\{\}]}
\DefineVerbatimEnvironment{Highlighting}{Verbatim}{commandchars=\\\{\}}
% Add ',fontsize=\small' for more characters per line
\newenvironment{Shaded}{}{}
\newcommand{\KeywordTok}[1]{\textcolor[rgb]{0.00,0.44,0.13}{\textbf{{#1}}}}
\newcommand{\DataTypeTok}[1]{\textcolor[rgb]{0.56,0.13,0.00}{{#1}}}
\newcommand{\DecValTok}[1]{\textcolor[rgb]{0.25,0.63,0.44}{{#1}}}
\newcommand{\BaseNTok}[1]{\textcolor[rgb]{0.25,0.63,0.44}{{#1}}}
\newcommand{\FloatTok}[1]{\textcolor[rgb]{0.25,0.63,0.44}{{#1}}}
\newcommand{\ConstantTok}[1]{\textcolor[rgb]{0.53,0.00,0.00}{{#1}}}
\newcommand{\CharTok}[1]{\textcolor[rgb]{0.25,0.44,0.63}{{#1}}}
\newcommand{\SpecialCharTok}[1]{\textcolor[rgb]{0.25,0.44,0.63}{{#1}}}
\newcommand{\StringTok}[1]{\textcolor[rgb]{0.25,0.44,0.63}{{#1}}}
\newcommand{\VerbatimStringTok}[1]{\textcolor[rgb]{0.25,0.44,0.63}{{#1}}}
\newcommand{\SpecialStringTok}[1]{\textcolor[rgb]{0.73,0.40,0.53}{{#1}}}
\newcommand{\ImportTok}[1]{{#1}}
\newcommand{\CommentTok}[1]{\textcolor[rgb]{0.38,0.63,0.69}{\textit{{#1}}}}
\newcommand{\DocumentationTok}[1]{\textcolor[rgb]{0.73,0.13,0.13}{\textit{{#1}}}}
\newcommand{\AnnotationTok}[1]{\textcolor[rgb]{0.38,0.63,0.69}{\textbf{\textit{{#1}}}}}
\newcommand{\CommentVarTok}[1]{\textcolor[rgb]{0.38,0.63,0.69}{\textbf{\textit{{#1}}}}}
\newcommand{\OtherTok}[1]{\textcolor[rgb]{0.00,0.44,0.13}{{#1}}}
\newcommand{\FunctionTok}[1]{\textcolor[rgb]{0.02,0.16,0.49}{{#1}}}
\newcommand{\VariableTok}[1]{\textcolor[rgb]{0.10,0.09,0.49}{{#1}}}
\newcommand{\ControlFlowTok}[1]{\textcolor[rgb]{0.00,0.44,0.13}{\textbf{{#1}}}}
\newcommand{\OperatorTok}[1]{\textcolor[rgb]{0.40,0.40,0.40}{{#1}}}
\newcommand{\BuiltInTok}[1]{{#1}}
\newcommand{\ExtensionTok}[1]{{#1}}
\newcommand{\PreprocessorTok}[1]{\textcolor[rgb]{0.74,0.48,0.00}{{#1}}}
\newcommand{\AttributeTok}[1]{\textcolor[rgb]{0.49,0.56,0.16}{{#1}}}
\newcommand{\RegionMarkerTok}[1]{{#1}}
\newcommand{\InformationTok}[1]{\textcolor[rgb]{0.38,0.63,0.69}{\textbf{\textit{{#1}}}}}
\newcommand{\WarningTok}[1]{\textcolor[rgb]{0.38,0.63,0.69}{\textbf{\textit{{#1}}}}}
\newcommand{\AlertTok}[1]{\textcolor[rgb]{1.00,0.00,0.00}{\textbf{{#1}}}}
\newcommand{\ErrorTok}[1]{\textcolor[rgb]{1.00,0.00,0.00}{\textbf{{#1}}}}
\newcommand{\NormalTok}[1]{{#1}}
\setlength{\emergencystretch}{3em}  % prevent overfull lines
\providecommand{\tightlist}{%
  \setlength{\itemsep}{0pt}\setlength{\parskip}{0pt}}
\setcounter{secnumdepth}{5}
% Redefines (sub)paragraphs to behave more like sections
\ifx\paragraph\undefined\else
\let\oldparagraph\paragraph
\renewcommand{\paragraph}[1]{\oldparagraph{#1}\mbox{}}
\fi
\ifx\subparagraph\undefined\else
\let\oldsubparagraph\subparagraph
\renewcommand{\subparagraph}[1]{\oldsubparagraph{#1}\mbox{}}
\fi

% set default figure placement to htbp
\makeatletter
\def\fps@figure{htbp}
\makeatother


\date{}

\begin{document}

{
\setcounter{tocdepth}{3}
\tableofcontents
}
\subsubsection{En résumé}\label{en-ruxe9sumuxe9}

\begin{enumerate}
\def\labelenumi{\arabic{enumi}.}
\tightlist
\item
  Un pointeur de fonction permet de stocker une référence vers une
  fonction ;
\item
  Il n'est pas nécessaire d'employer l'opérateur \texttt{\&} pour
  obtenir l'adresse d'une fonction ;
\item
  Le déréférencement n'est pas obligatoire lors de l'utilisation d'un
  pointeur de fonction ;
\item
  Une fonction employant un pointeur vers une autre fonction reçu en
  argument est appelée une fonction de rappel (\emph{callback function}
  en anglais) ;
\item
  Le recours à une structure permet d'éviter les problèmes de
  définitions récursives ;
\item
  Il est possible d'utiliser un pointeur « générique » de fonction en ne
  fournissant aucune information quant aux arguments lors de sa
  définition ;
\item
  Un pointeur « générique » de fonction peut être converti vers ou
  depuis n'importe quel autre type de pointeur de fonction du moment que
  le type de retour reste identique ;
\item
  Lors de l'utilisation d'un pointeur « générique » de fonction, les
  arguments transmis sont promus, mais aucune conversion implicite n'est
  réalisée par le compilateur ;
\item
  Lors de l'utilisation d'un pointeur « générique » de fonction, un
  pointeur nul ne peut être fourni qu'à l'aide d'une conversion
  explicite.À ce stade, vous pensiez sans doute avoir fait le tour des
  pointeurs, que ces derniers n'avaient plus de secrets pour vous et que
  vous maîtrisiez enfin tous leurs aspects ainsi que leur syntaxe
  parfois déroutante ? \emph{Eh} bien, pas encore ! :diable:
\end{enumerate}

Il reste un dernier type de pointeur (et non des moindres) que nous
avons tu jusqu'ici : les \textbf{pointeurs de fonction}.Jusqu'à
maintenant, nous avons manipulé des pointeurs sur objet, c'est-à-dire
des adresses vers des zones mémoires contenant des \emph{données} (des
entiers, des flottants, des structures, etc.). Toutefois, il est
également possible de référencer des \emph{instructions} et ceci est
réalisé en C à l'aide des pointeurs de fonction.

Un pointeur de fonction se définit à l'aide d'une syntaxe mélangeant
celle des pointeurs sur tableau et celles des prototypes de fonction.
Sans plus attendre, voici ci-dessous la définition d'un pointeur sur une
fonction retournant un \texttt{int} et attendant un \texttt{int} comme
argument.

\begin{Shaded}
\begin{Highlighting}[]
\DataTypeTok{int} \NormalTok{(*pf)(}\DataTypeTok{int}\NormalTok{);}
\end{Highlighting}
\end{Shaded}

Comme vous le voyez, il est nécessaire, tout comme les pointeurs sur
tableau, d'entourer le symbole \texttt{*} et l'identificateur de
parenthèses, ici afin d'éviter que cette déclaration ne soit vue comme
un prototype et non comme un pointeur de fonction. Autre particularité :
le type de retour, le nombre d'arguments et leur type doivent également
être spécifiés.

\section{Initialisation}\label{initialisation}

{[}{[}question{]}{]} \textbar{} Ok\ldots{} Et je lui affecte comment
l'adresse d'une fonction, moi, à ce machin ? D'ailleurs, elles ont une
adresse, les fonctions ? :euh:

Oui et comme d'habitude, cela est réalisé à l'aide de l'opérateur
\texttt{\&}. \^{}\^{}\\
En fait, dans le cas des fonctions, il n'est pas obligatoire de recourir
à cet opérateur, ainsi, les deux syntaxes suivantes sont correctes.

\begin{Shaded}
\begin{Highlighting}[]
\DataTypeTok{int} \NormalTok{(*pf)(}\DataTypeTok{int}\NormalTok{);}

\NormalTok{pf = &fonction;}
\NormalTok{pf = fonction;}
\end{Highlighting}
\end{Shaded}

Ceci est dû, à l'image des tableaux, à une conversion implicite : sauf
s'il est l'opérande de l'opérateur \texttt{\&}, un identificateur de
fonction est converti en un pointeur sur cette fonction. L'utilisation
de l'opérateur \texttt{\&} est donc facultative, mais elle a le mérite
de clarifier un peu les choses. Pour cette raison, nous utiliserons
cette syntaxe dans la suite de ce cours.\# Déréférencement

Un pointeur de fonction s'emploie de la même manière qu'un pointeur
classique, si ce n'est que l'opérateur \texttt{*} et l'identificateur
doivent à nouveau être entre parenthèses. Pour le reste, la liste des
arguments suit l'expression déréférencée, comme pour un appel de
fonction classique.

\begin{Shaded}
\begin{Highlighting}[]
\PreprocessorTok{#include }\ImportTok{<stdio.h>}


\DataTypeTok{static} \DataTypeTok{int} \NormalTok{triple(}\DataTypeTok{int} \NormalTok{a)}
\NormalTok{\{}
    \ControlFlowTok{return} \NormalTok{a * }\DecValTok{3}\NormalTok{;}
\NormalTok{\}}


\DataTypeTok{int} \NormalTok{main(}\DataTypeTok{void}\NormalTok{)}
\NormalTok{\{}
    \DataTypeTok{int} \NormalTok{(*pt)(}\DataTypeTok{int}\NormalTok{) = &triple;}

    \NormalTok{printf(}\StringTok{"%d.}\SpecialCharTok{\textbackslash{}n}\StringTok{"}\NormalTok{, (*pt)(}\DecValTok{3}\NormalTok{));}
    \ControlFlowTok{return} \DecValTok{0}\NormalTok{;}
\NormalTok{\}}
\end{Highlighting}
\end{Shaded}

\begin{verbatim}
9.
\end{verbatim}

Toutefois, particularité des fonctions oblige, sachez que le
déréférencement n'\emph{est pas nécessaire}. Ceci à cause de la
conversion implicite expliquée précédemment : un identificateur de
fonction est, sauf s'il est l'opérande de l'opérateur \texttt{\&},
converti en un pointeur sur cette fonction. L'appel \texttt{triple(3)}
cache donc en fait un pointeur de fonction qui, comme vous le voyez,
n'est \emph{pas} déréférencé.

{[}{[}question{]}{]} \textbar{} \emph{Heu}\ldots{} Mais pourquoi
l'expression \texttt{(*pt)(3)} ne provoque-t-elle pas une erreur si
c'est un pointeur de fonction qui est nécessaire lors d'un appel ?

Parce que la conversion implicite aura lieu juste après le
déréférencement. :-°\\
\emph{Eh} oui, déréférencé un pointeur de fonction, c'est un peu reculer
pour mieux sauter : nous obtenons une expression équivalente à un
identificateur de fonction qui sera ensuite convertie en un pointeur de
fonction. Les deux expressions suivantes sont donc équivalentes.

\begin{Shaded}
\begin{Highlighting}[]
\NormalTok{triple(}\DecValTok{3}\NormalTok{);}
\NormalTok{(*pt)(}\DecValTok{3}\NormalTok{);}
\end{Highlighting}
\end{Shaded}

L'intérêt d'employer le déréférencement est purement syntaxique : cela
permet de distinguer des appels effectuer via des pointeurs des appels
de fonction classiques.

\section{Passage en argument}\label{passage-en-argument}

Comme n'importe quel pointeur, un pointeur de fonction peut être passé
en argument d'une autre fonction (c'est d'ailleurs tout l'intérêt de
ceux-ci, comme nous le verrons bientôt). Pour ce faire, il vous suffit
d'employer la même syntaxe que pour une déclaration.

\begin{Shaded}
\begin{Highlighting}[]
\PreprocessorTok{#include }\ImportTok{<stdio.h>}


\DataTypeTok{static} \DataTypeTok{int} \NormalTok{triple(}\DataTypeTok{int} \NormalTok{a)}
\NormalTok{\{}
    \ControlFlowTok{return} \NormalTok{a * }\DecValTok{3}\NormalTok{;}
\NormalTok{\}}


\DataTypeTok{static} \DataTypeTok{int} \NormalTok{quadruple(}\DataTypeTok{int} \NormalTok{a)}
\NormalTok{\{}
    \ControlFlowTok{return} \NormalTok{a * }\DecValTok{4}\NormalTok{;}
\NormalTok{\}}


\DataTypeTok{static} \DataTypeTok{void} \NormalTok{affiche(}\DataTypeTok{int} \NormalTok{a, }\DataTypeTok{int} \NormalTok{(*pf)(}\DataTypeTok{int}\NormalTok{))}
\NormalTok{\{}
    \NormalTok{printf(}\StringTok{"%d.}\SpecialCharTok{\textbackslash{}n}\StringTok{"}\NormalTok{, (*pf)(a));}
\NormalTok{\}}


\DataTypeTok{int} \NormalTok{main(}\DataTypeTok{void}\NormalTok{)}
\NormalTok{\{}
    \NormalTok{affiche(}\DecValTok{3}\NormalTok{, &triple);}
    \NormalTok{affiche(}\DecValTok{3}\NormalTok{, &quadruple);}
    \ControlFlowTok{return} \DecValTok{0}\NormalTok{;}
\NormalTok{\}}
\end{Highlighting}
\end{Shaded}

\begin{verbatim}
9.
12.
\end{verbatim}

{[}{[}information{]}{]} \textbar{} La fonction \texttt{affiche()}
ci-dessus est ce que l'on appelle une \emph{fonction de rappel}
(\emph{callback function} en anglais), c'est-à-dire une fonction faisant
appel à une autre à l'aide de l'adresse qui lui est fournie en argument.

\section{Retour de fonction}\label{retour-de-fonction}

Dans l'autre sens, il est possible de retourner un pointeur de fonction
à l'aide d'une syntaxe\ldots{} un peu lourde. :-°

\begin{Shaded}
\begin{Highlighting}[]
\PreprocessorTok{#include }\ImportTok{<stddef.h>}
\PreprocessorTok{#include }\ImportTok{<stdio.h>}


\DataTypeTok{static} \DataTypeTok{void} \NormalTok{affiche_pair(}\DataTypeTok{int} \NormalTok{a)}
\NormalTok{\{}
    \NormalTok{printf(}\StringTok{"%d est pair.}\SpecialCharTok{\textbackslash{}n}\StringTok{"}\NormalTok{, a);}
\NormalTok{\}}


\DataTypeTok{static} \DataTypeTok{void} \NormalTok{affiche_impair(}\DataTypeTok{int} \NormalTok{a)}
\NormalTok{\{}
    \NormalTok{printf(}\StringTok{"%d est impair.}\SpecialCharTok{\textbackslash{}n}\StringTok{"}\NormalTok{, a);}
\NormalTok{\}}


\DataTypeTok{static} \DataTypeTok{void} \NormalTok{(*affiche(}\DataTypeTok{int} \NormalTok{a))(}\DataTypeTok{int}\NormalTok{)}
\NormalTok{\{}
    \ControlFlowTok{if} \NormalTok{(a % }\DecValTok{2} \NormalTok{== }\DecValTok{0}\NormalTok{)}
        \ControlFlowTok{return} \NormalTok{&affiche_pair;}
    \ControlFlowTok{else}
        \ControlFlowTok{return} \NormalTok{&affiche_impair;}
\NormalTok{\}}
    


\DataTypeTok{int} \NormalTok{main(}\DataTypeTok{void}\NormalTok{)}
\NormalTok{\{}
    \DataTypeTok{void} \NormalTok{(*pf)(}\DataTypeTok{int}\NormalTok{);}
    \DataTypeTok{int} \NormalTok{a = }\DecValTok{2}\NormalTok{;}

    \NormalTok{pf = affiche(a);}
    \NormalTok{(*pf)(a);}
    \ControlFlowTok{return} \DecValTok{0}\NormalTok{;}
\NormalTok{\}}
\end{Highlighting}
\end{Shaded}

\begin{verbatim}
2 est pair.
\end{verbatim}

Comme pour une variable de type pointeur de fonction, le symbole
\texttt{*} doit être entouré de parenthèses ainsi que l'identificateur
qui le suit. Toutefois, lorsqu'il s'agit du type de retour d'une
fonction, la liste des arguments doit également être placée entre ces
parenthèses.

Dans cet exemple, la fonction \texttt{affiche()} attend un \texttt{int}
et retourne un pointeur sur une fonction ne retournant rien et utilisant
un argument de type \texttt{int}. Suivant si \texttt{a} est pair ou
impair, la fonction \texttt{affiche()} retourne l'adresse de la fonction
\texttt{affichage\_pair()} ou \texttt{affichage\_impair()} qui est
recueillie par le pointeur \texttt{pf} de la fonction
\texttt{main()}.Vous le savez, le type \texttt{void} est employé en C
pour produire un pointeur générique qui peut se voir assigner n'importe
quel type de pointeur et être converti vers n'importe quel type de
pointeurs. Cette définition est toutefois incomplète car il doit en fait
être précisé que cela ne fonctionne que pour des pointeurs sur des
\emph{objets}. Le code ci-dessous est donc incorrect.

\begin{Shaded}
\begin{Highlighting}[]
\DataTypeTok{int} \NormalTok{(*pf)(}\DataTypeTok{int}\NormalTok{);}
\DataTypeTok{void} \NormalTok{*p;}

\NormalTok{pf = p; }\CommentTok{/* Faux. */}
\NormalTok{p = pf; }\CommentTok{/* Faux également. */}
\end{Highlighting}
\end{Shaded}

Pareillement, une conversion explicite d'un pointeur sur un objet vers
ou depuis un pointeur sur fonction est interdite (ou, plus précisément,
son résultat est indéterminé). Ceci exclut donc l'utilisation de
l'indicateur \texttt{p} de la fonction \texttt{printf()} pour afficher
un pointeur de fonction.

\begin{Shaded}
\begin{Highlighting}[]
\NormalTok{printf(}\StringTok{"%p.}\SpecialCharTok{\textbackslash{}n}\StringTok{"}\NormalTok{, (}\DataTypeTok{void} \NormalTok{*)pf); }\CommentTok{/* Faux. */}
\end{Highlighting}
\end{Shaded}

Toutefois, les pointeurs de fonction disposent de leur propre pointeur «
générique ». Nous utilisons ici les guillemets car il ne l'est pas tout
à fait puisqu'il peut notamment être utilisé à l'inverse d'un pointeur
sur \texttt{void}. Un pointeur « générique » de fonction se déclare
comme un pointeur de fonction, mais en ne spécifiant que le type de
retour.

\begin{Shaded}
\begin{Highlighting}[]
\DataTypeTok{int} \NormalTok{(*pf)();}
\end{Highlighting}
\end{Shaded}

Un tel pointeur peut se voir assigner n'importe quel pointeur sur
fonction du moment que le type de retour de celui-ci est identique au
sien. Inversement, ce pointeur « générique » peut être affecté à un
autre pointeur sous la même condition. Dans notre cas, le type de retour
doit donc être \texttt{int}.

\begin{Shaded}
\begin{Highlighting}[]
\DataTypeTok{int} \NormalTok{(*f)(}\DataTypeTok{int}\NormalTok{, }\DataTypeTok{int}\NormalTok{);}
\DataTypeTok{int} \NormalTok{(*g)(}\DataTypeTok{char}\NormalTok{, }\DataTypeTok{char}\NormalTok{, }\DataTypeTok{double}\NormalTok{);}
\DataTypeTok{void} \NormalTok{(*h)(}\DataTypeTok{void}\NormalTok{);}
\DataTypeTok{int} \NormalTok{(*pf)();}

\NormalTok{pf = f; }\CommentTok{/* Ok. */}
\NormalTok{pf = g; }\CommentTok{/* Ok. */}
\NormalTok{pf = h; }\CommentTok{/* Faux car le type de retour de `h` est `void`. */}
\NormalTok{f = pf; }\CommentTok{/* Ok. */}
\end{Highlighting}
\end{Shaded}

{[}{[}attention{]}{]} \textbar{} Il existe cependant une exception
supplémentaire : une fonction à nombre variables d'arguments ne peut pas
être affectée à un tel pointeur, même si le type de retour est
identique. Nous verrons bientôt de quoi il s'agit, mais pour l'heure,
sachez que les fonctions de la famille de \texttt{printf()} et de
\texttt{scanf()} sont concernées par cette règle.

\section{La promotion des arguments}\label{la-promotion-des-arguments}

{[}{[}question{]}{]} \textbar{} \emph{Hé} là, minute papillon ! Il se
passe quoi si j'utilise le pointeur \texttt{pf} avec une fonction qui
attend normalement des arguments ?

Excellente question ! :)

Vous vous en doutez : les arguments peuvent toujours être envoyé à la
fonction référencée. Cependant, il y a une subtilité. Étant donné qu'un
pointeur « générique » de fonction ne fournit aucune information quant
aux arguments, le compilateur ne peut pas convertir ceux-ci vers le type
attendu par la fonction. Ainsi, si vous fournissez un \texttt{int} et
que la fonction attend un \texttt{char}, le \texttt{int} ne sera pas
converti vers le type \texttt{char} par le compilateur.

Toutefois, dans un tel cas, plusieurs conversions implicites sont
appliquées afin de limiter les types possibles (on parle de « promotion
des arguments ») :

\begin{enumerate}
\def\labelenumi{\arabic{enumi}.}
\tightlist
\item
  Un argument de type entier de rang inférieur ou égal à celui du type
  \texttt{int} (soit \texttt{char} et \texttt{short} le plus souvent)
  est converti vers le type \texttt{int} (ou \texttt{unsigned\ int} si
  le type \texttt{int} ne peut pas représenter toutes les valeurs du
  type d'origine).
\item
  Un argument de type \texttt{float} et converti vers le type
  \texttt{double}.
\end{enumerate}

{[}{[}erreur{]}{]} \textbar{} Ceci signifie qu'une fonction appelée à
l'aide d'un pointeur « générique » de fonction ne pourra \emph{jamais}
recevoir des arguments de type \texttt{char}, \texttt{short} ou
\texttt{float}.

Illustration.

\begin{Shaded}
\begin{Highlighting}[]
\PreprocessorTok{#include }\ImportTok{<stdio.h>}


\DataTypeTok{static} \DataTypeTok{float} \NormalTok{triple(}\DataTypeTok{float} \NormalTok{f)}
\NormalTok{\{}
    \ControlFlowTok{return} \FloatTok{3.F} \NormalTok{* f;}
\NormalTok{\}}


\DataTypeTok{static} \DataTypeTok{short} \NormalTok{quadruple(}\DataTypeTok{short} \NormalTok{n)}
\NormalTok{\{}
    \ControlFlowTok{return} \DecValTok{4} \NormalTok{* n;}
\NormalTok{\}}


\DataTypeTok{int} \NormalTok{main(}\DataTypeTok{void}\NormalTok{)}
\NormalTok{\{}
    \DataTypeTok{float} \NormalTok{(*pt)() = &triple;}
    \DataTypeTok{short} \NormalTok{(*pq)() = &quadruple;}

    \NormalTok{printf(}\StringTok{"triple = %f.}\SpecialCharTok{\textbackslash{}n}\StringTok{"}\NormalTok{, (*pt)(}\FloatTok{3.F}\NormalTok{)); }\CommentTok{/* Faux. */}
    \NormalTok{printf(}\StringTok{"quadruple = %d.}\SpecialCharTok{\textbackslash{}n}\StringTok{"}\NormalTok{, (*pq)(}\DecValTok{2}\NormalTok{)); }\CommentTok{/* Faux. */}
    \ControlFlowTok{return} \DecValTok{0}\NormalTok{;}
\NormalTok{\}}
\end{Highlighting}
\end{Shaded}

\section{Les pointeurs nuls}\label{les-pointeurs-nuls}

L'absence de conversions par le compilateur dans le cas où aucune
information n'est fournie par rapport aux arguments pose un problème
particulier dans le cas des pointeurs nuls et tout spécialement lors de
l'usage de la macroconstante \texttt{NULL}.

Rappelez-vous : un pointeur nul est construit en convertissant, soit
explicitement, soit implicitement, zéro (entier) vers un type pointeur.
Or, étant donné que le compilateur n'effectuera aucune conversion
implicite dans notre cas, nous ne pouvons compter que sur les
conversions explicites.

Et c'est ici que le bât blesse : la macroconstante \texttt{NULL} a deux
valeurs possibles : \texttt{(void\ *)0} ou \texttt{0}, le choix étant
laissé aux différents systèmes. La première ne pose pas de problème,
mais la seconde en pose un plutôt gênant : c'est un \texttt{int} qui
sera passé comme argument et non un pointeur nul.

Dès lors, lorsque vous employez un pointeur « générique » de fonction,
vous \emph{devez} recourir à une conversion explicite si vous souhaitez
produire un pointeur nul.

\begin{Shaded}
\begin{Highlighting}[]
\DataTypeTok{static} \DataTypeTok{void} \NormalTok{affiche(}\DataTypeTok{char} \NormalTok{*chaine)}
\NormalTok{\{}
    \ControlFlowTok{if} \NormalTok{(chaine != NULL)}
        \NormalTok{puts(chaine);}
\NormalTok{\}}

\CommentTok{/* ... */}

\DataTypeTok{void} \NormalTok{(*pf)() = &affiche;}

\NormalTok{(*pf)(NULL); }\CommentTok{/* Faux. */}
\NormalTok{(*pf)(}\DecValTok{0}\NormalTok{); }\CommentTok{/* Faux. */}
\NormalTok{(*pf)((}\DataTypeTok{char}\NormalTok{*)}\DecValTok{0}\NormalTok{); }\CommentTok{/* Ok. */}
\end{Highlighting}
\end{Shaded}

\end{document}
