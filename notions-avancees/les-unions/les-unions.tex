\PassOptionsToPackage{unicode=true}{hyperref} % options for packages loaded elsewhere
\PassOptionsToPackage{hyphens}{url}
%
\documentclass[]{article}
\usepackage{lmodern}
\usepackage{amssymb,amsmath}
\usepackage{ifxetex,ifluatex}
\usepackage{fixltx2e} % provides \textsubscript
\ifnum 0\ifxetex 1\fi\ifluatex 1\fi=0 % if pdftex
  \usepackage[T1]{fontenc}
  \usepackage[utf8]{inputenc}
  \usepackage{textcomp} % provides euro and other symbols
\else % if luatex or xelatex
  \usepackage{unicode-math}
  \defaultfontfeatures{Ligatures=TeX,Scale=MatchLowercase}
\fi
% use upquote if available, for straight quotes in verbatim environments
\IfFileExists{upquote.sty}{\usepackage{upquote}}{}
% use microtype if available
\IfFileExists{microtype.sty}{%
\usepackage[]{microtype}
\UseMicrotypeSet[protrusion]{basicmath} % disable protrusion for tt fonts
}{}
\IfFileExists{parskip.sty}{%
\usepackage{parskip}
}{% else
\setlength{\parindent}{0pt}
\setlength{\parskip}{6pt plus 2pt minus 1pt}
}
\usepackage{hyperref}
\hypersetup{
            pdfborder={0 0 0},
            breaklinks=true}
\urlstyle{same}  % don't use monospace font for urls
\usepackage{color}
\usepackage{fancyvrb}
\newcommand{\VerbBar}{|}
\newcommand{\VERB}{\Verb[commandchars=\\\{\}]}
\DefineVerbatimEnvironment{Highlighting}{Verbatim}{commandchars=\\\{\}}
% Add ',fontsize=\small' for more characters per line
\newenvironment{Shaded}{}{}
\newcommand{\KeywordTok}[1]{\textcolor[rgb]{0.00,0.44,0.13}{\textbf{{#1}}}}
\newcommand{\DataTypeTok}[1]{\textcolor[rgb]{0.56,0.13,0.00}{{#1}}}
\newcommand{\DecValTok}[1]{\textcolor[rgb]{0.25,0.63,0.44}{{#1}}}
\newcommand{\BaseNTok}[1]{\textcolor[rgb]{0.25,0.63,0.44}{{#1}}}
\newcommand{\FloatTok}[1]{\textcolor[rgb]{0.25,0.63,0.44}{{#1}}}
\newcommand{\ConstantTok}[1]{\textcolor[rgb]{0.53,0.00,0.00}{{#1}}}
\newcommand{\CharTok}[1]{\textcolor[rgb]{0.25,0.44,0.63}{{#1}}}
\newcommand{\SpecialCharTok}[1]{\textcolor[rgb]{0.25,0.44,0.63}{{#1}}}
\newcommand{\StringTok}[1]{\textcolor[rgb]{0.25,0.44,0.63}{{#1}}}
\newcommand{\VerbatimStringTok}[1]{\textcolor[rgb]{0.25,0.44,0.63}{{#1}}}
\newcommand{\SpecialStringTok}[1]{\textcolor[rgb]{0.73,0.40,0.53}{{#1}}}
\newcommand{\ImportTok}[1]{{#1}}
\newcommand{\CommentTok}[1]{\textcolor[rgb]{0.38,0.63,0.69}{\textit{{#1}}}}
\newcommand{\DocumentationTok}[1]{\textcolor[rgb]{0.73,0.13,0.13}{\textit{{#1}}}}
\newcommand{\AnnotationTok}[1]{\textcolor[rgb]{0.38,0.63,0.69}{\textbf{\textit{{#1}}}}}
\newcommand{\CommentVarTok}[1]{\textcolor[rgb]{0.38,0.63,0.69}{\textbf{\textit{{#1}}}}}
\newcommand{\OtherTok}[1]{\textcolor[rgb]{0.00,0.44,0.13}{{#1}}}
\newcommand{\FunctionTok}[1]{\textcolor[rgb]{0.02,0.16,0.49}{{#1}}}
\newcommand{\VariableTok}[1]{\textcolor[rgb]{0.10,0.09,0.49}{{#1}}}
\newcommand{\ControlFlowTok}[1]{\textcolor[rgb]{0.00,0.44,0.13}{\textbf{{#1}}}}
\newcommand{\OperatorTok}[1]{\textcolor[rgb]{0.40,0.40,0.40}{{#1}}}
\newcommand{\BuiltInTok}[1]{{#1}}
\newcommand{\ExtensionTok}[1]{{#1}}
\newcommand{\PreprocessorTok}[1]{\textcolor[rgb]{0.74,0.48,0.00}{{#1}}}
\newcommand{\AttributeTok}[1]{\textcolor[rgb]{0.49,0.56,0.16}{{#1}}}
\newcommand{\RegionMarkerTok}[1]{{#1}}
\newcommand{\InformationTok}[1]{\textcolor[rgb]{0.38,0.63,0.69}{\textbf{\textit{{#1}}}}}
\newcommand{\WarningTok}[1]{\textcolor[rgb]{0.38,0.63,0.69}{\textbf{\textit{{#1}}}}}
\newcommand{\AlertTok}[1]{\textcolor[rgb]{1.00,0.00,0.00}{\textbf{{#1}}}}
\newcommand{\ErrorTok}[1]{\textcolor[rgb]{1.00,0.00,0.00}{\textbf{{#1}}}}
\newcommand{\NormalTok}[1]{{#1}}
\setlength{\emergencystretch}{3em}  % prevent overfull lines
\providecommand{\tightlist}{%
  \setlength{\itemsep}{0pt}\setlength{\parskip}{0pt}}
\setcounter{secnumdepth}{5}
% Redefines (sub)paragraphs to behave more like sections
\ifx\paragraph\undefined\else
\let\oldparagraph\paragraph
\renewcommand{\paragraph}[1]{\oldparagraph{#1}\mbox{}}
\fi
\ifx\subparagraph\undefined\else
\let\oldsubparagraph\subparagraph
\renewcommand{\subparagraph}[1]{\oldsubparagraph{#1}\mbox{}}
\fi

% set default figure placement to htbp
\makeatletter
\def\fps@figure{htbp}
\makeatother


\date{}

\begin{document}

{
\setcounter{tocdepth}{3}
\tableofcontents
}
\subsubsection{En résumé}\label{en-ruxe9sumuxe9}

\begin{enumerate}
\def\labelenumi{\arabic{enumi}.}
\tightlist
\item
  Une union est un agrégat regroupant des objets de types différents,
  mais ne pouvant en stocker qu'un seul à la fois ;
\item
  Comme les structures, les unions peuvent comprendre des \emph{bits} de
  bourrage, mais uniquement à leur fin ;
\item
  Une union acquiert l'alignement le plus strict parmi ceux de ses
  membres.Dans la deuxième partie de ce cours nous vous avons présenté
  la notion d'agrégat qui recouvrait les tableaux et les structures.
  Toutefois, nous avons passé sous silence un dernier agrégat plus
  discret moins utilisé : les \textbf{unions}.Étant donné leur
  singularité, les unions sont rarement employées. Leur principal
  intérêt est de réduire l'espace mémoire utilisé là où une structure ne
  le permet pas.
\end{enumerate}

Par exemple, imaginez que vous souhaitiez construire une structure
pouvant accueillir plusieurs types possibles, par exemple des entiers et
des flottants. Vous aurez besoin de trois champs : un indiquant quel
type est stocké dans la structure et deux permettant de stocker soit un
entier soit un flottant.

\begin{Shaded}
\begin{Highlighting}[]
\KeywordTok{struct} \NormalTok{nombre}
\NormalTok{\{}
    \DataTypeTok{unsigned} \NormalTok{entier : }\DecValTok{1}\NormalTok{;}
    \DataTypeTok{unsigned} \NormalTok{flottant : }\DecValTok{1}\NormalTok{;}
    \DataTypeTok{int} \NormalTok{e;}
    \DataTypeTok{double} \NormalTok{f;}
\NormalTok{\};}
\end{Highlighting}
\end{Shaded}

Toutefois, vous gaspiller ici de la mémoire puisque seul un des deux
objets sera stockés. Une union est ici la bienvenue afin d'économiser de
la mémoire.

\begin{Shaded}
\begin{Highlighting}[]
\KeywordTok{struct} \NormalTok{nombre}
\NormalTok{\{}
    \DataTypeTok{unsigned} \NormalTok{entier : }\DecValTok{1}\NormalTok{;}
    \DataTypeTok{unsigned} \NormalTok{flottant : }\DecValTok{1}\NormalTok{;}
    \KeywordTok{union}
    \NormalTok{\{}
        \DataTypeTok{int} \NormalTok{e;}
        \DataTypeTok{double} \NormalTok{f;}
    \NormalTok{\} u;}
\NormalTok{\};}
\end{Highlighting}
\end{Shaded}

Le code suivant illustre l'utilisation de cette construction.

\begin{Shaded}
\begin{Highlighting}[]
\PreprocessorTok{#include }\ImportTok{<stdio.h>}

\KeywordTok{struct} \NormalTok{nombre}
\NormalTok{\{}
    \DataTypeTok{unsigned} \NormalTok{entier : }\DecValTok{1}\NormalTok{;}
    \DataTypeTok{unsigned} \NormalTok{flottant : }\DecValTok{1}\NormalTok{;}
    \KeywordTok{union}
    \NormalTok{\{}
        \DataTypeTok{int} \NormalTok{e;}
        \DataTypeTok{double} \NormalTok{f;}
    \NormalTok{\} u;}
\NormalTok{\};}


\DataTypeTok{static} \DataTypeTok{void} \NormalTok{affiche_nombre(}\KeywordTok{struct} \NormalTok{nombre n)}
\NormalTok{\{}
    \ControlFlowTok{if} \NormalTok{(n.entier)}
        \NormalTok{printf(}\StringTok{"%d}\SpecialCharTok{\textbackslash{}n}\StringTok{"}\NormalTok{, n.u.e);}
    \ControlFlowTok{else} \ControlFlowTok{if} \NormalTok{(n.flottant)}
        \NormalTok{printf(}\StringTok{"%f}\SpecialCharTok{\textbackslash{}n}\StringTok{"}\NormalTok{, n.u.f);}
\NormalTok{\}}


\DataTypeTok{int} \NormalTok{main(}\DataTypeTok{void}\NormalTok{)}
\NormalTok{\{}
    \KeywordTok{struct} \NormalTok{nombre a = \{ }\DecValTok{0} \NormalTok{\};}
    \KeywordTok{struct} \NormalTok{nombre b = \{ }\DecValTok{0} \NormalTok{\};}

    \NormalTok{a.entier = }\DecValTok{1}\NormalTok{;}
    \NormalTok{a.u.e = }\DecValTok{10}\NormalTok{;}
    \NormalTok{b.flottant = }\DecValTok{1}\NormalTok{;}
    \NormalTok{b.u.f = }\FloatTok{10.56}\NormalTok{;}

    \NormalTok{affiche_nombre(a);}
    \NormalTok{affiche_nombre(b);}
    \ControlFlowTok{return} \DecValTok{0}\NormalTok{;}
\NormalTok{\}}
\end{Highlighting}
\end{Shaded}

\begin{verbatim}
10
10.560000
\end{verbatim}

Une autre utilisation fréquente des unions est de permettre de modifier
l'alignement d'un objet ou, plus précisément, d'augmenter l'alignement
d'un objet. En fait, il s'agit d'une des conséquences de l'union : étant
donné qu'elle doit pouvoir stocker n'importe lequel de ses membres, son
alignement doit être le plus élevé parmi celui de ses membres.

{[}{[}information{]}{]} \textbar{} Pour rappel, l'alignement d'un type
peut être connu à l'aide de la macrofonction \texttt{offsetof()} et d'un
type structure de la forme \texttt{struct\ \{\ char\ c;\ type\ x;\ \}}.

Ainsi, si nous souhaitons aligner un objet de type \texttt{int} de la
même manière qu'un objet de type \texttt{double}, il nous suffit de
construire une union qui comprendra les deux types. Le code suivant
vérifie ce qui vient d'être dit.

\begin{Shaded}
\begin{Highlighting}[]
\PreprocessorTok{#include }\ImportTok{<stddef.h>}
\PreprocessorTok{#include }\ImportTok{<stdio.h>}

\KeywordTok{union} \NormalTok{nombre}
\NormalTok{\{}
    \DataTypeTok{int} \NormalTok{e;}
    \DataTypeTok{double} \NormalTok{f;}
\NormalTok{\};}


\DataTypeTok{int} \NormalTok{main(}\DataTypeTok{void}\NormalTok{)}
\NormalTok{\{}
    \NormalTok{printf(}\StringTok{"int : %u}\SpecialCharTok{\textbackslash{}n}\StringTok{"}\NormalTok{, (}\DataTypeTok{unsigned}\NormalTok{)offsetof(}\KeywordTok{struct} \NormalTok{\{ }\DataTypeTok{char} \NormalTok{c; }\DataTypeTok{int} \NormalTok{n; \}, n));}
    \NormalTok{printf(}\StringTok{"union nombre : %u}\SpecialCharTok{\textbackslash{}n}\StringTok{"}\NormalTok{, (}\DataTypeTok{unsigned}\NormalTok{)offsetof(}\KeywordTok{struct} \NormalTok{\{ }\DataTypeTok{char} \NormalTok{c; }\KeywordTok{union} \NormalTok{nombre n; \}, n));}
    \ControlFlowTok{return} \DecValTok{0}\NormalTok{;}
\NormalTok{\}}
\end{Highlighting}
\end{Shaded}

\begin{verbatim}
int : 4
union nombre : 8
\end{verbatim}

Dans la même veine, il est ainsi possible de connaître l'alignement le
plus strict parmi les types natifs en construisant une union comportant
les types les plus contraignants, à savoir : le type \texttt{long}, le
type \texttt{long\ double} et le type \texttt{void\ *}.

\begin{Shaded}
\begin{Highlighting}[]
\PreprocessorTok{#include }\ImportTok{<stddef.h>}
\PreprocessorTok{#include }\ImportTok{<stdio.h>}

\KeywordTok{union} \NormalTok{align}
\NormalTok{\{}
    \DataTypeTok{long} \NormalTok{e;}
    \DataTypeTok{long} \DataTypeTok{double} \NormalTok{f;}
    \DataTypeTok{void} \NormalTok{*p;}
\NormalTok{\};}


\DataTypeTok{int} \NormalTok{main(}\DataTypeTok{void}\NormalTok{)}
\NormalTok{\{}
    \NormalTok{printf(}\StringTok{"union align : %u}\SpecialCharTok{\textbackslash{}n}\StringTok{"}\NormalTok{, (}\DataTypeTok{unsigned}\NormalTok{)offsetof(}\KeywordTok{struct} \NormalTok{\{ }\DataTypeTok{char} \NormalTok{c; }\KeywordTok{union} \NormalTok{align n; \}, n));}
    \ControlFlowTok{return} \DecValTok{0}\NormalTok{;}
\NormalTok{\}}
\end{Highlighting}
\end{Shaded}

\begin{verbatim}
union align : 16
\end{verbatim}

Comme pour l'exemple précédent, cette technique peut être utilisée pour
imposer l'alignement le plus strict à un objet ayant une contraine
d'alignement plus faible. Gardez bien ceci en mémoire, nous y
reviendrons lors du T.P. final. ;)Une union est, à l'image d'une
structure, un regroupement d'objet de type \emph{différents}. La nuance,
et elle est de taille, est qu'une union est un agrégat qui ne peut
contenir qu'\emph{un seul} de ses membres à la fois. Autrement dit, une
union peut accueillir la valeur de n'importe lequel de ses membres, mais
un seul à la fois.

Concernant la définition, elle est identique à celle d'une structure ci
ce n'est que le mot-clé \texttt{union} est employé en lieu et place de
\texttt{struct}.

\begin{Shaded}
\begin{Highlighting}[]
\KeywordTok{union} \NormalTok{type}
\NormalTok{\{}
    \DataTypeTok{int} \NormalTok{entier;}
    \DataTypeTok{double} \NormalTok{flottant;}
    \DataTypeTok{void} \NormalTok{*pointeur;}
    \DataTypeTok{char} \NormalTok{lettre;}
\NormalTok{\};}
\end{Highlighting}
\end{Shaded}

Le code ci-dessus défini une union \texttt{type} pouvant contenir un
objet de type \texttt{int} ou de type \texttt{double} ou de type
pointeur sur \texttt{void} ou de type \texttt{char}. Cette possiblité de
ne stocker qu'un objet à la fois est traduite par le résultat de
l'opérateur \texttt{sizeof}.

\begin{Shaded}
\begin{Highlighting}[]
\PreprocessorTok{#include }\ImportTok{<stdio.h>}

\KeywordTok{union} \NormalTok{type}
\NormalTok{\{}
    \DataTypeTok{int} \NormalTok{entier;}
    \DataTypeTok{double} \NormalTok{flottant;}
    \DataTypeTok{void} \NormalTok{*pointeur;}
    \DataTypeTok{char} \NormalTok{lettre;}
\NormalTok{\};}


\DataTypeTok{int} \NormalTok{main(}\DataTypeTok{void}\NormalTok{)}
\NormalTok{\{}
    \NormalTok{printf(}\StringTok{"%u.}\SpecialCharTok{\textbackslash{}n}\StringTok{"}\NormalTok{, }\KeywordTok{sizeof} \NormalTok{(}\KeywordTok{union} \NormalTok{type));}
    \ControlFlowTok{return} \DecValTok{0}\NormalTok{; }
\NormalTok{\}}
\end{Highlighting}
\end{Shaded}

\begin{verbatim}
8.
\end{verbatim}

Dans notre cas, la taille de l'union correspond à la taille du plus
grand type stocké à savoir les types \texttt{void\ *} et \texttt{double}
qui font huit octets. Ceci traduit bien l'impossiblité de stocker
plusieurs objets à la fois.

{[}{[}information{]}{]} \textbar{} Notez que, comme les structures, les
unions peuvent contenir des \emph{bits} de bourrage, mais uniquement à
leur fin.

Pour le surplus, une union s'utilise de la même manière qu'une structure
et l'accès aux membres s'effectue à l'aide des opérateurs \texttt{.} et
\texttt{-\textgreater{}}.

\end{document}
