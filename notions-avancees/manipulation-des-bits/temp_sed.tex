\chapter{Manipulation des bits}
\label{manipulation-des-bits}

Nous avons vu dans le chapitre précédent la représentation des différents 
type et notamment celle des types entiers. Ce chapitre est en quelque sorte 
le prolongement de celui-ci puisque nous allons à présent voir comment 
manipuler directement les \emph{bits} à l'aide des opérateurs de manipulation
des \emph{bits} et des champs de \emph{bits}.

\section{Les opérateurs de manipulation des bits}
\label{les-operateurs-de-manipulation-des-bits}

\subsection{Présentation}
\label{presentation-6}

Le langage C définit six opérateurs permettant de manipuler les
\emph{bits} :

\begin{itemize}
\item
  l'opérateur « et » : \mybox{\&} ;
\item
  l'opérateur « ou inclusif » : \mybox{\textbar{}} ;
\item
  l'opérateur « ou exclusif » : \mybox{\^{}} ;
\item
  l'opérateur de négation ou de complément : \mybox{\textasciitilde{}}
  ;
\item
  l'opérateur de décalage à droite :
  \mybox{\textgreater{}\textgreater{}} ;
\item
  l'opérateur de décalage à gauche : \mybox{\textless{}\textless{}}.
\end{itemize}

\begin{erreurbox}
  Veillez à ne pas confondre les opérateurs
de manipulations des \emph{bits} « et » (\mybox{\&}) et « ou inclusif »
(\mybox{\textbar{}}) avec leurs homologues « et » (\mybox{\&\&}) et «
ou » (\mybox{\textbar{}\textbar{}}) logiques. Il s'agit d'opérateurs
totalement différent au même titre que les opérateurs d'affectation
(\mybox{=}) et d'égalité (\mybox{==}). De même, l'opérateur de
manipulations des \emph{bits} « et » (\mybox{\&}) n'a pas de rapport
avec l'opérateur d'adressage (\mybox{\&}), ce dernier n'utilisant qu'un
opérande.
\end{erreurbox}


\begin{attentionbox}
  Notez que tous ces opérateurs
travaillent uniquement sur des \emph{entiers}. \textbar{} \textbar{} Si
néanmoins vous souhaitez modifier la représentation d'un type flottant
(ce que nous vous déconseillons), vous pouvez accéder à ses \emph{bits}
à l'aide d'un pointeur sur \mybox{unsigned\ char} (revoyez
\MYhref{https://zestedesavoir.com/tutoriels/755/le-langage-c-1
/1043_aggregats-memoire-et-fichiers/4285_lallocation-dynamique/
\#1-14293_la-notion-dobjet}{le chapitre sur l'allocation dynamique} 
si cela ne vous dit rien).
\end{attentionbox}


\subsection{Les opérateurs « et », « ou inclusif » et « ou exclusif »}
\label{les-operateurs-et-ou-inclusif-et-ou-exclusif}

Chacun de ces trois opérateurs attend deux opérandes entiers et produit
un nouvel entier en appliquant une table de vérité à chaque paire de
\emph{bits} formée à partir des \emph{bits} des deux nombres de départs.
Plus précisémment :

\begin{itemize}
\item
  L'opérateur « et » (\mybox{\&}) donnera 1 si les deux \emph{bits} de
  la paire sont à 1 ;
\item
  L'opérateur « ou inclusif » (\mybox{\textbar{}}) donnera 1 si
  \emph{au moins} un des deux \emph{bits} de la paire est à 1 ;
\item
  L'opérateur « ou exclusif » (\mybox{\^{}}) donnera 1 si \emph{un
  seul} des deux \emph{bits} de la paire est à 1.
\end{itemize}

Ceci est résumé dans le tableau ci-dessous, donnant le résultat des
trois opérateurs pour chaque paires de \emph{bits} possibles.

\begin{table}
\centering
\rowcolors{1}{}{gris-clair-tab}
\begin{tabular}{|l|l|}\hline
\rowcolor{gris-tab-entete}
\textbf{\emph{Bit} 1}& \textbf{\emph{Bit} 2} & \textbf{Opérateur « et » } & \textbf{Opérateur « ou inclusif »} & \textbf{Opérateur « ou exclusif »}\tabularnewline\hline
0&0&0&0&0
\tabularnewline\hline
1&0&0&1&1
\tabularnewline\hline
0&1&0&1&1
\tabularnewline\hline
1&1&1&1&0
\tabularnewline\hline
\end{tabular}
\end{table}

L'exemple ci-dessous utilise ces trois opérateurs. Comme vous le voyez,
les \emph{bits} des deux opérandes sont pris un à un pour former des
paires et chacune d'elle se voit appliquer la table de vérité
correspondante afin de produire un nouveau nombre entier.

\begin{C}
#include <stdio.h>


int main(void)
{
    int a = 0x63; /* 0x63 == 99 == 0110 0011 */
    int b = 0x2A; /* 0x2A == 42 == 0010 1010 */

    /* 0110 0011 & 0010 1010 == 0010 0010 == 0x22 == 34 */
    printf("%2X\n", a & b);
    /* 0110 0011 | 0010 1010 == 0110 1011 == 0x6B == 107 */
    printf("%2X\n", a | b);
    /* 0110 0011 ^ 0010 1010 == 0100 1001 == 0x49 == 73 */
    printf("%2X\n", a ^ b);
    return 0;
}
\end{C}

\begin{C}
22
6B
49
\end{C}

\subsection{L'opérateur de négation ou de complément}
\label{loperateur-de-negation-ou-de-complement}

L'opérateur de négation a un fonctionnement assez simple : il inverse
les \emph{bits} de son opérande (les uns deviennent des zéros et les
zéros des uns).

\begin{C}
#include <stdio.h>


int
main(void)
{
    unsigned a = 0x7F; /* 0111 1111 */

    /* ~0111 1111 == 1000 0000 */
    printf("%2X\n", ~a);
    return 0;
}
\end{C}

\begin{C}
FFFFFF80
\end{C}

\begin{attentionbox}
  Notez que \emph{tous} les \emph{bits}
ont été inversés, d'où le nombre élevé que nous obtenons puisque les
\emph{bits} de poids forts ont été mis à un. Ceci nous permet par
ailleurs de constater que, sur notre machine, il y a visiblement quatre
octets qui sont utilisés pour représenter la valeur d'un objet de type
\mybox{unsigned\ int}.
\end{attentionbox}


\begin{erreurbox}
  N'oubliez pas que les représentations
entières \emph{signées} ont chacune une représentation qui est
susceptible d'être invalide. Les opérateurs de manipulation des
\emph{bits} vous permettant de modifier directement la représentation,
vous devez éviter d'obtenir ces dernières.
\end{erreurbox}


Ainsi, les exemples suivants sont susceptibles de produire des valeurs
incorrectes (à supposer que la taille du type \mybox{int} soit de
quatre octets sans \emph{bits} de bourrages).

\begin{C}
/* Invalide en cas de représentation en complément à deux ou signe et magnitude */
int a = ~0x7FFFFFFF;
/* Idem */
int b = 0x00000000 | 0x80000000;
/* Invalide en cas de représentation en complément à un */
int c = ~0;
/* Idem */
int d = 0x11110000 ^ 0x00001111;
\end{C}

\begin{infobox}
  Notez toutefois que les entiers
\emph{non signés}, eux, ne subissent pas ces restrictions.
\end{infobox}


\subsection{Les opérateurs de décalage}
\label{les-operateurs-de-decalage}

Les opérateurs de décalage, comme leur nom l'indique, décalent la valeur
des \emph{bits} d'un objet d'une certaine quantité, soit vers la gauche
(c'est-à-dire vers le \emph{bit} de poids fort), soit vers la droite
(autrement dit, vers le \emph{bit} de poids faible). Ils attendent deux
opérandes : le nombre dont les \emph{bits} doivent être décalés et la
grandeur du décalage.

\begin{erreurbox}
  Un décalage ne peut être négatif ni être
supérieur \emph{ou égal} au nombre de \emph{bits} composant l'objet
décalé. Ainsi, si le type \mybox{int} utilise 32 \emph{bits} (sans
\emph{bits} de bourrage), le décalage ne peut être plus grand que 31.
\end{erreurbox}


\subsubsection{L'opérateur de décalage à gauche}
\label{loperateur-de-decalage-a-gauche}

L'opérateur de décalage à gauche translate la valeur des \emph{bits}
vers le \emph{bit} de poids forts. Les \emph{bits} de poids faibles
perdant leur valeur durant l'opération sont mis à zéro. Techniquement,
l'opération de décalage à gauche revient à calculer la valeur de
l'expression a × 2\textsuperscript{y}.

\begin{C}
#include <stdio.h>


int
main(void)
{
    /* 0000 0001 << 2 == 0000 0100 */
    int a = 1 << 2;
    /* 0010 1010 << 2 == 1010 1000 */
    int b = 42 << 2;

    printf("a = %d, b = %d\n", a, b);
    return 0;
}
\end{C}

\begin{C}
a = 4, b = 168
\end{C}

\begin{erreurbox}
  La première opérande ne peut être un
nombre négatif.
\end{erreurbox}


\begin{attentionbox}
  L'opération de décalage à gauche
revenant à effectuer une multiplication, celle-ci est soumise au risque
de dépassement de capacité que nous verrons au chapitre suivant.
\end{attentionbox}


\subsubsection{L'opérateur de décalage à droite}
\label{loperateur-de-decalage-a-droite}

L'opérateur de décalage à droite translate la valeur des \emph{bits}
vers le \emph{bit} de poids faible. Dans le cas où la première opérande
est un entier \emph{non signé} ou un entier signé \emph{positif}, les
\emph{bits} de poids forts perdant leur valeur durant l'opération sont
mis à zéro. Si en revanche il s'agit d'un nombre signé \emph{négatif},
les \emph{bits} perdant leur valeur se voient mis à zéro ou un suivant
la machine cible. Techniquement, l'opération de décalage à droite
revient à calculer la valeur de l'expression a / 2\textsuperscript{y}.

\begin{C}
#include <stdio.h>


int
main(void)
{
    /* 0001 0000 >> 2 == 0000 0100 */
    int a = 16 >> 2;
    /* 0010 1010 >> 2 == 0000 1010 */
    int b = 42 >> 2;

    printf("a = %d, b = %d\n", a, b);
    return 0;
}
\end{C}

Lexmark_CS317dn
a = 4, b = 10
\end{C}

\begin{infobox} Dans le cas où une valeur est
translatée au-delà du \emph{bit} de poids faible, elle est tout
simplement perdue.

\subsection{Opérateurs combinés}
\label{operateurs-combines}

Enfin, sachez que, comme pour les opérations arithmétiques usuelles, les
opérateurs de manipulation des \emph{bits} disposent d'opérateurs
combinés réalisant une affectation et une opération.

\begin{table}
\centering
\rowcolors{1}{}{gris-clair-tab}
\begin{tabular}{|l|l|}\hline
\rowcolor{gris-tab-entete}
\textbf{Opérateur combiné} & \textbf{Équivalent à}\tabularnewline\hline
variable \&= nombre & variable = variable \& nombre\tabularnewline\hline
variable \textbar{}= nombre & variable = variable \textbar{} nombre\tabularnewline\hline
variable \^{}=nombre & variable = variable \^{}nombre\tabularnewline\hline
variable\textless{}\textless{}=nombre & variable=variable\textless{}\textless{}nombre\tabularnewline\hline
variable\textgreater{}\textgreater{}=nombre & variable=variable\textgreater{}\textgreater{}nombre\tabularnewline\hline
\end{tabular}
\end{table}

\section{Masques et champs de bits}
\label{masques-et-champs-de-bits}

\subsection{Les masques}
\label{les-masques}

Une des utilisations fréquentes des opérateurs de manipulations des
\emph{bits} est l'emploi de \textbf{masques}. Un masque est un ensemble
de \emph{bits} appliqué à un second ensemble \emph{de même taille} lors
d'une opération de manipulation des \emph{bits} (plus précisément,
uniquement les opérations \mybox{\&}, \mybox{\textbar{}} et
\mybox{\^{}}) en vue soit de sélectionner un sous-ensemble, soit de
modifier un sous-ensemble.

\subsubsection{Modifier la valeur d'un bit}
\label{modifier-la-valeur-dun-bit}

\textbf{Mise à zéro}
\label{mise-a-zero}

Cette définition doit probablement vous paraître quelque peu abstraite,
aussi, prenons un exemple.

\begin{C}
unsigned short n;
\end{C}

Nous disposons d'une variable \mybox{n} de type
\mybox{unsigned\ short} (que nous supposerons composées de deux octets
pour nos exemples) et souhaiterions mettre le \emph{bit} de poids fort à
zéro.

Une solution consiste à appliquer les opérateurs de manipulation des
\emph{bits} afin d'obtenir la valeur voulue. Étant donné que nous
désirons mettre un \emph{bit} à zéro, nous pouvons déjà abandonné
l'opérateur \mybox{\textbar{}} au vu de sa table de vérité. Également,
l'opérateur \mybox{\^{}} ne convient pas tout à fait puisqu'il
inverserait la valeur du \emph{bit} au lieu de la mettre à zéro. Il nous
reste donc l'opérateur \mybox{\&}.

Avec cet opérateur, il nous est possible d'utiliser une valeur qui nous
donnera le bon résultat. Cette valeur, de même taille que celle de la
variable \mybox{n}, est précisément un masque qui va « cacher », «
masquer » une partie de la valeur.

\begin{C}
#include <stdio.h>
#include <stdlib.h>


int main(void)
{
    unsigned short n;

    if (scanf("%hx", &n) != 1)
    {
        perror("scanf");
        return EXIT_FAILURE;
    }

    printf("%X\n", n & 0x7FFF);
    return 0;
}
\end{C}

\begin{C}
8FFF
FFF

7F
7F
\end{C}

Comme vous le voyez, l'opérateur \mybox{\&} peut être utilisé pour
sélectionner une partie de la valeur de \mybox{n} en mettant à un les
\emph{bits} que nous souhaitons garder (en l'occurrence tous sauf le
\emph{bit} de poids fort) et les autres à zéro.

\textbf{Mise à un}
\label{mise-a-un}

À l'inverse, les opérateurs de manipulation des \emph{bits} peuvent être
employés pour mettre un ou plusieurs \emph{bits} à un. Dans ce cas,
c'est l'opérateur \mybox{\&} qui ne convient pas au vu de sa table de
vérité.

Si nous reprenons notre exemple précédent et que nous souhaitons
modifier la valeur de la variable \mybox{n} de sorte de mettre à un le
\emph{bit} de signe, nous pouvons procéder comme suit.

\begin{C}
#include <stdio.h>
#include <stdlib.h>


int main(void)
{
    unsigned short n;

    if (scanf("%hx", &n) != 1)
    {
        perror("scanf");
        return EXIT_FAILURE;
    }

    printf("%X\n", n | 0x8000);
    return 0;
}
\end{C}

\begin{C}
FFF
8FFF

7F
807F
\end{C}

Comme vous le voyez, l'opérateur \mybox{\textbar{}} peut être utilisé
de la même manière dans ce cas ci à l'aide d'un masque dont les
\emph{bits} qui doivent être mis à un sont\ldots{} à un.

\subsection{Les champs de bits}
\label{les-champs-de-bits}

\subsubsection{Mise en situation}
\label{mise-en-situation}

Une autre utilisation des opérateurs de manipulation des \emph{bits} est
le compactage de données entières.

Imaginons que nous souhaitions stocker la date courante sous la forme de
trois entiers : l'année, le mois et le jour. La première solution qui
vous viendra à l'esprit sera probablement de recourir à une structure,
par exemple comme celle ci-dessous, ce qui est un bon réflexe.

\begin{C}
struct date {
    unsigned char jour;
    unsigned char mois;
    unsigned short annee;
};
\end{C}

Toutefois, nous gaspillons finalement de la mémoire avec ce système. En
effet, techniquement, nous aurions besoin de 12 \emph{bits} pour stocker
l'année (afin d'avoir un peu de marge jusque l'an 4095 :p ), 4 pour le
mois et 5 pour le jour, ce qui nous fait un total de 21 \emph{bits}
contre 32 pour notre structure (à supposer que le type \mybox{short}
fasse deux octets et le type \mybox{char} un octet), sans compter les
multiplets de bourrage (revoyez le chapitre sur les structures si cela
ne vous dit rien).

Ceci n'est pas gênant dans la plupart des cas, mais cela peut le devenir
si la mémoire disponible vient à manquer ou si cette structure est
amenée à être créée un grand nombre de fois.

Avec les opérateurs de manipulations des \emph{bits}, il nous est
possible de stocker ces trois champs dans un tableau de trois
\mybox{unsigned\ char} afin d'économiser de la place.

\begin{C}
#include <stdio.h>
#include <stdlib.h>


static void modifie_jour(unsigned char *date, unsigned jour)
{
    /* Nous stockons le jour (cinq bits). */
    date[0] |= jour;
}


static void modifie_mois(unsigned char *date, unsigned mois)
{
    /* Nous ajoutons les trois premiers bits du mois après ceux du jour. */
    date[0] |= (mois & 0x07) << 5;
    /* Puis le bit restant dans le second octet. */
    date[1] |= (mois >> 3);
}


static void modifie_annee(unsigned char *date, unsigned annee)
{
    /* Nous ajoutons sept bits de l'année après le dernier bit du mois. */
    date[1] |= (annee & 0x7F) << 1;
    /* Et ensuite les cinq restants. */
    date[2] |= (annee) >> 7;
}


static unsigned calcule_jour(unsigned char *date)
{
    return date[0] & 0x1F;
}


static unsigned calcule_mois(unsigned char *date)
{
    return (date[0] >> 5) | ((date[1] & 0x1) << 3);
}


static unsigned calcule_annee(unsigned char *date)
{
    return (date[1] >> 1) | (date[2] << 7);
}


int
main(void)
{ 
    unsigned char date[3] = { 0 }; /* Initialisation à zéro. */
    unsigned jour, mois, annee;

    printf("Entrez une date au format jj/mm/aaaa : ");

    if (scanf("%u/%u/%u", &jour, &mois, &annee) != 3) {
        perror("fscanf");
        return EXIT_FAILURE;
    }

    modifie_jour(date, jour);
    modifie_mois(date, mois);
    modifie_annee(date, annee);
    printf("Le %u/%u/%u\n", calcule_jour(date), calcule_mois(date), calcule_annee(date));
    return 0;
}
\end{C}

\begin{C}
Entrez une date au format jj/mm/aaaa : 31/12/2042
Le 31/12/2042
\end{C}

Cet exemple amène quelques explications. Une fois les trois valeurs 
récupérées, il nous les compacter dans le tableau 
d'\mybox{unsigned\ char} :

\begin{enumerate}
\def
\labelenumi{\arabic{enumi}.}
\item
  Pour le jour, c'est assez simple, nous incorporons ses cinq
  \emph{bits} à l'aide de l'opérateur \mybox{\textbar{}} (les trois
  éléments du tableau étant à zéro au début, cela ne pose pas de
  problème).
\item
  Pour le mois, seuls trois \emph{bits} étant encore disponibles, il va
  nous falloir répartir ceux-ci sur deux éléments du tableau. Tout
  d'abord, nous sélectionnons les trois premiers \emph{bits} à l'aide du
  masque \mybox{0x07}, nous les décalons ensuite de cinq \emph{bits}
  vers la gauche (afin de ne pas écraser les cinq \emph{bits} du jour)
  et, enfin, nous les ajoutons à l'aide de l'opérateur
  \mybox{\textbar{}}. Le dernier \emph{bit} est lui stocké dans le
  second élément et est sélectionné à l'aide d'un décalage vers la
  droite afin d'éliminer les trois premiers \emph{bits} (qui ont déjà
  été traité).
\item
  Pour l'année, nous utilisons la même technique que pour le mois : nous
  sélectionnons les sept premiers \emph{bits} à l'aide du masque
  \mybox{0x7F}, les décalons d'un \emph{bit} vers la gauche en vue de
  ne pas écraser le \emph{bit} du mois et les intégrons avec l'opérateur
  \mybox{\textbar{}}. Les cinq \emph{bits} restants sont ensuite
  insérer en recourant préalablement à un décalage de sept \emph{bit}
  vers la droite.
\end{enumerate}

\subsubsection{Présentation}
\label{presentation}

Comme vous le voyez, si nous gagnons effectivement de la place en
mémoire, nous y perdons en temps de calcul et, plus important, notre
code est nettement plus complexe. C'est la raison pour laquelle cette
méthode n'est employée que dans le cas de contraintes particulières.

Bien entendu, nous pourrions recourir à des fonctions ou à des
macrofonctions pour simplifier la lecture du code, toutefois, nous ne
ferions que reporter la difficulté de compréhension sur ces dernières.
Heureusement, en vue de simplifier ce type d'écritures, le langage C met
à notre disposition les \textbf{champs de \emph{bits}}.

Un champ de \emph{bits} est une structure \emph{composée exclusivement}
de champs de type \mybox{int} ou \mybox{unsigned\ int} dont la taille
en \emph{bits} de chacun est précisée. Cette taille \emph{ne peut être}
supérieure à la taille en \emph{bits} du type \mybox{int}. L'exemple
ci-dessous défini une structure composée de trois champs de \emph{bits},
\mybox{a}, \mybox{b} et \mybox{c} de respectivement un, deux et trois
\emph{bits}.

\begin{C}
struct champ_de_bits
{
    unsigned a : 1;
    unsigned b : 2;
    unsigned c : 3;
};
\end{C}

Ainsi, notre exemple précédent peut être réécrit comme ceci.

\begin{C}
#include <stdio.h>
#include <stdlib.h>


struct date {
    unsigned jour : 5;
    unsigned mois : 4;
    unsigned annee : 12;
};


int
main(void)
{ 
    struct date date;
    unsigned jour, mois, annee;

    printf("Entrez une date au format jj/mm/aaaa : ");

    if (scanf("%u/%u/%u", &jour, &mois, &annee) != 3) {
        perror("fscanf");
        return EXIT_FAILURE;
    }

    date.jour = jour;
    date.mois = mois;
    date.annee = annee;

    printf("Le %u/%u/%u\n", date.jour, date.mois, date.annee);
    return 0;
}
\end{C}

\begin{erreurbox}
  Les champs de \emph{bits} ne disposent pas
d'une adresse et ne peuvent en conséquence se voir appliquer l'opérateur
d'adressage. Par ailleurs, nous vous conseillons de ne les employer que
pour stocker des nombres non signés, le support des nombres signés
n'étant pas garanti par la norme.
\end{erreurbox}


Si vous avez poussé la curiosité jusqu'à vérifier la taille de cette
structure, il y a de forte chance pour que celle-ci équivaille à celle
du type \mybox{int}. En fait, il s'agit de la méthode la plus courante
pour conserver les champs de \emph{bits} : ils sont stockés dans des
suites de \mybox{int}. Dès lors, si vous souhaitez économiser de la
place, faites en sorte que les données à stocker coïncident le plus
possible avec la taille d'un ou plusieurs objets de type \mybox{int}.

\begin{attentionbox}
  Gardez à l'esprit que le compactage de
données et les champs de \emph{bits} répondent à des besoins particulier
et complexifient votre code. Dès lors, ne les utilisez que lorsque cela
semble réellement se justifier.
\end{attentionbox}

\section{Les drapeaux}
\label{les-drapeaux}

Une autre utilisation régulière des opérateurs
de manipulation des \emph{bits} consiste en la gestion des
\textbf{drapeaux}. Un drapeau correspond en fait à un \emph{bit} qui est
soit « levé », soit « baissé » dans l'objectif d'indiquer si une
situation est vraie ou fausse.

Supposons que nous souhaitions fournir à une fonction un nombre et
plusieurs de ses propriétés, par exemple s'il est pair, s'il s'agit
d'une puissance de deux et s'il est premier. Nous pourrions bien entendu
lui fournir quatre paramètres, mais cela fait finalement beaucoup pour
simplement fournir un nombre et, foncièrement, trois \emph{bits}.

À la place, il nous est possible d'employer un entier dont trois
\emph{bits} seront utilisés pour représenter chaque condition. Par
exemple, le premier indiquera si le nombre est pair, le second s'il
s'agit d'une puissance de deux et le troisième s'il est premier.

\begin{C}
void traitement(int nombre, unsigned char drapeaux)
{
    if (drapeaux & 0x01) /* Si le nombre est pair */
    {
        /* ... */
    }
    if (drapeaux & 0x02) /* Si le nombre est une puissance de deux */
    {
        /*... */
    }
    if (drapeaux & 0x04) /* Si le nombre est premier */
    {
        /*... */
    }
}


int main(void)
{
    int nombre;
    unsigned char drapeaux;

    nombre = 2;
    drapeaux = 0x01 | 0x02; /* 0000 0011 */
    traitement(nombre, drapeaux);
    nombre = 17;
    drapeaux = 0x04; /* 0000 0100 */
    traitement(nombre, drapeaux);
    return 0;
}
\end{C}

Comme vous le voyez, nous utilisons l'opérateur \mybox{\textbar{}} pour
combiner plusieurs drapeaux et l'opérateur \mybox{\&} pour déterminer
si un drapeau est levé ou non.

\begin{infobox}
  Notez que, chaque drapeau
représentant un \emph{bit}, ceux-ci correspondent toujours à des
puissances de deux.
\end{infobox}


Voilà qui est plus efficace, mais en somme assez peu lisible\ldots{} En
effet, il serait bon de préciser dans notre code à quoi correspond
chaque drapeaux. Pour ce faire, nous pouvons recourir au préprocesseur
afin de clarifier un peu tout cela.

\begin{C}
#define PAIR        (1 << 0)
#define PUISSANCE   (1 << 1)
#define PREMIER     (1 << 2)


void traitement(int nombre, unsigned char drapeaux)
{
    if (drapeaux & PAIR) /* Si le nombre est pair */
    {
        /* ... */
    }
    if (drapeaux & PUISSANCE) /* Si le nombre est une puissance de deux */
    {
        /*... */
    }
    if (drapeaux & PREMIER) /* Si le nombre est premier */
    {
        /*... */
    }
}


int main(void)
{
    int nombre;
    unsigned char drapeaux;

    nombre = 2;
    drapeaux = PAIR | PUISSANCE; /* 0000 0011 */
    traitement(nombre, drapeaux);
    nombre = 17;
    drapeaux = PREMIER; /* 0000 0100 */
    traitement(nombre, drapeaux);
    return 0;
}
\end{C}

Voici qui est mieux.

Pour terminer, remarquez qu'il s'agit d'un bon cas d'utilisation des
champs de \emph{bits}, chacun d'entre eux représentant alors un drapeau.

\begin{C}
struct propriete
{
    unsigned pair : 1;
    unsigned puissance : 1;
    unsigned premier : 1;
}


void traitement(int nombre, struct propriete prop)
{
    if (prop.pair) /* Si le nombre est pair */
    {
        /* ... */
    }
    if (prop.puissance) /* Si le nombre est une puissance de deux */
    {
        /*... */
    }
    if (prop.premier) /* Si le nombre est premier */
    {
        /*... */
    }
}


int main(void)
{
    int nombre;
    struct propriete prop = { 0 }; /* Initialisation à zéro. */

    nombre = 2;
    prop.pair = 1;
    prop.puissance = 1;
    traitement(nombre, prop);
    memset(&prop, 0, sizeof prop); /* Mise à zéro. */
    nombre = 17;
    drapeaux = PREMIER; /* 0000 0100 */
    traitement(nombre, drapeaux);
    return 0;
}
\end{C}

\section{Exercices}
\label{exercices-3}

\subsection{Obtenir la valeur maximale d'un type non signé}
\label{obtenir-la-valeur-maximale-dun-type-non-signe}

Maintenant que nous connaissons la représentation des nombres non signés
ainsi que les opérateurs de manipulation des \emph{bits}, vos devriez
pouvoir trouver comment obtenir la plus grande valeur représentable par
le type \mybox{unsigned\ int}.

\subsubsection{Indice}
\label{indice-1}

\begin{secretbox}
Rappelez-vous : dans la représentation des entiers non signés, chaque 
\emph{bit} représente une puissance de deux.
\end{secretbox}


\subsubsection{Solution}
\label{solution-1}

\begin{C}
#include <stdio.h>


int main(void)
{
    printf("%u\n", ~0U);
    return 0;
}Cette technique n'est valable que pour les entiers
\emph{non signés}, la représentation où tous les \emph{bits} sont à un
étant potentiellement invalide dans le cas des entiers signés
(représentation en complément à un).
\end{C}

\begin{attentionbox} 
 Notez bien que nous avons utilisé le suffixe \mybox{U} afin que le type de la
constante \mybox{0} soit \mybox{unsigned\ int} et non \mybox{int}
(n'hésitez pas à revoir le chapitre relatif aux opérations mathématiques
si cela ne vous dit rien).
\end{attentionbox}

\begin{erreurbox}
 Cette technique n'est valable que pour les entiers
\emph{non signés}, la représentation où tous les \emph{bits} sont à un
étant potentiellement invalide dans le cas des entiers signés
(représentation en complément à un).
\end{erreurbox}

\subsection{Afficher la représentation en base deux d'un entier}
\label{afficher-la-representation-en-base-deux-dun-entier}

Vous le savez, il n'existe pas de format de la fonction
\mybox{printf()} qui permet d'afficher la représentation binaire d'un
nombre. Pourtant, cela pourait s'avérer bien pratique dans certains cas,
même si la représentation hexadécimale est disponible.

Dans ce second exercice, votre tâche sera de réaliser une fonction
capable d'afficher la représentation binaire d'un \mybox{unsigned\ int}
\emph{en gros-boutiste}.

\subsubsection{Indice}
\label{indice-2}

\begin{secretbox} 
 Pour afficher la représentation gros-boutiste, il va vous falloir commencer 
 par afficher le \emph{bit} de poids de fort suivit des autres. Pour ce 
 faire, vous allez avoir besoin d'un masque dont seul ce \emph{bit} sera à un. 
 Pour ce faire, vous pouvez vous aider de l'exercice précédent.
\end{secretbox}

\subsubsection{Solution}
\label{solution-2}

\begin{C}
#include <stdio.h>


void affiche_bin(unsigned n)
{
    unsigned mask = ~(~0U >> 1);
    unsigned i = 0;

    while (mask > 0)
    {
      if (i != 0 && i % 4 == 0)
          putchar(' ');

      putchar((n & mask) ? '1' : '0');
      mask >>= 1;
      ++i;
    }

    putchar('\n');
}


int main(void)
{
    affiche_bin(1);
    affiche_bin(42);
    return 0;
}

\end{C}

\begin{C}
0000 0000 0000 0000 0000 0000 0000 0001
0000 0000 0000 0000 0000 0000 0010 1010
\end{C}

L'expression \mybox{\textasciitilde{}(\textasciitilde{}0U\ \textgreater{}\textgreater{}\ 1)}
nous permet d'obtenir un masque ou seul le \emph{bit} de poids fort est
à un. Nous pouvons ensuite l'employer successivement en décalant le
\emph{bit} à un de sorte d'obtenir la représentation binaire d'un entier
\emph{non signé}.

\begin{erreurbox}
 À nouveau, faites bien attention que ceci n'est valable que
pour les entiers \emph{non signés}, une représentation dont tous les
\emph{bits} sont à un ou dont seul le \emph{bit} de poids fort est à un
étant possiblement incorrecte dans le cas des entiers signés.
\end{erreurbox}


\subsection{Déterminer si un nombre est une puissances de deux}
\label{determiner-si-un-nombre-est-une-puissances-de-deux}

Vous le savez : les puissances de deux ont la particularité de n'avoir
qu'un seul bit à un, tous les autres étant à zéro. Toutefois, elles ont
une autre propriété : si l'on soustrait un à une puissance de deux
\(n\), tous les \emph{bits} précédent celui de la puissance seront mis à
un (par exemple \mybox{0000\ 1000\ -\ 1\ ==\ 0000\ 0111}). En
particulier, on remarque que \(n\) et \(n-1\) n'ont aucun bit à 1 en
commun. Réciproquement, si \(n\) n'est pas une puissance de 2, alors le
bit à 1 le plus fort est aussi à 1 dans \(n-1\). par exemple
\mybox{0000\ 1010\ -\ 1\ ==\ 0000\ 1001}).

Sachant cela, il nous est possible de créer une fonction très simple
déterminant si un nombre est une puissance de 2 ou non.

\begin{C}
 #include <stdio.h>


int puissance_de_deux(unsigned int n)
{
    return n != 0 && (n & (n - 1)) == 0;
}


int main(void)
{
    if (puissance_de_deux(256))
        printf("256 est une puissance de deux\n");
    else
        printf("256 n'est pas une puissance de deux\n");

    if (puissance_de_deux(48))
        printf("48 est une puissance de deux\n");
    else
        printf("48 n'est pas une puissance de deux\n");

    return 0;
}
\end{C}

\begin{C}
256 est une puissance de deux
48 n'est pas une puissance de deux
\end{C}

\hrulefill

En résumé

\begin{enumerate}
\def
\labelenumi{\arabic{enumi}.}
\item
  Le C fourni six opérateurs de manipulaton des \emph{bits} ;
\item
  Ces derniers travaillant directement sur la représentation des
  entiers, il est \emph{impératif} d'éviter d'obtenir des
  représentations potentiellement invalide dans le cas des entiers
  signés ;
\item
  L'utilisation de masques permet de sélectionner ou modifier une
  portion d'un nombre entier ;
\item
  Les champs de \emph{bits} permettent de stocker des entiers de taille
  arbitraire, mais doivent \emph{toujours} avoir une taille inférieure à
  celle du type \mybox{int}. Par ailleurs, ils n'ont pas d'adresse et
  ne supportent pas forcément le stockage d'entiers signés ;
\item
  Les drapeaux constituent une solution élégante pour stocker des états
  binaires (« vrai » ou « faux »).
\end{enumerate}