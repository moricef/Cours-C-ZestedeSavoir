\chapter{Les définitions de type}
\label{les-definitions-de-type}

Ce chapitre sera relativement court et pour cause, nous allons aborder 
  un petit point du langage C, mais qui a toute son importance : les 
  \textbf{définitions de type}.
  
\section{Définition et utilisation }
\label{definition-et-utilisation}

  Une définition de type permet, comme son nom l'indique, de définir un
  type, c'est-à-dire d'en produire un nouveau ou, plus précisément, de
  créer un \emph{alias} (un synonyme si vous préférez) d'un type
  existant. Une définition de type est identique à une déclaration de
  variable, si ce n'est que celle-ci doit être précédée du mot-clé
  \mybox{typedef} (pour \emph{type definition}) et que l'identificateur
  ainsi choisi désignera un type et non une variable.


Ainsi, le code ci-dessous défini un nouveau type \mybox{entier} qui
sera un \emph{alias} pour le type \mybox{int}.

\begin{C}
typedef int entier;
\end{C}

Le synonyme ainsi créé peut être utilisé au même endroit que n'importe
quel autre type.

\begin{C}
#include <stdio.h>

typedef int entier;


entier main(void)
{
    entier a = 10;

    printf("%d.\n", a);
    return 0;
}
\end{C}

\begin{questionbox}
  D'accord, mais cela me sert à quoi de
créer un synonyme pour un type existant ?
\end{questionbox}

Les définitions de type permettent en premier lieu de raccourcir
certaines écritures, notamment afin de s'affranchir des mots-clés
\mybox{struct}, \mybox{union} et \mybox{enum} (bien que, ceci
constitue une pertes d'information aux yeux de certains).

\begin{C}
#include <stdio.h>

struct position
{
    int lgn;
    int col;
};

typedef struct position position;


int main(void)
{
    position pos;

    pos.lgn = 1;
    pos.col = 1;
    printf("%d, %d.\n", pos.lgn, pos.col);
    return 0;
}
\end{C}

Notez qu'une définition de type étant une délaration, il est
parfaitement possible de combiner la définition de la structure et la
définition de type (comme pour une variable, finalement).

\begin{C}
typedef struct position
{
    int lgn;
    int col;
} position;
\end{C}

Également, dans le même sens que ce qui a été dit au sujet des
énumérations, une définition de type peut être employée pour donner plus
d'informations via le typage. C'est une manière de désigner plus
finement le contenu d'un type en ne se contentant pas d'une information
plus générale comme « un entier » ou « un flottant ».

Par exemple, nous aurions pu définir un type \mybox{ligne} et un type
\mybox{colonne} afin de donner plus d'information sur le contenu de nos
variables.

\begin{C}
typedef short ligne;
typedef short colonne;

struct position
{
    ligne lgn;
    colonne col;
};
\end{C}

De même, les définitions de type sont couramment utilisée afin de créer
des asbstractions. Nous en avons vu un exemple avec l'en-tête
\mybox{\textless{}time.h\textgreater{}} qui défini le type
\mybox{time\_t}. Celui-ci permet de ne pas devoir modifier la fonction
\mybox{time()} et son utilisation suivant le type qui est sous-jacent.
Peu importe que \mybox{time\_t} soit un entier ou un flottant, la
fonction \mybox{time()} s'utilise de la même manière.

Enfin, les définitions de type permettent de résoudre quelques points de
syntaxe tordus.

\hrulefill

\subsubsection{En résumé}
\label{en-resume-5}

\begin{enumerate}
\def
\labelenumi{\arabic{enumi}.}
\item
  Une définition de type permet de créer un synonyme d'un type existant
  (en ce compris un autre \emph{alias}) ;
  \end{enumerate}
  
  