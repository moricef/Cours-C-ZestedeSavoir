\chapter{Les limites des types}
\label{les-limites-des-types-1}

Nous avons abordés les limites des types à plusieurs reprises dans les
chapitres précédents, mais nous ne nous sommes pas encore véritablement
arrêté sur ce point. Il est a présent temps pour nous de vous présenter
les implications de ces limites.

\section{Les limites des types}
\label{les-limites-des-types-2}

Au début de ce cours, nous vous avons précisés que tous les types 
avaient des bornes et qu'en conséquence, ils ne pouvaient stocker 
qu'un intervalle définit de valeurs. Nous vous avons également 
indiqué que la norme du langage imposait des minimas pour ces 
intervalles qui, pour rappel, sont les suivants.

\begin{table}
\centering
\rowcolors{1}{}{gris-clair-tab}
\begin{tabular}{|l|l|l|}\hline
\rowcolor{gris-tab-entete}\bf 
Type & Minimum & Maximum\tabularnewline\hline
\textbf{signed char} & -127 & 127\tabularnewline\hline
\textbf{unsigned char} & 0 & 255\tabularnewline\hline
\textbf{short} & -32 767 & 32 767\tabularnewline\hline
\textbf{unsigned short} & 0 & 65 535\tabularnewline\hline
\textbf{int} & -32 767 & 32 767\tabularnewline\hline
\textbf{unsigned int} & 0 & 65 535\tabularnewline\hline
\textbf{long} & -2 147 483 647 & 2 147 483 647\tabularnewline\hline
\textbf{unsigned long} & 0 & 4 294 967 295\tabularnewline\hline
\textbf{float} & -1 × 10\textsuperscript{37} & 1 ×
10\textsuperscript{37}\tabularnewline\hline
\textbf{double} & -1 × 10\textsuperscript{37} & 1 ×
10\textsuperscript{37}\tabularnewline\hline
\textbf{long double} & -1 × 10\textsuperscript{37} & 1 ×
10\textsuperscript{37}\tabularnewline\hline
\end{tabular}
\end{table}

Toutefois, mise à part pour le type \mybox{char}, les bornes des types
vont le plus souvent au-delà de ces minimums garantis. Cependant, pour
pouvoir manipuler nos types dans leur limites, encore faut-il les
connaître. Heureusement pour nous, la norme nous fourni deux en-têtes
définissant des macroconstantes correspondant aux limites des types
entiers et flottants : \mybox{\textless{}limits.h\textgreater{}} et
\mybox{\textless{}float.h\textgreater{}}.

\subsection{L'en-tête \textless{}limits.h\textgreater{}}
\label{len-tete-limits.h}

\begin{table}
\centering
\rowcolors{1}{}{gris-clair-tab}
\begin{tabular}{|l|l|l|}\hline
\rowcolor{gris-tab-entete}\bf 
Type & Minimum & Maximum\tabularnewline\hline
\textbf{signed char} & SCHAR\_MIN & SCHAR\_MAX\tabularnewline\hline
\textbf{unsigned char} & 0 & UCHAR\_MAX\tabularnewline\hline
\textbf{short} & SHRT\_MIN & SHRT\_MAX\tabularnewline\hline
\textbf{unsigned short} & 0 & USHRT\_MAX\tabularnewline\hline
\textbf{int} & INT\_MIN & INT\_MAX\tabularnewline\hline
\textbf{unsigned int} & 0 & UINT\_MAX\tabularnewline\hline
\textbf{long} & LONG\_MIN & LONG\_MAX\tabularnewline\hline
\textbf{unsigned long} & 0 & ULONG\_MAX\tabularnewline\hline
\end{tabular}
\end{table}

\begin{infobox}
Remarquez qu'il n'y a pas de macroconstantes pour le minimum des 
types entiers non signés puisque celui-ci est \emph{toujours} zéro.
\end{infobox}


À ces macroconstantes, il faut également ajouter \mybox{CHAR\_BIT} qui
vous indique le nombre \emph{bits} composant un multiplet. Celle-ci a
pour valeur minimale huit, c'est-à-dire qu'en C, un multiplet ne peut
avoir moins de huit \emph{bits}.

L'exemple suivant utilise ces macroconstantes et affiche leur valeur.

\begin{C}
#include <limits.h>
#include <stdio.h>


int
main(void)
{
    printf("Un multiplet se compose de %d bits.\n", CHAR_BIT);
    printf("signed char : min = %d ; max = %d.\n", SCHAR_MIN, SCHAR_MAX);
    printf("unsigned char : min = 0 ; max = %u.\n", UCHAR_MAX);
    printf("short : min = %d ; max = %d.\n", SHRT_MIN, SHRT_MAX);
    printf("unsigned short : min = 0 ; max = %u.\n", USHRT_MAX);
    printf("int : min = %d ; max = %d.\n", INT_MIN, INT_MAX);
    printf("unsigned int : min = 0 ; max = %u.\n", UINT_MAX);
    printf("long : min = %ld ; max = %ld.\n", LONG_MIN, LONG_MAX);
    printf("unsigned long : min = 0 ; max = %lu.\n", ULONG_MAX);
    return 0;
}
\end{C}

\begin{C}
Un multiplet se compose de 8 bits.
signed char : min = -128 ; max = 127.
unsigned char : min = 0 ; max = 255.
short : min = -32768 ; max = 32767.
unsigned short : min = 0 ; max = 65535.
int : min = -2147483648 ; max = 2147483647.
unsigned int : min = 0 ; max = 4294967295.
long : min = -9223372036854775808 ; max = 9223372036854775807.
unsigned long : min = 0 ; max = 18446744073709551615.
\end{C}

\begin{infobox} 
Bien entendu, les valeurs obtenues dépendent de votre machine, 
celles-ci sont simplement données à titre d'illustration.
\end{infobox}

\subsection{L'en-tête \textless{}float.h\textgreater{}}
\label{len-tete-float.h}

Les types flottants amènent une subtilité supplémentaire : ceux-ci
disposent en vérité de \emph{quatre} bornes : le plus grand nombre
représentable, le plus petit nombre représentable (qui est l'opposé du
précédent), la plus petite partie fractionnaire représentable (autrement
dit, le nombre le plus proche de zéro représentable) et son opposé. Dit
autrement, les bornes des types flottants peuvent être schématisées
comme suit.

\begin{C}
-MAX             -MIN   0   MIN              MAX
  +----------------+    +    +----------------+
\end{C}

Où \mybox{MAX} représente le plus grand nombre représentable et
\mybox{MIN} le plus petit nombre représentable avant zéro. Étant donné
que deux des bornes ne sont finalement que l'opposé des autres, deux
macroconstantes sont suffisantes pour chaque type flottant.

\begin{table}
\centering
\rowcolors{1}{}{gris-clair-tab}
\begin{tabular}{|l|l|l|}\hline
\rowcolor{gris-tab-entete}\bf 
Type & Maximum & Minimum de la partie fractionnaire\tabularnewline\hline
\textbf{float} & FLT\_MAX & FLT\_MIN\tabularnewline\hline
\textbf{double} & DBL\_MAX & DBL\_MIN\tabularnewline\hline
\textbf{long double} & LDBL\_MAX & LDBL\_MIN\tabularnewline\hline
\end{tabular}
\end{table}

\begin{C}
#include <float.h>
#include <stdio.h>


int
main(void)
{
    printf("float : min = %e ; max = %e.\n", FLT_MIN, FLT_MAX);
    printf("double : min = %e ; max = %e.\n", DBL_MIN, DBL_MAX);
    printf("long double : min = %Le ; max = %Le.\n", LDBL_MIN, LDBL_MAX);
    return 0;
}
\end{C}

\begin{C}
 float : min = 1.175494e-38 ; max = 3.402823e+38.
double : min = 2.225074e-308 ; max = 1.797693e+308.
long double : min = 3.362103e-4932 ; max = 1.189731e+4932.
\end{C}

\section{Les dépassements de capacité}
\label{les-depassements-de-capacite}

Nous savons à présent comment obtenir les limites des différents types
de base, c'est bien. Toutefois, nous ignorons toujours \emph{quand} la
limite d'un type est dépassée et ce qu'il se passe dans un tel cas.

Le franchissement de la limite d'un type est appelé un
\textbf{dépassement de capacitié} (\emph{overflow} ou \emph{underflow}
en anglais, suivant que la limite maximale ou minimale est outrepassée).
Un dépassement survient lorsqu'une opération produit un résultat dont la
valeur n'est pas représentable par le type de l'expression ou lorsqu'une
valeur est convertie vers un type qui ne peut la représenter.

Ainsi, dans l'exemple ci-dessous, l'expression \mybox{INT\_MAX\ +\ 1}
provoquent un dépassement de capacité, de même que la conversion
(implicite) de la valeur \mybox{INT\_MAX} vers le type
\mybox{signed\ char}.

\begin{C}

\end{C}

\begin{attentionbox}
  Gardez en mémoire que des conversions
implicites ont lieu lors des opérations. Revoyez le chapitre sur les
opérations mathématiques si cela vous paraît lointain.
\end{attentionbox}


\begin{questionbox}
  D'accord, mais que se passe-t-il dans un tel cas ?
\end{questionbox}


C'est ici que le bât blesse : ce n'est pas précisé. :-°\\
Ou, plus précisément, cela dépend des situations.

\subsection{Dépassement lors d'une opération}
\label{depassement-lors-dune-operation}

\subsubsection{Les types entiers}
\label{les-types-entiers}

\textbf{Les entiers signés}
\label{les-entiers-signes}

Si une opération travaillant avec des entiers signés dépasse la capacité
du type du résultat, le comportement est indéfini. Dans la pratique, il
y a trois possibilités :

\begin{enumerate}
\def
\labelenumi{\arabic{enumi}.}
\item
  La valeur boucle, c'est-à-dire qu'une fois une limite atteinte, la
  valeur revient à la seconde. Cette solution est la plus fréquente et
  s'explique par le résultat de l'arithmétique en complément à deux. En
  effet, à supposer qu'un \mybox{int} soit représenté sur 32
  \emph{bits} :

  \begin{itemize}
  \item
    \mybox{INT\_MAX\ +\ 1} se traduit par\\
    \mybox{0111\ 1111\ 1111\ 1111\ 1111\ 1111\ 1111\ 1111\ +\ 0000\ 0000\ 0000\ 0000\ 0000\ 0000\ 0000\ 0001},
    ce qui donne
    \mybox{1000\ 0000\ 0000\ 0000\ 0000\ 0000\ 0000\ 0000}, soit
    \mybox{INT\_MIN} ; et
  \item
    \mybox{INT\_MIN\ -\ 1} se traduit par\\
    \mybox{1000\ 0000\ 0000\ 0000\ 0000\ 0000\ 0000\ 0000\ +\ 1111\ 1111\ 1111\ 1111\ 1111\ 1111\ 1111\ 1111}
    ce qui donne
    \mybox{0111\ 1111\ 1111\ 1111\ 1111\ 1111\ 1111\ 1111}, soit
    \mybox{INT\_MAX}.
  \end{itemize}
\item
  La valeur sature, c'est-à-dire qu'une fois une limite atteinte, la
  valeur reste bloquée à celle-ci.
\item
  Le processeur considère l'opération comme invalide et stoppe
  l'exécution du programme.
\end{enumerate}

\textbf{Les entiers non signés}
\label{les-entiers-non-signes}

Si une opération travaillant avec des entiers non signés dépasse la
capacité du type du résultat, alors il n'y a a proprement parler
\emph{pas} de dépassement et la valeur boucle. Autremement dit,
l'expression \mybox{UINT\_MAX\ +\ 1} donne \emph{toujours} zéro.

\subsubsection{Les types flottants}
\label{les-types-flottants}

Si une opération travaillant avec des flottants dépasse la capacité du
type du résultat, alors, comme pour les entiers signés, le comportement
est indéfini. Toutefois, les flottants étant le plus souvent représenté
suivant la norme IEEE 754, le résultat sera souvent le suivant :

\begin{enumerate}
\def
\labelenumi{\arabic{enumi}.}
\item
  En cas de dépassement de la borne maximale ou minimale, le résultat
  sera égal à l'inifini (positif ou négatif).
\item
  En cas de dépassement de la limite de la partie fractionnaire, le
  résultat sera arrondi à zéro.
\end{enumerate}

Néanmoins, gardez à l'esprit que ceci \emph{n'est pas garantit} par la
norme. Il est donc parfaitement possible que votre programme soit tout
simplement arrêté.

\subsection{Dépassement lors d'une conversion}
\label{depassement-lors-dune-conversion}

\subsubsection{Valeur entière vers entier signé}
\label{valeur-entiere-vers-entier-signe}

Si la conversion d'une valeur entière vers un type signé produit un
dépassement, le résultat n'est pas déterminé.

\subsubsection{Valeur entière vers entier non signé}
\label{valeur-entiere-vers-entier-non-signe}

Dans le cas où la conversion d'une valeur entière vers un type non
signé, la valeur boucle, comme précisé dans le cas d'une opération
impliquant des entiers non signés.

\subsubsection{Valeur flottante vers entier}
\label{valeur-flottante-vers-entier}

Lors d'une conversion d'une valeur flottante vers un type entier (signé
ou non signé), la partie fractionnaire est ignorée. Si la valeur
restante n'est pas représentable, le résultat est indéfini (\emph{y
compris} pour les types non signés, donc).

\subsubsection{Valeur entière vers flottant}
\label{valeur-entiere-vers-flottant}

Si la conversion d'une valeur entière vers un type flottant implique un
dépassement, le comportement est indéfini.

\begin{infobox}
  Notez que si le résultat n'est pas
\emph{exactement} représentable (rappelez-vous des pertes de précision)
la valeur sera approchée, mais cela ne constitue pas un dépassement.
\end{infobox}


\subsubsection{Valeur flottante vers flottant}
\label{valeur-flottante-vers-flottant}

Enfin, si la conversion d'une valeur flottante vers un autre type
flottant produit un dépassement, le comportement est indéfini.

\begin{infobox}
 Comme pour le cas d'une conversion
d'un type entier vers un type flottant, si le résultat n'est pas
\emph{exactement} représentable, la valeur sera approchée.
\end{infobox}

\section{Gérer les dépassements entiers}
\label{gerer-les-dépassements-entiers }

Nous savons à présent ce qu'est un dépassement, quand ils se produisent 
et dans quel cas ils sont problématiques (en vérité, potentiellement 
tous, puisque même si la valeur boucle, ce n'est pas forcément ce que nous 
souhai\-tons). Reste à présent pour nous à les gérer, c'est-à-dire les détecter 
et les éviter. Malheureusement pour nous, la bibliothèque standard ne nous 
fourni aucun outil à cet effet, mise à part les macroconstantes qui vous ont été
présentées. Aussi il va nous être nécessaire de réaliser nos propres outils.

\subsection{Préparation}
\label{preparation}

Avant de nous lancer dans la construction de ces fonctions, il va nous
être nécessaire de déterminer leur forme. En effet, si, intuitivement,
il vous vient sans doute à l'esprit de créer des fonctions prenans deux
paramètres et fournissant un réslutat, reste le problème de la gestion
d'erreur.

\begin{C}
int safe_add(int a, int b);
\end{C}

En effet, comment précise-t-on à l'utilisateur qu'un dépassement a été
détecté, toutes les valeurs du type \mybox{int} pouvant être retournées
? \emph{Eh} bien, pour résoudre ce soucis, nous allons recourir dans les
exemples qui suivent à la variable \mybox{errno}, à laquelle nous
affecteront la valeur \mybox{ERANGE} en cas de dépassement. Ceci
implique qu'elle soit mise à zéro avant tout appel à ces fonctions
(revoyez le premier chapitre sur la gestion d'erreur si cela ne vous dit
rien).

Par ailleurs, afin que ces fonctions puissent éventuellement être
utilisées dans des suites de calculs, nous allons faire en sorte qu'elle
retourne la borne maximale ou minimale en cas de dépassement.

Ceci étant posé, passons à la réalisation de ces fonctions. :)

\subsection{L'addition}
\label{laddition-1}

Afin d'empêcher un dépassement, il va nous falloir le détecter et ceci,
bien entendu, à l'aide de condition qui ne doivent pas elle-même
produire de dépassement. Ainsi, ce genre de condition est à proscrire :\\
\mybox{if\ (a\ +\ b\ \textgreater{}\ INT\_MAX)}. Pour le cas d'une
addition deux cas de figures se présentent :

\begin{enumerate}
\def
\labelenumi{\arabic{enumi}.}
\item
  \mybox{b} est positif et il nous faut alors vérifier que la somme ne
  dépasse pas le maximum.
\item
  \mybox{b} est négatif et il nous dans ce cas déterminer si la somme
  dépasse ou non le minimum.
\end{enumerate}

Ceci peut être contrôler en vérifiant que \mybox{a} n'est pas supérieur
ou inférieur à, respectivement, \mybox{MAX\ -\ b} et
\mybox{MIN\ -\ b}. Ainsi, il nous est possible de rédiger une fonction
comme suit.

\begin{C}
int safe_add(int a, int b)
{
    int err = 1;

    if (b >= 0 && a > INT_MAX - b)
        goto overflow;
    else if (b < 0 && a < INT_MIN - b)
        goto underflow;

    return a + b;
underflow:
    err = -1;
overflow:
    errno = ERANGE;
    return err > 0 ? INT_MAX : INT_MIN;
}
\end{C}

Contrôlons à présent son fonctionnement.

\begin{C}
#include <errno.h>
#include <limits.h>
#include <stdio.h>
#include <string.h>

/* safe_add() */


static void test_add(int a, int b)
{
    errno = 0;
    printf("%d + %d = %d", a, b, safe_add(a, b));
    printf(" (%s).\n", strerror(errno));
}


int main(void)
{
    test_add(INT_MAX, 1);
    test_add(INT_MIN, -1);
    test_add(INT_MIN, 1);
    test_add(INT_MAX, -1);
    return 0;
}
\end{C}

\begin{C}
2147483647 + 1 = 2147483647 (Numerical result out of range).
-2147483648 + -1 = -2147483648 (Numerical result out of range).
-2147483648 + 1 = -2147483647 (Success).
2147483647 + -1 = 2147483646 (Success).
\end{C}

\subsection{La soustraction}
\label{la-soustraction-1}

Pour la soustraction, le principe est le même que pour l'addition si ce
n'est que les tests doivent être quelques peu modifiés. En guise
d'exercice, essayez de trouver la solution par vous-même. ;)

\begin{C}
int safe_sub(int a, int b)
{
    int err = 1;

    if (b >= 0 && a < INT_MIN + b)
        goto underflow;
    else if (b < 0 && a > INT_MAX + b)
        goto overflow;

    return a - b;
underflow:
    err = -1;
overflow:
    errno = ERANGE;
    return err > 0 ? INT_MAX : INT_MIN;
}
\end{C}

\begin{questionbox}
  Mais ?! Pourquoi ne pas simplement faire \mybox{safe\_add(a,\ -b)} ?
\end{questionbox}

Bonne question !

Cela a de quoi surprendre de prime à bord, mais l'opérateur de négation
\mybox{-} peut produire un dépassement. En effet, souvenez-vous : la
représentation en complément à deux ne dispose que d'une seule
représentation pour zéro, du coup, il reste une représentation qui est
possiblement invalide où seul le \emph{bit} de signe est à un.
Cependant, le plus souvent, cette représentation est utilisée pour
fournir un nombre négatif supplémentaire. Or, si c'est le cas, il y a
une asymétrie entre la borne inférieure et supérieure et la négation de
la limite inférieure produira un dépassement. Il nous est donc
nécessaire de prendre ce cas en considération.

\subsection{La négation}
\label{la-negation}

À ce sujet, pour réaliser cette opération, deux solutions s'offrent à
nous :

\begin{enumerate}
\def
\labelenumi{\arabic{enumi}.}
\item
  Vérifier si l'opposé de la borne supérieure est plus grand que la
  borne inférieur (autrement dit que
  \mybox{-INT\_MAX\ \textgreater{}\ INT\_MIN}) et, si c'est le cas,
  vérifier que le paramètre n'est pas la borne inférieure ; ou
\item
  Vérifier \mybox{sub(0,\ a)} où \mybox{a} est l'opérande qui doit
  subir la négation.
\end{enumerate}

Notre préférence allant à la seconde solution, notre fonction sera donc
la suivante.

\begin{C}
int safe_neg(int a)
{
    return safe_sub(0, a);
}
\end{C}

\subsection{La multiplication}
\label{la-multiplication-1}

Pour la multiplication, cela se corse un peu, notamment au vu de ce qui
a été dit précédemment au sujet de la représentation en complément à
deux. Procédons par ordre :

\begin{enumerate}
\def
\labelenumi{\arabic{enumi}.}
\item
  Tout d'abord, si l'un des deux opérandes est postif, nous devons
  vérifier que le minimum n'est pas atteint (puisque le résultat sera
  négatif) et, si les deux sont positifs, que le maximum n'est pas
  dépassé. Toutefois, vu que nous allons recourir à la division pour le
  déterminer, nous devons en plus vérifier que le diviseur n'est pas
  nul.
\end{enumerate}

\begin{C}
if (b > 0)
{
    if (a > 0 && a > INT_MAX / b)
        goto overflow;
    else if (a < 0 && a < INT_MIN / b)
        goto underflow;
}
\end{C}

Ensuite, si l'un des deux opérandes est négatif, nous devons effectué la
même vérification que précédemment et, si les deux sont négatifs,
vérifier que le résultat ne dépasse pas le maximum. Toutefois, nous
devons également faire attention à ce que les opérandes ne soient pas
nuls ainsi qu'au cas de dépassement possible par l'expression
\mybox{INT\_MIN\ /\ -1} (en cas de représentation en complément à
deux).

\begin{C}
else if (b < 0)
{
    if (a < 0 && a < INT_MAX / b)
        goto overflow;
    else if ((-INT_MAX > INT_MIN && b < -1) && a > INT_MIN / b)
        goto underflow;
}
\end{C}

Ce qui nous donne finalement la fonction suivante.

\begin{C}
int safe_mul(int a, int b)
{
    int err = 1;

    if (b > 0)
    {
        if (a > 0 && a > INT_MAX / b)
            goto overflow;
        else if (a < 0 && a < INT_MIN / b)
            goto underflow;
    }
    else if (b < 0)
    {
        if (a < 0 && a < INT_MAX / b)
            goto overflow;
        else if ((-INT_MAX > INT_MIN && b < -1) && a > INT_MIN / b)
            goto underflow;
    }

    return a * b;
underflow:
    err = -1;
overflow:
    errno = ERANGE;
    return err > 0 ? INT_MAX : INT_MIN;
}
\end{C}

\subsection{La division}
\label{la-division-1}

Rassurez-vous, la division est nettement moins pénible à vérifier.
En fait, il y a un seul cas problématique : si les deux opérandes sont
\mybox{INT\_MIN} et \mybox{-1} et que l'opposé du maximum est
inférieur à la borne minimale.

\begin{C}
int safe_div(int a, int b)
{
    if (-INT_MAX > INT_MIN && b == -1 && a == INT_MIN)
    {
        errno = ERANGE;
        return INT_MIN;
    }

    return a / b;
}
\end{C}

\subsection{Le modulo}
\label{le-modulo}

Enfin, pour le modulo, les mêmes règles que celles de la division
s'appliquent.

\begin{C}
int safe_mod(int a, int b)
{
    if (-INT_MAX > INT_MIN && b == -1 && a == INT_MIN)
    {
        errno = ERANGE;
        return INT_MIN;
    }

    return a % b;
}
\end{C}

\section{Gérer les dépassements flottants}
\label{gerer-les-depassements-flottants}

La tâche se complique un peu avec les flottants, puisqu'en plus de tester 
les limites supérieures (cas d'\emph{overflow}), il faudra aussi 
s'intéresser à celles de la partie fractionnaire (cas d'\emph{underflow}).

\subsection{L'addition}
\label{laddition}

Comme toujours, il faut prendre garde à ce que les codes de vérification
ne provoquent pas eux-mêmes des dépassements. Pour implémenter
l'opération d'addition, on commence donc par se ramener au cas plus
simple où on connaît le signe des opérandes \mybox{a} et \mybox{b}.

\begin{itemize}
\item
  Si \mybox{a} et \mybox{b} sont de signes opposés (par exemple :
  \mybox{a\ \textgreater{}=\ 0} et \mybox{b\ \textgreater{}=\ 0},
  alors il ne peut pas se produire d'\emph{overflow}.
\item
  Si \mybox{a} et \mybox{b} sont de même signes, alors il ne peut pas
  se produire d'\emph{underflow}.
\end{itemize}

Dans le premier cas, tester l'\emph{underflow} revient à tester si
\mybox{a\ +\ b} est censé (dans l'arithmétique usuelle) être compris
entre \mybox{-DBL\_MIN} et \mybox{DBL\_MIN}. Il faut cependant exclure
0 de cet intervalle, de manière à ce que \mybox{1.0\ +\ (-1.0)}, par
exemple, ne soit pas considéré comme un \emph{underflow}. Si par exemple
\mybox{a\ \textgreater{}=\ 0} (et dans ce cas
\mybox{b\ \textless{}=\ 0}), il suffit alors de tester si
\mybox{-DBL\_MIN\ -\ a\ \textless{}\ b} et
\mybox{a\ \textless{}\ DBL\_MIN\ -\ b}.

Dans le deuxième cas, il reste à vérifier l'absence d'\emph{overflow}.
Si par exemple \mybox{a} et \mybox{b} sont tous deux positifs, il
s'agit de vérifier que \mybox{a\ \textless{}=\ DBL\_MAX\ -\ b}.

\begin{C}
/* Retourne a+b. */
/* En cas d'overflow, +-DBL_MAX est retourné (suivant le signe de a+b) */
/* En cas d'underflow, +- DBL_MIN est retourné */
/* errno est fixé en conséquence */
double safe_addf(double a, double b)
{
    int positive_result = a > -b;
    double ret_value = positive_result ? DBL_MAX : -DBL_MAX;

    if (a == -b)
        return 0;
    else if ((a < 0) != (b < 0))
    {
        /* On se ramène au cas où a <= 0 et b >= 0 */
        if (a <= 0)
        {
            double t = a;
            a = b;
            b = t;
        }
        if (-DBL_MIN - a < b && a < DBL_MIN - b) 
            goto underflow;
    }
    else 
    {
        int negative = 0;

        /* On se ramène au cas où a et b sont positifs */   
        if (a < 0)
        {
            a = -a;
            b = -b;
            negative = 1;
        }

        if (a > DBL_MAX - b)    
            goto overflow;

        if (negative)
        {
            a = -a;
            b = -b;
        }
    }

    return a + b;

underflow:
    ret_value = positive_result ? DBL_MIN : -DBL_MIN;       
overflow:
    errno = ERANGE;
    return ret_value;
}
\end{C}

Quelques tests :

\begin{C}
static void test_addf(double a, double b)
{
    errno = 0;
    printf("%e + %e = %e", a, b, safe_addf(a, b)); 
    printf(" (%s)\n", strerror(errno));
}

int main(void)
{
    test_addf(1, 1);
    test_addf(5, -6);
    test_addf(DBL_MIN, -3./2*DBL_MIN); /* underflow */
    test_addf(DBL_MAX, 1);
    test_addf(DBL_MAX, -DBL_MAX);
    test_addf(DBL_MAX, DBL_MAX); /* overflow */
    test_addf(-DBL_MAX, -1./2*DBL_MAX); /* overflow */
    return 0;
}

`

\end{C}

Exemple de résultat :

\begin{minted}
1.000000e+00 + 1.000000e+00 = 2.000000e+00 (Success)
5.000000e+00 + -6.000000e+00 = -1.000000e+00 (Success)
2.225074e-308 + -3.337611e-308 = -2.225074e-308 (Numerical result out of range)
1.797693e+308 + 1.000000e+00 = 1.797693e+308 (Success)
1.797693e+308 + -1.797693e+308 = 0.000000e+00 (Success)
1.797693e+308 + 1.797693e+308 = 1.797693e+308 (Numerical result out of range)
-1.797693e+308 + -8.988466e+307 = -1.797693e+308 (Numerical result out of range)
\end{minted}

On peut faire deux remarques sur la fonction \mybox{safe\_addf}
ci-dessus :

\begin{itemize}
\item
  \mybox{addf(DBL\_MAX,\ 1)} peut ne pas être considéré comme un
  \emph{overflow} du fait des pertes de précision.
\item
  à l'inverse, si un \emph{underflow} se déclare, cela ne veut pas
  forcément dire qu'il y a une erreur irrécupérable dans l'algorithme.
  Il est possible que les pertes de précision dans les représentations
  des flottants les causent elles-mêmes.
\end{itemize}

\begin{infobox}
  Pour être tout à fait rigoureux, il
faut remarquer que des expressions telles que \mybox{-DBL\_MIN\ -\ a}
et \mybox{DBL\_MAX\ -\ b} (avec \mybox{a\ \textgreater{}=\ 0} et
\mybox{b\ \textgreater{}=\ 0}) peuvent théoriquement générer
respectivement un \emph{overflow} et un \emph{underflow}. Cela ne peut
cependant se produire que sur des implémentations marginales des
flottants, puisque cela signifierait que la partie fractionnaire (la
mantisse) est représentée sur plus de bits que la différence entre les
valeurs extrémales de l'exposant. Nous avons donc omis ces vérifications
pour des raisons de concision
\end{infobox}

\subsection{La soustraction}
\label{la-soustraction}

Pour notre plus grand bonheur, étant donné qu'il n'y a pas d'asymétrie
entre les bornes, la soustraction revient à faire
\mybox{safe\_addf(a,\ -b)}.

\subsection{La multiplication}
\label{la-multiplication}

Heureusement, la multiplication est un peu plus simple que l'addition !
Comme tout à l'heure, on va essayer de contrôler la valeur des arguments
de manière à savoir dans quel cas on se situe potentiellement
(\emph{underflow} ou \emph{overflow}).

Ici, ce n'est plus le signe de \mybox{a} et de \mybox{b} qui va nous
intéresser : au contraire, on va même le gommer en utilisant la valeur
absolue. Cependant, la position de \mybox{a} et de \mybox{b} par
rapport à +1 et -1 est essentielle :

\begin{itemize}
\item
  si \mybox{fabs(b)\ \textless{}=\ 1}, \mybox{a\ *\ b} peut produire
  un \emph{underflow} mais pas d'\emph{overflow}.
\item
  si \mybox{fabs(b)\ \textgreater{}=\ 1}, \mybox{a\ *\ b} peut produit
  un \emph{overflow} mais pas d'\emph{underflow}.
\end{itemize}

\begin{infobox}
  La fonction \mybox{fabs()} est
définie dans l'en-tête \mybox{\textless{}math.h\textgreater{}} et,
comme son nom l'indique, retourne la valeur absolue de son argument.
N'oubliez pas à cet égard de préciser l'option \mybox{-lm} lors de la
compilation.
\end{infobox}


Ainsi, dans le premier cas, on vérifiera si
\mybox{fabs(a)\ \textless{}\ DBL\_MIN\ /\ fabs(b)}. L'expression
\mybox{DBL\_MIN\ /\ fabs(b)} ne peut ni produire d'\emph{underflow}
(puisque \mybox{fabs(b)\ \textless{}=\ 1}) ni d'\emph{overflow}
(puisque \mybox{DBL\_MIN\ \textless{}=\ fabs(b)}). De même, dans le
deuxième cas, la condition pourra s'écrire
\mybox{fabs(a)\ \textgreater{}\ DBL\_MAX\ /\ fabs(b)}.

\begin{C}
double safe_mulf(double a, double b)
{
    int different_signs = (a < 0) != (b < 0);
    double ret_value = different_signs ? -DBL_MAX : DBL_MAX;

    if (fabs(b) < 1) 
    {
        if (fabs(a) < DBL_MIN / fabs(b))
            goto underflow;
    }
    else
    {
        if (fabs(a) > DBL_MAX / fabs(b))
            goto overflow;
    }   

    return a * b;

underflow:
    ret_value = different_signs ? -DBL_MIN : DBL_MIN;
overflow:
    errno = ERANGE;
    return ret_value;
}
\end{C}

Quelques tests :

\begin{C}
static void test_add(double a, double b)
{
    errno = 0;
    printf("%e + %e = %e", a, b, safe_addf(a, b)); 
    printf(" (%s)\n", strerror(errno));
}

int main(void)
{
    test_add(1, 1);
    test_add(5, -6);
    test_add(DBL_MIN, -3./2*DBL_MIN); /* underflow */
    test_add(DBL_MAX, 1);
    test_add(DBL_MAX, -DBL_MAX);
    test_add(DBL_MAX, DBL_MAX); /* overflow */
    test_add(-DBL_MAX, -1./2*DBL_MAX); /* overflow */
    return 0;
}
\end{C}

Exemple de résultat :

\begin{C}
-1.000000e+00 * -1.000000e+00 = 1.000000e+00 (Success)
2.000000e+00 * 3.000000e+00 = 6.000000e+00 (Success)
1.797693e+308 * 2.000000e+00 = 1.797693e+308 (Numerical result out of range)
2.225074e-308 * 5.000000e-01 = 2.225074e-308 (Numerical result out of range)
-3.337611e-308 * 6.666667e-01 = -2.225074e-308 (Success)
-8.988466e+307 * 1.500000e+00 = -1.348270e+308 (Success)
-8.988466e+307 * 3.000000e+00 = -1.797693e+308 (Numerical result out of range)
\end{C}

\subsection{La division}
\label{la-division}

La division ressemble sensiblement à la multiplication. On traite
cependant en plus le cas où le dénominateur est nul.

\begin{C}
double safe_divf(double a, double b)
{
    int different_signs = (a < 0) != (b < 0);
    double ret_value = different_signs ? -DBL_MAX : DBL_MAX;

    if (b == 0)
    {
        errno = EDOM;
        return ret_value;   
    }
    else if (fabs(b) < 1)
    {
        if (fabs(a) > DBL_MAX * fabs(b))
            goto overflow;
    }
    else
    {
        if (fabs(a) < DBL_MIN * fabs(b))
            goto underflow;
    }

    return a / b;

underflow:
    ret_value = different_signs ? -DBL_MIN : DBL_MIN;
overflow:
    errno = ERANGE;
    return ret_value;
}
\end{C}

Ce chapitre aura été fastidieux, n'hésitez pas à vous poser après sa 
lecture et à tester les différents codes fourni afin de correctement 
appréhender les différentes limites.

\subsubsection{En résumé}
\label{en-resume}

\begin{enumerate}
\def
\labelenumi{\arabic{enumi}.}
\item
  Les limites des différents types sont fournies par des macroconstantes
  définies dans les en-tête \mybox{\textless{}limits.h\textgreater{}}
  et \mybox{\textless{}float.h\textgreater{}} ;
\item
  Les types flottants disposent de quatre limites et non de deux ;
\item
  Un dépassement de capacité se produit soit quand une opération produit
  un résultat hors des limites du type de son expression ou lors de la
  conversion d'une valeur vers un type qui ne peut représenter celle-ci
  ;
\item
  En cas de dépassement, le comportement est indéfini \emph{sauf} pour
  les calculs impliquant des entiers non signés et pour les conversions
  d'un type entier vers un type entier non signé, auquel cas la valeur
  boucle.
\item
  La bibliothèque standard ne fourni aucun outils permettant de détecter
  et/ou de gérer ces dépassements.  
\end{enumerate}