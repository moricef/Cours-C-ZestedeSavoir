\part{Notions avancées}
\label{notions-avancees}

\chapter{La représentation des types }
\label{la-representation-des-types }

Dans le chapitre sur l'allocation
  dynamique, nous avons effleuré la notion de représentation en parlant
  de celle d'objet. Pour rappel, la représentation d'un type correspond
  à la manière dont les données sont réparties en mémoire, plus
  précisémment comment les multiplets et les \emph{bits} les composant
  sont agencés et utilisés.
  
  Dans ce chapitre, nous allons plonger au coeur de ce concept et vous
exposer la représentation des types.
  
\section{La représentation des entiers}
\label{la-representation-des-entiers}

\subsection{Les entiers non signés}
\label{les-entiers-non-signes}

La représentation des entiers non signés étant la plus simple, nous
allons commencer par celle-ci. :)

Dans un entier non signé, les différents \emph{bits} correspondent à une
puissance de deux. Plus précisément, le premier correspond à
2\textsuperscript{0}, le second à 2\textsuperscript{1}, le troisième à
2\textsuperscript{2} et ainsi de suite jusqu'au dernier \emph{bit}
composant le type utilisé. Pour calculer la valeur d'un entier non
signé, il suffit donc d'additionner les puissances de deux correspondant
aux bits à 1.

Pour illustrer notre propos, voici un tableau comprenant la
représentation de plusieurs nombres au sein d'un objet de type
\mybox{unsigned\ char} (ici sur un octet).

\begin{table}
\centering
\begin{tabular}{|l|l|l|l|l|l|l|l|l|}\hline
  \rowcolor{gris-tab-entete}\bf Nombre&\multicolumn{8}{l|}{\bf Représentation}\\
  \hline
  \multirow{3}{2cm}{1} & 0 & 0 & 0 & 0 & 0 & 0 & 0 & 1 \\ 
  \cline{2-9}
  &2\textsuperscript{7}&2\textsuperscript{6}&2\textsuperscript{5}&2\textsuperscript{4}&2\textsuperscript{3}&2\textsuperscript{2}&2\textsuperscript{1}&2\textsuperscript{0}\\ 
  \cline{2-9}
   &\multicolumn{8}{c|}{2\textsuperscript{0}=1}\\ 
  \hline
  \multirow{3}{2cm}{3} & 0 & 0 & 0 & 0 & 0 & 0 & 1 & 1 \\
  \cline{2-9}
  &2\textsuperscript{7}&2\textsuperscript{6}&2\textsuperscript{5}&2\textsuperscript{4}&2\textsuperscript{3}&2\textsuperscript{2}&2\textsuperscript{1}&2\textsuperscript{0}\\
  \cline{2-9}
  &\multicolumn{8}{c|}{2\textsuperscript{0}+2\textsuperscript{1}=3}\\
  \hline
  \multirow{3}{2cm}{42} & 0 & 0 & 1 & 0 & 1 & 0 & 1 & 0 \\
  \cline{2-9}
  &2\textsuperscript{7}&2\textsuperscript{6}&2\textsuperscript{5}&2\textsuperscript{4}&2\textsuperscript{3}&2\textsuperscript{2}&2\textsuperscript{1}&2\textsuperscript{0}\\
  \cline{2-9}
  &\multicolumn{8}{c|}{2\textsuperscript{1}+2\textsuperscript{3}+2\textsuperscript{5}=42}\\
  \hline
  \multirow{3}{2cm}{255} & 1 & 1 & 1 & 1 & 1 & 1 & 1 & 1 \\
  \cline{2-9}
  &2\textsuperscript{7}&2\textsuperscript{6}&2\textsuperscript{5}&2\textsuperscript{4}&2\textsuperscript{3}&2\textsuperscript{2}&2\textsuperscript{1}&2\textsuperscript{0}\\
  \cline{2-9}
  &\multicolumn{8}{c|}{2\textsuperscript{0}+2\textsuperscript{1}+2\textsuperscript{2}+2\textsuperscript{3}+2\textsuperscript{4}+2\textsuperscript{5}+2\textsuperscript{6}+2\textsuperscript{7}=255}\\
  \hline
\end{tabular}
\end{table}

\subsection{Les entiers signés}
\label{les-entiers-signes}

Les choses se compliquent un peu avec les entiers signés. En effet, il
nous faut représenter une information supplémentaire : le signe de la
valeur.

\subsubsection{La représentation en signe et
magnitude}
\label{la-representation-en-signe-et-magnitude}

La première solution qui vous est peut-être venue à l'esprit serait de
réserver un \emph{bit}, par exemple le dernier, pour représenter le
signe. Ainsi, la représentation est identique à celle des entiers non
signés si ce n'est que le dernier \emph{bit} est réservé pour le signe
(ce qui diminue donc en conséquence la valeur maximale représentable).
Cette méthode est appelée \textbf{représentation en signe et magnitude}.

\begin{table}
\centering
\begin{tabular}{|l|l|l|l|l|l|l|l|l|}\hline
\rowcolor{gris-tab-entete}\bf Nombre & \multicolumn{8}{l|}{\bf Représentation en signe et magnitude}\\
  \hline
  \multirow{3}{2cm}{1} & 0 & 0 & 0 & 0 & 0 & 0 & 0 & 1 \\ 
  \cline{2-9}
    &+&2\textsuperscript{6}&2\textsuperscript{5}&2\textsuperscript{4}&2\textsuperscript{3}&2\textsuperscript{2}&2\textsuperscript{1}&2\textsuperscript{0}\\ 
  \cline{2-9}
   &\multicolumn{8}{c|}{2\textsuperscript{0}=1}\\ 
  \hline
  \multirow{3}{2cm}{1} & 0 & 0 & 0 & 0 & 0 & 0 & 0 & 1 \\ 
  \cline{2-9}
    &-&2\textsuperscript{6}&2\textsuperscript{5}&2\textsuperscript{4}&2\textsuperscript{3}&2\textsuperscript{2}&2\textsuperscript{1}&2\textsuperscript{0}\\ 
  \cline{2-9}
   &\multicolumn{8}{c|}{-2\textsuperscript{0}=-1}\\ 
  \hline
  \end{tabular}
\end{table}

Cependant, si la représentation en signe et magnitude paraît simple,
elle n'est en vérité pas facile à mettre en œuvre au sein d'un
processeur car elle implique plusieurs vérifications (notamment au
niveau du signe) qui se traduisent par des circuits supplémentaires et
un surcoût en calcul. De plus, elle laisse deux représentations
possibles pour le zéro (\mybox{0000\ 0000} et \mybox{1000\ 0000}), ce
qui est gênant pour l'évaluation des conditions.

\subsubsection{La représentation en complément à un}
\label{la-representation-en-complement-a-un}

Dès lors, comment pourrions nous faire pour simplifier nos calculs en
évitant des vérifications liées au signe ? \emph{Eh} bien, sachant que
le maximum représentable dans notre exemple est \mybox{255} (soit
\mybox{1111\ 1111}), il nous est possible de représenter chaque nombre
négatif comme la différence entre le maximum et sa valeur absolue. Par
exemple \mybox{-1} sera représenté par \mybox{255\ -\ 1}, soit
\mybox{254} (\mybox{1111\ 1110}). Cette représentation est appelée
\textbf{représentation en complément à un}.

Ainsi, si nous additionnons \mybox{1} et \mybox{-1}, nous n'avons pas
de vérifications à faire et obtenons le maximum, \mybox{255}.
Toutefois, ceci implique, comme pour la représentation en signe et
magnitude, qu'il existe deux représentations pour le zéro :
\mybox{0000\ 0000} et \mybox{1111\ 1111}.

\begin{table}
\centering
\begin{tabular}{|l|l|l|l|l|l|l|l|l|}\hline
\rowcolor{gris-tab-entete}\bf Nombre& \multicolumn{8}{l|}{\bf Représentation en complément à un}\\
 \hline
  \multirow{3}{2cm}{1} & 0 & 0 & 0 & 0 & 0 & 0 & 0 & 1 \\ 
  \cline{2-9}
  &+&2\textsuperscript{6}&2\textsuperscript{5}&2\textsuperscript{4}&2\textsuperscript{3}&2\textsuperscript{2}&2\textsuperscript{1}&2\textsuperscript{0}\\ 
  \cline{2-9}
   &\multicolumn{8}{c|}{2\textsuperscript{0}=1}\\ 
  \hline
  \multirow{3}{2cm}{-1} & 1 & 1 & 1 & 1 & 1 & 1 & 1 & 0 \\ 
  \cline{2-9}
    &-/2\textsuperscript{7}&2\textsuperscript{6}&2\textsuperscript{5}&2\textsuperscript{4}&2\textsuperscript{3}&2\textsuperscript{2}&2\textsuperscript{1}&2\textsuperscript{0}\\ 
  \cline{2-9}
   &\multicolumn{8}{c|}{-(255-2\textsuperscript{7}-2\textsuperscript{6}-2\textsuperscript{5}-2\textsuperscript{4}-2\textsuperscript{3}-2\textsuperscript{2}-2\textsuperscript{1})=-1}\\ 
  \hline
\end{tabular}
\end{table}

\begin{infobox} 
 Notez qu'en fait, cette
représentation revient à inverser tous les \emph{bits} d'un nombre
positif pour obtenir son équivalent négatif. Par exemple, si nous
inversons \mybox{1} (\mybox{0000\ 0001}), nous obtenons bien
\mybox{-1} (\mybox{1111\ 1110}) comme ci-dessus.
\end{infobox}

Par ailleurs, il subsiste un second problème : dans le cas où deux
nombres négatifs sont additionnés, le résultat obtenu n'est pas valide.
En effet, \mybox{-1\ +\ -1} nous donne
\mybox{1111\ 1110\ +\ 1111\ 1110} soit \mybox{1\ 1111\ 1100} ;
autrement dit, comme nous travaillons sur huit \emph{bits} pour nos
exemples, \mybox{-3}\ldots{} Mince !

Pour résoudre ce problème, il nous faut reporter la dernière retenue
(soit ici le dernier \emph{bit} que nous avons ignoré) au début (ce qui
revient à ajouté un) ce qui nous permet d'obtenir \mybox{1111\ 1101},
soit \mybox{-2}.

\subsubsection{La représentation en complément à deux}
\label{la-representation-en-complement-a-deux}

Ainsi est apparue une troisième représentation : celle en
\textbf{complément à deux}. Celle-ci conserve les qualités de la
représentation en complément à un, mais lui corrige certains défauts. En
fait, elle représente les nombres négatifs de la même manière que la
représentation en complément à un, si ce n'est que la soustraction
s'effectue entre le maximum augmenté de un (soit \mybox{256} dans notre
cas) et non le maximum.

Ainsi, par exemple, la représentation en complément à deux de
\mybox{-1} est \mybox{256\ -\ 1}, soit \mybox{255}
(\mybox{1111\ 1111}).

\begin{table}
\centering
\begin{tabular}{|l|l|l|l|l|l|l|l|l|}\hline
\rowcolor{gris-tab-entete}\bf Nombre& \multicolumn{8}{l|}{\bf Représentation en complément à deux}\\
 \hline
  \multirow{3}{2cm}{1} & 0 & 0 & 0 & 0 & 0 & 0 & 0 & 1 \\ 
  \cline{2-9}
    &+&2\textsuperscript{6}&2\textsuperscript{5}&2\textsuperscript{4}&2\textsuperscript{3}&2\textsuperscript{2}&2\textsuperscript{1}&2\textsuperscript{0}\\ 
  \cline{2-9}
    &\multicolumn{8}{c|}{2\textsuperscript{0}=1}\\ 
  \hline
  \multirow{3}{2cm}{-1} & 1 & 1 & 1 & 1 & 1 & 1 & 1 & 0 \\ 
  \cline{2-9}
    &-/2\textsuperscript{7}&2\textsuperscript{6}&2\textsuperscript{5}&2\textsuperscript{4}&2\textsuperscript{3}&2\textsuperscript{2}&2\textsuperscript{1}&2\textsuperscript{0}\\ 
  \cline{2-9}
    &\multicolumn{8}{c|}{-(256-2\textsuperscript{7}-2\textsuperscript{6}-2\textsuperscript{5}-2\textsuperscript{4}-2\textsuperscript{3}-2\textsuperscript{2}-2\textsuperscript{1}-2\textsuperscript{0})=-1}\\ 
  \hline
\end{tabular}
\end{table}

\begin{infobox}
  Remarquez que cette représentation
revient finalement à inverser tous les \emph{bits} d'un nombre positif
et à augmenter le résultat de un en vue d'obtenir son équivalent
négatif. Par exemple, si nous inversons les \emph{bits} du nombre
\mybox{1} nous obtenons \mybox{1111\ 1110} et si nous lui ajoutons un,
nous avons bien \mybox{1111\ 1111}.
\end{infobox}


Cette fois ci, l'objectif recherché est atteint ! :)

En effet, si nous additionnons \mybox{1} et \mybox{-1} (soit
\mybox{0000\ 0001} et \mybox{1111\ 1111}) et que nous ignorons la
retenue, nous obtenons bien zéro. De plus, il n'y a plus de cas
particulier de retenue à déplacer comme en complément à un, puisque, par
exemple, la somme \mybox{-1\ +\ -2}, soit
\mybox{1111\ 1111\ +\ 1111\ 1110} donne \mybox{1\ 1111\ 1101},
autrement dit \mybox{-3} \emph{sans la retenue}. Enfin, nous n'avons
plus qu'une seule représentation pour le zéro.

\begin{infobox}
  \textbf{Point de norme (1)}
  La norme\footnote{\footnotesize{la norme C89 ne précise pas les
  représentations possibles pour les types entiers (Programming Language
  C, X3J11/88-090, § A.6.3.4, Integers, al. 1), mais les normes
  suivantes bien. Voyez à cet égard : ISO/IEC 9899:TC3, doc.
  \MYhref{http://open-std.org/JTC1/SC22/WG14/www/docs/n1256.pdf}{N1256}, §
  6.2.6.2, Integer types, p.~38 et ISO/IEC 9899:201x, doc.
  \MYhref{http://open-std.org/JTC1/SC22/WG14/www/docs/n1570.pdf}{N1570}, §
  6.2.6.2, Integer types, p.~45.}} du langage C ne précise pas quelle
  représentation doit être utilisée pour les entiers signés. Elle impose
  uniquement qu'il s'agisse d'une de ces trois. Cependant, dans les faits,
  il s'agit presque toujours de la représentation en complément à deux.
\end{infobox}


\begin{erreurbox}
  \textbf{Point de norme (2)}
  Notez que chacune de ces représentations disposent toutefois
d'une suite de \emph{bits} qui est susceptible de ne pas représenter une
valeur et de produire une erreur en cas d'utilisation\footnote{\footnotesize{La norme
  C89 est muette sur ce point, mais les normes suivantes sont plus
  locaces : ISO/IEC 9899:TC3, doc.
  \MYhref{http://open-std.org/JTC1/SC22/WG14/www/docs/n1256.pdf}{N1256}, §
  6.2.6.1 , General, al. 5, pp.~37-38 et ISO/IEC 9899:201x, doc.
  \MYhref{http://open-std.org/JTC1/SC22/WG14/www/docs/n1570.pdf}{N1570}, §
  6.2.6.1, General, al. 5, p.~44.}}.
  \begin{itemize}
    \item 
      Pour les représentations en signe et magnitude et en complément à deux, il s'agit
      de la suite où tous les \emph{bits} sont à zéro et le \emph{bit} de
      signe à un : \mybox{1000\ 0000} ; 
     \item 
	Pour la représentation en complément à un, il s'agit de la suite où tous les \emph{bits} sont à
	un, y compris le \emph{bit} de signe : \mybox{1111\ 1111}.
  \end{itemize}

Dans le cas des représentations en signe et magnitude et en
complément à un, il s'agit des représentations possibles pour le « zéro
négatif ». Pour la représentation en complément à deux, cette suite est
le plus souvent utilisée pour représenter un nombre négatif
supplémentaire (dans le cas de \mybox{1000\ 0000}, il s'agira de
\mybox{-128}). Toutefois, même si ces représentations peuvent être
utilisées pour représenter une valeur valide, ce n'est pas forcément le
cas.

Néanmoins, rassurez-vous, ces valeurs ne peuvent être produites dans le cas 
de calculs ordinaires, sauf survenance d'une condition exceptionnelle comme 
un dépassement de capacité (nous en parlerons bientôt). Vous n'avez donc pas 
de vérifications supplémentaires à effectuer à leur égard. Évitez
simplement d'en produire une délibérément, par exemple en l'affectant
directement à une variable.
\end{erreurbox}


\subsection{Les bits de bourrages}
\label{les-bits-de-bourrages}

Il est important de préciser que tous les \emph{bits} composant un type
entier ne sont pas forcément utilisés pour représenter la valeur qui y
est stockée. Nous l'avons vu avec le \emph{bit} de signe, qui ne
correspond pas à une valeur. La norme prévoit également qu'il peut
exister des \emph{bits} de bourrages, et ce \emph{aussi bien pour les
entiers signés que pour les entiers non signés} à l'exception du type
\mybox{char}. Ceux-ci peuvent par exemple être employés pour maintenir
d'autres informations (par exemple : le type de la donnée stockée, ou
encore un
\MYhref{https://fr.wikipedia.org/wiki/Somme_de_contrôle\#Exemple_:_bit_de_parité}{bit
de parité} pour vérifier l'intégrité de celle-ci).

\begin{attentionbox}
  Par conséquent, il n'y a pas forcément
une corrélation parfaite entre le nombres de \emph{bits} composant un
type entier et la valeur maximale qui peut y être stockée.
\end{attentionbox}


\begin{infobox}
  Sachez toutefois que les \emph{bits}
de bourrages au sein des types entiers sont assez rares, les
architectures les plus courantes n'en emploient pas.
\end{infobox}

\section{La représentation des flottants}
\label{la-representation-des-flottants}

La représentation des types flottants amène deux difficultés 
supplémentaires :

\begin{itemize}
\item
  la gestion de nombres réels, c'est-à-dire potentiellement composés
  d'une partie entière et d'une suite de décimales ;
\item
  la possibilité de stocker des nombres de différents ordres de
  grandeur, entre 10\textsuperscript{-37} et 10\textsuperscript{37}.
\end{itemize}

\subsection{Première approche}
\label{premiere-approche-2}

\subsubsection{Avec des entiers}
\label{avec-des-entiers}

Une première solution consisterait à garantir une précision à
10\textsuperscript{-37} en utilisant deux nombres entiers : un, signé,
pour la partie entière et un, non signé, pour stocker la suite de
décimales.

Cependant, si cette approche a le mérite d'être simple, elle a le
désavantage d'utiliser \emph{beaucoup} de mémoire. En effet, pour
stocker un entier de l'ordre de 10\textsuperscript{37}, il serait
nécessaire d'utiliser un peu moins de 128 \emph{bits}, soit 16 octets
(et donc environ 32 octets pour la représentation globale). Un tel coût
est inconcevable, même à l'époque actuelle.

Une autre limite provient de la difficulté d'effectuer des calculs sur
les nombres flottants avec une telle représentation : il faudrait tenir
compte du passage des décimales vers la partie entière et inversement.

\subsubsection{Avec des puissances de deux négatives}
\label{avec-des-puissances-de-deux-negatives}

Une autre représentation possible consiste à attribuer des puissances de
deux \emph{négatives} à une partie des \emph{bits}. Les calculs sur les
nombres flottants obtenus de cette manière sont similaires à ceux sur
les nombres entiers. Le tableau ci-dessous illustre ce concept en
divisant un octet en deux : quatre \emph{bits} pour la partie entière et
quatre \emph{bits} pour la partie fractionnaire.

\begin{table}
\centering
\begin{tabular}{|l|l|l|l|l|l|l|l|l|}\hline
  \rowcolor{gris-tab-entete}\bf Nombre&\multicolumn{8}{l|}{\bf Représentation}\\
  \hline
  \multirow{3}{2cm}{1} & 0 & 0 & 0 & 1 & 0 & 0 & 0 & 0 \\ 
  \cline{2-9}
  &2\textsuperscript{3}&2\textsuperscript{2}&2\textsuperscript{1}&2\textsuperscript{0}&2\textsuperscript{-1}&2\textsuperscript{-2}&2\textsuperscript{-3}&2\textsuperscript{-4}\\ 
  \cline{2-9}
   &\multicolumn{8}{c|}{2\textsuperscript{0}=1}\\ 
  \hline
  \multirow{3}{2cm}{1,5} & 0 & 0 & 0 & 1 & 1 & 0 & 0 & 0 \\
  \cline{2-9}
  &2\textsuperscript{3}&2\textsuperscript{2}&2\textsuperscript{1}&2\textsuperscript{0}&2\textsuperscript{-1}&2\textsuperscript{-2}&2\textsuperscript{-3}&2\textsuperscript{-4}\\ 
  \cline{2-9}
  &\multicolumn{8}{c|}{2\textsuperscript{-1}+2\textsuperscript{0}=1,5}\\
  \hline
  \multirow{3}{2cm}{0,875} & 0 & 0 & 0 & 1 & 1 & 1 & 1 & 0 \\
  \cline{2-9}
  &2\textsuperscript{3}&2\textsuperscript{2}&2\textsuperscript{1}&2\textsuperscript{0}&2\textsuperscript{-1}&2\textsuperscript{-2}&2\textsuperscript{-3}&2\textsuperscript{-4}\\ 
  \cline{2-9}
  &\multicolumn{8}{c|}{2\textsuperscript{-3}+2\textsuperscript{-2}+2\textsuperscript{-1}=0,875}\\
  \hline
\end{tabular}
\end{table}

Toutefois, notre problème de capacité persiste : il nous faudra toujours
une grande quantité de mémoire pour pouvoir stocker des nombres d'ordre
de grandeur aussi différents.

\subsection{Une histoire de virgule flottante}
\label{une-histoire-de-virgule-flottante}

Par conséquent, la solution qui a été retenue historiquement consiste à
réserver quelques \emph{bits} qui contiendront la valeur d'un exposant.
Celui-ci sera ensuite utilisé pour déterminer à quelles puissances de
deux correspondent les \emph{bits} restants. Ainsi, la virgule séparant
la partie entière de sa suite décimales est dite « flottante », car sa
position est ajustée par l'exposant. Toutefois, comme nous allons le
voir, ce gain en mémoire ne se réalise pas sans sacrifice : la précision
des calculs va en pâtir.

Le tableau suivant utilise deux octets : un pour stocker un exposant sur
sept \emph{bits} avec un \emph{bit} de signe ; un autre pour stocker la
valeur du nombre. L'exposant est lui aussi signé et est représenté en
signe et magnitude (par souci de facilité). Par ailleurs, cet exposant
est attribué au premier \emph{bit} du deuxième octet ; les autres
correspondent à une puissance de deux chaque fois inférieure d'une
unité.

\begin{table}
\centering
\begin{tabular}{p{1,5cm}|p{1,5cm}|l|l|l|l|l|l|l|l|l|l|l|l|l|l|l|l|}\hline
  \rowcolor{gris-tab-entete}\bf Nombre&\bf Signe&\multicolumn{7}{l|}{\bf Exposant}&\multicolumn{8}{l|}{\bf Bits du nombre} \\
  \hline
  \multirow{3}{2cm}{1} & 0 & 0 & 0 & 0 & 0 & 0 & 0 & 0 & 1 & 0 & 0 & 0 & 0 & 0 & 0 & 0\\ 
  \cline{2-17}
    &+&+&2\textsuperscript{5}&2\textsuperscript{4}&2\textsuperscript{3}&2\textsuperscript{2}&2\textsuperscript{1}&2\textsuperscript{0}&2\textsuperscript{0}&2\textsuperscript{-1}&2\textsuperscript{-2}&2\textsuperscript{-3}&2\textsuperscript{-4}&2\textsuperscript{-5}&2\textsuperscript{0-6}&2\textsuperscript{-7}\\ 
  \cline{2-17}
    &+&\multicolumn{7}{c|}{0}&\multicolumn{8}{c|}{2\textsuperscript{0}=1}\\ 
  \hline
  
  \multirow{3}{2cm}{-1} & 1 & 0 & 0 & 0 & 0 & 0 & 0 & 0 & 1 & 0 & 0 & 0 & 0 & 0 & 0 & 0\\ 
  \cline{2-17}
    &-&+&2\textsuperscript{5}&2\textsuperscript{4}&2\textsuperscript{3}&2\textsuperscript{2}&2\textsuperscript{1}&2\textsuperscript{0}&2\textsuperscript{0}&2\textsuperscript{-1}&2\textsuperscript{-2}&2\textsuperscript{-3}&2\textsuperscript{-4}&2\textsuperscript{-5}&2\textsuperscript{0-6}&2\textsuperscript{-7}\\ 
  \cline{2-17}
    &-&\multicolumn{7}{c|}{0}&\multicolumn{8}{c|}{-2\textsuperscript{0}=-1}\\ 
  \hline
  
  \multirow{3}{2cm}{0,5} & 0 & 0 & 0 & 0 & 0 & 0 & 0 & 0 & 0 & 1 & 0 & 0 & 0 & 0 & 0 & 0\\ 
  \cline{2-17}
     &+&+&2\textsuperscript{5}&2\textsuperscript{4}&2\textsuperscript{3}&2\textsuperscript{2}&2\textsuperscript{1}&2\textsuperscript{0}&2\textsuperscript{0}&2\textsuperscript{-1}&2\textsuperscript{-2}&2\textsuperscript{-3}&2\textsuperscript{-4}&2\textsuperscript{-5}&2\textsuperscript{0-6}&2\textsuperscript{-7}\\ 
  \cline{2-17}
     &+&\multicolumn{7}{c|}{0}&\multicolumn{8}{c|}{2\textsuperscript{-1}=0,5}\\ 
  \hline
  
  \multirow{3}{2cm}{10} & 0 & 0 & 0 & 0 & 0 & 0 & 1 & 1 & 1 & 0 & 1 & 0 & 0 & 0 & 0 & 0\\ 
  \cline{2-17}
     &+&+&2\textsuperscript{5}&2\textsuperscript{4}&2\textsuperscript{3}&2\textsuperscript{2}&2\textsuperscript{1}&2\textsuperscript{0}&2\textsuperscript{3}&2\textsuperscript{2}&2\textsuperscript{1}&2\textsuperscript{0}&2\textsuperscript{-1}&2\textsuperscript{-2}&2\textsuperscript{-3}&2\textsuperscript{-4}\\ 
  \cline{2-17}
     &+&\multicolumn{7}{c|}{2\textsuperscript{1}+2\textsuperscript{0}=3}&\multicolumn{8}{c|}{2\textsuperscript{3}+2\textsuperscript{1}=10}\\ 
  \hline
  
  \multirow{3}{2cm}{0,00001} & 0 & 1 & 0 & 1 & 0 & 0 & 1 & 0 & 1 & 0 & 1 & 1 & 0 & 1 & 1 & 1\\ 
  \cline{2-17}
    &+&-&2\textsuperscript{5}&2\textsuperscript{4}&2\textsuperscript{3}&2\textsuperscript{2}&2\textsuperscript{1}&2\textsuperscript{0}&2\textsuperscript{-17}&2\textsuperscript{-18}&2\textsuperscript{-19}&2\textsuperscript{-20}&2\textsuperscript{-21}&2\textsuperscript{-22}&2\textsuperscript{-23}&2\textsuperscript{-24}\\ 
  \cline{2-17}
     &+&\multicolumn{7}{c|}{-2\textsuperscript{1}+-2\textsuperscript{4}=-17}&\multicolumn{8}{c|}{2\textsuperscript{-24}+2\textsuperscript{-23}+2\textsuperscript{-22}+2\textsuperscript{-21}+2\textsuperscript{-20}+2\textsuperscript{-19}+2\textsuperscript{-18}+2\textsuperscript{-17}=\~0,00001}\\ 
  \hline
  
  \multirow{3}{2cm}{32769} & 0 & 0 & 1 & 1 & 1 & 1 & 1 & 0 & 0 & 0 & 0 & 0 & 0 & 0 & 0 & 0\\ 
  \cline{2-17}
    &+&+&2\textsuperscript{5}&2\textsuperscript{4}&2\textsuperscript{3}&2\textsuperscript{2}&2\textsuperscript{1}&2\textsuperscript{0}&2\textsuperscript{15}&2\textsuperscript{14}&2\textsuperscript{13}&2\textsuperscript{12}&2\textsuperscript{11}&2\textsuperscript{10}&2\textsuperscript{9}&2\textsuperscript{8}\\ 
  \cline{2-17}
     &+&\multicolumn{7}{c|}{2\textsuperscript{0}+2\textsuperscript{1}+2\textsuperscript{2}+2\textsuperscript{3}=15}&\multicolumn{8}{c|}{2\textsuperscript{15}= 32768}\\ 
  \hline
  \end{tabular}
\end{table}
  
Les quatre premiers exemples n'amènent normalement pas de commentaires
particuliers ; néanmoins, les deux derniers illustre deux problèmes
posés par les nombres flottants.

\subsubsection{Les approximations}
\label{les-approximations}

Les nombres réels mettent en évidence la difficulté (voir
l'impossibilité) de représenter une suite de décimales à l'aide d'une
somme de puissances de deux. En effet, le plus souvent, les valeurs ne
peuvent être qu'approchées et obtenues suite à des arrondis. Dans le cas
de 0,00001, la somme 2\textsuperscript{-24} + 2\textsuperscript{-23} +
2\textsuperscript{-22} + 2\textsuperscript{-20} + 2\textsuperscript{-19}
+ 2\textsuperscript{-17} donne en vérité 0,000010907649993896484375 qui,
une fois arrondie, donnera bien 0,00001.

\subsubsection{Les pertes de précision}
\label{les-pertes-de-precision}

Dans le cas où la valeur à représenter ne peut l'être entièrement avec
le nombre de \emph{bits} disponible (autrement dit, le nombre de
puissances de deux disponible), la partie la moins significative sera
abandonnée et, corrélativement, de la précision.

Ainsi, dans notre exemple, si nous souhaitons représenter le nombre
32769, nous avons besoin d'un exposant de 15 afin d'obtenir 32768.
Toutefois, vu que nous ne possédons que de 8 \emph{bits} pour
représenter notre nombre, il nous est impossible d'attribuer l'exposant
0 à un des \emph{bits}. Cette information est donc perdue et seule la
valeur la plus significative (ici 32768) est conservée.

\subsection{Formalisation}
\label{formalisation}

L'exemple simplifié que nous vous avons montré illustre dans les grandes
lignes la représentation des nombres flottants. Cependant, vous vous en
doutez, la réalité est un peu plus complexe. De manière plus générale,
un nombre flottant est représenté à l'aide : d'un signe, d'un exposant
et d'une \textbf{mantisse} qui est en fait la suite de puissances qui
sera additionnée.

Par ailleurs, nous avons utilisés une suite de puissance de deux, mais
il est parfaitement possible d'utiliser une autre base. Beaucoup de
calculatrices utilisent par exemple des suites de puissances de dix et
non de deux.

\begin{infobox}
  À l'heure actuelle, les nombres
flottants sont presque toujours représentés suivant la norme
\MYhref{https://fr.wikipedia.org/wiki/IEEE_754}{IEEE 754} qui utilise une
représentation en base deux.
\end{infobox}


\section{La représentation des pointeurs}
\label{la-representation-des-pointeurs}

Cet extrait sera relativement court : le représentation des pointeurs est 
indéterminée. Sur la plupart des architectures, il s'agit en vérité d'entiers 
non signés, mais il n'y a absolument aucune garantie à ce sujet.

\section{Ordre des multiplets et des bits}
\label{ordre-des-multiplets-et-des-bits}

Jusqu'à présent, nous vous avons montré les représentations binaires en
ordonnant les \emph{bits} et les multiplets par puissance de deux
croissantes. Cependant, s'il s'agit d'une représentation possible, elle
n'est pas la seule. En vérité, l'ordre des multiplets et des \emph{bits}
peut varier d'une machine à l'autre.

\subsection{Ordre des multiplets}
\label{ordre-des-multiplets}

Dans le cas où un type est composé de plus d'un multiplet (ce qui, à
l'exception du type \mybox{char}, est pour ainsi dire toujours le cas),
ceux-ci peuvent être agencés de différentes manières. L'ordre des
multiplets d'une machine est appelé son \textbf{boutisme}
(\emph{endianness} en anglais).

Il en existe principalement deux : le gros-boutisme et le
petit-boutisme.

\subsubsection{Le gros-boutisme}
\label{le-gros-boutisme}

Sur une architecture gros-boutiste, les multiplets sont ordonnés par
\textbf{poids} décroissant.

Le poids d'un \emph{bit} se détermine suivant la puissance de deux qu'il
représente : plus elle est élevée, plus le \emph{bit} a un poids
important. Le \emph{bit} représentant la puissance de deux la plus basse
est appelé de \emph{bit} de \textbf{poids faible} et celui correspondant
à la puissance la plus grande est nommé \emph{bit} de \textbf{poids
fort}. Pour les multiplets, le raisonnement est identique : son poids
est déterminé par celui des \emph{bits} le composant.

Autrement dit, une machine gros-boutiste place les multiplets comprenant
les puissance de deux les plus élevées en premier (nos exemple
précédents utilisaient donc cette représentation).

\subsubsection{Le petit-boutisme}
\label{le-petit-boutisme}

Une architecture petit-boutiste fonctionne de manière inverse : les
multiplets sont ordonnés par poids croissant.

Ainsi, si nous souhaitons stocker le nombre \mybox{1} dans une variable
de type \mybox{unsigned\ short} (qui sera, pour notre exemple, d'une
taille de deux octets), nous aurons les deux résultats suivants, selon
le boutisme de la machine.

\begin{table}
\centering
\begin{tabular}{|l|l|l|l|l|l|l|l|l|l|l|l|l|l|l|l|l|}\hline
\rowcolor{gris-tab-entete}\footnotesize \bf Nombre& \multicolumn{16}{l|}{\footnotesize \bf Représentation gros-boutiste}\\
 \hline
  \multirow{4}{2cm}{1} &\multicolumn{8}{c|}{Octet de poids fort}&\multicolumn{8}{c|}{Octet de poids faible}\\
  & 0 & 0 & 0 & 0 & 0 & 0 & 0 & 0 & 0 & 0 & 0 & 0 & 0 & 0 & 0 & 1 \\ 
  \cline{2-17}
    &2\textsuperscript{15}&2\textsuperscript{14}&2\textsuperscript{13}&2\textsuperscript{12}&2\textsuperscript{11}&2\textsuperscript{10}&2\textsuperscript{9}&2\textsuperscript{8}&2\textsuperscript{7}&2\textsuperscript{6}&2\textsuperscript{5}&2\textsuperscript{4}&2\textsuperscript{3}&2\textsuperscript{2}&2\textsuperscript{1}&2\textsuperscript{0}\\ 
  \cline{2-17}
    &\multicolumn{16}{c|}{2\textsuperscript{0}=1}\\ 
  \hline
  \end{tabular}
\end{table}

\begin{table}
\centering
\begin{tabular}{|l|l|l|l|l|l|l|l|l|l|l|l|l|l|l|l|l|}\hline
\rowcolor{gris-tab-entete}\footnotesize \bf Nombre& \multicolumn{16}{l|}{\footnotesize \bf Représentation petit-boutiste}\\
 \hline
  \rowcolor{gris-clair-tab}\multirow{4}{2cm}{1} &\multicolumn{8}{c|}{Octet de poids faible}&\multicolumn{8}{c|}{Octet de poids fort}\\
  & 0 & 0 & 0 & 0 & 0 & 0 & 0 & 1 & 0 & 0 & 0 & 0 & 0 & 0 & 0 & 0 \\ 
  \cline{2-17}
    &2\textsuperscript{7}&2\textsuperscript{6}&2\textsuperscript{5}&2\textsuperscript{4}&2\textsuperscript{3}&2\textsuperscript{2}&2\textsuperscript{1}&2\textsuperscript{0}&2\textsuperscript{15}&2\textsuperscript{14}&2\textsuperscript{13}&2\textsuperscript{12}&2\textsuperscript{11}&2\textsuperscript{10}&2\textsuperscript{9}&2\textsuperscript{8}\\ 
  \cline{2-17}
     \rowcolor{gris-clair-tab}&\multicolumn{16}{c|}{2\textsuperscript{0}=1}\\ 
  \hline
  \end{tabular}
\end{table}

\begin{infobox} 
 Le boutisme est relativement
transparent du point de vue du programmeur, puisqu'il s'agit d'une
propriété de la mémoire. En pratique, de tels problèmes ne se posent que
lorsqu'on accède à la mémoire à travers des types différents de celui de
l'objet stocké initialement (par exemple via un pointeur sur
\mybox{char}). En particulier, la communication entre plusieurs
machines doit les prendre en compte.
\end{infobox}


\subsection{Ordre des bits}
\label{ordre-des-bits}

Cependant, ce serait trop simple si le problème en restait là\ldots{}
:-°\\
Malheureusement, l'ordre des \emph{bits} varie également d'une machine à
l'autre et ce, \emph{indépendamment de l'ordre des multiplets}. Comme
pour les multiplets, il existe différente possibilités, mais les plus
courante sont le gros-boutisme et le petit-boutisme. Ainsi, il est par
exemple possible que la représentation des multiplets soit
gros-boutiste, mais que celle des \emph{bits} soit petit-boutiste (et
inversément).

\begin{table}
\centering
\begin{tabular}{|l|l|l|l|l|l|l|l|l|l|l|l|l|l|l|l|l|}\hline
\rowcolor{gris-tab-entete}\footnotesize \bf Nombre& \multicolumn{16}{l|}{\footnotesize \bf Représentation gros-boutiste pour les multiplets et petit-boutiste pour les \emph{bits}}\\
 \hline
  \multirow{4}{2cm}{1} &\multicolumn{8}{c|}{Octet de poids fort}&\multicolumn{8}{c|}{Octet de poids faible}\\
  & 0 & 0 & 0 & 0 & 0 & 0 & 0 & 0& 1 & 0 & 0 & 0 & 0 & 0 & 0 & 0 \\ 
  \cline{2-17}
    &2\textsuperscript{8}&2\textsuperscript{9}&2\textsuperscript{10}&2\textsuperscript{11}&2\textsuperscript{12}&2\textsuperscript{13}&2\textsuperscript{14}&2\textsuperscript{15}&2\textsuperscript{0}&2\textsuperscript{1}&2\textsuperscript{2}&2\textsuperscript{3}&2\textsuperscript{4}&2\textsuperscript{5}&2\textsuperscript{6}&2\textsuperscript{7}\\ 
  \cline{2-17}
    &\multicolumn{16}{c|}{2\textsuperscript{0}=1}\\ 
  \hline
  \end{tabular}
\end{table}

\begin{table}
\centering
\begin{tabular}{|l|l|l|l|l|l|l|l|l|l|l|l|l|l|l|l|l|}\hline
\rowcolor{gris-tab-entete}\footnotesize \bf Nombre& \multicolumn{16}{l|}{\footnotesize \bf Représentation petit-boutiste pour les multiplets et petit-boutiste pour les \emph{bits}}\\
 \hline
  \rowcolor{gris-clair-tab}\multirow{4}{2cm}{1} &\multicolumn{8}{c|}{Octet de poids faible}&\multicolumn{8}{c|}{Octet de poids fort}\\
  & 1 & 0 & 0 & 0 & 0 & 0 & 0 & 0& 0 & 0 & 0 & 0 & 0 & 0 & 0 & 0 \\ 
  \cline{2-17}
    &2\textsuperscript{0}&2\textsuperscript{1}&2\textsuperscript{2}&2\textsuperscript{3}&2\textsuperscript{4}&2\textsuperscript{5}&2\textsuperscript{6}&2\textsuperscript{7}&2\textsuperscript{8}&2\textsuperscript{9}&2\textsuperscript{10}&2\textsuperscript{11}&2\textsuperscript{12}&2\textsuperscript{13}&2\textsuperscript{14}&2\textsuperscript{15}\\ 
  \cline{2-17}
     \rowcolor{gris-clair-tab}&\multicolumn{16}{c|}{2\textsuperscript{0}=1}\\ 
  \hline
  \end{tabular}
\end{table}


\begin{infobox}
  Notez toutefois que, le plus souvent,
une représentation petit-boutiste ou gros-boutiste s'applique aussi bien
aux octets qu'aux \emph{bits}.\\
\textbar{} Par ailleurs, sachez qu'à quelques exceptions près, l'ordre
du stockage des \emph{bits} n'apparaît pas en C. En particulier, les
opérateurs de manipulation de \emph{bits}, vus au chapitre suivant, n'en
dépendent pas.
\end{infobox}


\subsection{Applications}
\label{applications-1}

\subsubsection{Connaître le boutisme d'une machine}
\label{connaitre-le-boutisme-dune-machine}

Le plus souvent, il est possible de connaître le boutisme employé pour
les \emph{multiplets} à l'aide du code suivant. Ce dernier stocke la
valeur \mybox{1} dans une variable de type \mybox{unsigned\ short}, et
accède à son premier octet à l'aide d'un pointeur sur
\mybox{unsigned\ char}. Si celui-ci est nul, c'est que la machine est
gros-boutiste, sinon, c'est qu'elle est petit-boutiste.

\begin{C}
#include <stdio.h>


int main(void)
{
    unsigned short n = 1;
    unsigned char *p = (unsigned char *)&n;

    if (*p != 0)
        printf("Je suis petit-boutiste.\n");
    else
        printf("Je suis gros-boutiste.\n");

    return 0;
}
\end{C}

\begin{attentionbox}
  Notez bien que cette technique n'est
pas entièrement fiable. En effet, rappelez-vous : d'une part, les deux
boutismes présentés, s'ils sont les plus fréquents, ne sont pas les
seuls et, d'autre part, la présence de \emph{bits} de bourrage au sein
du type entier, bien que rare, pourrait fausser le résultat.
\end{attentionbox}


\subsubsection{Le boutisme des constantes octales et hexadécimales}
\label{le-boutisme-des-constantes-octales-et-hexadecimales}

Suivant ce qui vient de vous être présenter, peut-être vous êtes vous
posé la question suivante : si les boutismes varient d'une machine à
l'autre, alors que vaut finalement la constante \mybox{0x01} ? En
effet, suivant les boutismes employés, celle-ci peut valoir 1 ou 16.

Heureusement, cette question est réglée par la norme\footnote{Programming
  Language C, X3J11/88-090, § 3.1.3.2, Integer constants, al. 3Pour
  terminer ce chapitre, il nous reste à vous présenter quatre fonctions
  que nous avions passé sous silence lors de notre présentation de
  l'en-tête \mybox{\textless{}string.h\textgreater{}} :
  \mybox{memset()}, \mybox{memcpy()}, \mybox{memmove()} et
  \mybox{memcmp()}. Bien qu'elles soient définies dans cet en-tête, ces
  fonctions ne sont pas véritablement liées aux chaînes de caractères et
  opèrent en vérité sur les multiplets composant les objets. De telles
  modifications impliquant la représentation des types, nous avons
  attendu ce chapitre pour vous en parler.} : le boutisme ne change
\emph{rien} à l'écriture des constantes octales et hexadécimales, elle
est toujours réalisée suivant la convention classique des chiffres de
poids fort en premier. Ainsi, la constante \mybox{0x01} vaut 1 et la
constante \mybox{0x8000} vaut 32768 (en non signé).

De même, les indicateurs de conversion \mybox{x}, \mybox{X} et
\mybox{o} affichent toujours leurs résultats suivant cette écriture.

\begin{C}
printf("%02x\n", 1);
\end{C}

\begin{C}
 01
\end{C}

\section{Les fonctions memset, memcpy, memmove et memcmp}
\label{les-fonctions-memset,-memcpy,-memmove-et-memcmp}

\subsection{La fonction memset}
\label{la-fonction-memset}

\begin{C}
void *memset(void *obj, int val, size_t taille);
\end{C}

La fonction \mybox{memset()} initialise les \mybox{taille} premiers
multiplets de l'objet référencé par \mybox{obj} avec la valeur
\mybox{val} (qui est convertie vers le type \mybox{unsigned\ char}).
Cette fonction est très utile pour (ré)initialiser les différents
éléments d'un aggrégats sans devoir recourir à une boucle.

\begin{C}
#include <stdio.h>
#include <string.h>


int main(void)
{
    int tab[] = { 10, 20, 30, 40, 50 };
    int i;

    memset(tab, 0, sizeof tab);

    for (i = 0; i < sizeof tab / sizeof tab[0]; ++i)
        printf("%d : %d\n", i, tab[i]);

    return 0;
}
\end{C}

\begin{C}
0 : 0
1 : 0
2 : 0
3 : 0
4 : 0
\end{C}

\begin{erreurbox}
  Faites attention ! Manipuler ainsi les
objets \emph{byte} par \emph{byte} nécessite de connaître leur
représentation. Dans un souci de portabilité, on ne devrait donc pas
utiliser \mybox{memset} sur des pointeurs ou des flottants. De même,
pour éviter d'obtenir les représentations problématiques dans le cas
d'entiers signés, il est conseillé de n'employer cette fonction que pour
mettre ces derniers à zéro.
\end{erreurbox}


\subsection{Les fonctions memcpy et memmove}
\label{les-fonctions-memcpy-et-memmove}

\begin{C}
void *memcpy(void *destination, void *source, size_t taille);
void *memmove(void *destination, void *source, size_t taille);
\end{C}

Les fonction \mybox{memcpy()} et \mybox{memmove()} copient toutes deux
les \mybox{taille} premiers multiplets de l'objet pointé par
\mybox{source} vers les premiers multiplets de l'objet référencé par
\mybox{destination}. La différence entre ces deux fonctions est que
\mybox{memcpy()} ne doit être utilisée qu'avec deux objets qui ne se
chevauchent pas (autrement dit, les deux pointeurs ne doivent pas
accéder ou modifier une même zone mémoire ou une partie d'une même zone
mémoire) alors que \mybox{memmove()} ne souffre pas de cette
restriction.

\begin{C}
#include <stdio.h>
#include <string.h>


int main(void)
{
    int n = 10;
    int m;

    memcpy(&m , &n, sizeof n);
    printf("m = n = %d\n", m);

    memmove(&n, ((unsigned char *)&n) + 1, sizeof n - 1);
    printf("n = %d\n", n);
    return 0;
}
\end{C}

\begin{C}
m = n = 10
n = 0
\end{C}

L'utilisation de \mybox{memmove()} amène quelques précisions : nous
copions ici les trois derniers multiplets de l'objet \mybox{n} vers ses
trois premiers multiplets. Étant donné que notre machine est
petit-boutiste, les trois derniers multiplets sont à zéro et la variable
\mybox{n} a donc pour valeur finale zéro.

\subsection{La fonction memcmp}
\label{la-fonction-memcmp}

\begin{C}
int memcmp(void *obj1, void *obj2, size_t taille);
\end{C}

La fonction \mybox{memcmp()} compare les \mybox{taille} premiers
multiplets des objets référencés par \mybox{obj1} et \mybox{obj2} et
retourne un nombre entier inférieur, égal ou supérieur à zéro suivant
que la valeur du dernier multiplet comparé (convertie en
\mybox{unsigned\ char}) de \mybox{obj1} est inférieure, égales ou
supérieure à celle du dernier multiplet comparé de \mybox{obj2}.

\begin{C}
#include <stdio.h>
#include <string.h>


int main(void)
{
    int n = 10;
    int m = 10;
    int o = 20;

    if (memcmp(&n, &m, sizeof n) == 0)
        printf("n et m sont égaux\n");

    if (memcmp(&n, &o, sizeof n) != 0)
        printf("n et o ne sont pas égaux\n");

    return 0;
}
\end{C}

\begin{C}
n et m sont égaux
n et o ne sont pas égaux
\end{C}

\begin{infobox}
  Dit autrement, la fonction
\mybox{memcmp()} retourne zéro si les deux portions comparées ont la
même représentation.
\end{infobox}

\hrulefill

\subsubsection{En résumé}
\label{en-resume}

\begin{enumerate}
\def
\labelenumi{\arabic{enumi}.}
\item
  les entiers non signés sont représentés sous forme d'une somme de
  puissance de deux ;
\item
  il existe trois représentations possibles pour les entiers signés,
  mais la plus fréquente est celle en complément à deux ;
\item
  les types entiers peuvent contenir des \emph{bits} de bourrage
  \emph{sauf} le type \mybox{char} ;
\item
  les types flottants sont représentés sous forme d'un signe, d'un
  exposant et d'une mantisse, mais sont susceptibles d'engendrer des
  approximations et des pertes de précision ;
\item
  la représentation des pointeurs est indéterminée ;
\item
  le boutisme utilisé dépend de la machine cible ;
\item
  les constantes sont toujours écrites suivant la représentation
  gros-boutiste.
\end{enumerate}