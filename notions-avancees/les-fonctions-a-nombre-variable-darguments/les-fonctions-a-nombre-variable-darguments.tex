\PassOptionsToPackage{unicode=true}{hyperref} % options for packages loaded elsewhere
\PassOptionsToPackage{hyphens}{url}
%
\documentclass[]{article}
\usepackage{lmodern}
\usepackage{amssymb,amsmath}
\usepackage{ifxetex,ifluatex}
\usepackage{fixltx2e} % provides \textsubscript
\ifnum 0\ifxetex 1\fi\ifluatex 1\fi=0 % if pdftex
  \usepackage[T1]{fontenc}
  \usepackage[utf8]{inputenc}
  \usepackage{textcomp} % provides euro and other symbols
\else % if luatex or xelatex
  \usepackage{unicode-math}
  \defaultfontfeatures{Ligatures=TeX,Scale=MatchLowercase}
\fi
% use upquote if available, for straight quotes in verbatim environments
\IfFileExists{upquote.sty}{\usepackage{upquote}}{}
% use microtype if available
\IfFileExists{microtype.sty}{%
\usepackage[]{microtype}
\UseMicrotypeSet[protrusion]{basicmath} % disable protrusion for tt fonts
}{}
\IfFileExists{parskip.sty}{%
\usepackage{parskip}
}{% else
\setlength{\parindent}{0pt}
\setlength{\parskip}{6pt plus 2pt minus 1pt}
}
\usepackage{hyperref}
\hypersetup{
            pdfborder={0 0 0},
            breaklinks=true}
\urlstyle{same}  % don't use monospace font for urls
\usepackage{color}
\usepackage{fancyvrb}
\newcommand{\VerbBar}{|}
\newcommand{\VERB}{\Verb[commandchars=\\\{\}]}
\DefineVerbatimEnvironment{Highlighting}{Verbatim}{commandchars=\\\{\}}
% Add ',fontsize=\small' for more characters per line
\newenvironment{Shaded}{}{}
\newcommand{\KeywordTok}[1]{\textcolor[rgb]{0.00,0.44,0.13}{\textbf{{#1}}}}
\newcommand{\DataTypeTok}[1]{\textcolor[rgb]{0.56,0.13,0.00}{{#1}}}
\newcommand{\DecValTok}[1]{\textcolor[rgb]{0.25,0.63,0.44}{{#1}}}
\newcommand{\BaseNTok}[1]{\textcolor[rgb]{0.25,0.63,0.44}{{#1}}}
\newcommand{\FloatTok}[1]{\textcolor[rgb]{0.25,0.63,0.44}{{#1}}}
\newcommand{\ConstantTok}[1]{\textcolor[rgb]{0.53,0.00,0.00}{{#1}}}
\newcommand{\CharTok}[1]{\textcolor[rgb]{0.25,0.44,0.63}{{#1}}}
\newcommand{\SpecialCharTok}[1]{\textcolor[rgb]{0.25,0.44,0.63}{{#1}}}
\newcommand{\StringTok}[1]{\textcolor[rgb]{0.25,0.44,0.63}{{#1}}}
\newcommand{\VerbatimStringTok}[1]{\textcolor[rgb]{0.25,0.44,0.63}{{#1}}}
\newcommand{\SpecialStringTok}[1]{\textcolor[rgb]{0.73,0.40,0.53}{{#1}}}
\newcommand{\ImportTok}[1]{{#1}}
\newcommand{\CommentTok}[1]{\textcolor[rgb]{0.38,0.63,0.69}{\textit{{#1}}}}
\newcommand{\DocumentationTok}[1]{\textcolor[rgb]{0.73,0.13,0.13}{\textit{{#1}}}}
\newcommand{\AnnotationTok}[1]{\textcolor[rgb]{0.38,0.63,0.69}{\textbf{\textit{{#1}}}}}
\newcommand{\CommentVarTok}[1]{\textcolor[rgb]{0.38,0.63,0.69}{\textbf{\textit{{#1}}}}}
\newcommand{\OtherTok}[1]{\textcolor[rgb]{0.00,0.44,0.13}{{#1}}}
\newcommand{\FunctionTok}[1]{\textcolor[rgb]{0.02,0.16,0.49}{{#1}}}
\newcommand{\VariableTok}[1]{\textcolor[rgb]{0.10,0.09,0.49}{{#1}}}
\newcommand{\ControlFlowTok}[1]{\textcolor[rgb]{0.00,0.44,0.13}{\textbf{{#1}}}}
\newcommand{\OperatorTok}[1]{\textcolor[rgb]{0.40,0.40,0.40}{{#1}}}
\newcommand{\BuiltInTok}[1]{{#1}}
\newcommand{\ExtensionTok}[1]{{#1}}
\newcommand{\PreprocessorTok}[1]{\textcolor[rgb]{0.74,0.48,0.00}{{#1}}}
\newcommand{\AttributeTok}[1]{\textcolor[rgb]{0.49,0.56,0.16}{{#1}}}
\newcommand{\RegionMarkerTok}[1]{{#1}}
\newcommand{\InformationTok}[1]{\textcolor[rgb]{0.38,0.63,0.69}{\textbf{\textit{{#1}}}}}
\newcommand{\WarningTok}[1]{\textcolor[rgb]{0.38,0.63,0.69}{\textbf{\textit{{#1}}}}}
\newcommand{\AlertTok}[1]{\textcolor[rgb]{1.00,0.00,0.00}{\textbf{{#1}}}}
\newcommand{\ErrorTok}[1]{\textcolor[rgb]{1.00,0.00,0.00}{\textbf{{#1}}}}
\newcommand{\NormalTok}[1]{{#1}}
\setlength{\emergencystretch}{3em}  % prevent overfull lines
\providecommand{\tightlist}{%
  \setlength{\itemsep}{0pt}\setlength{\parskip}{0pt}}
\setcounter{secnumdepth}{5}
% Redefines (sub)paragraphs to behave more like sections
\ifx\paragraph\undefined\else
\let\oldparagraph\paragraph
\renewcommand{\paragraph}[1]{\oldparagraph{#1}\mbox{}}
\fi
\ifx\subparagraph\undefined\else
\let\oldsubparagraph\subparagraph
\renewcommand{\subparagraph}[1]{\oldsubparagraph{#1}\mbox{}}
\fi

% set default figure placement to htbp
\makeatletter
\def\fps@figure{htbp}
\makeatother


\date{}

\begin{document}

{
\setcounter{tocdepth}{3}
\tableofcontents
}
De manière générale, il existe trois grandes méthodes pour gérer le
nombre et le type des arguments optionnels.

\section{Les chaînes de formats}\label{les-chauxeenes-de-formats}

Cette méthode, vous la connaissez déjà, il s'agit de celle employée par
les fonctions \texttt{printf()}, \texttt{scanf()} et consœurs. Elle
consiste à décrire le nombre et le type des arguments à l'aide d'une
chaîne de caractères comprenant des indicateurs.

\section{Les suites d'arguments de même
type}\label{les-suites-darguments-de-muxeame-type}

La seconde solution ne peut être utilisée que si tous les arguments
optionnels sont de même type. Elle consiste soit à indiquer le nombre
d'arguments transmis, soit à utiliser un délimitateur. La fonction
\texttt{affiche\_suite()} recourt par exemple à un paramètre pour
déterminer le nombre de paramètres optionnels qu'elle a reçu.

Un délimitateur pourrait par exemple être un pointeur nul dans le cas
d'une fonction similaire affichant une suite de chaîne de caractères.

\begin{Shaded}
\begin{Highlighting}[]
\PreprocessorTok{#include }\ImportTok{<stdarg.h>}
\PreprocessorTok{#include }\ImportTok{<stddef.h>}
\PreprocessorTok{#include }\ImportTok{<stdio.h>}


\DataTypeTok{static} \DataTypeTok{void} \NormalTok{affiche_suite(}\DataTypeTok{char} \NormalTok{*chaine, ...)}
\NormalTok{\{}
    \NormalTok{va_list ap;}

    \NormalTok{va_start(ap, chaine);}

    \ControlFlowTok{do}
    \NormalTok{\{}
        \NormalTok{puts(chaine);}
        \NormalTok{chaine = va_arg(ap, }\DataTypeTok{char} \NormalTok{*);}
    \NormalTok{\} }\ControlFlowTok{while}\NormalTok{(chaine != NULL);}

    \NormalTok{va_end(ap);}
\NormalTok{\}}


\DataTypeTok{int} \NormalTok{main(}\DataTypeTok{void}\NormalTok{)}
\NormalTok{\{}
    \NormalTok{affiche_suite(}\StringTok{"un"}\NormalTok{, }\StringTok{"deux"}\NormalTok{, }\StringTok{"trois"}\NormalTok{, (}\DataTypeTok{char} \NormalTok{*)}\DecValTok{0}\NormalTok{);}
    \ControlFlowTok{return} \DecValTok{0}\NormalTok{;}
\NormalTok{\}}
\end{Highlighting}
\end{Shaded}

\begin{verbatim}
un
deux
trois
\end{verbatim}

\section{Emploi d'un pivot}\label{emploi-dun-pivot}

La dernière pratique consiste à recourir à un paramètre « pivot » qui
fera varier le parcours des paramètres optionnels en fonction de sa
valeur. Notez qu'un type énuméré se prête très bien pour ce genre de
paramètre. Par exemple, la fonction suivante affiche un \texttt{int} ou
un \texttt{double} suivant la valeur du paramètre \texttt{type}.

\begin{Shaded}
\begin{Highlighting}[]
\PreprocessorTok{#include }\ImportTok{<stdarg.h>}
\PreprocessorTok{#include }\ImportTok{<stdio.h>}

\KeywordTok{enum} \NormalTok{type \{ TYPE_INT, TYPE_DOUBLE \};}


\DataTypeTok{static} \DataTypeTok{void} \NormalTok{affiche(}\KeywordTok{enum} \NormalTok{type type, ...)}
\NormalTok{\{}
    \NormalTok{va_list ap;}

    \NormalTok{va_start(ap, type);}

    \ControlFlowTok{switch} \NormalTok{(type)}
    \NormalTok{\{}
    \ControlFlowTok{case} \NormalTok{TYPE_INT:}
        \NormalTok{printf(}\StringTok{"Un entier : %d.}\SpecialCharTok{\textbackslash{}n}\StringTok{"}\NormalTok{, va_arg(ap, }\DataTypeTok{int}\NormalTok{));}
        \ControlFlowTok{break}\NormalTok{;}

    \ControlFlowTok{case} \NormalTok{TYPE_DOUBLE:}
        \NormalTok{printf(}\StringTok{"Un flottant : %f.}\SpecialCharTok{\textbackslash{}n}\StringTok{"}\NormalTok{, va_arg(ap, }\DataTypeTok{double}\NormalTok{));}
        \ControlFlowTok{break}\NormalTok{;}
    \NormalTok{\}}

    \NormalTok{va_end(ap);}
\NormalTok{\}}


\DataTypeTok{int} \NormalTok{main(}\DataTypeTok{void}\NormalTok{)}
\NormalTok{\{}
    \NormalTok{affiche(TYPE_INT, }\DecValTok{10}\NormalTok{);}
    \NormalTok{affiche(TYPE_DOUBLE, }\FloatTok{3.14}\NormalTok{);}
    \ControlFlowTok{return} \DecValTok{0}\NormalTok{;}
\NormalTok{\}}
\end{Highlighting}
\end{Shaded}

\texttt{text\ Un\ entier\ :\ 10.\ Un\ flottant\ :\ 3.140000.}Une
fonction à nombre variable d'arguments est, comme son nom l'indique, une
fonction capable de recevoir et de manipuler un nombre variable
d'arguments, mais en plus, et cela son nom ne le spécifie pas,
potentiellement de types différents.

Une telle fonction se définit en employant comme \emph{dernier}
paramètre une suite de trois points appelée une ellipse indiquant que
des arguments supplémentaires peuvent être transmis.

\begin{Shaded}
\begin{Highlighting}[]
\DataTypeTok{void} \NormalTok{affiche_suite(}\DataTypeTok{int} \NormalTok{n, ...);}
\end{Highlighting}
\end{Shaded}

Le prototype ci-dessus déclare une fonction recevant un \texttt{int} et,
éventuellement, un nombre indéterminé d'autres arguments. La partie
précédant l'ellipse décrit donc les paramètres attendus par la fonction
et qui, comme pour n'importe quelle autre fonction, \emph{doivent} lui
être transmis.

{[}{[}erreur{]}{]} \textbar{} L'ellipse ne peut être placée qu'à la fin
de la liste des paramètres.

Une fois définie, la fonction peut être appelée en lui fournissant zéro
ou plusieurs arguments supplémentaires.

\begin{Shaded}
\begin{Highlighting}[]
\NormalTok{affiche_suite(}\DecValTok{10}\NormalTok{, }\DecValTok{20}\NormalTok{);}
\NormalTok{affiche_suite(}\DecValTok{10}\NormalTok{, }\DecValTok{20}\NormalTok{, }\DecValTok{30}\NormalTok{, }\DecValTok{40}\NormalTok{, }\DecValTok{50}\NormalTok{, }\DecValTok{60}\NormalTok{, }\DecValTok{70}\NormalTok{, }\DecValTok{80}\NormalTok{, }\DecValTok{90}\NormalTok{, }\DecValTok{100}\NormalTok{);}
\NormalTok{affiche_suite(}\DecValTok{10}\NormalTok{);}
\end{Highlighting}
\end{Shaded}

{[}{[}attention{]}{]} \textbar{} Comme pour les pointeurs génériques de
fonction, le nombre et le types des arguments est inconnu du compilateur
(c'est d'ailleurs bien le but de la manœuvre). Dès lors, les même règles
de promotion s'appliquent ainsi que les problèmes qui y sont inhérent,
notamment la problématique des pointeurs nuls. N'employer donc pas la
macroconstante \texttt{NULL} pour fournir un pointeur nul comme argument
optionnel.Une fois le prototype de la fonction déterminé, encore faut-il
pouvoir manipuler ces arguments supplémentaires au sein de sa
définition.Pour ce faire, la bibliothèque standard nous fourni trois
macrofonctions : \texttt{va\_start()}, \texttt{va\_arg()} et
\texttt{va\_end()} définie dans l'ent-tête
\texttt{\textless{}stdarg.h\textgreater{}}. Ces trois macrofonctions
attendent comme premier argument une variable de type \texttt{va\_list}
définit dans le même en-tête.

Afin d'illustrer leur fonctionnement, nous vous proposons directement un
exemple mettant en œuvre une fonction \texttt{affiche\_suite()} qui
reçoit comme premier paramètre le nombre d'entiers qui vont lui être
transmis et les affiche ensuite tous, un par ligne.

\begin{Shaded}
\begin{Highlighting}[]
\PreprocessorTok{#include }\ImportTok{<stdarg.h>}
\PreprocessorTok{#include }\ImportTok{<stdio.h>}


\DataTypeTok{void} \NormalTok{affiche_suite(}\DataTypeTok{int} \NormalTok{nb, ...)}
\NormalTok{\{}
    \NormalTok{va_list ap;}

    \NormalTok{va_start(ap, nb);}

    \ControlFlowTok{while} \NormalTok{(nb > }\DecValTok{0}\NormalTok{)}
    \NormalTok{\{}
        \DataTypeTok{int} \NormalTok{n;}

        \NormalTok{n = va_arg(ap, }\DataTypeTok{int}\NormalTok{);}
        \NormalTok{printf(}\StringTok{"%d.}\SpecialCharTok{\textbackslash{}n}\StringTok{"}\NormalTok{, n);}
        \NormalTok{--nb;}
    \NormalTok{\}}

    \NormalTok{va_end(ap);}
\NormalTok{\}}


\DataTypeTok{int} \NormalTok{main(}\DataTypeTok{void}\NormalTok{)}
\NormalTok{\{}
    \NormalTok{affiche_suite(}\DecValTok{10}\NormalTok{, }\DecValTok{10}\NormalTok{, }\DecValTok{20}\NormalTok{, }\DecValTok{30}\NormalTok{, }\DecValTok{40}\NormalTok{, }\DecValTok{50}\NormalTok{, }\DecValTok{60}\NormalTok{, }\DecValTok{70}\NormalTok{, }\DecValTok{80}\NormalTok{, }\DecValTok{90}\NormalTok{, }\DecValTok{100}\NormalTok{);}
    \ControlFlowTok{return} \DecValTok{0}\NormalTok{;}
\NormalTok{\}}
\end{Highlighting}
\end{Shaded}

\begin{verbatim}
10.
20.
30.
40.
50.
60.
70.
80.
90.
100.
\end{verbatim}

La macrofonction \texttt{va\_start} initialise le parcours des
paramètres optionnels. Elle attend deux arguments : une variable de type
\texttt{va\_list} et le nom du dernier paramètre obligatoire de la
fonction courante (dans notre cas \texttt{nb}). Il est impératif de
l'appeler avant toute opération sur les paramètres optionnels.

La marcofonction \texttt{va\_arg()} retourne le paramètre optionnel
suivant en considérant celui-ci comme de type \texttt{type}. Elle attend
deux arguments : une variable de type \texttt{va\_list} précédemment
initialisée par la macrofonction \texttt{va\_start()} et le type du
paramètre optionnel suivant.

Enfin, la fonction \texttt{va\_end()} met fin au parcours des arguments
optionnels. Elle doit \emph{toujours} être appelée après un appel à la
macrofonction \texttt{va\_start()}.

{[}{[}erreur{]}{]} \textbar{} La macrofonction \texttt{va\_arg()}
n'effectue \emph{aucune} vérification ! Cela signifie que si vous
renseigner un type qui ne correspond pas au paramètre optionnel suivant
ou si vous tentez de récupérer un paramètre optionnel qui n'existe pas,
vous allez au devant de comportements indéfinis.

Étant donné cet état de fait, il est impératif de pouvoir déterminer le
nombre d'arguments optionnels envoyé à la fonction. Dans notre cas, nous
avons opté pour l'emploi d'un paramètre obligatoire indiquant le nombre
d'arguments optionnels envoyés.

{[}{[}question{]}{]} \textbar{} Mais ?! Et si l'utilisateur de la
fonction se plante en ne précisant pas le bon nombre ou le bon type
d'arguments ?

\emph{Eh} bien\ldots{} C'est foutu. :-°\\
Il s'agit là d'une lacune similaire à celle des pointeurs « génériques »
de fonction : étant donné que le type et le nombre des arguments
optionnels est inconnu, le compilateur ne peut effectuer aucune
vérification ou conversion.\#\#\# En résumé

\begin{enumerate}
\def\labelenumi{\arabic{enumi}.}
\tightlist
\item
  Une fonction à nombre variable d'arguments est capable de recevoir et
  de manipuler une suite indéterminée d'arguments optionnels de types
  potentiellement différents.
\item
  L'ellipse ne peut être placée qu'à la fin de la liste des paramètres
  et doit impérativement être précédée d'un paramètre non optionnel.
\item
  La macrofonction \texttt{va\_arg()} n'effectue \emph{aucune}
  vérification, il est donc impératif de contrôler le nombre et le types
  des arguments reçus, par exemple à l'aide d'une chaîne de formats ou
  d'un paramètre décrivant le nombre d'arguments transmis.Dans le
  chapitre précédent, il a été question des pointeurs de fonction et,
  notamment, des pointeurs « génériques » de fonction. Néanmoins, ces
  derniers ne permettent pas d'expliquer le fonctionnement des fonctions
  \texttt{printf()} et \texttt{scanf()}. C'est ce que nous allons voir
  dans ce chapitre en étudiant les \textbf{fonctions à nombre variable
  d'arguments}.
\end{enumerate}

\end{document}
