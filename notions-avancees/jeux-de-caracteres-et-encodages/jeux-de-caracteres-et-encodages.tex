\chapter{Jeux de caractères et encodages}
\label{jeux-de-caracteres-et-encodages}

  Lorsque nous vous   avons présenté les chaînes de caractères, nous vous 
  avons précisé que celle-ci étaient des tableaux de \mybox{char} terminé 
  par un caractère nul, en vous laissant sous-entendre qu'à chaque 
  « caractère » (\mybox{\textquotesingle{}a\textquotesingle{}},
  \mybox{\textquotesingle{};\textquotesingle{}},
  \mybox{\textquotesingle{}1\textquotesingle{}}, etc.) correspondait un
  \mybox{char}. Toutefois, ce n'est pas tout à fait exacte ou, plus
  précisément, ce n'est pas toujours vrai.
  
  Dans ce chapitre, nous allons découvrir comment les chaînes de
caractères sont réellement représentées et pour quelles raisons cette
représentation est susceptible de varier.

\section{Les jeux de caractères et les encodages}
\label{les-jeux-de-caracteres-et-les-encodages}

\subsection{Introduction}
\label{introduction-}

\begin{infobox}
  Avant de poursuivre ce chapitre, nous
vous invitons à lire au moins les deux premiers chapitres du
\MYhref{https://zestedesavoir.com/tutoriels/1114/comprendre-les-encodages/}{cours
de Maëlan sur les encodages}. Ceux-ci constitueront une base solide sur
laquelle nous nous appuierons dans la suite.
\end{infobox}


\subsection{Ce que dit la norme}
\label{ce-que-dit-la-norme}

Maintenant que les notions de jeux de caractères et d'encodages vous
sont connues, voyons comment celles-ci s'agencent en C. La norme définit
deux jeux de caractères\footnote{\footnotesize{Programming Language C, X3J11/88-090, §
  2.2.1, Character sets}} :

\begin{enumerate}
\def
\labelenumi{\arabic{enumi}.}
\item
  Le jeu de caractères source qui, comme son nom ne l'indique pas,
  correspond au jeu de caractères vers lequel votre fichier source va
  être converti. Il ne s'agit donc pas de l'encodage de votre fichier
  source (que vous déterminez à l'aide de votre éditeur de texte), mais
  d'un encodage interne au compilateur.
\item
  Le jeu de caractères d'exécution qui correspond à celui utilisé par
  votre système (il est par exemple utilisé par votre console) vers
  lequel votre programme sera finalement traduit. Ce dernier dépend de
  la \emph{locale} employée (nous y reviendrons un peu plus tard dans ce
  chapitre).
\end{enumerate}

\begin{infobox}
  Autrement dit, il y a possiblement
deux conversions lors de la compilation : une qui a lieu du jeu de
caractères de vos fichiers sources vers le jeu employé en interne par le
compilateur et une depuis le jeu de caractères du compilateur vers celui
du système.
\end{infobox}


\subsection{Caractères garantis}
\label{caracteres-garantis}

La norme précise que ces deux jeux comprennent au minimum les caractères
suivants.

\begin{C}
A  B  C  D  E  F  G  H  I  J  K  L  M
N  O  P  Q  R  S  T  U  V  W  X  Y  Z
a  b  c  d  e  f  g  h  i  j  k  l  m
n  o  p  q  r  s  t  u  v  w  x  y  z
0  1  2  3  4  5  6  7  8  9
!  "  #  %  &  '  (  )  *  +  ,  -  .  /  :
;  <  =  >  ?  [  \  ]  ^  _  {  |  }  ~
\end{C}

À ceux-ci s'ajoutent l'espace, les tabulations horizontales et
verticales et le saut de page.

Il est également précisé que les points de codes des dix chiffres
(\mybox{\textquotesingle{}0\textquotesingle{}},
\mybox{\textquotesingle{}1\textquotesingle{}},
\mybox{\textquotesingle{}2\textquotesingle{}},
\mybox{\textquotesingle{}3\textquotesingle{}},
\mybox{\textquotesingle{}4\textquotesingle{}},
\mybox{\textquotesingle{}5\textquotesingle{}},
\mybox{\textquotesingle{}6\textquotesingle{}},
\mybox{\textquotesingle{}7\textquotesingle{}},
\mybox{\textquotesingle{}8\textquotesingle{}} et
\mybox{\textquotesingle{}9\textquotesingle{}}) doivent se suivre de
manière croissante.

\subsubsection{Jeu de caractère source}
\label{jeu-de-caractere-source}

De plus, le jeu de caractères source doit comprendre un ou plusieurs
caractères permettant d'indiquer la fin d'une ligne de texte.

\subsubsection{Jeu de caractère d'exécution}
\label{jeu-de-caractere-dexecution}

Enfin, le jeu de caractères d'exécution doit comprendre le caractère
nul, le caractère d'appel, l'espacement arrière, le retour chariot et le
saut de ligne.

\begin{questionbox}
   Certes, mais encore ?\\
C'est super cette description détaillée, mais ça m'apporte
quoi de savoir ça ? J'en fais quoi, moi, de vos deux jeux ?
\end{questionbox}


Pour résumer, la norme nous décrit ici les caractères qui peuvent être
employés dans vos fichiers sources et les caractères que votre système
doit supporter au minimum.

Ceci est impératif pour assurer d'une part la compilation de vos
programmes et la bonne exécution de ceux-ci. En effet, imaginer par
exemple que le compilateur utilise en interne un jeu de caractères ne
comprenant pas le caractère \mybox{p}, vous serez bien ennuyer ensuite
pour faire appel à \mybox{printf()}\ldots{} De même, il serait fort
gênant que votre console ne sache pas afficher les retours à la ligne.

Toutefois, cela a également une seconde conséquence : si vous utilisez
un autre caractère que ceux énumérés ci-dessus (par exemple un « e »
accent, un caractère cyrillique ou un idéogramme japonais), la norme ne
vous garantit pas d'une part que la compilation réussisse (la conversion
du jeu utilisé par vos fichiers sources vers celui du compilateur
pourrait par exemple échouer) et, d'autre part, que ceux-ci seront
supportés par votre système (si ce n'est pas le cas, cela se traduira le
plus souvent par un affichage chaotique).

Aussi, dans un soucis de portabilité, il est nécessaire de se contenter
le plus souvent de ces derniers, ce qui exclut donc l'emploi (ou à tout
le moins l'emploi correct) de la plupart des langues du monde à
l'exception de l'anglais et du latin.

\begin{infobox}
  \begin{enumerate}  
    \item
    Cette restriction doit toutefois être relativisée puisque la plupart 
    des compilateurs utilisent l'UTF-8 en interne, de même que les systèmes 
    d'exploitations modernes. De plus, cette restriction n'a pas d'objet pour 
    les commentaires puisque ceux-ci sont ignorés lors de la compilation.
    \item
    Par ailleurs, sachez
    qu'il existe pas mal de solutions permettant de contourner cette
    limitation. À ce sujet, si cela vous intéresse, nous vous recommendons
    \MYhref{https://www.gnu.org/software/gettext/manual/html_node/index.html}{GNU
    gettext} qui est une solution libre et gratuite très utilisée sous
    GNU/Linux et *BSD. 
  \end{enumerate}
\end{infobox}


\begin{attentionbox}
 Les plus attentifs (ou fourbes, c'est
selon) d'entre-vous se rappelerons sans doute que plusieurs des
codes présentés dans ce cours comportent des caractères accentués
et\ldots{} oui, nous n'avons pas respecté la norme sur ce point. Mais
bon, un exemple n'est-il pas plus agréable avec ?
\end{attentionbox}

\section{Les caractères larges}
\label{les-caractères-larges}

\subsection{Mise en situation}
\label{mise-en-situation-2}

Si nous devons tâcher de respecter la norme en n'employant pas de
caractères en dehors de ceux garantis, cela n'est pas vrai pour les
utilisateurs de nos programmes qui, eux, ne se gêneront pas pour
utiliser ceux supportés par leur système (c'est d'ailleurs bien là
l'intérêt de choisir la langue de son système).

Or, jusqu'à présent, nous sommes toujours partis du principe que les
caractères entrés tenaient sur un seul \mybox{char}, ce qui n'est pas
toujours vrai, comme vous avez pu le voir au travers du cours de Maëlan.

En fait, il s'agit du comportement \emph{par défaut} des fonctions de la
bibliothèque standard. Chaque \mybox{char} est supposé représenter un
caractère et une chaîne de caractères est censée n'être qu'une suite de
\mybox{char} finie par un zéro.

Il est possible de s'en rendre compte à l'aide du code suivant.

\begin{C}
#include <stdio.h>
#include <stdlib.h>
#include <string.h>


int main(void)
{
    char chaine[255];
    char *nl;

    if (fgets(chaine, sizeof chaine, stdin) == NULL)
    {
        perror("fgets");
        return EXIT_FAILURE;
    }

    nl = strchr(chaine, '\n');

    if (nl != NULL)
        *nl = '\0';

    printf("Longueur : %u\n", (unsigned)strlen(chaine));
    return 0;
}
\end{C}

\begin{C}
Bonjour   
Longueur : 7

Élégament trouvé
Longueur : 19
\end{C}

Comme vous le voyez, la taille de la chaîne « Élégament trouvé » n'est
pas celle attendue dans notre cas (notre exemple emploie l'UTF-8 comme
encodage) car ce sont les multiplets (les \mybox{char}, donc) qui ont
été comptés et non les caractères. La bonne réponse aurait dû être 16.

\subsection{Les caractères larges}
\label{les-caracteres-larges}

Afin de résoudre ce problème, la norme C89 a introduit les
\textbf{caractères larges}. Pour ce faire, un nouveau type a été
introduit : le type \mybox{wchar\_t} (pour \emph{\textbf{w}ide
\textbf{c}haracter}), défini dans l'en-tête
\mybox{\textless{}stddef.h\textgreater{}}. Celui-ci n'est rien d'autre
qu'un type entier (signé ou non signé) capable de représenter le point
de code le plus élevé supporté par le système.

L'objectif recherché est de traduire une chaîne de caractères classique
recourant à un encodage avec un nombre variable de multiplets (comme
l'UTF-8 ou l'ISO-2022) vers une chaîne de caractères larges dont chacun
représentera exactement un caractère. Dans le cas de chaînes de
caractères en UTF-8 par exemple, celles-ci seront le plus souvent
converties en UTF-16 ou en UTF-32.

À cet effet, plusieurs fonctions de conversions sont mises à notre
disposition et sont définies dans l'en-tête
\mybox{\textless{}stdlib.h\textgreater{}}. Toutefois, aucune fonction
de traitement de ces chaînes n'est fournie, c'est-à-dire que celles-ci
doivent être manipulées « à la main » (il n'y a donc par exemple pas de
fonction du type \mybox{strlen()} qui manipule une chaîne de
\mybox{wchar\_t}).

\begin{infobox}
  En vérité, les normes suivantes du
langage C (à commencer par un amendement adopté en 1994) ont ajouté des
fonctions de traitements des chaînes de caractères larges ainsi que
d'autres fonctions de conversions dites « réentrantes » (c'est-à-dire
qui peuvent être appelées simultanément par plusieurs fils d'exécutions
ou \emph{threads} en anglais). Toutefois, nous ne les aborderons pas
dans ce cours, d'une part parce que celui-ci se fonde sur la norme C89
et, d'autre part, parce que leur présentation mériterait plusieurs
chapitres à elle seule.
\end{infobox}


\subsubsection{Les fonctions mbtowc et wctomb}
\label{les-fonctions-mbtowc-et-wctomb}

\begin{C}
int mbtowc(wchar_t *destination, const char *chaine, size_t max);
int wctomb(char *destination, wchar_t source);
\end{C}

La fonction \mybox{mbtowc()} (pour \emph{\textbf{m}ulti\textbf{b}yte
character \textbf{to} \textbf{w}ide \textbf{c}haracter}) convertit une
suite de multiplets en un caractère large (qu'elle stocke dans l'objet
référencé par \mybox{destination}) en lisant au plus \mybox{max}
multiplets depuis la chaîne \mybox{source}. Elle retourne le nombre de
multiplets utilisés pour produire le caractère large ou \mybox{-1} en
cas d'erreur.

La fonction \mybox{wctomb()} (pour \emph{\textbf{w}ide
\textbf{c}haracter \textbf{to} \textbf{m}ulti\textbf{b}yte character})
effectue l'opération inverse : elle convertit le caractère large
\mybox{source} en une suite de multiplets qui sera stockée dans le
tableau \mybox{destination}. Elle retourne le nombre de multiplets
produits en cas de succès et \mybox{-1} en cas d'erreur.

\textbf{La fonction mbtowc}
\label{la-fonction-mbtowc}

L'exemple ci-dessous lit une ligne depuis l'entrée standard et convertit
la première suite de multiplets représentant un caractère en un
caractère large.

\begin{C}
#include <locale.h>
#include <stddef.h>
#include <stdio.h>
#include <stdlib.h>
#include <string.h>


int main(void)
{
    char chaine[255];
    wchar_t wc;
    char *nl;
    int n;

    if (setlocale(LC_CTYPE, "") == NULL)
    {
        perror("setlocale");
        return EXIT_FAILURE;
    }
    if (fgets(chaine, sizeof chaine, stdin) == NULL)
    {
        perror("fgets");
        return EXIT_FAILURE;
    }

    nl = strchr(chaine, '\n');

    if (nl != NULL)
        *nl = '\0';

    n = mbtowc(&wc, chaine, MB_CUR_MAX);

    if (n <= 0)
    {
        perror("mbtowc");
        return EXIT_FAILURE;
    }


    printf("%d multiplet(s) a(ont) été lu pour produire la valeur %u.\n", \
    n, (unsigned)wc);
    return 0;
}
\end{C}

\begin{C}
Élégamment trouvé
2 multiplet(s) a(ont) été lu pour produire la valeur 201.

ASCII
1 multiplet(s) a(ont) été lu pour produire la valeur 65.
\end{C}

\begin{infobox}
  Les résultats obtenus dépendent bien
entendu du jeu de caractères utilisé par votre système. Si ce dernier
n'emploie pas l'Unicode ou n'utilise pas l'UTF-8 comme encodage, la
sortie du programme peut être différente.
\end{infobox}


Comme vous le voyez, étant donné que notre système emploie de l'UTF-8,
deux multiplets ont été lus depuis la chaîne \mybox{chaine} pour
construire la valeur du caractère \mybox{É}, soit 201 (qui correspond à
son point de code dans le jeu de caractères Unicode). Le caractère
\mybox{A} étant quant à lui représenté sur un seul multiplet en UTF-8,
la lecture d'un seul suffit pour obtenir sa valeur, à savoir
\mybox{65}.

L'en-tête \mybox{\textless{}locale.h\textgreater{}} a été ajouté en vue
d'utiliser la fonction \mybox{setlocale()} dont nous parlerons dans la
prochaine section. Sachez pour l'instant qu'elle doit être appelée
\emph{avant} d'utiliser les fonctions de conversions.

\mybox{MB\_CUR\_MAX} est une macro (elle est éfinie dans l'en-tête
\mybox{\textless{}stdlib.h\textgreater{}}) dont la valeur est
déterminée par la \emph{locale} courante (nous y reviendrons lorsque
nous aborderons la fonction \mybox{setlocale()}) et correspond au
nombre maximum de multiplets nécessaires pour construire un caractère
large \emph{dans la} locale \emph{actuelle}.

\textbf{La fonction wctomb}
\label{la-fonction-wctomb}

Comme nous vous l'avons dit, la fonction \mybox{wctomb()} effectue
l'exact inverse de la fonction \mybox{mbtowc()}. L'exemple suivant
tente de convertir le caractère large \mybox{É} en la suite de
multiplets correspondante.

\begin{C}
#include <limits.h>
#include <locale.h>
#include <stddef.h>
#include <stdio.h>
#include <stdlib.h>


int main(void)
{
    char tab[MB_LEN_MAX];
    int i;
    int n;

    if (setlocale(LC_CTYPE, "") == NULL)
    {
        perror("setlocale");
        return EXIT_FAILURE;
    }

    n = wctomb(tab, L'É');

    if (n <= 0)
    {
        perror("wctomb");
        return EXIT_FAILURE;
    }

    for (i = 0; i < n; ++i)
        printf("%x ", (unsigned char)tab[i]);

    putchar('\n');
    return 0;
}
\end{C}

\begin{C}
c3 89
\end{C}

Ici, nous essayons de convertir le caractère large \mybox{É} (notez
qu'une constante de type caractère large peut être définie en la faisant
précédé de la lettre \mybox{l} ou \mybox{L}) en une suite de
multiplets qui sera stockée dans le tableau \mybox{tab}.

La taille du tableau \mybox{tab} a été fixée à \mybox{MB\_LEN\_MAX},
une macroconstante définie dans l'en-tête
\mybox{\textless{}limits.h\textgreater{}} correspondant à la plus
grande suite de multiplets pouvant représenter un caractère sur le
système. Notez bien la différence avec la macro \mybox{MB\_CUR\_MAX}
qui se limite à la \emph{locale} courante. Par ailleurs, la valeur de
\mybox{MB\_CUR\_MAX} dépendant des appels à la fonction
\mybox{setlocale()}, elle ne peut être utilisée pour déterminer la
taille d'un tableau lors de sa définition.

Une fois la conversion effectuée, nous affichons la valeur des
différents multiplets en hexadécimal (remarquez la conversion en
\mybox{unsigned\ char} afin d'éviter l'affichage de nombres négatifs).

\begin{attentionbox}
  Dans le cas où vous insérez une
constante de type caractère large dans votre code source comme
\mybox{L\textquotesingle{}É\textquotesingle{}}, les mêmes restrictions
s'appliquent que pour les caractères simples : si le caractère en
question ne fait pas partie du jeu de caractères source ou d'exécution,
le résultat n'est pas déterminé par la norme. Une telle pratique est
donc à proscrire également, pour les mêmes motifs que précédemment.
\end{attentionbox}


\subsubsection{Convertir une chaîne complète}
\label{convertir-une-chaine-complete}

Bien entendu, ces deux fonctions peuvent être utilisées en combinaison
avec une boucle en vue de convertir une chaîne de caractères entière
(c'est même là tout l'intérêt de la chose).

\begin{C}
#include <locale.h>
#include <stddef.h>
#include <stdio.h>
#include <stdlib.h>
#include <string.h>


int main(void)
{
    char mbs[255];
    wchar_t wcs[255];
    char *nl;
    char *pc;
    int n;
    int i;

    if (setlocale(LC_CTYPE, "") == NULL)
    {
        perror("setlocale");
        return EXIT_FAILURE;
    }
    if (fgets(mbs, sizeof mbs, stdin) == NULL)
    {
        perror("fgets");
        return EXIT_FAILURE;
    }

    nl = strchr(mbs, '\n');

    if (nl != NULL)
        *nl = '\0';

    pc = mbs;

    for (i = 0; (n = mbtowc(&wcs[i], pc, MB_CUR_MAX)) > 0; ++i)
    {
        if (*pc == '\0')
            break;

        pc += n;
    }

    if (*pc != '\0')
    {
        perror("mbtowc");
        return EXIT_FAILURE;
    }

    for (i = 0; wcs[i] != L'\0'; ++i)
        printf("%u ", (unsigned)wcs[i]);

    putchar('\n');
    return 0;
}
\end{C}

\begin{C}
Élégament trouvé                   
201 108 233 103 97 109 101 110 116 32 116 114 111 117 118 233 
\end{C}

Comme vous le voyez, nous appelons la fonction \mybox{mbtowc()} tant
que celle-ci ne rencontre pas une erreur ou que nous ne rencontrons pas
le caractère de fin de chaîne (ce dernier devant également être
converti, cette seconde condition est placée au sein de la boucle).
Ensuite, suivant le nombre de multiplets lus par \mybox{mbtowc()}, nous
augmentons la valeur du pointeur \mybox{pc} afin de référencer les
prochains caractères à lire. À la sortie de la boucle, nous vérifions
que le pointeur \mybox{pc} pointe bien vers le caractère nul sans quoi
cela signifie que la fonction \mybox{mbtowc()} a rencontré une erreur.
Enfin, nous parcourons la chaînes large \mybox{wcs} pour afficher les
différents points de code des caractères la composant.

\begin{infobox}
  Notez que comme la comparaison
\mybox{wcs{[}i{]}\ !=\ L\textquotesingle{}\textbackslash{}0\textquotesingle{}}
porte sur le caractère large \mybox{wcs{[}i{]}}, nous avons fait du
second opérande un caractère large également. Le caractère nul étant
garanti de faire partie du jeu de caractères d'exécution, cela ne pose
pas de problèmes
\end{infobox}
.

\subsubsection{Les fonction mbstowcs et wcstombs}
\label{les-fonction-mbstowcs-et-wcstombs}

\begin{C}
size_t mbstowcs(wchar_t *destination, char *source, size_t max);
size_t wcstombs(char *destination, wchar_t *source, size_t max);
\end{C}

C'est chouette de pouvoir employer des boucles, mais cela reste assez
fastidieux\ldots{} Heureusement pour nous, il existe des fonctions qui
se chargent de le faire pour nous. \^{}\^{}\\
Les fonctions \mybox{mbstowcs()} et \mybox{wcstombs()} convertissent
une chaîne de caractères vers une chaîne de caractères larges et
inversement. Elles stockent les \mybox{max} premiers caractères
produits dans la chaîne \mybox{destination}.

Les deux fonctions retournent le nombre de caractères convertis (hors
caractère nul final) ou \mybox{(size\_t)-1} en cas d'erreurs.

Ci-dessous, un exemple de conversion recourant à la fonction
\mybox{mbstowcs()} qui nous permet de rendre notre programme initial
correct. Notez que nous avons dû réaliser notre propre version de
\mybox{strlen()} afin de calculer la longueur de la chaîne de
caractères larges obtenues.

\begin{C}
#include <locale.h>
#include <stdio.h>
#include <stdlib.h>


static size_t
wchar_len(wchar_t *wcs)
{
    size_t i = 0;

    while (wcs[i] != L'\0')
        ++i;

    return i;
}


int main(void)
{
    wchar_t wcs[255];
    char mbs[255];
    char *nl;

    if (setlocale(LC_CTYPE, "") == NULL)
    {
        perror("setlocale");
        return EXIT_FAILURE;
    }
    if (fgets(mbs, sizeof mbs, stdin) == NULL)
    {
        perror("fgets");
        return EXIT_FAILURE;
    }

    nl = strchr(mbs, '\n');

    if (nl != NULL)
        *nl = '\0';

    if (mbstowcs(wcs, mbs, sizeof mbs) == (size_t)-1)
    {
        perror("mbstowcs");
        return EXIT_FAILURE;
    }

    printf("Nombre de multiplets : %u\n", (unsigned)strlen(mbs));
    printf("Nombre de caractères : %u\n", (unsigned)wchar_len(wcs));
    return 0;
}
\end{C}

\begin{C}
Élégament trouvé
Nombre de multiplets : 19
Nombre de caractères : 16
\end{C}

\subsubsection{La fonction mblen}
\label{la-fonction-mblen}

\begin{C}
int mblen(char *chaine, size_t max);
\end{C}

La fonction \mybox{mblen()} lit au plus \mybox{max} multiplets de la
chaîne \mybox{chaine}. Si ceux-ci forment un caractère large valide,
elle retourne le nombre de multiplets qui seront utilisés pour le
composer. Dans le cas contraire, elle retourne soit \mybox{0} (si le
premier caractère est le caractère nul) ou \mybox{-1} (la suite de
multiplets ne correspond pas à un caractère large).

\begin{C}
#include <locale.h>
#include <stdio.h>
#include <stdlib.h>


int
main(void)
{
    int n;

    if (setlocale(LC_CTYPE, "") == NULL)
    {
        perror("setlocale");
        return EXIT_FAILURE;
    }

    n = mblen("Élégamment trouvé", MB_CUR_MAX);

    if (n > 0)
        printf("Le prochain caractère large sera composé de %d multiplet(s).\n", n);

    return 0;
}
\end{C}

\begin{C}
Le prochain caractère large sera composé de 2 multiplet(s).
\end{C}

\subsection{Des fonctions dans tous leurs états}
\label{des-fonctions-dans-tous-leurs-etats}

Il est important de vous préciser une chose en rapport avec ces
fonctions : elles disposent d'un \textbf{état} interne. En vérité, ceci
est essentiellement important dans le cas où vous utilisez un jeu de
caractères avec état comme l'ISO-2022-JP. Dans un tel cas, les fonctions
de conversions doivent mémoriser la dernière séquence d'échappement afin
d'effectuer correctement leur travail, ce qu'elles réalisent à l'aide de
variables internes. Or, si une erreur est rencontrée ou si une autre
suite de multiplets leur est donnée en cours de route, le résultat
risque de s'en trouver compromis.

Dès lors, il est nécessaire de réinitialiser cet état interne après la
rencontre d'une erreur ou lors d'un changement de données à traiter.
Cela se réalise très simplement en fournissant comme argument une chaîne
nulle à l'une des fonctions \mybox{mbtowc()}, \mybox{wctomb()} ou
\mybox{mblen()}. Dans un tel cas, l'état interne est remis à zéro et
une valeur nulle est retournée si le jeu de caractères courant n'utilise
pas d'état et un nombre strictement positif sinon.\#\#\# 

\section{Internationalisation et localisation}
\label{internationalisation-et-localisation}

Cela est resté finalement assez discret jusqu'à ce chapitre, mais en 
y regardant de plus près, les programmes que nous avons réalisés sont 
en fait destinés à un environnement anglophone. En effet, prenez par 
exemple les entrées : si nous souhaitons fournir un nombre flottant 
à notre programme, nous devons utiliser le point comme séparateur 
entre la partie entière et décimale. Or, dans certains pays, on pourrait 
vouloir utiliser la virgule à la place. Cela nous paraît moins étrange 
étant donné que les constantes flottantes sont écrites de cette manière 
en C, mais il n'en va pas de même pour nos utilisateurs.

Ce qu'il faudrait finalement, c'est que nos programmes puissent
s'adapter aux usages, coutumes et langues de notre utilisateur, ce que
nous avons entrevu dans la section précédente.

\subsection{L'internationalisation}
\label{linternationalisation}

L'\textbf{internationalisation} (parfois abrégée « i18n ») est un
procédé par lequel un programme est rendu capable de s'adapter aux
préférences linguistiques et régionales d'un utilisateur.

\subsection{La localisation}
\label{la-localisation}

La \textbf{localisation} (parfois abrégée « l10n ») est une opération
par laquelle un programme internationalisé se voit fournir les
informations nécessaires pour s'adapter aux préférences linguistiques et
régionales d'un utilisateur.

\subsection{La fonction setlocale}
\label{la-fonction-setlocale}

De manière générale, les programmes que nous avons conçus jusqu'ici
étaient déjà partiellement internationalisés, car la bibliothèque
standard du langage C l'est dans une certaine mesure. Toutefois, nous
n'avons jamais recouru à un processus de localisation pour que ceux-ci
s'adaptent à nos usages. Nous vous le donnons en mille : la localisation
en C s'effectue à l'aide de\ldots{} la fonction \mybox{setlocale()}.

\begin{C}
 char *setlocale(int categorie, char *localisation);
\end{C}

Cette fonction attends deux arguments : une catégorie et la localisation
qui doit être employée pour cette catégorie.

\subsubsection{Les catégories}
\label{les-categories}

La bibliothèque standard du C divise la localisation en plusieurs
catégories, plus précisément cinq :

\begin{enumerate}
\def
\labelenumi{\arabic{enumi}.}
\item
  La catégorie \mybox{LC\_COLLATE} qui modifie le comportement des
  fonctions \mybox{strcoll()} et \mybox{strxfrm()} ;
\item
  La catégorie \mybox{LC\_CTYPE} qui adapte le comportement des
  fonctions de conversions que nous venons de voir, ainsi que les
  fonctions de l'en-tête \mybox{\textless{}ctype.h\textgreater{}} ;
\item
  La catégorie \mybox{LC\_MONETARY} qui influence le comportement de la
  fonction \mybox{localeconv()} ;
\item
  La catégorie \mybox{LC\_NUMERIC} qui altère le comportement des
  fonctions \mybox{*printf()} et \mybox{*scanf()} ainsi que des
  fonctions de conversions des chaînes de caractères (\mybox{atof()},
  \mybox{strtod()}, etc.) en ce qui concerne les nombres flottants ;
\item
  La catégorie \mybox{LC\_TIME} qui change le comportement de la
  fonction \mybox{strftime()}.
\end{enumerate}

Enfin, la catégorie \mybox{LC\_ALL} (qui n'en est pas vraiment une)
représente toutes les catégories en même temps. Nous ne attarderons
toutefois que sur \mybox{LC\_CTYPE} et \mybox{LC\_NUMERIC} dans cette
section, les fonctions affectées par les autres catégories n'ayant pas
été présentées à ce stade.

\subsubsection{Les localisations}
\label{les-localisations}

La bibliothèque standard prévoit deux localisations possibles :

\begin{enumerate}
\def
\labelenumi{\arabic{enumi}.}
\item
  La localisation \mybox{"C"} qui correspond à celle par défaut.
  Celle-ci utilise les usages anglophones et part du principe que les
  caractères employés se limitent à ceux garantis par la norme et qu'ils
  sont tous représentés sur un multiplet ;
\item
  La localisation \mybox{""} (une chaîne vide) qui correspond à celle
  utilisée par votre système.
\end{enumerate}

Il est également possible de fournir un pointeur nul comme localisation,
auquel cas la localisation actuelle est retournée.

\begin{C}
#include <locale.h>
#include <stdio.h>
#include <stdlib.h>


int main(void)
{
    char *s;

    s = setlocale(LC_ALL, NULL);
    puts(s);

    if (setlocale(LC_ALL, "") == NULL)
    {
        perror("setlocale");
        return EXIT_FAILURE;
    }

    s = setlocale(LC_ALL, NULL);
    puts(s);
    return 0;
}
\end{C}

\begin{C}
C
fr_BE.UTF-8
\end{C}

Comme vous le voyez, la localisation de départ est bien \mybox{C}.

\begin{infobox}
  La forme que prend la localisation
dépend de votre système. Sous unixoïdes et dans notre exemple, elle
prend la forme de la langue en minuscule (au format
\MYhref{https://fr.wikipedia.org/wiki/ISO_639}{ISO 639}) suivie d'un tiret
bas et du pays en majuscule (au format
\MYhref{https://fr.wikipedia.org/wiki/ISO_3166}{ISO 3166-1}) et,
éventuellement, terminée par un point et par l'encodage utilisé.
\end{infobox}


\subsection{Exemple}
\label{exemple-15}

Nous avons déjà eu l'occasion d'expérimenter la modification de la
localisation de la catégorie \mybox{LC\_CTYPE}. Ainsi, nous avons pu
préciser aux fonctions de conversion que la traduction devait s'opérer
depuis et vers le jeu de caractères d'exécution complet et non
uniquement le sous-ensemble défini par la norme.

L'exemple ci-dessous modifie la localisation de la catégorie
\mybox{LC\_NUMERIC} pour que les fonctions \mybox{scanf()} et
\mybox{printf()} adaptent leur gestion et affichage des nombres
flottants.

\begin{C}
#include <locale.h>
#include <stdio.h>
#include <stdlib.h>


int main(void)
{
    double f;

    if (setlocale(LC_NUMERIC, "") == NULL)
    {
        perror("setlocale");
        return EXIT_FAILURE;
    }

    printf("Veuillez entrer un nombre flottant : ");

    if (scanf("%lf", &f) != 1)
    {
        perror("scanf");
        return EXIT_FAILURE;
    }

    printf("Vous avez entré : %f.\n", f);
    return 0;
}
\end{C}

\begin{C}
Veuillez entrer un nombre flottant : 45,5
Vous avez entré : 45,500000.

Veuillez entrer un nombre flottant : 45.5
Vous avez entré : 45,000000.
\end{C}

Comme vous le voyez, après l'appel à \mybox{setlocale()}, seule la
virgule est considérée comme séparateur de la partie entière et de la
partie décimale.

\hrulefill

\textbf{En résumé}

\begin{enumerate}
\def
\labelenumi{\arabic{enumi}.}
\item
  La norme définit deux jeux de caractères : le jeu de caractère source,
  utilisé en interne par le compilateur, vers lequel vos fichiers
  sources sont convertis et le jeu de caractère d'exécution, utilisé par
  votre système, vers lequel la conversion finale aura lieu.
\item
  Mise à part les caractères garantis par la norme, les caractères
  supportés par les jeux de caractères source et d'exécution sont
  indéterminés.
\item
  Le type \mybox{wchar\_t} et les fonctions de conversion associées
  permettent de construire des chaînes de caractères larges à partir de
  chaîne simple, ce qui est particulièrement utile lorsque le jeu
  employé par le système encode les caractères sur un nombre variable de
  multiplets (comme l'ISO-2022 ou l'UTF-8).
\item
  Ces fonctions de conversions disposent d'un était interne qui doit
  être réinitialisé après une erreur ou avant chaque changement de
  données à traiter.
\item
  La fonction \mybox{setlocale()} permet de modifier la localisation de
  certaines fonctions de la bibliothèque standard.
\end{enumerate}