\chapter{T.P. : un allocateur statique de mémoire}
\label{T.P.-:-un-allocateur-statique-de-memoire}

Dans ce dernier chapitre, nous allons mettre en œuvre une partie des notions 
présentées dans cette partie afin de réaliser un allocateur statique de
mémoire.


\section{Objectif}
\label{objectif}

Votre objectif sera de parvenir à réaliser un petit allocateur statique de 
mémoire en mettant en œuvre deux fonctions : \mybox{static\_malloc()} et
\mybox{static\_free()} comparables aux fonctions \mybox{malloc()} et
\mybox{free()} de la bibliothèque standard. Toutefois, afin d'éviter
d'entrer dans les méandres de l'allocation dynamique de mémoire, ces
fonctions fourniront de la mémoire préalablement allouée
statiquement.


\section{Première étape : allouer de la mémoire}
\label{premiere-etape-:-allouer-de-la-memoire}


Pour commencer, vous allez devoir construire une fonction
\mybox{static\_malloc()} dont le prototype sera le suivant.

\begin{C}
void *static_malloc(size_t taille);
\end{C}

À la lumière de la fonction \mybox{malloc()}, celle-ci reçoit une
taille en multiplet et retourne un pointeur vers un bloc de mémoire d'au
moins la taille demandée ou un pointeur nul s'il n'y a plus de mémoire
disponible.

\subsection{Mémoire statique}
\label{memoire-statique}

Afin de réaliser ses allocations, la fonction \mybox{static\_malloc()}
ira piocher dans un bloc de mémoire statique. Celui-ci consistera
simplement en un tableau de \mybox{char} de classe de stockage statique
d'une taille prédéterminée. Pour cet exercice, nous partirons avec un
bloc de un mébimultiplets, soit 1.048.576 multiplets (1024 × 1024).

\begin{C}
static char mem[1048576UL];
\end{C}

\subsubsection{Alignement}
\label{alignement}

Toutefois, retourner un bloc de \mybox{char} de la taille demandée ne
suffit pas. En effet, si nous allouons par exemple treize multiplets,
disons pour une chaîne de caractères, puis quatre multiplets pour un
\mybox{int}, nous allons retourner une adresse qui ne respecte
potentiellement pas l'alignement requis par le type \mybox{int}. Étant
donné que la fonction \mybox{static\_malloc()} ne connaît pas les
contraintes d'alignements que doit respecter l'objet qui sera stocké
dans le bloc qu'elle fourni, elle doit retourner un bloc respectant les
contraintes les plus strictes. Dit autrement, la taille de chaque bloc
devra être un multiple de l'alignement le plus rigoureux.

Cet alignement peut être connu à l'aide d'une union comprenant les types
les plus contraignants et de la macrofonction \mybox{offsetof()}, comme
précisé dans le chapitre sur les unions.

\begin{C}
union align
{
    long e;
    long double f;
    void *p;
};
\end{C}

Mais\ldots{} ce n'est pas tout ! Le tableau alloué statiquement doit lui
aussi être aligné suivant l'alignement le plus sévère. En effet, si
celui-ci commence à une adresse non multiple de cet alignement, notre
stratégie précédente tombe à l'eau. Pour ce faire, il nous suffit
d'inclure le tableau dans une union avec comme autre membre l'union
décrite ci-dessus.

\begin{C}
union reserve
{
    union align align
    char mem[1048576UL]
};

static union reserve reserve;
\end{C}

Avec ceci, vous devriez pouvoir réaliser la fonction
\mybox{static\_malloc()} sans encombre.\\
\emph{Hop} \emph{hop} ! Au travail !

\section{Correction}
\label{correction-21}

\begin{C}
#include <assert.h>
#include <stddef.h>
#include <stdio.h>

#define MEM_SIZE (1024UL * 1024UL)
#define ALIGNEMENT(type) (offsetof(struct { char c; type t; }, t))

union align
{
    long e;
    long double f;
    void *p;
};

union reserve
{
    union align align;
    char mem[MEM_SIZE];
};

static union reserve reserve;

static size_t calcule_multiple_align(size_t);
static void *static_malloc(size_t);


static size_t calcule_multiple_align(size_t a)
{
    /*
     * Calcule le plus petit multiple de l'alignement maximal égal ou supérieur à `a`.
     */

    static size_t align_max = ALIGNEMENT(union align);
    size_t multiple = a / align_max;

    return ((a % align_max == 0) ? multiple : multiple + 1) * align_max;
}


static void *static_malloc(size_t taille)
{
    /*
     * Alloue un bloc de mémoire au moins de taille `taille`.
     */

    static size_t alloue = 0UL;
    void *ret;

    assert(taille > 0);
    taille = calcule_multiple_align(taille);

    if (MEM_SIZE - alloue < taille)
    {
        fprintf(stderr, "Il n'y a plus assez de mémoire disponible.\n");
        return NULL;
    }

    ret = &reserve.mem[alloue];
    alloue += taille;
    return ret;
}
\end{C}

La fonction \mybox{calcule\_multiple\_align()} détermine le plus petit
multiple de l'alignement maximal égal ou supérieur à \mybox{a}.
Celle-ci nous permet d'« arrondir » la taille demandée de sorte que les
blocs alloués soit toujours correctement alignés.

La macrofonction \mybox{ALIGNEMENT()} calcule l'alignement d'un type
donné en recourant à la macrofonction \mybox{offsetof()} comme expliqué
précédemment.

La fonction \mybox{static\_malloc()} utilise une variable statique :
\mybox{alloue} qui représente la quantité de mémoire déjà allouée. Dans
le cas où la taille demandée (arrondie au multiple de l'alignement qui
lui est égal ou supérieur) n'est pas disponible, un pointeur nul est
retourné. Sinon, la variable \mybox{alloue} est mise à jour et un
pointeur vers la position courante du tableau \mybox{reserve.mem} est
retourné.

Notez que nous avons utilisé une assertion afin d'éviter qu'une taille
nulle ne soit fournie.

\section{Deuxième étape : libérer de la mémoire}
\label{deuxieme-etape-:-liberer-de-la-memoire}

Pouvoir allouer de la mémoire, c'est bien, mais pouvoir en
libérer, ce serait mieux. Parce que pour le moment, dès que l'on arrive
à la fin du tableau, que la mémoire précédemment allouée soit encore
utilisée ou non, c'est grillé. :-°

Toutefois, pour mettre en place un système de libération de la mémoire,
il va nous falloir changer un peu de tactique.

\subsection{Ajout d'un en-tête}
\label{ajout-dun-en-tete}

Tout d'abord, si nous voulons libérer des blocs puis les réutiliser, il
nous faut conserver d'une manière ou d'une autre leur taille. En effet,
sans cette information, il nous sera impossible de les réaffecter.

Pour ce faire, il nous est possible d'ajouter un en-tête lors de
l'allocation d'un bloc, autrement dit un objet (par exemple un entier de
type \mybox{size\_t}) qui précédera le bloc alloué et contiendra la
taille du bloc.

\begin{C}
+---------+-----------------+
| En-tête |   Bloc alloué   |
+---------+-----------------+
\end{C}

Ainsi, lors de l'allocation, il nous suffit de transmettre à
l'utilisateur le bloc alloué sans son en-tête à l'aide d'une expression
du type \mybox{(char\ *)ptr\ +\ sizeof\ (size\_t)} et, à l'inverse,
lorsque l'utilisateur souhaite libérer un bloc, de le récupérer à l'aide
d'une expression comme \mybox{(char\ *)ptr\ -\ sizeof\ (size\_t)}.
Notez que cette technique nous coûte un peu plus de mémoire puisqu'il
est nécessaire d'allouer le bloc \emph{et} l'en-tête.

\subsection{Les listes chaînées}
\label{les-listes-chainees}

Mais, ce n'est pas tout. Pour pouvoir retrouver un bloc libéré, il nous
faut également un moyen pour les conserver et en parcourir la liste.
Afin d'atteindre cet objectif, nous allons employer une structure de
donnée appelée une \textbf{liste chaînée}. Celle-ci consiste simplement
en une structure comprenant des données ainsi qu'un pointeur vers une
structure du même type.

\begin{C}
struct list
{
    struct list *suivante;
};
\end{C}

Ainsi, il est possible de créer des \emph{chaines} de structure, chaque
maillon de la chaîne référencant le suivant. L'idée pour notre
allocateur va être de conserver une liste des blocs libérés, liste qu'il
parcourera en vue de rechercher un bloc de taille suffisante
\emph{avant} d'allouer de la mémoire provenant de la réserve. De cette
manière, il nous sera possible de réutiliser de la mémoire précédemment
allouée avant d'aller puiser dans la réserve.

Nous aurons donc a priori besoin d'une liste comprenant une référence
vers le bloc libéré et une référence vers le bloc suivant (il ne nous
est pas nécessaire d'employer un membre pour la taille du bloc puisque
l'en-tête la contient).

\begin{C}
struct bloc
{
    void *p;
    struct bloc *suivant;
};
\end{C}

\subsubsection{Le serpent qui se mange la queue}
\label{le-serpent-qui-se-mange-la-queue}

Cependant, avec une telle technique, nous risquons d'entrer dans un
cercle vicieux. En effet, imaginons qu'un utilisateur libère un bloc de
64 multiplets. Pour l'ajouter à la liste des blocs libérés, nous avons
besoin d'espace pour stocker une structure \mybox{bloc}, nous allons
donc allouer un peu de mémoire. De plus, si par la suite ce bloc est
réutilisé, la structure \mybox{bloc} ne nous est plus utile, sauf que
si nous la libérons, nous allons devoir allouer une autre structure
\mybox{bloc} pour référencé\ldots{} une structure \mybox{bloc} (à
moins que la structure ne s'autoréférence). Voilà qui n'est pas très
optimal.

À la place, il nous est possible de recourir à une autre stratégie :
inclure la structure \mybox{bloc} dans l'en-tête ou, plus précisémment,
son champ \mybox{suivant}, la référence vers le bloc n'étant plus
nécessaire dans ce cas. Autrement dit, notre en-tête sera le suivant.

\begin{C}
struct bloc
{
    size_t taille;
    struct bloc *suivant;
};
\end{C}

Lors de la libération du bloc, il nous suffit d'employer le champ
\mybox{suivant} de l'en-tête pour ajouter le bloc à la liste des blocs
libres.

\begin{C}
       En-tête                                          En-tête
<------------------->                            <------------------->
+---------+---------+-----------------+          +---------+---------+-----------------+
| Taille  | Suivant |   Bloc alloué   |          | Taille  | Suivant |   Bloc alloué   |
+---------+---------+-----------------+          +---------+---------+-----------------+
               +                                 ^
               |                                 |
               +---------------------------------+
\end{C}

\begin{attentionbox}
  Étant donné que l'en-tête précède le
bloc alloué, il est impératif que sa taille soit « arrondie » de sorte
que les données qui seront stockées dans le bloc respectent l'alignement
le plus strict.
\end{attentionbox}


Avec ceci, vous devriez pouvoir réaliser une fonction
\mybox{static\_free()} et modifier la fonction
\mybox{static\_malloc()} en conséquence.


\section{Correction}
\label{correction-22}

\begin{C}
#include <assert.h>
#include <stddef.h>
#include <stdio.h>
#include <stdlib.h>

#define MEM_SIZE (1024UL * 1024UL)
#define ALIGNEMENT(type) (offsetof(struct { char c; type t; }, t))
#define TAILLE_EN_TETE (offsetof(struct { struct bloc b; union align u; }, u))

union align
{
    long e;
    long double f;
    void *p;
};

union reserve
{
    union align align;
    char mem[MEM_SIZE];
};

struct bloc
{
    size_t taille;
    struct bloc *suivant;
};


static union reserve reserve;
static struct bloc *libre;

static void bloc_init(struct bloc *, size_t);
static size_t calcule_multiple_align(size_t);
static struct bloc *recherche_bloc_libre(size_t);
static void static_free(void *);
static void *static_malloc(size_t);


static void bloc_init(struct bloc *b, size_t taille)
{
    /*
     * Initialise les membres d'une structure `bloc`.
     */

    b->taille = taille;
    b->suivant = NULL;
}


static size_t calcule_multiple_align(size_t a)
{
    /*
     * Calcule le plus petit multiple de l'alignement maximal égal ou supérieur à `a`.
     */

    static size_t align_max = ALIGNEMENT(union align);
    size_t multiple = a / align_max;

    return ((a % align_max == 0) ? multiple : multiple + 1) * align_max;
}


static struct bloc *recherche_bloc_libre(size_t taille)
{
    /*
     * Recherche un bloc libre ou une suite de bloc libres dont la taille
     * est au moins égale à `taille`.
     */

    struct bloc *bloc = libre;
    struct bloc *precedent = NULL;
    struct bloc *ret = NULL;

    while (bloc != NULL)
    {
        if (bloc->taille >= taille)
        {
            if (precedent != NULL)
                precedent->suivant = bloc->suivant;
            else
                libre = bloc->suivant;

            ret = bloc;
            break;
        }

        precedent = bloc;
        bloc = bloc->suivant;
    }

    return ret;
}


static void *static_malloc(size_t taille)
{
    /*
     * Alloue un bloc de mémoire au moins de taille `taille`.
     */

    static size_t alloue = 0UL;
    void *ret;

    assert(taille > 0);
    taille = calcule_multiple_align(taille);
    ret = recherche_bloc_libre(taille);

    if (ret == NULL)
    {
        if (MEM_SIZE - alloue < TAILLE_EN_TETE + taille)
        {
            fprintf(stderr, "Il n'y a plus assez de mémoire disponible.\n");
            return NULL;
        }

        ret = &reserve.mem[alloue];
        alloue += TAILLE_EN_TETE + taille;
        bloc_init(ret, taille);
    }

    return (char *)ret + TAILLE_EN_TETE;
}


static void static_free(void *ptr)
{
    /*
     * Ajoute le bloc fourni à la liste des blocs libres.
     */

    struct bloc *bloc = libre;
    struct bloc *nouveau;

    if (ptr == NULL)
         return;

    nouveau = (struct bloc *)((char *)ptr - TAILLE_EN_TETE);

    while (bloc != NULL && bloc->suivant != NULL)
        bloc = bloc->suivant;

    if (bloc == NULL)
        libre = nouveau;
    else
        bloc->suivant = nouveau;
}
\end{C}

Désormais, la fonction \mybox{static\_malloc()} fait d'abord appel à la
fonction \mybox{recherche\_bloc\_libre()} \emph{avant} d'allouer de la
mémoire de la réserve. Comme son nom l'indique, cette fonction parcours
la liste des blocs libres à la recherche d'un bloc d'une taille égale ou
supérieure à celle demandée. Cette liste est matérialisée par la
variable globale \mybox{libre} qui est un pointeur sur une structure de
type \mybox{bloc}. Cette variable correspond au premier maillon de la
liste et est employée pour initialiser le parcours.

Si aucun bloc libre n'est trouvé, alors la fonction
\mybox{static\_malloc()} pioche dans la réserve un bloc de la taille
demandée augmentée de la taille de l'en-tête. Ces deux tailles sont «
arrondies » au multiple égal ou directement supérieur de l'alignement le
plus contraignant. Étant donné que la taille de l'en-tête est fixe, nous
avons représenté sa taille « arrondie » à l'aide de la macroconstante
\mybox{TAILLE\_EN\_TETE} et de la macrofonction \mybox{offsetof()}.
Une fois le tout alloué, l'en-tête est « retiré » en ajoutant la taille
de l'en-tête au pointeur référencant la mémoire allouée et le bloc
demandé est retourné.

La fonction \mybox{static\_free()} se rend à la fin de la liste et y
ajoute le bloc qui lui est fourni en argument. Pour ce faire, elle «
récupère » l'en-tête en soustrayant la taille de l'en-tête au pointeur
fourni en argument. Notez également que cette fonction n'effectue aucune
opération dans le cas où un pointeur nul lui est fourni (tout comme la
fonction \mybox{free()} standard), ce qui peut être intéressant pour
simplifier la gestion d'erreur dans certains cas.

\section{Troisième étape : fragmentation et défragmentation}
\label{troisieme-etape-:-fragmentation-et-défragmentation}

Bien, notre allocateur recycle à présent les blocs qu'il a précédemment 
alloués, c'est une bonne chose. Toutefois, un problème subsiste : la 
fragmentation de la mémoire allouée, autrement dit sa division en une 
multitude de petits blocs.

S'il s'agit d'un effet partiellement voulu (nous allouons par petits
blocs pour préserver la réserve), il peut avoir des conséquences
fâcheuses non désirées. En effet, imaginez que nous ayons alloué toute
la mémoire sous forme de blocs de 16, 32 et 64 multiplets, si même tous
ces blocs sont libres, notre allocateur retournera un pointeur nul dans
le cas d'une demande de par exemple 80 multiplets\ldots{} Voilà qui est
plutôt gênant.

Une solution consiste à défragmenter la liste des blocs libres,
c'est-à-dire fusionner plusieurs blocs pour en reconstruire d'autre avec
une taille plus importante. Dans notre cas, nous allons mettre en œuvre
ce système lors de la recherche d'un bloc libre : désormais, nous allons
regarder si un bloc est d'une taille suffisante \emph{ou} si
\emph{plusieurs} blocs, une fois fusionnés, seront de taille suffisante.

\subsection{Fusion de blocs}
\label{fusion-de-blocs}

Toutefois, une fusion de blocs n'est possible que si ceux-ci sont
adjacents, c'est-à-dire s'ils se suivent en mémoire. Plus précisémment,
l'adresse suivant le premier bloc à fusionner doit être celle de début
du second (autrement dit
\mybox{(char\ *)ptr\_bloc1\ +\ taille\_bloc1\ ==\ (char\ *)ptr\_bloc2}).

Néanmoins, il ne nous est pas possible de vérifier cela facilement si
notre liste de blocs libres n'est pas un minimum triée. En effet, sans
tri, il nous serait nécessaire de parcourir toute la liste à la
recherche d'éventuels blocs adjacents au premier, puis, de faire de même
pour le deuxième et ainsi de suite, ce qui n'est pas particulièrement
efficace.

À la place, il nous est possible de trier notre liste par adresses
croissantes (ou décroissantes, le résultat sera le même) de sorte que si
un bloc n'est pas adjacent au suivant, la recherche peut être
immédiatement arrêtée pour ce bloc ainsi que tous ceux qui lui étaient
adjacents. Ce tri peut être réalisé simplement lors de l'insertion d'un
nouveau bloc libre en placant celui-ci correctement dans la liste à
l'aide de comparaisons : s'il a une adresse inférieure à celle d'un
élément de la liste, il est placé avant cet élément, sinon, le parcours
continue.

En effet, deux pointeurs peuvent tout à fait être comparés du moment que
ceux-ci référencent un même objet ou un même aggrégat (c'est notre cas
ici puisqu'ils référenceront tous des éléments du tableau \mybox{mem}
de l'union \mybox{reserve}) et qu'ils sont du même type (une conversion
explicite vers le type pointeur sur \mybox{char} sera donc nécessaire
comme explicité auparavant).

\begin{infobox}
  N'oubliez pas que si deux ou
plusieurs blocs sont fusionnés, il n'y a plus besoin que d'un seul
en-tête, les autres peuvent donc être comptés comme de la mémoire
utilisable.
\end{infobox}

\section{Correction}
\label{correction-23}

\begin{C}
#include <assert.h>
#include <stddef.h>
#include <stdio.h>
#include <stdlib.h>

#define MEM_SIZE (1024UL * 1024UL)
#define ALIGNEMENT(type) (offsetof(struct { char c; type t; }, t))
#define TAILLE_EN_TETE (offsetof(struct { struct bloc b; union align u; }, u))

union align
{
    long e;
    long double f;
    void *p;
};

union reserve
{
    union align align;
    char mem[MEM_SIZE];
};

struct bloc
{
    size_t taille;
    struct bloc *suivant;
};

struct suite
{
    struct bloc *precedent;
    struct bloc *premier;
    struct bloc *dernier;
    size_t taille;
};


static union reserve reserve;
static struct bloc *libre;

static void bloc_init(struct bloc *, size_t);
static size_t calcule_multiple_align(size_t);
static struct bloc *recherche_bloc_libre(size_t);
static void static_free(void *);
static void *static_malloc(size_t);


static void bloc_init(struct bloc *b, size_t taille)
{
    /*
     * Initialise les membres d'une structure `bloc`.
     */

    b->taille = taille;
    b->suivant = NULL;
}


static size_t calcule_multiple_align(size_t a)
{
    /*
     * Calcule le plus petit multiple de l'alignement maximal égal ou supérieur à `a`.
     */

    static size_t align_max = ALIGNEMENT(union align);
    size_t multiple = a / align_max;

    return ((a % align_max == 0) ? multiple : multiple + 1) * align_max;
}


static struct bloc *recherche_bloc_libre(size_t taille)
{
    /*
     * Recherche un bloc libre ou une suite de bloc libres dont la taille
     * est au moins égale à `taille`.
     */

    struct suite suite = { 0 };
    struct bloc *bloc = libre;
    struct bloc *precedent = NULL;
    struct bloc *ret = NULL;

    while (bloc != NULL)
    {
        if (bloc->taille >= taille)
        {
            if (precedent != NULL)
                precedent->suivant = bloc->suivant;
            else
                libre = bloc->suivant;

            ret = bloc;
            break;
        }
        else if (suite.dernier != NULL && (char *)suite.dernier + TAILLE_EN_TETE \
        + suite.dernier->taille == (char *)bloc)
        {
            suite.dernier = bloc;
            suite.taille += TAILLE_EN_TETE + bloc->taille;
        }
        else
        {
            suite.precedent = precedent;
            suite.premier = bloc;
            suite.dernier = bloc;
            suite.taille = bloc->taille;
        }

        if (suite.taille >= taille)
        {
            if (suite.precedent != NULL)
                suite.precedent->suivant = suite.dernier->suivant;
            else
                libre = suite.dernier->suivant;

            suite.premier->taille = suite.taille;
            ret = suite.premier;
            break;
        }

        precedent = bloc;
        bloc = bloc->suivant;
    }

    return ret;
}


static void *static_malloc(size_t taille)
{
    /*
     * Alloue un bloc de mémoire au moins de taille `taille`.
     */

    static size_t alloue = 0UL;
    void *ret;

    assert(taille > 0);
    taille = calcule_multiple_align(taille);
    ret = recherche_bloc_libre(taille);

    if (ret == NULL)
    {
        if (MEM_SIZE - alloue < TAILLE_EN_TETE + taille)
        {
            fprintf(stderr, "Il n'y a plus assez de mémoire disponible.\n");
            return NULL;
        }

        ret = &reserve.mem[alloue];
        alloue += TAILLE_EN_TETE + taille;
        bloc_init(ret, taille);
    }

    return (char *)ret + TAILLE_EN_TETE;
}


static void static_free(void *ptr)
{
    /*
     * Ajoute le bloc fourni à la liste des blocs libres.
     */

    struct bloc *bloc = libre;
    struct bloc *precedent = NULL;
    struct bloc *nouveau;

    if (ptr == NULL)
         return;

    nouveau = (struct bloc *)((char *)ptr - TAILLE_EN_TETE);

    while (bloc != NULL)
    {
        if (bloc < nouveau)
        {
            if (precedent != NULL)
                precedent->suivant = nouveau;
            else
                libre = nouveau;
                

            nouveau->suivant = bloc;
            break;
        }

        precedent = bloc;
        bloc = bloc->suivant;
    }

    if (bloc == NULL)
        libre = nouveau;
}
\end{C}

La fonction \mybox{static\_free()} a été aménagée afin que les adresses
des différents blocs soient triées par ordre croissant. Si le nouveau
bloc libéré a une adresse inférieur à un autre bloc, il est placé avant
lui, sinon il est placé à la fin de la liste.

Quant à la fonction \mybox{recherche\_bloc\_libre()}, elle emploie
désormais une structure \mybox{suite} qui comprend un pointeur vers le
bloc précédent le début de la suite (champ \mybox{precedent}), un
pointeur vers le premier bloc de la suite (champ \mybox{premier}), un
pointeur vers le dernier bloc de la suite (champ \mybox{dernier}) et la
taille totale de la suite (champ \mybox{taille}).

Dans le cas où cette structure n'a pas encore été modifiée ou que le
dernier bloc de la suite n'est pas adjacent à celui qui le suit, le
premier et le dernier blocs de la suite deviennent le bloc courant, la
taille est réinitialisée à la taille du bloc courant et le bloc
précédent la suite est\ldots{} celui qui précède le bloc courant (ou un
pointeur nul s'il n'y en a pas).

Si la suite atteint une taille suffisante, alors les blocs la composant
sont fusionnés et le nouveau bloc ainsi construit est
retourné.

\hrulefill

Ce chapitre clôt cette troisième partie ainsi que ce cours. N'hésitez pas à 
revoir l'un ou l'autre passage qui vous ont semblé difficiles ou flous avant
de prendre le large vers de nouveaux horizons.