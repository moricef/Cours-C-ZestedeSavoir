\chapter{La gestion d'erreurs (1)}
\label{la-gestion-erreurs-(1)}

Dans les chapitres précédents, nous vous avons présenté des exemples simplifiés afin de
vous familiariser avec le langage. Aussi, nous avons pris soin de ne pas
effectuer de vérifications quant à d'éventuelles rencontres d'erreurs.

Mais à présent, c'est fini ! Vous disposez désormais d'un bagage
suffisant pour affronter la dure réalité d'un programmeur : des fois, il
y a des trucs qui foirent et il est nécessaire de le prévoir. Nous
allons voir comment dans ce chapitre.

\section{Détection d'erreurs}
\label{detection-derreurs}

La première chose à faire pour gérer d'éventuelles erreurs lors de l'exécution, c'est avant tout de les
détecter. Par exemple, quand vous executez une fonction et qu'une erreur
a lieu lors de son exécution, celle-ci doit vous prévenir d'une manière
ou d'une autre. Et elle peut le faire de deux manières différentes.

\subsection{Valeurs de retour}
\label{valeurs-de-retour}

Nous l'avons vu dans les chapitres précédents : certaines fonctions,
comme \mybox{scanf()}, retournent un nombre (souvent un entier) alors
qu'elles ne calculent pas un résultat comme la fonction \mybox{pow()}
par exemple. Vous savez dans le cas de \mybox{scanf()} que cette valeur
représente le nombre de conversions réussies, cependant cela va plus
loin que cela : cette valeur vous signifie si l'exécution de la fonction
s'est bien déroulée.

\subsubsection{Scanf}
\label{scanf}

En fait, la fonction \mybox{scanf()} retourne le nombre de conversions
réussies ou un nombre inférieur si elles n'ont pas toutes été réalisée
ou, enfin, \emph{un nombre négatif en cas d'erreur}.

Ainsi, si nous souhaitons récupérer deux entiers et être certains que
\mybox{scanf()} les a récupéré, nous pouvons utiliser le code suivant.

\begin{C}
#include <stdio.h>


int main(void)
{
    int x;
    int y;

    printf("Entrez deux nombres : ");

    if (scanf("%d %d", &x, &y) == 2)
        printf("Vous avez entre : %d et %d\n", x, y);

    return 0;
}
\end{C}

\begin{C}
Entrez deux nombres : 1 2
Vous avez entre : 1 et 2

Entrez deux nombres : 1 a
\end{C}

Comme vous pouvez le constater, le programme n'exécute pas l'affichage
des nombres dans le dernier cas, car \mybox{scanf()} n'a pas réussi à
réaliser deux conversions.

\subsubsection{Main}
\label{main}

Maintenant que vous savez cela, regarder bien votre fonction
\mybox{main()}.

\begin{C}
int main(void)
{
    return 0;
}
\end{C}

Vous ne voyez rien qui vous interpelle ? :)

Oui, vous avez bien vu, elle retourne un entier qui, comme pour
\mybox{scanf()}, sert à indiquer la présence d'erreur. En fait, il y a
deux valeurs possibles :

\begin{itemize}
\item
  \mybox{EXIT\_SUCCESS} (ou zéro, cela revient au même), qui indique
  que tout s'est bien passé ; et
\item
  \mybox{EXIT\_FAILURE}, qui indique un échec du programme.
\end{itemize}

Ces deux constantes sont définies dans l'en-tête
\mybox{\textless{}stdlib.h\textgreater{}}.

\subsubsection{Les autres fonctions}
\label{les-autres-fonctions}

Sachez que \mybox{scanf()}, \mybox{printf()} et \mybox{main()} ne
sont pas les seules fonctions qui retournent des entiers, en fait
quasiment toutes les fonctions de la bibliothèque standard le font.

\begin{questionbox}
 Ok, mais je fais comment pour savoir ce que retourne une fonction ?
\end{questionbox}


À l'aide de la documentation. Vous disposez de
\MYhref{http://flash-gordon.me.uk/ansi.c.txt}{la norme} (enfin, du
brouillon de celle-ci) qui reste la référence ultime, sinon vous pouvez
également utiliser un moteur de recherche avec la requête
\mybox{man\ nom\_de\_fonction} afin d'obtenir les informations dont
vous avez besoin.

\begin{infobox}
 Si vous êtes anglophobe, une traduction française de diverses descriptions
 est disponible à \MYhref{http://perkamon.traduc.org/}{cette adresse}, vous
 les trouverez à la section trois.
\end{infobox}


\subsection{Variable globale errno}
\label{variable-globale-errno}

Le retour des fonctions est un vecteur très pratique pour signaler une
erreur. Cependant, il n'est pas toujours utilisable. En effet, nous
avons vu lors du second TP la fonction mathématique \mybox{pow()}. Or,
cette dernière utilise \emph{déjà} son retour pour transmettre le
résultat d'une opération. Comment faire dès lors pour signaler un
problème ?

Une première idée serait d'utiliser une valeur particulière, comme zéro
par exemple. Toutefois, ce n'est pas satisfaisant puisque, dans le cas
de la fonction \mybox{pow()}, elle peut parfaitement retourner zéro
lors d'un fonctionnement normal. Que faire alors ?

Dans une telle situation, il ne reste qu'une seule solution : utiliser
un autre canal, en l'occurrence une variable globale. La bibliothèque
standard fourni une variable globale nomée \mybox{errno} (elle est
déclarée dans l'en-tête \mybox{\textless{}errno.h\textgreater{}}) qui
permet à différentes fonctions d'indiquer une erreur en modifiant la
valeur de celle-ci.

\begin{infobox}
Une valeur de zéro indique qu'aucune
erreur n'est survenue.
\end{infobox}
 

Les fonctions mathématiques recourent abondamment à cette fonction.
Prenons l'exemple suivant.

\begin{C}

#include <errno.h>
#include <stdio.h>


int main(void)
{
    double x;

    errno = 0;
    x = pow(-1, 0.5);

    if (errno == 0)
        printf("x = %f\n", x);

    return 0;
}
\end{C}

L'appel revient à demander le résultat de l'expression
\(-1^\frac{1}{2}\), autrement dit, de cette expression : \(\sqrt{-1}\),
ce qui est impossible dans l'essemble des réels. Aussi, la fonction
\mybox{pow()} modifie la variable \mybox{errno} pour vous signifier
qu'elle n'a pas pu calculer l'expression demandée.

Une petite précision concernant ce code et la variable \mybox{errno} :
celle-ci doit \emph{toujours} être mise à zéro \emph{avant} d'appeler
une fonction qui est susceptible de la modifier, ceci afin de vous
assurez qu'elle ne contient pas la valeur qu'une autre fonction lui a
assignée auparavant. Imaginez que vous ayez précédemment appelé la
fonction \mybox{pow()} et que cette dernière a échoué, si vous
l'appelez à nouveau, la valeur de \mybox{errno} sera toujours celle
assignée lors de l'appel précédent.

\begin{infobox}
 Notez que la bibliothèque standard ne
prévoit en fait que deux valeurs d'erreur possibles pour \mybox{errno}
: \mybox{EDOM} (pour le cas où le résultat d'une fonction mathématique
est impossible) et \mybox{ERANGE} (en cas de dépassement de capacité,
nous y reviendrons plus tard). Ces deux constantes sont définies dans
l'en-tête \mybox{\textless{}errno.h\textgreater{}}. 
\end{infobox}

\section{Prévenir l'utilisateur}
\label{prevenir-l-utilisateur}

Savoir qu'une erreur s'est produite, c'est bien, le signaler à
l'utilisateur, c'est mieux. Ne laisser pas votre utilisateur dans le
vide, s'il se passe quelque chose, dites le lui.

\begin{C}
#include <stdio.h>


int main(void)
{
    int x;
    int y;

    printf("Entrez deux nombres : ");

    if (scanf("%d %d", &x, &y) == 2)
        printf("Vous avez entré : %d et %d\n", x, y);
    else
        printf("Vous devez saisir deux nombres !\n");

    return 0;
}

\end{C}

\begin{C}
Entrez deux nombres : a b
Vous devez saisir deux nombres !

Entrez deux nombres : 1 2
Vous avez entre : 1 et 2
\end{C}

Simple, mais tellement plus agréable.


\section{Un exemple d'utilisation des valeurs de retour }
\label{un-exemple-d-utilisation-des-valeurs-de-retour }

Maintenant que vous savez tout cela, il vous est possible de modifier le code utilisant la fonction
\emph{scanf}() pour vérifier si celle-ci a réussi et, si ce n'est pas le
cas, préciser à l'utilisateur qu'une erreur est survenue \emph{et}
quitter la fonction \emph{main}() en retournant la valeur EXIT\_FAILURE.

\begin{C}
#include <stdio.h>
#include <stdlib.h>


int main(void)
{
    int x;
    int y;

    printf("Entrez deux nombres : ");

    if (scanf("%d %d", &x, &y) != 2)
    {
        printf("Vous devez saisir deux nombres !\n");
        return EXIT_FAILURE;
    }

    printf("Vous avez entré : %d et %d\n", x, y);
    return 0;
}
\end{C}

Ceci nous permet de réduire un peu la taille de notre code en éliminant
directement les cas d'erreurs.

\hrulefill

Bien, vous voilà à présent fin prêt pour la deuxième partie du cours et
ses \emph{vrais} exemples. Plus de pitié donc : gare à vos fesses si vous ne vérifiez pas le comportement des fonctions que vous appelez !