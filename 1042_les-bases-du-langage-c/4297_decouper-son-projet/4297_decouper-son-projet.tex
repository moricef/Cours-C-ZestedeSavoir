\chapter{Découper son projet}
\label{decouper-son-projet }

Ce chapitre est la suite directe de celui consacré aux fonctions : nous allons voir comment découper nos
projets en plusieurs fichiers. En effet, même si l'on découpe bien son
projet en fonctions, ce dernier est difficile à relire si tout est
contenu dans le même fichier. Ce chapitre a donc pour but de vous
apprendre à découper vos projets efficacement.

\section{Portée et masquage}
\label{portee-et-masquage}

\subsection{La notion de portée}
\label{la-notion-de-portee}

Avant de voir comment diviser nos programmes en plusieurs fichiers, il
est nécessaire de vous présenter une notion importante, celle de
\textbf{portée}. La portée d'une variable ou d'une fonction est la
partie du programme où cette dernière est utilisable. Il existe
plusieurs types de portées, cependant nous n'en verrons que deux :

\begin{itemize}
\item
  au niveau d'un bloc ;
\item
  au niveau d'un fichier.
\end{itemize}

\subsubsection{Au niveau d'un bloc}
\label{au-niveau-dun-bloc}

Une portée au niveau d'un bloc signifie qu'une variable n'est
utilisable, visible que de sa déclaration jusqu'à la fin du bloc dans
lequel elle est déclarée. Illustration.

\begin{C}
#include <stdio.h>

int main(void)
{
    {
        int nombre = 3;
        printf("%d\n", nombre);
    }

    /* Incorrect ! */
    printf("%d\n", nombre);
    return 0;
}
\end{C}

Dans ce code, la variable \mybox{nombre} est déclarée dans un
sous-bloc. Sa portée est donc limitée à ce dernier et elle ne peut pas
être utilisée en dehors.

\subsection{Au niveau d'un fichier}
\label{au-niveau-dun-fichier}

Une portée au niveau d'un fichier signifie qu'une variable n'est
utilisable, visible, que de sa déclaration jusqu'à la fin du fichier
dans lequel elle est déclarée. En fait, il s'agit de la portée des
variables « globales » dont nous avons parlé dans le chapitre sur les
fonctions.

\begin{C}
#include <stdio.h>

int nombre = 3;

int triple(void)
{
    return nombre * 3;
}

int main(void)
{
    nombre = triple();
    printf("%d\n", nombre);
    return 0;
}
\end{C}

Dans ce code, la variable \mybox{nombre} a une portée au niveau du
fichier et peut par conséquent être aussi bien utilisée dans la fonction
\mybox{triple()} que dans la fonction \mybox{main()}.

\section{La notion de masquage}
\label{la-notion-de-masquage}

En voyant les deux types de portées, vous vous êtes peut-être posé la
question suivante : que se passe-t-il s'il existe plusieurs variables
et/ou plusieurs fonctions de même nom ? \emph{Hé} bien, cela dépend de
la portée de ces dernières. Si elles ont la même portée comme dans
l'exemple ci-dessous, alors le compilateur sera incapable de déterminer
à quelle variable ou à quelle fonction le nom fait référence et, dès
lors, retournera une erreur.

\begin{C}
int main(void)
{
    int nombre = 10;
    int nombre = 20;

    return 0;
}
\end{C}

En revanche, si elles ont des portées différentes, alors celle ayant la
portée la plus faible sera privilégiée, on dit qu'elle \textbf{masque}
celle(s) de portée plus élevée. Autrement dit, dans l'exemple qui suit,
c'est la variable du bloc de la fonction \mybox{main()} qui sera
affichée.

\begin{C}
#include <stdio.h>

int nombre = 10;

int main(void)
{
    int nombre = 20;

    printf("%d\n", nombre);
    return 0;
}
\end{C}

Notez que nous disons : « celle(s) de portée plus élevée » car les
variables déclarées dans un sous-bloc ont une portée plus faible que
celle déclarée dans un bloc supérieur. Ainsi, le code ci-dessous est
parfaitement valide et affichera 30.

\begin{C}
 #include <stdio.h>

int nombre = 10;

int main(void)
{
    int nombre = 20;

    if (nombre == 20)
    {
        int nombre = 30;

        printf("%d\n", nombre);
    }

    return 0;
}
\end{C}



\section{Diviser pour mieux régner }
\label{diviser-pour-mieux-régner }

\subsection{Les fonctions}
\label{les-fonctions-1}

Dans l'extrait précédent, nous avions, entre autres, créé une fonction
\mybox{triple()} que nous avons placée dans le même fichier que la
fonction \mybox{main()}. Essayons à présent de les répartir dans deux
fichiers distincts. Pour ce faire, il vous suffit de créer un second
fichier avec l'extension « .c ». Dans notre cas, il s'agira de « main.c
» et de « autre.c ».

\begin{C}
/* Fichier autre.c */

int triple(int nombre)
{
    return nombre * 3;
}

\end{C}

\begin{C}
/* Fichier main.c */

int main(void)
{
    int nombre = triple(3);
    return 0;
}
\end{C}

La compilation se réalise de la même manière qu'auparavant, si ce n'est
qu'il vous est nécessaire de spécifier les deux noms de fichier :
\mybox{zcc\ main.c\ autre.c}. À noter que vous pouvez également
utiliser une forme raccourcie : \mybox{zcc\ *.c}, où \mybox{*.c}
correspond à tous les fichiers portant l'extension « .c » du dossier
courant.

Si vous testez ce code, vous aurez droit à un bel avertissement de votre
compilateur du type « \emph{implicit declaration of function
`triple'} ». Quel est le problème ? Le problème est que la fonction
\mybox{triple()} n'est pas déclarée dans le fichier \mybox{main.c} et
que le compilateur ne la connaît donc pas lorsqu'il compile le fichier.
Pour corriger cette situation, nous devons déclarer la fonction en
signalant au compilateur que cette dernière se situe dans un autre
fichier. Pour ce faire, nous allons inclure le prototype de la fonction
\mybox{triple()} dans le fichier \mybox{main.c} en le précédant du
mot-clé \mybox{extern}, qui signifie que la fonction est externe au
fichier.

\begin{C}
/* Fichier autre.c */

int triple(int nombre)
{
    return nombre * 3;

\end{C}

\begin{C}
/* Fichier main.c */

extern int triple(int nombre);

int main(void)
{
    int nombre = triple(3);

    return 0;
}
\end{C}

En terme technique, on dit que la fonction \mybox{triple()} est
\textbf{définie} dans le fichier « autre.c » (car c'est là que se situe
le corps de la fonction) et qu'elle est \textbf{déclarée} dans le
fichier « main.c ». Sachez qu'une fonction ne peut être définie
qu'\emph{une seule et unique fois}.

Pour information, notez que le mot-clé \mybox{extern} est facultatif
devant un prototype (il est implicitement inséré par le compilateur).
Nous vous conseillons cependant de l'utiliser, dans un soucis de clarté
et de symétrie avec les déclarations de variables (voyez ci-dessous).

\subsection{Les variables}
\label{les-variables-2}

La même méthode peut être appliquée aux variables, mais \emph{uniquement
à celle ayant une portée au niveau d'un fichier}. Également, à l'inverse
des fonctions, il est plus difficile de distinguer une définition d'une
déclaration de variable (elles n'ont pas de corps comme les fonctions).
La règle pour les différencier est qu'une déclaration sera précédée du
mot-clé \mybox{extern} alors que la définition non. C'est à vous de
voir dans quel fichier vous souhaitez définir la variable, mais elle ne
peut être définie qu'\emph{une seule et unique fois}. Enfin, sachez que
seule la définition peut comporter une initialisation. Ainsi, cet
exemple est tout à fait valide.

\begin{C}
/* Fichier autre.c */

int nombre = 10;    /* Une définition */
extern int autre;   /* Une déclaration */
\end{C}

\begin{C}
 /* Fichier main.c */
extern int nombre;  /* Une déclaration */
int autre = 10;     /* Une définition */
\end{C}


Alors que celui-ci, non.

\begin{C}
/* Fichier autre.c */

int nombre = 10;    /* Il existe une autre définition */
extern int autre = 10;  /* Une déclaration ne peut pas comprendre une initialisation */
\end{C}

\begin{C}
/* Fichier main.c */

int nombre = 20;    /* Il existe une autre définition */
int autre = 10;     /* Une définition */
\end{C}

\section{On m'aurait donc menti ?}
\label{on-maurait-donc-menti}

Nous vous avons dit plus haut qu'il n'était possible de définir une
variable ou une fonction qu'une seule fois, mais en fait, ce n'est pas
tout à fait vrai
\ldots{} Il est possible de rendre une variable (ayant
une portée au niveau d'un fichier) ou une fonction locale à un fichier
en précédant sa définition du mot-clé \mybox{static}. De cette manière,
la variable ou la fonction est interne au fichier où elle est définie et
n'entre pas en conflit avec les autres variables ou fonctions locales à
d'autres fichiers. La contrepartie est que la variable ou la fonction ne
peut être utilisée que dans le fichier où elle est définie (c'est assez
logique). Ainsi, l'exemple suivant est tout à fait correct et affichera
20.

\begin{C}
/* Fichier autre.c */
static int nombre = 10;
\end{C}

\begin{C}
/* Fichier main.c */
#include <stdio.h>

static int nombre = 20;

int main(void)
{
    printf("%d\n", nombre);
    return 0;
}
\end{C}

\begin{erreurbox}
Ne confondez pas l'utilisation du mot-clé
\mybox{static} visant à modifier la classe de stockage d'une variable
automatique avec celle permettant de limiter l'utilisation d'une
variable globale à un seul fichier !
\end{erreurbox}


\section{Les fichiers d'en-têtes }
\label{les-fichiers-den-têtes }

Pour terminer ce chapitre,
il ne nous reste plus qu'à voir les fichiers d'en-têtes.

Jusqu'à présent, lorsque vous voulez utiliser une fonction ou une
variable définie dans un autre fichier, vous insérez sa déclaration dans
le fichier ciblé. Seulement voilà, si vous utilisez dix fichiers et que
vous décidez un jour d'ajouter ou de supprimer une fonction ou une
variable ou encore de modifier une déclaration, vous vous retrouvez
Gros-Jean comme devant et vous êtes bon pour modifier les dix fichiers,
ce qui n'est pas très pratique\ldots{}

Pour résoudre ce problème, on utilise des fichiers d'en-têtes
(d'extension « .h »). Ces derniers contiennent conventionnellement des
déclarations de fonctions et de variables et sont inclus via la
directive \mybox{\#include} dans les fichiers qui utilisent les
fonctions et variables en question.

\begin{infobox}
 Les fichiers d'en-têtes n'ont pas
besoin d'être spécifiés lors de la compilation, ils seront
automatiquement inclus.
\end{infobox}


La structure d'un fichier d'en-tête est généralement de la forme
suivante.

\begin{C}
#ifndef CONSTANTE_H
#define CONSTANTE_H

/* Les déclarations */

#endif

\end{C}

Les directives du préprocesseur sont là pour éviter les inclusions
multiples : vous devez les utiliser pour chacun de vos fichiers
d'en-têtes. Vous pouvez remplacer \mybox{CONSTANTE} par ce que vous
voulez, le plus simple et le plus fréquent étant le nom de votre
fichier, par exemple \mybox{AUTRE\_H} si votre fichier se nomme «
autre.h ». Voici un exemple d'utilisation de fichiers d'en-têtes.

\begin{C}
/* Fichier d'en-tête autre.h */

#ifndef AUTRE_H
#define AUTRE_H

extern int triple(int nombre);

#endif
\end{C}

\begin{C}
/* Fichier source autre.c */

#include "autre.h"

int triple(int nombre)
{
    return nombre * 3;
}
\end{C}

\begin{C}
/* Fichier source main.c */

#include "autre.h"

int main(void)
{
    int nombre = triple(3);
    return 0;

\end{C}

Plusieurs remarques à propos de ce code :

\begin{itemize}
\item
  dans la directive d'inclusion, les fichiers d'en-têtes sont entre
  guillemets et non entre crochets comme les fichiers d'en-têtes de la
  bibliothèque standard ;
\item
  les fichiers sources et d'en-têtes correspondants portent le même nom
  ;
\item
  nous vous conseillons d'inclure le fichier d'en-tête dans le fichier
  source correspondant (dans mon cas « autre.h » dans « autre.c ») afin
  d'éviter des problèmes de portée.
\end{itemize}

\hrulefill

Dans le chapitre suivant, nous aborderons un point essentiel que
nous verrons en deux temps : la gestion d'erreurs.
