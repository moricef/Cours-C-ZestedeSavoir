\chapter{TP : une calculatrice basique}
\label{TP-:-une-calculatrice-basique}

Après tout ce que vous venez de découvrir, il est temps de faire une
petit pause et de mettre en pratique vos nouveaux acquis. Pour ce faire,
rien de tel qu'un exercice récapitulatif : réaliser une calculatrice
basique.

\section{Objectif}
\label{objectif}

Votre objectif sera de réaliser une calculatrice basique pouvant
calculer une somme, une soustraction, une multiplication, une division,
le reste d'une division entière, une puissance, une factorielle, le PGCD
et le PPCD.

Celle-ci attendra une entrée formatée suivant la
\href{http://fr.wikipedia.org/wiki/Notation_polonaise_inverse}{notation
polonaise inverse}. Autrement dit, les opérandes d'une opération seront
entrés \emph{avant} l'opérateur, par exemple comme ceci pour la somme de
quatre et cinq : \mybox{4\ 5\ +}.

Elle devra également retenir le résultat de l'opération précédente et
déduire l'utilisation de celui-ci en cas d'omission d'un opérande. Plus
précisément, si l'utilisateur entre par exemple \mybox{5\ +}, vous
devrez déduire que le premier opérande de la somme est le résultat de
l'opération précédente (ou zéro s'il n'y en a pas encore eu).

Chaque opération se verra attribuer un symbole ou une lettre, comme suit~:

\begin{itemize}
\item
  addition : \mybox{+} ;
\item
  soustraction : \mybox{-} ;
\item
  multiplication : \mybox{*} ;
\item
  division : \mybox{/} ;
\item
  reste de la division entière : \mybox{\%} ;
\item
  puissance : \mybox{\^{}} ;
\item
  factorielle : \mybox{!} ;
\item
  PGCD : \mybox{g} ;
\item
  PPCD : \mybox{p}.
\end{itemize}

Le programme doit s'arrêter lorsque la lettre « q » est spécifiée comme
opération (avec ou sans opérande).

\section{Préparation}
\label{preparation}

\subsection{Précisions concernant scanf}
\label{precisions-concernant-scanf}

\begin{questionbox}
Pourquoi utiliser la notation polonaise
inverse et non l'écriture habituelle ?
\end{questionbox}

Parce qu'elle va vous permettre de bénéficier d'une caractéristique
intéressante de la fonction \mybox{scanf()} : sa valeur de retour. Nous
anticipons un peu sur les chapitres suivants, mais sachez que la
fonction \mybox{scanf()} retourne une valeur entière qui correspond au
nombre de conversions réussies. Une conversion est réussie si ce
qu'entre l'utilisateur correspond à l'indicateur de conversion.

Ainsi, si nous souhaitons récupérer un entier à l'aide de l'indicateur
\mybox{d}, la conversion sera réussie si l'utilisateur entre un nombre
(par exemple 2) alors qu'elle échouera s'il entre une lettre ou un signe
de ponctuation.

Grâce à cela, vous pourrez détecter facilement s'il manque ou non un
opérande pour une opération.

\begin{attentionbox}
  Lorsqu'une conversion échoue, la
fonction \mybox{scanf()} arrête son exécution. Aussi, s'il y avait
d'autres conversions à effectuer après celle qui a avorté, elles ne
seront pas réalisée
\end{attentionbox}
.

\begin{C}
double a;
double b;
char op;

scanf("%lf %lf %c", &a, &b, &op);
\end{C}

Dans le code ci-dessus, si l'utilisateur entre \mybox{7\ *}, la
fonction \mybox{scanf()} retournera 1 et n'aura lu que le nombre 7. Il
sera nécessaire de l'appeler une seconde fois pour que le symbole
\mybox{*} soit récupéré.

\begin{infobox}
Petit bémol tout de même : les
symboles \mybox{+} et \mybox{-} sont considérés comme des débuts de
nombre valables (puisque vous pouvez par exemple entrer -2). Dès lors,
si vous souhaitez additionner ou soustraire un nombre au résultat de
l'opération précédente, vous devrez doubler ce symbole. Pour ajouter
cinq cela donnera donc : \mybox{5\ ++}. 
\end{infobox}


\subsection{Les puissances}
\label{les-puissances}

Pour élever une nombre à une puissance donnée (autrement dit, pour
calculer \(x^y\)), nous allons avoir besoin d'une nouvelle partie de la
bibliothèque standard dédiée aux fonctions mathématiques de base. Le
fichier d'en-tête de la bibliothèque mathématique se nomme
\mybox{\textless{}math.h\textgreater{}} et contient, entre autre, la
déclaration de la fonction \mybox{pow()}.

\begin{C}
\enddouble pow(double x, double y);
\end{C}

Cette dernière prends deux arguments : la base et l'exposant.

\begin{attentionbox}
L'utilisation de la bibliothèque
mathématique requiert d'ajouter l'option \mybox{-lm} lors de la
compilation, par exemple comme ceci : \mybox{zcc\ -lm\ main.c}
\end{attentionbox}


\subsection{La factorielle}
\label{la-factorielle}

La factorielle d'un nombre est égal au produit des nombres entiers
\emph{positifs et non nuls} inférieurs ou égaux à ce nombre. La
factorielle de quatre équivaut donc à \mybox{1\ *\ 2\ *\ 3\ *\ 4}, donc
vingt-quatre. Cette fonction n'est pas fournie par la bibliothèque
standard, il vous faudra donc la programmer par vous-même (pareil pour
le PGCD et le PPCD que nous avons vus dans les chapitres précédents).

\begin{infobox}
Par convention, la factorielle de
zéro est égale à un.
\end{infobox}


\subsection{Exemple d'utilisation}
\label{exemple-dutilisation}

\begin{C}
> 5 6 +
11.000000
> 4 *
44.000000
> 2 /
22.000000
> 5 2 %
1.000000
> 2 5 ^
32.000000
> 1 ++
33.000000
> 5 !
120.000000
\end{C}

\section{Derniers conseils}
\label{derniers-conseils}

Nous vous conseillons de récupérer les nombres sous forme de
\mybox{double}. Cependant, gardez bien à l'esprit que certaines
opérations ne peuvent s'appliquer qu'à des entiers : le reste de la
division entière, la factorielle, le PGCD et le PPCD. Il vous sera donc
nécessaire d'effectuer des conversions.

Également, notez bien que la factorielle ne s'applique qu'à \emph{un
seul} opérande à l'inverse de toutes les autres opérations.

Bien, vous avez à présent toutes les cartes en main : au travail !

\section{Correction}
\label{correction-13}

Alors ? Pas trop secoué ?
Bien, voyons à présent la correction.

\begin{C}
#include <math.h>
#include <stdio.h>
#include <stdlib.h>

unsigned long pgcd(unsigned long, unsigned long);
unsigned long ppcd(unsigned long, unsigned long);
unsigned long factorielle(unsigned long);


unsigned long pgcd(unsigned long a, unsigned long b)
{
    unsigned long r = a % b;

    while (r != 0)
    {
            a = b;
            b = r;
            r = a % b;
    }

    return b;
}


unsigned long ppcd(unsigned long a, unsigned long b)
{
    unsigned long i;
    unsigned long min = (a < b) ? a : b;

    for (i = 2; i <= min; ++i)
        if (a % i == 0 && b % i == 0)
            return i;

    return 0;
}


unsigned long factorielle(unsigned long a)
{
    unsigned long i;
    unsigned long r = 1;

    for (i = 2; i <= a; ++i)
        r *= i;

    return r;
}


int
main(void)
{
    double a;
    double b;
    double res = 0;
    int n;
    char op;

    while (1)
    {
        printf("> ");
        n = scanf("%lf %lf %c", &a, &b, &op);

        if (n <= 1)
        {
            scanf("%c", &op);
            b = a;
            a = res;
        }
        if (op == 'q')
            break;

        switch (op)
        {
        case '+':
            res = a + b;
            break;

        case '-':
            res = a - b;
            break;

        case '*':
            res = a * b;
            break;

        case '/':
            res = a / b;
            break;

        case '%':
            res = (unsigned long)a % (unsigned long)b;
            break;

        case '^':
            res = pow(a, b);
            break;

        case '!':
            res = factorielle((n == 0) ? a : b);
            break;

        case 'g':
            res = pgcd(a, b);
            break;

        case 'p':
            res = ppcd(a, b);
            break;
        }

        printf("%lf\n", res);
    }
    return 0;
}
\end{C}

Commençons par la fonction \mybox{main()}. Nous définissons plusieurs
variables :

\begin{itemize}
\item
  \mybox{a} et \mybox{b}, qui représentent les éventuels opérandes
  fournis ;
\item
  \mybox{res}, qui correspond au résultat de la dernière opération
  réalisée (ou zéro s'il n'y en a pas encore eu) ;
\item
  \mybox{n}, qui est utilisée pour retenir le retour de la fonction
  \mybox{scanf()} ; et
\item
  \mybox{op}, qui retient l'opération demandée.
\end{itemize}

Ensuite, nous entrons dans une boucle infinie (la condition étant
toujours vraie puisque valant un) où nous demandons à l'utilisateur
d'entrer l'opération à réaliser et les éventuels opérandes. Nous
vérifions ensuite si un seul opérande est fourni ou aucune (ce qui ce
déduit, respectivement, d'un retour de la fonction \mybox{scanf()}
valant un ou zéro). Si c'est le cas, nous appelons une seconde fois
\mybox{scanf()} pour récupérer l'opérateur. Puis, la valeur de
\mybox{a} est attribuée à \mybox{b} et la valeur de \mybox{res} à
\mybox{a}.

Si l'opérateur utilisé est \mybox{q}, alors nous quittons la boucle et
par la même occasion le programme. Notez que nous n'avons pas pu
effectuer cette vérification dans le corps de l'instruction
\mybox{switch} qui suit puisque l'instruction \mybox{break} nous
aurait fait quitter celui-ci et non la boucle.

Enfin, nous réalisons l'opération demandée au sein de l'instruction
\mybox{switch}, nous stockons le résultat dans la variable \mybox{res}
et l'affichons. Remarquez que l'utilisation de conversions explicites
n'a été nécessaire que pour le calcul du reste de la division entière.
En effet, dans les autres cas (par exemple lors de l'affectation à la
variable \mybox{res}), il y a des conversions implicites.

\begin{infobox}
 Nous avons utilisé le type \mybox{unsigned\ long} lors des calculs nécessitant des nombres entiers
afin de disposer de la plus grande capacité possible et parce que
l'usage de nombres négatifs n'a pas beaucoup d'intérêt dans ce cadre.
Cependant, l'usage de nombres non signés n'est obligatoire que pour la
fonction factorielle (puisque celle-ci n'opère que sur des nombres
strictement positifs).
\end{infobox}

\hrulefill

Ce chapitre nous aura permit de revoir la plupart
des notions des chapitres précédents. Dans le chapitre suivant, nous
verrons comment découper nos projets en plusieurs fichiers.