\chapter{Tests et conditions}
\label{tests-et-conditions}

Jusqu'à présent, vous avez appris à écrire du texte, manipuler des nombres
et interagir un tout petit peu avec l'utilisateur.

En gros, pour le moment, un programme est quelque chose de sacrément
simple et linéaire, ce dernier ne nous permettant que d'exécuter des
instructions dans un ordre donné. Techniquement, une simple calculatrice
peut en faire autant (voire plus). Cependant et heureusement, les
langages de programmation actuels fournissent des moyens permettant de
réaliser des tâches plus évoluées.

Pour ce faire, diverses \textbf{structures de contrôle} ont été
inventées. Celles-ci permettent de modifier le comportement d'un
programme suivant la réalisation de différentes conditions. Ainsi, si
une condition est vraie, le programme se comportera d'une telle façon et
à l'inverse, si elle est fausse, le programme fera telle ou telle chose.

Dans ce chapitre, nous allons voir comment rédiger des conditions à
l'aide de deux catégories d'opérateurs :

\begin{itemize}

\item
  les \textbf{opérateurs de comparaison}, qui comparent deux nombres ;
\item
  les \textbf{opérateurs logiques}, qui permettent de combiner plusieurs
  conditions.
\end{itemize}
  
\section{Les booléens}

Comme les opérateurs que nous avons vu
précédemment (\mybox{+}, \mybox{-}, \mybox{*}, etc), les opérateurs
de comparaison et les opérateurs logiques donnent un résultat : « vrai »
si la condition est vérifiée, et « faux » si la condition est fausse.
Toutefois, comme vous le savez, notre ordinateur ne voit que des
nombres. Aussi, il est nécessaire de représenter ces valeurs à l'aide de
ceux-ci.

Certains langages fournissent pour cela un type distinct pour stocker le
résultat des opérations de comparaison et deux valeurs spécifiques :
\mybox{true} (vrai) et \mybox{false} (faux). Néanmoins, dans les
premières versions du langage C, ce type spécial n'existait
pas\footnote{\footnotesize{Depuis la norme C99, le type \mybox{\_Bool} a été
  introduit ainsi que l'en-tête \mybox{\textless{}stdbool.h\textgreater{}} qui fournit un synonyme
  pour ce nouveau type : \mybox{bool}, et deux constantes entières
  \mybox{true} (qui vaut 1) et \mybox{false} (qui vaut zéro).}}. Il a donc fallu
ruser et trouver une solution pour représenter les valeurs « vrai » et «
faux ». Pour cela, la méthode la plus simple a été privilégiée :
utiliser directement des nombres pour représenter ces deux valeurs.
Ainsi, le langage C impose que :

\begin{itemize}
\item
  la valeur « faux » soit représentée par zéro ;
\item
  et que la valeur « vrai » soit représentée par tout sauf zéro.
\end{itemize}

Les opérateurs de comparaison et les opérateurs logiques suivent cette
convention pour représenter leur résultat. Dès lors, une condition
vaudra zéro si elle est fausse et un si elle est vraie.

\section{Les opérateurs de comparaison}
\label{les-operateurs-de-comparaison}

  Le langage C fournit différents opérateurs qui permettent
  d'effectuer des comparaisons entre des nombres. Ces opérateurs peuvent
  s'appliquer aussi bien à des constantes qu'à des variables (ou un
  mélange des deux). Ces derniers permettent donc par exemple de
  vérifier si une variable est supérieure à une autre, si elles sont
  égales, etc.

\subsection{Comparaisons}
\label{comparaisons}

L'écriture de conditions est similaire aux écritures mathématiques que
vous voyez en cours : l'opérateur est entre les deux expressions à
comparer. Par exemple, dans le cas de l'opérateur
\mybox{\textgreater{}} (« strictement supérieur à »), il est possible
d'écrire des expressions du type \mybox{a\ \textgreater{}\ b}, qui
vérifie si la variable \mybox{a} est strictement supérieure à la
variable \mybox{b}.

Le tableau ci-dessous reprend les différents opérateurs de comparaisons.

\begin{table}[ht!]
\centering
\begin{tabular}{|l|l|}\hline
\rowcolor{gris-tab-entete}\textbf{Opérateur} & \textbf{Signification}\tabularnewline\hline
\rowcolor{gris-clair-tab}\== & Égalité\tabularnewline\hline
!= & Inégalité\tabularnewline\hline
\rowcolor{gris-clair-tab}\textless{} & Strictement inférieur à\tabularnewline\hline
\textless{}= & Inférieur ou égal à\tabularnewline\hline
\rowcolor{gris-clair-tab}\textgreater{} & Strictement supérieur à\tabularnewline\hline
\textgreater{}= & Supérieur ou égal à\tabularnewline\hline
\end{tabular}
\end{table}

Ces opérateurs ne semblent pas très folichons. En effet, avec, nous ne
pouvons faire que quelques tests basiques sur des nombres. Cependant,
rappelez-vous : pour un ordinateur, tout n'est que nombre et comme pour
le stockage des données (revoyez le début du chapitre sur les variables
si cela ne vous dit rien) il est possible de ruser et d'exprimer toutes
les conditions possibles avec ces opérateurs (cela vous paraîtra plus
clair quand nous passerons aux exercices).

\subsection{Résultat d'une comparaison}
\label{resultat-dune-comparaison}

Comme dit dans l'extrait plus haut, une opération de comparaison va
donner zéro si elle est fausse et un si elle est vraie. Pour illustrer
ceci, vérifions quels sont les résultats donnés par différentes
comparaisons.

\begin{C}
int main(void)
{
    printf("10 == 20 vaut %d\n", 10 == 20);
    printf("10 != 20 vaut %d\n", 10 != 20);
    printf("10 < 20 vaut %d\n", 10 < 20);
    printf("10 > 20 vaut %d\n", 10 > 20);

    return 0;
}
\end{C}

\begin{C}
10 == 20 vaut 0
10 != 20 vaut 1
10 < 20 vaut 1
10 > 20 vaut 0
\end{C}

Le résultat confirme bien ce que nous avons dit ci-dessus.
  
\section{Les opérateurs logiques}
\label{les-operateurs-logiques}

Toutes ces comparaisons sont toutefois un peu faibles seules car il y a
des choses qui ne sont pas possibles à vérifier en utilisant une seule
comparaison. Par exemple, si un nombre est situé entre zéro et mille
(inclus). Pour ce faire, il serait nécessaire de vérifier que celui-ci
est supérieur ou égal à zéro \textbf{et} inférieur ou égal à mille.

Il nous faudrait donc trouver un moyen de combiner plusieurs
comparaisons entre elles pour résoudre ce problème. \emph{Hé} bien
rassurez-vous, le langage C fournit de quoi associer plusieurs résultats
de comparaisons : les \textbf{opérateurs logiques}.

\subsection{Les opérateurs logiques de base}
\label{les-operateurs-logiques-de-base}

Il existe trois opérateurs logiques. L'opérateur « \textbf{et} »,
l'opérateur « \textbf{ou} », et l'opérateur de \textbf{négation}. Les
deux premiers permettent de combiner deux conditions alors que le
troisième permet d'inverser le sens d'une condition.

\subsubsection{L'opérateur « et »}
\label{loperateur-et}

L'opérateur « et » va manipuler deux conditions. Il va donner un si
elles sont vraies, et zéro sinon.

\begin{table}[ht!]
\centering
\begin{tabular}{|l|l|l|}\hline
\rowcolor{gris-tab-entete}\textbf{Première condition} & \textbf{Seconde condition} & \textbf{Résultat de l'opérateur « et »}\tabularnewline\hline
Fausse & Fausse & 0\tabularnewline\hline
Fausse & Vraie & 0\tabularnewline\hline
Vraie & Fausse & 0\tabularnewline\hline
Vraie & Vraie & 1\tabularnewline\hline
\end{tabular}
\end{table}


Cet opérateur s'écrit \mybox{\&\&} et s'intercale entre les deux
conditions à combiner. Si nous reprenons l'exemple vu plus haut, pour
combiner les comparaisons \mybox{a\ \textgreater{}=\ 0} et
\mybox{a\ \textless{}=\ 1000}, nous devons placer l'opérateur
\mybox{\&\&} entre les deux, ce qui donne l'expression
\mybox{a\ \textgreater{}=\ 0\ \&\&\ a\ \textless{}=\ 1000}.

\subsubsection{L'opérateur « ou »}
\label{loperateur-ou}

L'opérateur « ou » fonctionne exactement comme l'opérateur « et », il
prend deux conditions et les combine pour former un résultat. Cependant,
l'opérateur « ou » vérifie que l'une des deux conditions (ou que les
deux) est (sont) vraie(s).


\begin{table}[ht!]
\centering
\begin{tabular}{|l|l|l|}\hline
\rowcolor{gris-tab-entete}\textbf{Première condition} & \textbf{Seconde condition} & \textbf{Résultat de l'opérateur « ou »}\tabularnewline\hline
Fausse & Fausse & 0\tabularnewline\hline
Fausse & Vraie & 1\tabularnewline\hline
Vraie & Fausse & 1\tabularnewline\hline
Vraie & Vraie & 1\tabularnewline\hline
\end{tabular}
\end{table}

Cet opérateur s'écrit \mybox{\textbar{}\textbar{}} et s'intercale entre
les deux conditions à combiner. L'exemple suivant permet de déterminer
si un nombre est divisible par trois ou par cinq (ou les deux) :
\mybox{(a\ \%\ 3\ ==\ 0)\ \textbar{}\textbar{}\ (a\ \%\ 5\ ==\ 0)}.
Notez que les parenthèses ont été placées par soucis de lisibilité.

\subsubsection{L'opérateur de négation}
\label{loperateur-de-negation}

Cet opérateur est un peu spécial : il manipule une seule condition et en
inverse le sens.

\begin{table}[ht!]
\centering
\begin{tabular}{|l|l|}\hline
\rowcolor{gris-tab-entete}\textbf{Condition} & \textbf{Résultat de l'opérateur de négation}\tabularnewline\hline
Fausse & 1\tabularnewline\hline
Vraie & 0\tabularnewline\hline
\end{tabular}
\end{table}


Cet opérateur se note \mybox{!}. Son utilité ? Simplifier certaines
expressions. Par exemple, si nous voulons vérifier cette fois qu'un
nombre \textbf{n'est pas} situé entre zéro et mille, nous pouvons
utiliser la condition
\mybox{a\ \textgreater{}=\ 0\ \&\&\ a\ \textless{}=\ 1000} et lui
appliquer l'opérateur de négation, ce qui donne
\mybox{!(a\ \textgreater{}=\ 0\ \&\&\ a\ \textless{}=\ 1000)}.

\begin{attentionbox}
 Faites bien attention à l'utilisation
des parenthèses ! L'opérateur de négation s'applique à l'opérande le
plus proche (sans parenthèses, il s'agirait de \mybox{a}). Veillez donc
a bien entourer de parenthèses l'expression concernée par la négation.
\end{attentionbox}

\begin{infobox}
Notez que pour cet exemple, il est parfaitement possible de se passer de cet opérateur à l'aide de
l'expression \mybox{a\ \textless{}\ 0\ \textbar{}\textbar{}\ a\ \textgreater{}\ 1000}.
Il est d'ailleurs souvent possible d'exprimer une condition de
différentes manières.
\end{infobox}

\subsection{Évaluation en court-circuit}
\label{evaluation-en-court-circuit}

Les opérateurs \mybox{\&\&} et \mybox{\textbar{}\textbar{}} evaluent
toujours la première condition avant la seconde. Cela paraît évident,
mais ce n'est pas le cas dans tous les langages de programmation. Ce
genre de détail permet à ces opérateurs de disposer d'un comportement
assez intéressant : \textbf{l'évaluation en court-circuit}.

De quoi s'agit-il ? Pour illustrer cette notion, reprenons l'exemple
précédent : nous voulons vérifier si un nombre est compris entre zéro et
mille, ce qui donne l'expression
\mybox{a\ \textgreater{}=\ 0\ \&\&\ a\ \textless{}=\ 1000}. Si jamais
\mybox{a} est inférieur à zéro, nous savons dès la vérification de la
première condition qu'il n'est pas situé dans l'intervalle voulu. Il
n'est donc pas nécessaire de vérifier la seconde. \emph{Hé} bien, c'est
exactement ce qu'il se passe en langage C. Si le résultat de la première
condition suffit pour déterminer le résultat de l'opérateur
\mybox{\&\&} ou \mybox{\textbar{}\textbar{}}, \emph{alors la seconde
condition n'est pas évaluée}. Voilà pourquoi l'on parle d'évaluation en
court-circuit.

Plus précisément, ce sera le cas pour l'opérateur \mybox{\&\&} si la
première condition est fausse et pour l'opérateur
\mybox{\textbar{}\textbar{}} si la première condition est vraie
(relisez les tableaux précédents si cela ne vous semble pas évident).

Ce genre de propriété peut-être utilisé efficacement pour éviter de
faire certains calculs, en choisissant intelligemment quelle sera la
première condition.

\subsection{Combinaisons}
\label{combinaisons}

Bien sûr, il est possible de mélanger ces opérateurs pour créer des
conditions plus complexes. Voici un exemple un peu plus long (et
inutile, soit dit en passant :-° ).

\begin{C}
int a = 3;
double b = 64.67;
double c = 12.89;
int d = 8;
int e = -5;
int f = 42;
int r;

r = ((a < b && b > 32) || (c < d + b || e == 0)) && (f > d);
printf("La valeur logique est égale à : %d\n", r);
\end{C}

Ici, la variable \mybox{r} est égale à 1, la condition est donc vraie.

\subsubsection{Parenthèses}
\label{parentheses}

En regardant le code écrit plus haut, vous avez surement remarqué la
présence de plusieurs parenthèses. Celles-ci permettent d'enlever toute
ambigüité dans les expressions créées avec des opérateurs logiques. En
effet, comme pour les opérateurs mathématiques, les opérateurs logiques
ont une priorité (revoyez le chapitre sur les opérations mathématiques
si cela ne vous dit rien) qui fait que l'opérateur \mybox{\&\&} passe
\emph{avant} l'opérateur \mybox{\textbar{}\textbar{}}. Ainsi, le
premier code est équivalent au second car l'opérateur \mybox{\&\&} est
évalué \emph{avant} l'opérateur \mybox{\textbar{}\textbar{}}.

\begin{C}
printf("%d\n", (a && b) || (c && d));
\end{C}

\begin{C}
printf("%d\n", a && (b || c) && d );
\end{C}

Si vous souhaitez un autre résultat, il est nécessare d'ajouter des
parenthèses pour modifier la priorité par défaut, par exemple comme
ceci.

\mybox{c\ printf("\%d\textbackslash{}n",\ a\ \&\&\ (b\ \textbar{}\textbar{}\ c)\ \&\&\ d\ );}Au
prochain chapitre, nous allons combiner les conditions avec une seconde
notion : les \textbf{sélections}.

\hrulefill

Au prochain chapitre, nous allons combiner les conditions avec une seconde notion : les sélections.