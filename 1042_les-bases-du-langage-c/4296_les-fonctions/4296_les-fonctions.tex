\chapter{Les fonctions}
\label{les-fonctions}

Nous avons découvert beaucoup de nouveautés dans les chapitres précédents et nos
programmes commencent à grossir. C'est pourquoi il est important
d'apprendre à les découper en \textbf{fonctions}.

\section{Qu'est-ce qu'une fonction ?}
\label{Qu-est-ce-qu-une-fonction-?}

Le concept de fonction ne vous est pas inconnu : \mybox{printf()},
\mybox{scanf()}, et \mybox{main()} sont des \textbf{fonctions}.

\begin{questionbox}
  Mais qu'est-ce qu'une fonction
exactement et quel est leur rôle exactement ?
\end{questionbox}


Une fonction est :

\begin{itemize}
\item
  une suite d'instructions ;
\item
  marquée à l'aide d'un nom (comme une variable finalement) ;
\item
  qui a vocation à être exécutée à plusieurs reprises ;
\item
  qui rassemble des instructions qui permettent d'effectuer une tâche
  précise (comme afficher du texte à l'écran, calculer la racine carrée
  d'un nombre, etc).
\end{itemize}

Pour mieux saisir leur intérêt, prenons un exemple concret.

\begin{C}
#include <stdio.h>


int main(void)
{
    int a;
    int b;
    int i;
    int min;

    printf("Entrez deux nombres : ");
    scanf("%d %d", &a, &b);
    min = (a < b) ? a : b;

    for (i = 2; i <= min; ++i)
    {
        if (a % i == 0 && b % i == 0)
        {
            printf("Le plus petit diviseur de %d et %d est %d\n", a, b, i);
            break;
        }
    }

    return 0;
}
\end{C}

Ce code, repris du chapitre précédent, permet de calculer le plus petit
commun diviseur de deux nombres donnés. Imaginons à présent que nous
souhaitions faire la même chose, mais avec deux paires de nombres. Le
code ressemblerait alors à ceci.

\begin{C}
#include <stdio.h>


int main(void)
{
    int a;
    int b;
    int i;
    int min;

    printf("Entrez deux nombres : ");
    scanf("%d %d", &a, &b);
    min = (a < b) ? a : b;

    for (i = 2; i <= min; ++i)
    {
        if (a % i == 0 && b % i == 0)
        {
            printf("Le plus petit diviseur de %d et %d est %d\n", a, b, i);
            break;
        }
    }

    printf("Entrez deux autres nombres : ");
    scanf("%d %d", &a, &b);
    min = (a < b) ? a : b;

    for (i = 2; i <= min; ++i)
    {
        if (a % i == 0 && b % i == 0)
        {
            printf("Le plus petit diviseur de %d et %d est %d\n", a, b, i);
            break;
        }
    }

    return 0;
}
\end{C}

Comme vous le voyez, ce n'est pas très pratique : nous devons recopier
les instructions de calcul deux fois, ce qui est assez dommage et qui
plus est source d'erreurs. C'est ici que les fonctions entre en jeu en
nous permettant par exemple de rassembler les instructions dédiées au
calcul du plus petit diviseur commun en un seul point que nous
solliciterons autant de fois que nécessaire.

\begin{infobox}
  Oui, il est aussi possible d'utiliser
une boucle pour éviter la répétition, mais l'exemple aurait été moins
parlant.
\end{infobox}

\section{Définir et utiliser une fonction}
\label{definir-et-utiliser-une-fonction}

Pour définir une fonction, nous allons devoir donner quatre informations
sur celle-ci :

\begin{itemize}
\item
  son \textbf{nom} : les règles sont les mêmes que pour les variables ;
\item
  son \textbf{corps} (son contenu) : le bloc d'instructions à exécuter ;
\item
  son \textbf{type de retour} : le type du résultat de la fonction ;
\item
  d'éventuels \textbf{paramètres} : des valeurs reçues par la fonction
  lors de l'appel.
\end{itemize}

La syntaxe est la suivante.

\begin{C}
type nom(paramètres)
{
     /* Corps de la fonction */
}
\end{C}

Prenons un exemple en créant une fonction qui affiche « bonjour ! » à
l'écran.

\begin{C}
#include <stdio.h>


void bonjour(void)
{
    printf("Bonjour !\n");
}


int main(void)
{
    bonjour();
    return 0;
}
\end{C}

Comme vous le voyez, la fonction se nomme « bonjour » et est composée
d'un appel à \mybox{printf()}. Reste les deux mots-clés \mybox{void} :

\begin{itemize}
\item
  dans le cas du type de retour, il spécifie que la fonction ne retourne
  rien ;
\item
  dans le cas des paramètres, il spécifie que la fonction n'en reçoit
  aucun (cela se manifeste lors de l'appel : il n'y a rien entre les
  parenthèses).
\end{itemize}

\subsection{Le type de retour}
\label{le-type-de-retour}

Le type de retour permet d'indiquer deux choses : si la fonction
retourne une valeur et le type de cette valeur.

\begin{C}
#include <stdio.h>


int deux(void)
{
    return 2;
}


int main(void)
{
    printf("Retour : %d\n", deux());
    return 0;

\end{C}

\begin{C}
Retour : 2
\end{C}

Dans l'exemple ci-dessus, la fonction \mybox{deux()} est définie comme
retournant une valeur de type \mybox{int}. Vous retrouvez l'instruction
\mybox{return}, une instruction de saut (comme \mybox{break},
\mybox{continue} et \mybox{goto}). Ce \mybox{return} arrête
l'exécution de la fonction courante et provoque un retour
(techniquement, un saut) vers l'appel à cette fonction qui se voit alors
attribuer la valeur de retour (s'il y en a une). Autrement dit, dans
notre exemple, l'instruction \mybox{return\ 2} stoppe l'exécution de la
fonction \mybox{deux()} et ramène l'exécution du programme à l'appel
qui vaut désormais 2, ce qui donne finalement
\mybox{printf("Retour\ :\ \%d\textbackslash{}n",\ 2)}

\subsection{Les paramètres}
\label{les-parametres}

Un paramètre sert à fournir des informations à la fonction lors de son
exécution. La fonction \mybox{printf()} par exemple récupère ce qu'elle
doit afficher dans la console à l'aide de paramètres. Ceux-ci sont
définis de la même manière que les variables si ce n'est que les
définitions sont séparées par des virgules.

\begin{C}
type nom(type paramètres1, type paramètres2, ...)
{
    /* Corps de la fonction */
}
\end{C}

\begin{infobox}
  Vous pouvez utiliser un maximum de
trente et un paramètres, toutefois nous vous conseillons de vous limiter
à \emph{cinq} afin de conserver un code concis et lisible.
\end{infobox}


Maintenant que nous savons tout cela, nous pouvons réaliser une fonction
qui calcul le plus petit commun diviseur entre deux nombres et ainsi
simplifier l'exemple du dessus.

\begin{C}
#include <stdio.h>


int ppcd(int a, int b)
{
    int min = (a < b) ? a : b;
    int i;

    for (i = 2; i <= min; ++i)
        if (a % i == 0 && b % i == 0)
            return i;

    return 0;
}


int main(void)
{
    int a;
    int b;
    int resultat;

    printf("Entrez deux nombres : ");
    scanf("%d %d", &a, &b);
    resultat = ppcd(a, b);

    if (resultat != 0)
        printf("Le plus petit diviseur de %d et %d est %d\n", a, b, resultat);

    printf("Entrez deux autres nombres : ");
    scanf("%d %d", &a, &b);
    resultat = ppcd(a, b);

    if (resultat != 0)
        printf("Le plus petit diviseur de %d et %d est %d\n", a, b, resultat);

    return 0;
}
\end{C}

Plus simple et plus lisible, non ?

\begin{infobox}
 Remarquez la présence de deux
instructions \mybox{return} dans la fonction \mybox{ppcd()}. La valeur
zéro est retournée afin d'indiquer l'absence d'un diviseur commun.
\end{infobox}


\section{Les arguments et les paramètres}
\label{les-arguments-et-les-parameres}

À ce stade, il est important de préciser qu'un paramètre est propre à
une fonction, il n'est \emph{pas} utilisable en dehors de celle-ci. Par
exemple, la variable \mybox{a} de la fonction \mybox{ppcd()} n'a aucun
rapport avec la variable \mybox{a} de la fonction \mybox{main()}.

Voici un autre exemple plus explicite à ce sujet.

\begin{C}
#include <stdio.h>


void fonction(int nombre)
{
    ++nombre;
    printf("Variable nombre dans `fonction' : %d\n", nombre);
}


int main(void)
{
    int nombre = 5;

    fonction(nombre);
    printf("Variable nombre dans `main' : %d\n", nombre);
    return 0;

\end{C}

\begin{C}
Variable nombre dans `fonction' : 6
Variable nombre dans `main' : 5
\end{C}

Comme vous le voyez, les deux variables \mybox{nombre} sont bel et bien
distinctes. En fait, lors d'un appel de fonction, vous spécifiez des
\textbf{arguments} à la fonction appelée. Ces arguments ne sont rien
d'autres que des expressions dont les résultats seront ensuite affectés
aux différents \textbf{paramètres} de la fonction.

\begin{attentionbox}
 Notez bien cette différence car elle
est très importante : un argument est une \emph{expression} alors qu'un
paramètre est une \emph{variable}.
\end{attentionbox}


Ainsi, la valeur de la variable \mybox{nombre} de la fonction
\mybox{main()} est passée en argument à la fonction \mybox{fonction()}
et est ensuite affectée au paramètre \mybox{nombre}. La variable
\mybox{nombre} de la fonction \mybox{main()} n'est donc en rien
modifiée.

\section{Les prototypes}
\label{les-prototypes}

Jusqu'à présent, nous avons toujours défini notre fonction
\emph{avant} la fonction \mybox{main()}. Cela paraît de prime abord
logique (nous définissons la fonction avant de l'utiliser), cependant
cela est surtout indispensable. En effet, si nous déplaçons la
définition après la fonction \mybox{main()}, le compilateur se retrouve
dans une situation délicate : il est face à un appel de fonction dont il
ne sait rien (nombres d'arguments, type des arguments et type de
retour). Que faire ? \emph{Hé} bien, il serait possible de stopper la
compilation, mais ce n'est pas ce qui a été retenu, le compilateur va
considérer que la fonction retourne une valeur de type \mybox{int} et
qu'elle reçoit un nombre indéterminé d'arguments.

Toutefois, si cette décision à l'avantage d'éviter un arrêt de la
compilation, elle peut en revanche conduire à des problèmes lors de
l'exécution si cette supposition du compilateur s'avère inadéquate. Or,
il serait pratique de pouvoir définir les fonctions dans l'ordre que
nous souhaitons sans se soucier de qui doit être défini avant qui.

Pour résoudre ce problème, il est possible de \textbf{déclarer} une
fonction à l'aide d'un \textbf{prototype}. Celui-ci permet de spécifier
le type de retour de la fonction, son nombre d'arguments et leur type,
mais ne comporte pas le corps de cette fonction. La syntaxe d'un
prototype est la suivante.

\begin{C}
type nom(paramètres);
\end{C}

Ce qui donne par exemple ceci.

\begin{C}
#include <stdio.h>

void bonjour(void);


int main(void)
{
    bonjour();
    return 0;
}


void bonjour(void)
{
    printf("Bonjour !\n");
}
\end{C}

\begin{attentionbox}
 Notez bien le point-virgule à la fin du
prototype qui est obligatoire.
\end{attentionbox}


\begin{infobox}
  Étant donné qu'un prototype ne
comprends pas le corps de la fonction qu'il déclare, il n'est pas
obligatoire de préciser le nom des paramètres de celles-ci. Ainsi, le
prototype suivant est parfaitement correct. 
\begin{C}
int ppcd(int, int);
\end{C}
\end{infobox}

\section{Variables globales et classes de stockage}
\label{variables-globales-et-classes-de-stockage}


\subsection{Les variables globales}
\label{les-variables-globales}

Il arrive parfois que l'utilisation de paramètres ne soit pas adaptée et
que des fonctions soient amenées à travailler sur des données qui
doivent leur être communes. Prenons un exemple simple : vous souhaitez
compter le nombre d'appels de fonction réalisé durant l'exécution de
votre programme. Ceci est impossible à réaliser, sauf à définir une
variable dans la fonction \mybox{main()}, la passé en argument de
chaque fonction et de faire en sorte que chaque fonction retourne sa
valeur augmentée de un, ce qui est très peu pratique.

À la place, il est possible de définir une variable dite «
\textbf{globale} » qui sera utilisable par toutes les fonctions. Pour
définir une variable globale, il vous suffit de définir une variable
\emph{en dehors de tout bloc}, autrement dit en dehors de toute
fonction.

\begin{C}
#include <stdio.h>

void fonction(void);

int appels = 0;


void fonction(void)
{
    ++appels;
}


int main(void)
{
    fonction();
    fonction();
    printf("Ce programme a réalisé %d appel(s) de fonction\n", appels);
    return 0;
}

\end{C}

\begin{C}
Ce programme a réalisé 2 appel(s) de fonction
\end{C}

Comme vous le voyez, nous avons simplement placé la définition de la
variable \emph{appels} en dehors de toute fonction et \emph{avant} toute
définition de fonction de sorte qu'elle soit partagée entres-elles.

\begin{infobox}
Le terme « global » est en fait un peu trompeur étant donné que la variable
n'est pas globale au programme, mais tout simplement disponible pour toutes
les fonctions du fichier dans lequel elle est située. Ce terme est utilisé
en opposition aux paramètres et variables des fonctions qui sont dits
« \textbf{locaux} ».
\end{infobox}


\begin{attentionbox}
  N'utilisez les variables globales que
lorsque cela vous paraît \emph{vraiment} nécessaire. Ces dernières étant
utilisables dans un fichier entier (voire dans plusieurs, nous le
verrons un peu plus tard), elles ont tendances à rendre la lecture du
code plus difficile.
\end{attentionbox}


\section{Les classes de stockage}
\label{les-classes-de-stockage}

Les variables locales et les variables globales ont une autre différence
de taille : leur \textbf{classe de stockage}. La classe de stockage
détermine (entre autre) la \textbf{durée de vie} d'un objet,
c'est-à-dire le temps durant lequel celui-ci existera en mémoire.

\subsection{Classe de stockage automatique}
\label{classe-de-stockage-automatique}

Les variables locales sont par défaut de classe de stockage
\textbf{automatique}. Cela signifie qu'elles sont allouées
automatiquement à chaque fois que le bloc auquel elles appartiennent est
exécuté et qu'elles sont détruites une fois son exécution terminée.

\begin{C}
int ppcd(int a, int b)
{
    int min = (a < b) ? a : b;
    int i;

    for (i = 2; i <= min; ++i)
        if (a % i == 0 && b % i == 0)
            return i;

    return 0;
}
\end{C}

Par exemple, à chaque fois que la fonction \mybox{ppcd()} est appelée,
les variables \mybox{a}, \mybox{b}, \mybox{min} et \mybox{i} sont
allouées en mémoires et détruites à la fin de l'exécution de la
fonction.

\subsection{Classe de stockage statique}
\label{classe-de-stockage-statique}

Les variables globales sont \emph{toujours} de classe de stockage
\textbf{statique}. Ceci signifie qu'elles sont allouées au début de
l'exécution du programme et sont détruites à la fin de l'exécution de
celui-ci. En conséquence, elles conservent leur valeur tout au long de
l'exécution du programme.

Également, à l'inverse des autres variables, celles-ci sont initialisées
à zéro si elles ne font pas l'objet d'une initialisation. L'exemple
ci-dessous est donc correct et utilise deux variables valant zéro.

\begin{C}
#include <stdio.h>

int a;
double b;


int main(void)
{
    printf("%d, %f\n", a, b);
    return 0;
}
\end{C}

\begin{C}
0, 0.000000
\end{C}

Petit bémol tout de même : étant donné que ces variables sont créées au
début du programme, elles ne peuvent être initialisée qu'à l'aide de
\emph{constantes}. La présence de variables au sein de l'expression
d'initialisation est donc proscrite.

\begin{C}
#include <stdio.h>

int a = 20; /* Correct */
double b = a; /* Incorrect */


int main(void)
{
    printf("%d, %f\n", a, b);
    return 0;
}
\end{C}

\subsection{Modification de la classe de stockage}
\label{modification-de-la-classe-de-stockage}

Il est possible de modifier la classe de stockage d'une variable
automatique en précédant sa définition du mot-clé \mybox{static} afin
d'en faire une variable statique.

\begin{C}
#include <stdio.h>


int compteur(void)
{
    static int n;

    return ++n;
}


int main(void)
{
    compteur();
    printf("n = %d\n", compteur());
    return 0;
}

\end{C}

\mybox{text\ n\ =\ 2}

\section{Exercices}
\label{exercices-3}

\subsection{Afficher un rectangle}
\label{afficher-un-rectangle}

Le premier exercice que nous vous proposons consiste à afficher un
rectangle dans la console. Voici ce que devra donner l'exécution de
votre programme.

\begin{C}
Donnez la longueur : 5
Donnez la largeur : 3

***
***
***
***
***
\end{C}

\subsection{Correction}
\label{correction-10}

\begin{C}
 #include <stdio.h>

void rectangle(int, int);


int main(void)
{
    int longueur;
    int largeur;

    printf("Donnez la longueur : ");
    scanf("%d", &longueur);
    printf("Donnez la largeur : ");
    scanf("%d", &largeur);
    printf("\n");
    rectangle(longueur, largeur);
    return 0;
}


void rectangle(int longueur, int largeur)
{
    int i;
    int j;

    for (i = 0; i < longueur; i++)
    {
        for (j = 0; j < largeur; j++)
            printf("*");

        printf("\n");
    }
}
\end{C}

\begin{infobox}
  Vous pouvez aussi essayer d'afficher
le rectangle dans l'autre sens
\end{infobox}
.

\section{Afficher un triangle}
\label{afficher-un-triangle}

Même principe, mais cette fois-ci avec un triangle (rectangle). Le
programme devra donner ceci.

\begin{C}
Donnez un nombre : 5

*
**
***
****
*****
\end{C}

Bien entendu, la taille du triangle variera en fonction du nombre entré.

\subsection{Correction}
\label{correction-11}

\begin{C}
 #include <stdio.h>

void triangle(int);


int main(void)
{
    int nombre;

    printf("Donnez un nombre : ");
    scanf("%d", &nombre);
    printf("\n");
    triangle(nombre);
    return 0;
}

void triangle(int nombre)
{
    int i;
    int j;

    for (i = 0; i < nombre; i++)
    {
        for (j = 0; j <= i; j++)
            printf("*");

        printf("\n");
    }
}
\end{C}

\section{En petites coupures ?}
\label{en-petites-coupures}

Pour ce dernier exercice, vous allez devoir réaliser un programme qui
reçoit en entrée une somme d'argent et donne en sortie la plus petite
quantité de coupures nécessaires pour reconstituer cette somme.

Pour cet exercice, vous utiliserez les coupures suivantes :

\begin{itemize}
\item
  des billets de 100\texteuro{} ;
\item
  des billets de 50\texteuro{} ;
\item
  des billets de 20\texteuro{} ;
\item
  des billets de 10\texteuro{} ;
\item
  des billets de 5\texteuro{} ;
\item
  des pièces de 2\texteuro{} ;
\item
  des pièces de 1\texteuro{} ;
\end{itemize}

Ci dessous un exemple de ce que devra donner votre programme une fois
terminé.

\begin{C}
Entrez une somme : 285
2 billet(s) de 100.
1 billet(s) de 50.
1 billet(s) de 20.
1 billet(s) de 10.
1 billet(s) de 5.
\end{C}

\subsection{Correction}
\label{correction-12}

\begin{C}
 #include <stdio.h>


int coupure_inferieure(int valeur)
{
    switch (valeur)
    {
    case 100:
        return 50;
    
    case 50:
        return 20;
    
    case 20:
        return 10;
    
    case 10:
        return 5;
    
    case 5:
        return 2;
    
    case 2:
        return 1;
    
    default:
        return 0;
    }
}


void coupure(int somme)
{
    int valeur;
    int nb_coupure;

    valeur = 100;

    while (valeur != 0)
    {
        nb_coupure = somme / valeur;
        
        if (nb_coupure > 0)
        {
            if (valeur >= 5)
                printf("%d billet(s) de %d.\n", nb_coupure, valeur);
            else
                printf("%d pièce(s) de %d.\n", nb_coupure, valeur);

            somme -= nb_coupure * valeur;
        }

        valeur = coupure_inferieure(valeur);
    }
}


int main(void)
{
    int somme;

    printf("Entrez une somme : ");
    scanf("%d", &somme);
    coupure(somme);
    return 0;
}
\end{C}

\hrulefill

Le prochain chapitre sera l'occasion de mettre en
pratique ce que nous venons de voir à l'aide d'un second TP.