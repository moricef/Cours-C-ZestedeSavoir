\chapter{Manipulations basiques des entrées/sorties}

Durant l'exécutiond'un programme, le processeur a besoin de communiquer 
avec le reste du matériel. Ces échanges d'informations sont les \textbf{entrées} et les
\textbf{sorties} (ou \emph{input} et \emph{output} pour les
anglophones), souvent abrégées E/S (ou I/O par les anglophones).

Les entrées permettent de recevoir une donnée en provenance de certains
périphériques. Les données fournies par ces entrées peuvent être une
information envoyée par le disque dur, la carte réseau, le clavier, la
souris, un CD, un écran tactile, bref par n'importe quel périphérique.
Par exemple, notre clavier va transmettre des informations sur les
touches enfoncées au processeur : notre clavier est donc une entrée.

À l'inverse, les sorties vont transmettre des données vers ces
périphériques. On pourrait citer l'exemple de l'écran : notre ordinateur
lui envoie des informations pour qu'elles soient affichées.

Dans ce chapitre, nous allons apprendre différentes fonctions fournies
par le langage C qui vont nous permettre d'envoyer des informations vers
nos sorties et d'en recevoir depuis nos entrées. Vous saurez ainsi
comment demander à un utilisateur de fournir une information au clavier
et comment afficher quelque chose sur la console.

\section{Les sorties}

Intéressons-nous dans un premier temps aux sorties. Afin
d'afficher du texte, nous avons besoin d'une \textbf{fonction}.

Une fonction est un morceau de code qui a un but, une \emph{fonction}
particulière et qui peut être \textbf{appelée} à l'aide d'une référence,
le plus souvent le nom de cette fonction (comme pour une variable,
finalement). En l'occurrence, nous allons utiliser une fonction qui a
pour objectif d'afficher du texte dans la console : la fonction
\mybox{printf()}.

\subsection{Première approche}
\label{premiere-approche}

Un exemple valant mieux qu'un long discours, voici un premier exemple.


\begin{C}
#include<stdio.h>}

int main(void)
{
     printf("Bonjour tout le monde !\n");
    return 0;
}
\end{C}


\begin{C}
Bonjour tout le monde !
\end{C}

Deux remarques au sujet de ce code.


\begin{C}
#include<stdio.h>
\end{C}


Il s'agit d'une \textbf{directive du préprocesseur}, facilement
reconnaissable car elles commencent toutes par le symbole \mybox{\#}.
Celle-ci sert à inclure un fichier (« stdio.h ») qui contient les
références de différentes fonctions d'entrée et sortie (« \emph{stdio} »
est une abbréviation pour « \emph{\textbf{St}andar\textbf{d i}nput
\textbf{o}utput} », soit « Entrée-sortie standard »).

Un fichier se terminant par l'extension « .h » est appelé un
\textbf{fichier d'en-tête} (\emph{header} en anglais) et fait partie
avec d'autre d'un ensemble plus large appelée la \textbf{bibliothèque
standard} (« standard » car elle est prévue par la
norme{\footnote{\footnotesize{Programming Language C, X3J11/88-090, §
4, Library.}}.


\begin{C}
 printf("Bonjour tout le monde !\n");
\end{C}


Ici, nous \textbf{appelons} la fonction \mybox{printf()} (un appel de
fonction est toujours suivi d'un groupe de parenthèses) avec comme
\textbf{argument} (ce qu'il y a entre les parenthèses de l'appel) un
texte (il s'agit plus précisément d'une \textbf{chaîne de caractères},
qui est toujours comprise entre deux guillemets double). Le
\mybox{\textbackslash n} est un caractère spécial qui représente un
retour à la ligne, cela est plus commode pour l'affichage.

Le reste devrait vous être familier.

\subsection{Les formats}
\label{les-formats}

Bien, nous savons maintenant afficher une phrase, mais ce serait quand
même mieux de pouvoir voir les valeurs de nos variables. Comment faire ?
\emph{Hé} bien, pour y parvenir, la fonction \mybox{printf()} met à
notre disposition des \textbf{formats}. Ceux-ci sont en fait des sortes
de repères au sein d'un texte qui indique à \mybox{printf()} que la
valeur d'une variable est attendue à cet endroit. Voici un exemple pour
une variable de type \mybox{int}.


\begin{C}
#include<stdio.h>}

int main(void)
{
    int variable = 20 ;

    printf("%d\n", variable);
    return 0 ;
}
\end{C}


\begin{C}
20
\end{C}

Nous pouvons voir que le texte de l'exemple précédent a été remplacé par
\mybox{\%d}, seul le \mybox{\textbackslash n} a été conservé. Un
format commence toujours par le symbole \mybox{\%} et est suivi par une
ou plusieurs lettres qui indiquent le type de données que nous
souhaitons voir afficher. Cette suite de lettre est appelée un
\textbf{indicateur de conversion}. En voici une liste non
exhaustive\footnote{\footnotesize{Pour le type \mybox{long long}
introduit en C99, les indicateurs de conversions sont \mybox{lld} et
\mybox{llu}. Il en va de même pour \mybox{scanf()}.}}

\clearpage

\begin{table}[ht!]
\centering
\rowcolors{1}{}{gris-clair-tab}
\begin{tabular}{|l|l|}\hline
\rowcolor{gris-tab-entete}\textbf{Type} & \textbf{Indicateur(s) de conversion}\tabularnewline\hline
\textbf{char} & c, d (ou i)\tabularnewline\hline
\textbf{short} & d (ou i)\tabularnewline\hline
\textbf{int} & d (ou i)\tabularnewline\hline
\textbf{long} & ld (ou li)\tabularnewline\hline
\textbf{unsigned char} & u, x (ou X) ou o\tabularnewline\hline
\textbf{unsigned short} & u, x (ou X) ou o\tabularnewline\hline
\textbf{unsigned int} & u, x (ou X) ou o\tabularnewline\hline
\textbf{unsigned long} & lu, lx (ou lX) ou lo\tabularnewline\hline
\textbf{float} & f, e (ou E) ou g (ou G)\tabularnewline\hline
\textbf{double} & f, e (ou E) ou g (ou G)\tabularnewline\hline
\textbf{long double} & Lf, Le (ou LE) ou Lg (ou LG)\tabularnewline\hline
\end{tabular}
\caption{Liste des indicateurs de conversion}
\end{table}


\begin{infobox}
Notez que pour le type \mybox{char},
l'indicateur est soit \mybox{c}, soit \mybox{d} (ou \mybox{i}). Cela
dépend si vous l'utilisez pour contenir un caractère ou un entier.
Également, notez que les indicateurs de conversions sont identiques pour
les types \mybox{char} (s'il stocke un entier), \mybox{short} et
\mybox{int} (pareil pour leurs équivalents non signés) ainsi que pour
les types \mybox{float} et \mybox{double}.
\end{infobox}

Les indicateurs \mybox{x}, \mybox{X} et \mybox{o} permettent
d'afficher un nombre en représentation hexadécimale ou octale
(l'indicateur \mybox{x} affiche les lettres en minuscules alors que
l'indicateur \mybox{X} les affiches en majuscules).

Les indicateurs \mybox{f}, \mybox{e} et \mybox{g} permettent quant à
eux d'afficher un nombre flottant. L'indicateur \mybox{f} affiche un
nombre en notation simple avec, par défaut, six décimales ; l'indicateur
\mybox{e} affiche un nombre flottant en notation scientifique
(l'indicateur \mybox{e} utilise la lettre « e » avant l'exposant alors
que l'indicateur « E » emploie la lettre « E ») et l'indicateur
\mybox{g} choisi quant à lui entre les deux notations précédentes
suivant le nombre fourni et supprime la partie fractionnaire si elle est
nulle de sorte que l'écriture soit concise (la différence entre les
indicateurs \mybox{g} et \mybox{G} est identique à celle entre les
indicateurs \mybox{e} et \mybox{E}).

Allez, un petit exemple pour reprendre tout cela et retravailler le
chapitre précédent par la même occasion.


\begin{C}
#include<stdio.h>}

int main(void )
{
    char z = 'z';
    char a = 10;
    unsigned} short b = 20;
    int c = 30;
    long d = 40;
    float e = 50 .;
    double f = 60.0;
    long double g = 70.0;

     printf("%c\n", z);
     printf("%d\n", a);
     printf("%u\n", b);
     printf("%o\n", b);
     printf("%x\n", b);
     printf("%d\n", c);
     printf("%li\n", d);
     printf("%f\n", e);
     printf( "%e\n", f);
     g = 80.0;
     printf( "%Lg\n", g);
    return 0;
}
\end{C}


\begin{C}
z
10
20
24
14
30
40
50.000000
6.000000e+01
80
\end{C}

\begin{infobox}
Si vous souhaitez afficher le caractère \mybox{\%} vous devez le doubler : \mybox{\%\%}.
\end{infobox}

\subsection{Précision des nombres flottants}\label{precision-des-nombres-flottants}

Vous avez peut-être remarquer que lorsqu'un flottant est affiché avec le
format \mybox{f}, il y a un certain nombre de zéros qui suivent (par
défaut six) et ce, peu importe qu'ils soient utiles ou non. Afin d'en
supprimer certains, vous pouvez spécifier une \textbf{précision}.
Celle-ci correspond au nombre de chiffre suivant la virgule qui seront
affichés. Elle prend la forme d'un point suivi par un nombre : la
quantité de chiffres qu'il y aura derrière la virgule.


\begin{C}
double x = 42.42734;

printf("%.2f", x);
\end{C}


\begin{C}
42.43
\end{C}

\subsection{Les caractères spéciaux}\label{les-caracteres-speciaux}

Dans certains cas, nous souhaitons obtenir un résultat à l'affichage
(saut de ligne, une tabulation, un retour chariot, etc.). Cependant, ils
ne sont pas particulièrement pratiques à insérer dans une chaîne de
caractères. Aussi, le C nous permet de le faire en utilisant une
\textbf{séquence d'échappement}. Il s'agit en fait d'une suite de
caractères commençant par le symbole \mybox{\textbackslash} et suivie
d'une lettre. En voici une liste non exhaustive.

\begin{table}[ht!]
\centering
\begin{tabular}{|l|l|}\hline
\rowcolor{gris-tab-entete}\textbf{Séquence d'échappement} & \textbf{Signification}\tabularnewline\hline
\rowcolor{gris-clair-tab} \mybox{\textbackslash}& Caractère d'appel\tabularnewline\hline
\mybox{\textbackslash b} & Espacement arrière\tabularnewline\hline
\rowcolor{gris-clair-tab} \mybox{\textbackslash f} & Saut de page\tabularnewline\hline
\mybox{\textbackslash n} & Saut de ligne\tabularnewline\hline
\rowcolor{gris-clair-tab} \mybox{\textbackslash r} & Retour chariot\tabularnewline\hline
\mybox{\textbackslash t} & Tabulation horizontale\tabularnewline\hline
\rowcolor{gris-clair-tab} \mybox{\textbackslash v} & Tabulation verticale\tabularnewline\hline
\mybox{\textbackslash "} & Le symbole « \mybox{"} »\tabularnewline\hline
\rowcolor{gris-clair-tab} \mybox{\textbackslash \textbackslash} & Le symbole « \mybox{\textbackslash} » lui-même\tabularnewline\hline
\end{tabular}
\caption{Les nombres de 1 à 16 en binaire, octal, décimal et hexadécimal.}
\end{table}

En général, vous n'utiliserez que le saut de ligne, la tabulation
horizontale et de temps à autre le retour chariot, les autres n'ont
quasiment plus d'intérêt. Un petit exemple pour illustrer leurs effets.


\begin{C}
#include<stdio.h>

int main(void)
{
     printf( "Quelques sauts de ligne\n\n\n");
     printf( " \tIl y a une tabulation avant moi !\n");
     printf( "Je voulais dire que... \r ");
     printf( "Hey ! Vous pourriez me laisser parler !\n");
    return 0 ;
}
\end{C}


\begin{C}
Quelques sauts de ligne


        Il y a une tabulation avant moi !
Hey ! Vous pourriez me laisser parler !
\end{C}

\begin{infobox}
Le retour chariot provoque un retour
au début de la ligne courante. Ainsi, il est possible d'écrire
par-dessus un texte affiché.
\end{infobox}

\subsection{Sur plusieurs lignes}\label{sur-plusieurs-lignes}

Notez qu'il est possible d'écrire un long texte sans appeler plusieurs
fois la fonction \mybox{printf()}. Pour ce faire, il suffit de le
diviser en plusieurs chaînes de caractères.


\begin{C}
#include<stdio.h>

int main(void )
{
     printf("Texte écrit sur plusieurs "
	    "lignes dans le code source "
	    "mais sur une seule dans la console.\n");
    return 0;
}
\end{C}


\begin{C}
Texte écrit sur plusieurs lignes dans le code source mais sur une seule dans la console.
\end{C}

\section{Interagir avec l'utilisateur }
Maintenant que nous savons déclarer, utiliser et même
afficher des variables, nous sommes fin prêts pour interagir avec
l'utilisateur.
En effet, jusqu'à maintenant, nous nous sommes contentés
d'afficher des informations. Nous allons à présent voir comment en
récupérer grâce à la fonction \mybox{scanf()}, dont l'utilisation est
assez semblable à \mybox{printf()}.


\begin{C}
#include<stdio.h>

int main(void)
{
    int age;

     printf("Quel âge avez-vous ? ");
     scanf("%d", &age);
     printf("Vous avez %d an(s)\n ", age);
    return 0 ;
}
\end{C}


\begin{C}
Quel age avez-vous ? 15
Vous avez 15 an(s).
\end{C}

Comme vous le voyez, l'appel à \mybox{scanf()} ressemble très fort à
celui de \mybox{printf()} mise à part l'absence du caractère spécial\mybox{\textbackslash n} (qui n'a pas d'intérêt puisque nous
récupérons des informations) et le symbole \mybox{\&}.

À son sujet, souvenez-vous de la brève explication sur la mémoire au
début du chapitre précédent. Celle-ci fonctionne comme une armoire avec
des tiroirs (les adresses mémoires) et des objets dans ces tiroirs (nos
variables). La fonction \mybox{scanf()} a besoin de connaitre
l'emplacement en mémoire de nos variables afin de les modifier. Afin
d'effectuer cette opération, nous utilisons le symbole \mybox{\&} (qui
est en fait un opérateur que nous verrons en détail plus tard). Ce
concept de transmission d'adresses mémoires est un petit peu difficile à
comprendre au début, mais ne vous inquiétez pas, vous aurez l'occasion
de bien revoir tout cela en profondeur dans le chapitre traîtant des
pointeurs.

Ici, \mybox{scanf()} attend patiemment que l'utilisateur saisisse un
nombre au clavier afin de modifier la valeur de la variable \emph{age}.
On dit que c'est une \textbf{fonction bloquante}, car elle suspend
l'exécution du programme tant que l'utilisateur n'a rien entré.

Pour ce qui est des indicateurs de conversions, ils sont un peu
différents de ceux de \mybox{printf()}.

\begin{table}[ht!]
\centering
\begin{tabular}{|l|l|}\hline
\rowcolor{gris-tab-entete}\textbf{Type} & \textbf{Indicateur(s) de conversion}\tabularnewline\hline
\rowcolor{gris-clair-tab}\textbf{char} & c\tabularnewline\hline
\textbf{short} & hd ou hi\tabularnewline\hline
\rowcolor{gris-clair-tab}\textbf{int} & d ou i\tabularnewline\hline
\textbf{long} & ld ou li\tabularnewline\hline
\rowcolor{gris-clair-tab}\textbf{unsigned short} & hu, hx ou ho\tabularnewline\hline
\textbf{unsigned int} & u, x ou o\tabularnewline\hline
\rowcolor{gris-clair-tab}\textbf{unsigned long} & lu, lx ou lo\tabularnewline\hline
\textbf{float} & f\tabularnewline\hline
\rowcolor{gris-clair-tab}\textbf{double} & lf\tabularnewline\hline
\textbf{long double} & Lf\tabularnewline\hline
\end{tabular}
\end{table}

\begin{erreurbox}
Faites bien attention aux différences ! Si
vous utilisez le mauvais format, le résultat ne sera pas celui que vous
attendez. Les changements concernent les types \mybox{char},
\mybox{short} et \mybox{double}.
\end{erreurbox}

\begin{infobox}
Notez que l'indicateur \mybox{c} ne
peut être employé que pour récupérer un caractère et non un nombre.
\end{infobox}

\begin{infobox}
Remarquez également qu'il n'y a plus
qu'un seul indicateur pour récupérer un nombre hexadécimal : \mybox{x}
(l'utilisation de lettres majuscules ou minuscules n'a pas
d'importance).
\end{infobox}

En passant, sachez qu'il est possible de lire plusieurs entrées en même
temps, par exemple comme ceci.


\begin{C}
int x, y;

 scanf("%d %d" , &x, &y);
 printf("x = %d | y = %d \n" , x, y);
\end{C}


L'utilisateur a deux possibilités, soit insérer un (ou plusieurs)
espace(s) entre les valeurs, soit insérer un (ou plusieurs) retour(s) à
la ligne entre les valeurs.

\begin{C}
14
6
x = 14 | y = 6
\end{C}

\begin{C}
14 6
x = 14 | y = 6
\end{C}

La fonction \mybox{scanf()} est en apparence simple (oui, \emph{en
apparence}), mais son utilisation peut devenir très complexe lorsqu'il
est nécessaire de vérifier les entrées de l'utilisateur (entre autres).
Cependant, à votre niveau, vous ne pouvez pas encore effectuer de telles
vérifications. Ce n'est toutefois pas très grave, nous verrons cela en
temps voulu.;)

\hrulefill

Maintenant, vous êtes capable de communiquer avec l'utilisateur.
Cependant, nos actions sont encore un peu limitées. Nous verrons dans
les prochains chapitres comment mieux interagir avec
l'utilisateur.