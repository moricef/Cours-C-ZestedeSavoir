\chapter{TP : déterminer le jour de la semaine}
\label{TP-:-determiner-le-jour-de-la-semaine }

Avant de poursuivre notre périple, il est à présent temps de nous poser un
instant afin de réaliser un petit exercice reprenant tout ce qui vient
d'être vu.

\section{Objectif}
\label{objectif }

Votre objectif est de parvenir à réaliser un programme qui,
suivant une date fournie par l'utilisateur sous la forme « jj/mm/aaaa »,
donne le jour de la semaine correspondant. Autrement dit, voici ce que
devrait donner l'exécution de ce programme.

\begin{C}
Entrez une date : 11/2/2015
C'est un mercredi

Entrez une date : 13/7/1970
C'est un lundi
\end{C}

\section{Première étape}
\label{premiere-etape }

Pour cette première étape, vous allez devoir réaliser un programme qui
demande à l'utilisateur un jour du mois de janvier de l'an un et qui lui
précise de quel jour de la semaine il s'agit, le tout à l'aide de la
méthode présentée ci-dessous.

\begin{C}
Entrez un jour : 27
C’est un jeudi
\end{C}

\subsection{Déterminer le jour de la semaine}
\label{determiner-le-jour-de-la-semaine}

Pour déterminer le jour de la semaine correspondant à une date, vous
allez devoir partir du premier janvier de l'an un (c'était un samedi) et
calculer le nombre de jours qui sépare cette date de celle fournie par
l'utilisateur. Une fois ce nombre obtenu, il nous est possible d'obtenir
le jour de la semaine correspondant à l'aide de l'opérateur modulo.

En effet, comme vous le savez, les jours de la semaine suivent un cycle
et se répètent tous les sept jours. Or, le reste de la division entière
est justement un nombre cyclique allant de zéro jusqu'au diviseur
diminué de un. Voici ce que donne le reste de la division entière des
chiffres 1 à 9 par 3.

\begin{C}
1 % 3 = 1
2 % 3 = 2
3 % 3 = 0
4 % 3 = 1
5 % 3 = 2
6 % 3 = 0
7 % 3 = 1
8 % 3 = 2
9 % 3 = 0
\end{C}

Comme vous le voyez, le reste de la division oscille toujours entre zéro
et deux. Ainsi, si nous attribuons un chiffre de zéro à six à chaque
jour de la semaine (par exemple zéro pour samedi et ainsi de suite pour
les autres) nous pouvons déduire le jour de la semaine correspondant à
un nombre de jours depuis le premier janvier de l'an un.

Prenons un exemple : l'utilisateur entre la date du vingt-sept janvier
de l'an un. Il y a vingt-six jours qui le sépare du premier janvier. Le
reste de la division entière de vingt-six par 7 est 5, il s'agit donc
d'un jeudi.

Si vous le souhaitez, vous pouvez vous aider du calendrier suivant.

\begin{C}
     Janvier 1        
di lu ma me je ve sa  
                   1  
 2  3  4  5  6  7  8  
 9 10 11 12 13 14 15  
16 17 18 19 20 21 22  
23 24 25 26 27 28 29  
30 31
\end{C}

À présent, à vous de jouer. ;)

\section{Correction}
\label{correction-1}

Alors, cela s'est bien passé ? Si oui, félicitations, si non, la
correction devrait vous aider à y voir plus clair.

\begin{C}
 #include <stdio.h>


int
main(void)
{
    unsigned jour;
    int njours;

    printf("Entrez un jour : ");
    scanf("%u", &jour);

    njours = (jour - 1);

    switch (njours % 7)
    {
    case 0:
        printf("C'est un samedi\n");
        break;

    case 1:
        printf("C'est un dimanche\n");
        break;

    case 2:
        printf("C'est un lundi\n");
        break;

    case 3:
        printf("C'est un mardi\n");
        break;

    case 4:
        printf("C'est un mercredi\n");
        break;

    case 5:
        printf("C'est un jeudi\n");
        break;

    case 6:
        printf("C'est un vendredi\n");
        break;
    }
    
    return 0;
}
\end{C}

Tout d'abord, nous demandons à l'utilisateur d'entrer un jour du mois de
janvier que nous affectons à la variable \mybox{jour}. Ensuite, nous
calculons la différence de jours séparant le premier janvier de celui
entré par l'utilisateur. Enfin, nous appliquons le modulo à ce résultat
afin d'obtenir le jour de la semaine correspondant.

\section{Deuxième étape}
\label{deuxieme-etape }

Bien, complexifions à présent un peu notre programme
et demandons à l'utilisateur de nous fournir un jour \emph{et} un mois
de l'an un.

\begin{C}
Entrez une date (jj/mm) : 20/4
C'est un mercredi
\end{C}

Pour ce faire, vous allez devoir convertir chaque mois en son nombre de
jours et ajouter ensuite celui-ci au nombre de jours séparant la date
entrée du premier du mois. À cette fin, vous pouvez considérer dans un
premier temps que chaque mois compte trente et un jours et ensuite
retrancher les jours que vous avez compté en trop suivant le mois
fourni.

Par exemple, si l'utilisateur vous demande quel jour de la semaine était
le vingt avril de l'an un :

\begin{itemize}
\item
  vous multipliez trente et un par trois puisque trois mois séparent le
  mois d'avril du mois de janvier (janvier, février et mars) ;
\item
  vous retranchez trois jours (puisque le mois de février ne comporte
  que vingt-huit jours les années non bissextiles) ;
\item
  enfin, vous ajoutez les dix-neuf jours qui séparent la date fournie du
  premier du mois.
\end{itemize}

Au total, vous obtenez alors cent et neuf jours, ce qui nous donne,
modulo sept, le nombre quatre, c'est donc un mercredi.

À toutes fins utiles, voici le calendrier complet de l'an un.

\begin{C}
  Janvier               Février                 Mars          
di lu ma me je ve sa  di lu ma me je ve sa  di lu ma me je ve sa  
                   1         1  2  3  4  5         1  2  3  4  5  
 2  3  4  5  6  7  8   6  7  8  9 10 11 12   6  7  8  9 10 11 12  
 9 10 11 12 13 14 15  13 14 15 16 17 18 19  13 14 15 16 17 18 19  
16 17 18 19 20 21 22  20 21 22 23 24 25 26  20 21 22 23 24 25 26  
23 24 25 26 27 28 29  27 28                 27 28 29 30 31        
30 31                                                             

       Avril                  Mai                   Juin          
di lu ma me je ve sa  di lu ma me je ve sa  di lu ma me je ve sa  
                1  2   1  2  3  4  5  6  7            1  2  3  4  
 3  4  5  6  7  8  9   8  9 10 11 12 13 14   5  6  7  8  9 10 11  
10 11 12 13 14 15 16  15 16 17 18 19 20 21  12 13 14 15 16 17 18  
17 18 19 20 21 22 23  22 23 24 25 26 27 28  19 20 21 22 23 24 25  
24 25 26 27 28 29 30  29 30 31              26 27 28 29 30        
                                                                  

      Juillet                 Août               Septembre        
di lu ma me je ve sa  di lu ma me je ve sa  di lu ma me je ve sa  
                1  2      1  2  3  4  5  6               1  2  3  
 3  4  5  6  7  8  9   7  8  9 10 11 12 13   4  5  6  7  8  9 10  
10 11 12 13 14 15 16  14 15 16 17 18 19 20  11 12 13 14 15 16 17  
17 18 19 20 21 22 23  21 22 23 24 25 26 27  18 19 20 21 22 23 24  
24 25 26 27 28 29 30  28 29 30 31           25 26 27 28 29 30     
31                                                                

      Octobre               Novembre              Décembre        
di lu ma me je ve sa  di lu ma me je ve sa  di lu ma me je ve sa  
                   1         1  2  3  4  5               1  2  3  
 2  3  4  5  6  7  8   6  7  8  9 10 11 12   4  5  6  7  8  9 10  
 9 10 11 12 13 14 15  13 14 15 16 17 18 19  11 12 13 14 15 16 17  
16 17 18 19 20 21 22  20 21 22 23 24 25 26  18 19 20 21 22 23 24  
23 24 25 26 27 28 29  27 28 29 30           25 26 27 28 29 30 31  
30 31
\end{C}

À vos claviers !

\section{Correction}
\label{correction-2}

Bien, passons à la correction.

\begin{C}
 #include <stdio.h>


int
main(void)
{
    unsigned jour;
    unsigned mois;
    int njours;

    printf("Entrez une date (jj/mm) : ");
    scanf("%u/%u", &jour, &mois);

    njours = (mois - 1) * 31;

    switch (mois)
    {
    case 12:
        --njours;
    case 11:
    case 10:
        --njours;
    case 9:
    case 8:
    case 7:
        --njours;
    case 6:
    case 5:
        --njours;
    case 4:
    case 3:
        njours -= 3;
        break;
    }

    njours += (jour - 1);

    switch (njours % 7)
    {
    case 0:
        printf("C'est un samedi\n");
        break;

    case 1:
        printf("C'est un dimanche\n");
        break;

    case 2:
        printf("C'est un lundi\n");
        break;

    case 3:
        printf("C'est un mardi\n");
        break;

    case 4:
        printf("C'est un mercredi\n");
        break;

    case 5:
        printf("C'est un jeudi\n");
        break;

    case 6:
        printf("C'est un vendredi\n");
        break;
    }

    return 0;
}
\end{C}

Nous commencons par demander deux nombres à l'utilisateur qui sont
affectés aux variables \mybox{jours} et \mybox{mois}. Ensuite, nous
multiplions le nombre de mois séparant celui entré par l'utilisateur du
mois de janvier par trente et un. Après quoi, nous soustrayons les jours
comptés en trop suivant le mois fourni. Enfin, nous ajoutons le nombre
de jours séparant celui entré du premier du mois, comme pour la première
étape.

\begin{infobox}
Notez que nous avons utilisé ici une
propriété intéressante de l'instruction \mybox{switch} : si la valeur
de contrôle correspond à celle d'une entrée, alors les instructions sont
exécutées \emph{jusqu'à rencontrer une instruction} \mybox{break} (ou
jusqu'à la fin du \mybox{switch}). Ainsi, si le mois entré est celui de
mai, l'instruction \mybox{$-$$-$njours} va être exécutée, puis
l'instruction \mybox{njours $-$= 3} va également être exécutée. 
\end{infobox}

\section{Troisième et dernière étape}
\label{troisieme-et-derniere-etape }

À présent, il est temps de réaliser un programme complet qui correspond aux objectifs du TP. Vous
allez donc devoir demander à l'utilisateur une date entière et lui
donner le jour de la semaine correspondant.

\begin{C}
Entrez une date : 11/2/2015
C'est un mercredi
\end{C}

Toutefois, avant de vous lancer dans la réalisation de celui-ci, nous
allons parler calendriers et années bissextiles.

\subsection{Les calendriers Julien et Grégorien}
\label{les-calendriers-julien-et-gregorien}

Vous le savez certainement, une
\MYhref{http://fr.wikipedia.org/wiki/Ann\%C3\%A9e_bissextile}{année
bissextile} est une année qui comporte 366 jours au lieu de 365 et qui
se voit ainsi ajouter un vingt-neuf février. Ce que vous ne savez en
revanche peut-être pas, c'est que la détermination des années bissextile
a varié au cours du temps.

Jusqu'en 1582, date d'adoption du
\MYhref{http://fr.wikipedia.org/wiki/Calendrier_gr\%C3\%A9gorien}{calendrier
Grégorien} (celui qui est en vigueur un peu près partout actuellement),
c'est le
\MYhref{http://fr.wikipedia.org/wiki/Calendrier_julien}{calendrier Julien}
qui était en application. Ce dernier considérait une année comme
bissextile si celle-ci était multiple de quatre. Cette méthode serait
correcte si une année comportait 365,25 jours. Cependant, il s'est avéré
plus tard qu'une année comportait en fait 365,2422 jours.

Dès lors, un décalage par rapport au cycle terrestre s'était lentement
installé ce qui posa problème à l'Église catholique pour le calcul de la
date de Pâques qui glissait doucement vers l'été. Le calendrier
Grégorien fût alors instauré en 1582 pour corriger cet écart en
modifiant la règle de calcul des années bissextile : il s'agit d'une
année multiple de quatre \emph{et}, s'il s'agit d'une année multiple de
100, également multiple de 400. Par exemple, les années 1000 et 1100 ne
sont plus bissextiles à l'inverse de l'année 1200 qui, elle, est
divisible par 400.

Toutefois, ce ne sont pas douze années bissextiles qui ont été
supprimées lors de l'adoption du calendrier Grégorien (100, 200, 300,
500, 600, 700, 900, 1000, 1100, 1300, 1400, 1500), mais seulement dix
afin de rapprocher la date de Pâques de l'équinoxe de printemps.

\subsection{Mode de calcul}
\label{mode-de-calcul}

Pour réaliser votre programme, vous devrez donc vérifier si la date
demandée est antérieure ou postérieure à l'an 1582. Si elle est
inférieure ou égale à l'an 1582, alors vous devrez appliquer le
calendrier Julien. Si elle est supérieure, vous devrez utiliser le
calendrier Grégorien.

Pour vous aider, voici un schéma que vous pouvez suivre.

\begin{C}
Si l'année est supérieure à 1582
    Multipler la différence d'années par 365
    Ajouter au résultat le nombre d'années multiples de 4
    Soustraire à cette valeur le nombre d'années multiples de 100
    Ajouter au résultat le nombre d'années multiples de 400
    Ajouter deux à ce nombre (du fait que seules dix années ont été supprimées en 1582)
Si l'année est inférieure ou égale à 1582
    Multipler la différence d'années par 365
    Ajouter au résultat le nombre d'années multiples de 4

Au nombre de jours obtenus, ajouter la différence de jours entre
le mois de janvier et le mois fourni. N'oubliez pas que les mois comportent
trente et un ou trente jours et que le mois de février comporte pour sa
part vingt-huit jours sauf les années bisextiles où il s'en voit ajouter un
vingt-neuvième. Également, faites attention au calendrier en application pour
la détermination des années bissextiles !

Au résultat obtenu ajouter le nombre de jour qui sépare celui entré du premier
du mois.

Appliquer le modulo et déterminer le jour de la semaine.
\end{C}

À vous de jouer !

\section{Correction}
\label{correction-3}

Ça va, vous tenez bon ?

\begin{C}
 #include <stdio.h>


int
main(void)
{
    unsigned jour;
    unsigned mois;
    unsigned an;
    int njours;

    printf("Entrez une date (jj/mm/aaaa) : ");
    scanf("%u/%u/%u", &jour, &mois, &an);
    njours = (an - 1) * 365;

    if (an > 1582) /* Calendrier Grégorien */
    {
        njours += ((an - 1) / 4);
        njours -= ((an - 1) / 100);
        njours += ((an - 1) / 400);
        njours += 2;
    }
    else /* Calendrier Julien */
        njours += ((an - 1) / 4);

    njours += (mois - 1) * 31;

    switch (mois)
    {
    case 12:
        --njours;
    case 11:
    case 10:
        --njours;
    case 9:
    case 8:
    case 7:
        --njours;
    case 6:
    case 5:
        --njours;
    case 4:
    case 3:
        if (an > 1582)
        {
            if (an % 4 == 0 && (an % 100 != 0 || an % 400 == 0))
                njours -= 2;
            else
                njours -= 3;
        }
        else
        {
            if (an % 4 == 0)
                njours -= 2;
            else
                njours -= 3;
        }
        break;
    }


    njours += (jour - 1);

    switch (njours % 7)
    {
    case 0:
        printf("C'est un samedi\n");
        break;

    case 1:
        printf("C'est un dimanche\n");
        break;

    case 2:
        printf("C'est un lundi\n");
        break;

    case 3:
        printf("C'est un mardi\n");
        break;

    case 4:
        printf("C'est un mercredi\n");
        break;

    case 5:
        printf("C'est un jeudi\n");
        break;

    case 6:
        printf("C'est un vendredi\n");
        break;
    }
    
    return 0;
}
\end{C}

Tout d'abord, nous demandons à l'utilisateur d'entrer une date au format
jj/mm/aaaa et nous attribuons chaque partie aux variables \mybox{jour},
\mybox{mois} et \mybox{an}. Ensuite, nous multiplions par 365 la
différence d'années séparant l'année fournie de l'an un. Toutefois, il
nous faut encore prendre en compte les années bissextiles pour que le
nombre de jours obtenus soit correct. Nous ajoutons donc un jour par
année bissextile en prenant soin d'appliquer les règles du calendrier en
vigueur à la date fournie.

Maintenant, il nous faut ajouter le nombre de jours séparant le mois de
janvier du mois spécifié par l'utilisateur. Pour ce faire, nous
utilisons la même méthode que celle vue lors de la deuxième étape
\emph{à une différence près} : nous vérifions si l'année courante est
bissextile afin de retrancher le bon nombre de jours (le mois de février
comportant dans ce cas vingt-neuf jours et non vingt-huit).

Enfin, nous utilisons le même code que celui de la première étape.



\hrulefill

Ce chapitre nous aura permis de faire une petite pause et de mettre en application ce que nous
avons vu dans les chapitres précédents. Reprenons à présent notre route
en attaquant la notion de \textbf{boucle}.