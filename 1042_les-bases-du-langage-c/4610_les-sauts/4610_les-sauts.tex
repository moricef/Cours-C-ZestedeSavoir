\chapter{Les sauts}
\label{les-sauts}

Dans les chapitres précédents, nous avons vu comment modifier
l'exécution de notre programme en fonction du résultat d'une ou
plusieurs conditions. Ainsi, nous avons pu réaliser des tâches plus
complexes que de simplement exécuter une suite d'instructions de
manière linéaire.

Cette exécution non linéaire est possible grâce à ce que l'on appel
des \textbf{sauts}. Un saut correspond au passage d'un point à un
autre d'un programme. Bien que cela vous ait été caché, sachez que
vous en avez déjà rencontré ! En effet, une instruction \mybox{if}
réalise par exemple un saut à votre insu.

\begin{C}
  if (/* Condition */) { /* Bloc */ }

  /* Suite du programme */
\end{C}

Dans le cas où la condition est fausse, l'exécution du programme passe
le bloc de l'instruction \mybox{if} et exécute ce qui suit. Autrement
dit, il y a un \textbf{saut} jusqu'à la suite du bloc.

Dans la même veine, une boucle \mybox{while} réalise également des
sauts.

\begin{C}
  while (/* Condition */) { /* Bloc à répéter */ }

  /* Suite du programme */
\end{C}

Dans cet exemple, si la condition est vraie, le bloc qui suit est
exécuté puis il y a un saut pour revenir à l'évaluation de la
condition.  Si en revanche elle est fausse, comme pour l'instruction
\mybox{if}, il y a un saut au-delà du bloc d'instructions.

Tous ces sauts sont cependant automatiques et vous sont cachés. Dans
ce chapitre, nous allons voir comment réaliser manuellement des sauts
à l'aide de trois instructions : \mybox{break}, \mybox{continue} et
\mybox{goto}.

\section{L'instruction break}
\label{Linstruction-break}


Nous avons déjà recontré l'instruction \mybox{break} lors de la
présentation de l'instruction \mybox{switch}, cette dernière
permettait de quitter le bloc d'un \mybox{switch} pour reprendre
immédiatement après.  Cependant, l'instruction \mybox{break} peut
également être utilisée au sein d'une boucle pour stopper son
exécution (autrement dit pour effectuer un saut au-delà du bloc à
répéter).

\subsection{Exemple}
\label{exemple-9}

Le plus souvent, une instruction \mybox{break} est employée pour
sortir d'une itération lorsqu'une condition (différente de celle
contrôlant l'exécution de la boucle) est remplie. Par exemple, si nous
souhaitons réaliser un programme qui détermine le plus petit diviseur
commun de deux nombres, nous pouvons utiliser cette instruction comme
suit.

\begin{C}

\end{C}

\begin{C}
  Entrez deux nombres : 112 567 le plus petit diviseur de 112 et 567
  est 7

  Entrez deux nombres : 13 17
\end{C}

Comme vous le voyez, la condition principale permet de progresser
parmis les diviseurs possibles alors que la seconde détermine si la
valeur courante de \mybox{i} est un diviseur commun. Si c'est le cas,
l'exécution de la boucle est stoppée et le résultat affiché. Dans le
cas où il n'y a aucun diviseur commun, la boucle s'arrête lorsque le
plus petit des deux nombres est atteint.

\section{L’instruction continue}
\label{Linstruction-continue}

L'instruction \mybox{continue} permet d'arrêter l'exécution de
l'itération courante. Autrement dit, celle-ci vous permet de retourner
(sauter) directement à l'évaluation de la condition.

\subsubsection*{Exemple}
\label{exemple-10}

Afin d'améliorer un peu l'exemple précédent, nous pourrions passer les
cas où le diviseur testé est un multiple de deux (puisque si un des
deux nombres n'est pas divisible par deux, il ne peut pas l'être par
quatre, par exemple).

Ceci peut s'exprimer à l'aide de l'instruction \mybox{continue}.

\begin{C}
  #include <stdio.h>


  int main(void) { int a; int b; int i; int min;

    printf("Entrez deux nombres : "); scanf("%d %d", &a, &b);
    min = (a < b) ? a : b;

    for (i = 2; i <= min; ++i) { if (i != 2 && i % 2 == 0)
      { printf("je passe %d\n", i);
        continue; } if (a % i == 0 && b % i == 0)
      { printf("le plus petit diviseur
        de %d et %d est %d\n", a, b, i);
        break; } }

    return 0; }
\end{C}

\begin{C}
  je passe 4 je passe 6 le plus petit diviseur de 112 et 567 est 7
\end{C}

\begin{infobox}
  Dans le cas de la boucle \mybox{for}, l'exécution reprend à
  l'évaluation de sa deuxième expression (ici \mybox{++i}) et non à
  l'évaluation de la condition (qui a lieu juste après). Il serait en
  effet mal venu que la variable \mybox{i} ne soit pas incrémentée
  lors de l'utilisation de l'instruction \mybox{continue}.Notez bien
  que les instructions \mybox{break} et \mybox{continue} n'affecte que
  l'exécution de la boucle dans laquelle elles sont situées. Ainsi, si
  vous utilisez l'instruction \mybox{break} dans une boucle imbriquée
  dans une autre, vous sortirez de la première, mais pas de la
  seconde.
\end{infobox}

\begin{C}
  #include <stdio.h>


  int main(void) { int i; int j;

    for (i = 0 ; i <= 1000 ; ++i) { for (j = i ; j <= 1000 ; ++j) { if
        (i * j == 1000) { printf ("%d * %d = 1000 \n", i, j);
          break; /* Quitte la boucle courante, mais pas la
          première. */ } } }

    return 0; }
\end{C}

\begin{C}
  1 * 1000 = 1000 2 * 500 = 1000 4 * 250 = 1000 5 * 200 = 1000 8 * 125
  = 1000 10 * 100 = 1000 20 * 50 = 1000 25 * 40 = 1000
\end{C}



\section{L'instruction goto}
\label{Linstruction-goto}

Nous venons de voir qu'il était possible de réaliser des sauts à
l'aide des instructions \mybox{break} et \mybox{continue}. Cependant,
d'une part ces instructions sont confinées à une boucle ou à une
instruction \mybox{switch} et, d'autre part, la destination du saut
nous est imposée (la condition avec \mybox{continue}, la fin du bloc
d'instructions avec \mybox{break}).

L'instruction \mybox{goto} permet de sauter à un point précis du
programme que nous aurons déterminé à l'avance. Pour ce faire, le
langage C nous permet de marquer des instructions à l'aide
d'étiquettes (\emph{labels} en anglais). Une étiquette n'est rien
d'autre qu'un nom choisis par nos soins suivi du catactère
\mybox{:}. Généralement, par soucis de lisibilité, les étiquettes sont
placées en retrait des instructions qu'elles désignent.

\section{Exemple}
\label{exemple-11}

Reprenons (encore) l'exemple du calcul du plus petit commun
diviseur. Ce dernier aurait pu être écrit comme suit à l'aide d'une
instruction \mybox{goto}.

\begin{C}
  #include <stdio.h>


  int main(void) { int a; int b; int i; int min;

    printf("Entrez deux nombres : "); scanf("%d %d", &a, &b);
    min = (a < b) ? a : b;

    for (i = 2; i <= min; ++i) { if (a % i == 0 && b % i == 0)
      { goto trouve; } }

    return 0; trouve: printf("le plus petit diviseur
    de %d et %d est %d\n", a, b, i);
    return 0; }
\end{C}

Comme vous le voyez, l'appel à la fonction \mybox{printf()} a été
marqué avec une étiquette nommée \mybox{trouve}. Celle-ci est utilisée
avec l'instruction \mybox{goto} pour spécifier que c'est à cet endroit
que nous souhaitons nous rendre si un diviseur commun est trouvé. Vous
remarquerez également que nous avons désormais deux instructions
\mybox{return}, la première étant executée dans le cas où aucun
diviseur commun n'est trouvé.

\section{Le dessous des boucles}
\label{le-dessous-des-boucles}

Maintenant que vous savez cela, vous devriez être capable de réecrire
n'importe quelle boucle à l'aide de cette instruction. En effet, une
boucle ne consiste jamais qu'en deux sauts : un vers une condition et
l'autre vers l'instruction qui suit le corps de la boucle. Ainsi, les
deux codes suivants sont équivalents.

\begin{C}
  #include <stdio.h>


  int main(void) { int i = 0;

    while (i < 5) { printf("La variable i vaut %d\n", i);
      i++; }

    return 0; }
\end{C}

\begin{C}
  #include <stdio.h>


  int main(void) { int i = 0;


    condition: if (i < 5) { printf("La variable i vaut %d\n", i);
      i++; goto condition; }

    return 0;

  \end{C}

  \section{Goto Hell ?}
  \label{goto-hell}

Bien qu'utile dans certaines circonstances, sachez que l'instruction
  \mybox{goto} est fortement décriée, principalement pour deux raisons
  :

\begin{itemize}
  \item mise à part dans des cas spécifiques, il est possible de
    réaliser la même action de manière plus claire à l'aide de
    structures de contrôles ;
  \item l'utilisation de cette instruction peut amener votre code à
    être plus difficilement lisible et, dans les pire cas, en faire un
    \href{http://fr.wikipedia.org/wiki/Programmation_spaghetti}{code
      spaghetti}.
  \end{itemize}

À vrai dire, elle est aujourd'hui surtout utilisée dans le cas de la
gestion d'erreur, ce que nous verrons plus tard dans ce
cours. Aussi, en attendant, nous vous conseillons d'éviter son
utilisation.

\hrulefill

Dans le chapitre suivant, nous aborderons la notion de \textbf{fonction}.