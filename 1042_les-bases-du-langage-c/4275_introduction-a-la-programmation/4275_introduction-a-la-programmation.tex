\part{Les bases du langage C}
\label{les-bases-du-langage-C}

\chapter{Introduction à la programmation}
\label{introduction-a-la-programmation}

La programmation est un sujet qui fascine énormément. Si vous lisez ce
cours, c’est que vous avez décidé de franchir le pas et de découvrir
de quoi il s’agit.  Cependant, avant de commencer à apprendre quoi que
ce soit sur le C et la programmation, il est d’abord nécessaire de
découvrir en quoi la programmation consiste. En effet, pour le moment,
vous ne savez pas réellement ce qu’est la programmation, ce que
signifie « programmer » ou encore ce qui caractérise le langage C. Ce
chapitre va donc consister en une introduction au monde de la
programmation, et plus particulièrement au langage C.

\section{Avant-propos}
\label{avant-propos}

\subsection{Esprit et but du tutoriel}
\label{esprit-et-but-du-tutoriel}

Ce cours a été écrit dans un seul but : vous enseigner le langage C de
la manière la plus complète, la plus rigoureuse et la plus instructive
possible. Pour ce faire, celui-ci combinera théorie, détails
techniques et exercices pratiques. Dès lors, nous ne vous le cachons
pas : cette approche va réclamer de votre part des \textbf{efforts},
certains passages étant assez complexes.

Nous avons choisi cette méthode d'apprentissage, car c'est celle que
nous jugeons la plus profitable. Elle s'oppose à une autre, plus
fréquente et plus superficielle, qui permet certes d'acquérir des
connaissances rapidement, mais qui s'avère bien souvent peu payante
sur le long terme.

En effet, beaucoup de programmeurs débutants se retrouvent ainsi
perdus lorsqu'ils sont jetés dans la jungle de la programmation à la
fin d'un cours, ceux-ci manquant souvent de connaissances techniques,
de (bonnes) pratique(s) et de rigueur.

Ne soyez toutefois pas apeuré, notre objectif n'est pas de vous noyer
d'informations ou de vous perdre avec des termes techniques. Nous vous
précisons simplement que ce cours nécessite d'avoir les doigts sur le
clavier et non dans le nez et que le début risque d'être un peu moins
« \emph{cool} » que ce que vous pourrez trouver ailleurs. ;)

\subsection{À qui est destiné ce cours ?}
\label{a-qui-est-destinuxe-ce-cours}

À n'importe quelle personne intéressée : que vous soyez un(e)
programmeur(euse) expérimenté(e), un(e) total(e) débutant(e) ou que
vous vouliez réviser certaines notions du C, vous êtes tous et toutes
les bienvenus(es). Les explications seront les plus claires possibles
afin de rendre la lecture accessible à tous.

Toutefois, quelques qualités sont opportunes pour arriver au bout de
ce cours :

\begin{itemize}
\item De la \textbf{motivation} : ce cours va présenter de nombreuses
  notions, souvent théoriques, et qui sembleront parfois complexes. Il
  vous faut donc être bien motivés pour profiter pleinement de cet
  apprentissage.
\item De la \textbf{logique} : apprendre la programmation, c'est aussi
  être logique. Bien sûr, ce cours vous apprendra à mieux l'être, mais
  il faut néanmoins savoir réfléchir par soi-même et ne pas compter
  sur les autres pour faire le travail à sa place.
\item De la \textbf{patience} : vous vous apprêtez à apprendre un
  langage de programmation. Pour arriver à un sentiment de maîtrise,
  il va vous falloir de la patience pour apprendre, comprendre, vous
  entraîner, faire des erreurs et les corriger.
\item De la \textbf{rigueur} : cette qualité, nous allons tenter de
  vous l'inculquer à travers ce cours. Elle est très importante, car
  c'est elle qui fera la différence entre un bon et un mauvais
  programmeur.
\item De la \textbf{curiosité} : n'hésitez pas à apporter des
  modifications aux codes proposés et à sortir un peu des balises du
  cours, cela ne vous sera que profitable.
\item De la \textbf{passion} : le plus important pour suivre ce
  tutoriel, c'est de prendre plaisir à programmer. Amusez-vous en
  codant, c'est le meilleur moyen de progresser !
\end{itemize}

À noter qu'un niveau acceptable en anglais est un plus indéniable,
beaucoup de cours, de forums et de documentations étant rédigés en
anglais. Si ce n'est pas le cas, gardez ceci à l'esprit : en
programmation, vous y serez confrontés tôt ou tard.

Enfin, un dernier point au sujet des mathématiques : contrairement à
la croyance populaire, un bon niveau en maths n'est absolument pas
nécessaire pour faire de la programmation. Certes, cela peut vous
aider en développant votre logique, mais si les mathématiques ne sont
pas votre fort, vous pourrez suivre ce cours sans problèmes.

\section{Aller plus loin}
\label{aller-plus-loin}

Un des concepts fondamentaux de l'apprentissage de notions
informatiques sur Internet est le \emph{croisement des sources}. Il
permet de voir la programmation sous un angle différent. Par exemple,
quelques cours de
\MYhref{http://c.developpez.com/cours/?page=lang-c}{Developpez}
recourant à des approches différentes sont à votre entière
disposition. N'hésitez pas non plus à lire des livres sur le C,
notamment le
\MYhref{http://en.wikipedia.org/wiki/The_C_Programming_Language}{K\&R},
écrit par les auteurs du langage (une version traduite en français est
disponible
\MYhref{http://www.dunod.com/informatique-multimedia/developpement/cc/ouvrages-denseignement/le-langage-c}{aux
  éditions Dunod}). C'est un livre qui pourra vous être utile.

\section{La programmation, qu’est-ce que c’est ?}
\label{la-programmation-quest-ce-que-cest}

\begin{infobox}
Dans cette section, nous nous
      contenterons d'une présentation succinte qui est suffisante pour
      vous permettre de poursuivre la lecture de ce cours. Toutefois,
      si vous souhaitez un propos plus étayé, nous vous conseillons la
      lecture du
      \MYhref{https://zestedesavoir.com/tutoriels/531/les-bases-de-la-programmation/}{cours
        d'introduction à la programmation} présent sur ce site.
\end{infobox}

La programmation est une branche de l'informatique qui sert à créer
des \textbf{programmes}. Tout ce que vous possédez sur votre
ordinateur est un programme : votre navigateur Internet (Internet
Explorer, Firefox, Opera, etc.), votre système d'exploitation
(Windows, GNU/Linux, Mac OS X, etc.) qui est un regroupement de
plusieurs programmes appelés \textbf{logiciels}, votre lecteur MP3,
votre logiciel de discussion instantanée, vos jeux vidéos, etc.

\subsection{Les programmes expliqués en long, en large et en   travers}
  \label{les-programmes-expliques-en-long-en-large-et-en-travers}

Un programme est une séquence d'\textbf{instructions}, d'ordres,
donnés à l'ordinateur afin qu'il exécute des actions. Ces instructions
sont généralement assez basiques. On trouve ainsi des instructions
d'addition, de multiplication, ou d'autres opérations mathématiques de
base, qui font que notre ordinateur est une vraie machine à calculer.
D'autres instructions plus complexes peuvent exister, comme des
opérations permettant de comparer des valeurs, traiter des caractères,
etc.

Créer un programme, c'est tout simplement utiliser une suite
d'instructions de base qui permettra de faire ce que l'on veut. Tous
les programmes sont créés ainsi : votre lecteur MP3 donne des
instructions à l'ordinateur pour écouter de la musique, le \emph{chat}
donne des instructions pour discuter avec d'autres gens sur le réseau,
le système d'exploitation donne des instructions pour dire à
l'ordinateur comment utiliser le matériel, etc.

\begin{infobox}
Notez qu'il n'est pas possible
    de créer des instructions. Ces dernières sont imprimées dans les
    circuits de l'ordinateur ce qui fait qu'il ne peut en gérer qu'un
    nombre précis et qu'il ne vous est donc pas loisible d'en
    construire de nouvelles (sauf cas particuliers vraiment tordus).
\end{infobox}

Notre ordinateur contient un composant électronique particulier,
spécialement conçu pour exécuter ces instructions : le
\textbf{processeur}. Ce qu'il faut retenir, c'est que notre ordinateur
contient un circuit, le processeur, qui permet d'effectuer de petits
traitements de base qu'on appelle des instructions et qui sont la base
de tout ce qu'on trouve sur un ordinateur.

Les instructions sont stockées dans notre ordinateur sous la forme de
chiffres binaires (appelés \emph{bits} en anglais), autrement dit sous
forme de zéros ou de uns. Ainsi, nos instructions ne sont rien d'autre
que des suites de zéros et de uns conservées dans notre ordinateur et
que notre processeur va interpréter comme étant des ordres à exécuter.
Ces suites de zéros et de uns sont difficilement compréhensibles pour
nous, humains, et parler à l'ordinateur avec des zéros et des uns est
très fastidieux et très long. Autant vous dire que créer des
programmes de cette façon revient à se tirer une balle dans le pied.

Pour vous donner un exemple, imaginez que vous deviez communiquer avec
un étranger alors que vous ne connaissez pas sa langue. Communiquer
avec un ordinateur reviendrait à devoir lui donner une suite de zéros
et de uns, ce dernier étant incapable de comprendre autre chose. Ce
langage s'appelle le \textbf{langage machine}.

Une question doit certainement vous venir à l'esprit : comment
communiquer avec notre processeur sans avoir à apprendre sa langue ?

L'idéal serait de parler à notre processeur en français, en anglais,
etc, mais disons-le clairement : notre technologie n'est pas
suffisamment évoluée et nous avons dû trouver autre chose. La solution
retenue a été de créer des langages de programmation plus évolués que
le langage machine, plus faciles à apprendre et de fournir le
traducteur qui va avec. Il s'agit de langages assez simplifiés,
souvent proches des langages naturels et dans lesquels on peut écrire
nos programmes beaucoup plus simplement qu'en utilisant le langage
machine. Grâce à eux, il est possible d'écrire nos programmes sous
forme de texte, sans avoir à se débrouiller avec des suites de zéros
et de uns totalement incompréhensibles. Il existe de nombreux langages
de programmation et l'un d'entre-eux est le \textbf{C}.

Reste que notre processeur ne comprend pas ces langages évolués et
n'en connaît qu'un seul : le sien. Aussi, pour utiliser un langage de
programmation, il faut disposer d'un traducteur qui fera le lien entre
celui-ci et le langage machine du processeur. Ainsi, il ne vous est
plus nécessaire de connaître la langue de votre processeur. En
informatique, ce traducteur est appelé un \textbf{compilateur}.

Pour illustrer notre propos, voici un code écrit en C (que nous
apprendrons à connaître).

\begin{C}
  #include <stdio.h>

  int main(void) { printf("Salut !\n"); return 0; }
\end{C}

\clearpage

Et le même en langage machine (plus précisémment pour un processeur de
la famille x86-64).

\begin{C}
  01010101 01001000 10001001 11100101 10111111 00100100 00101100
  01001000 00000000 10111000 00000000 00000000 00000000 00000000
  11101000 10011101 00001011 00000000 00000000 10111000 00000000
  00000000 00000000 00000000 01011101 11000011 01010011 01100001
  01101100 01110101 01110100 00100000 00100001 00001010 00000000
\end{C}

Nous y gagnons tout de même au change, non ? :pMalgré tous ces
langages de programmation disponibles nous allons, dans ce tutoriel,
nous concentrer sur un seul d'entre-eux : le C. Avant de parler des
caractéristiques de ce langage et des choix qui nous amènent à
l'étudier dans ce cours, faisons un peu d'histoire.

\section{Le langage C}
\label{le-langage-C}

\subsection{L'histoire du C}
\label{lhistoire-du-c}

Le langage C est né au début des années 1970 dans les laboratoires de
la société AT\&T aux États-Unis. Son concepteur,
\MYhref{http://fr.wikipedia.org/wiki/Dennis_Ritchie}{Dennis MacAlistair
  Ritchie}, souhaitait améliorer un langage existant, le B, afin de
lui adjoindre des nouveautés. En 1973, le C était pratiquement au
point et il commença à être distribué l'année suivante. Son succès fut
tel auprès des informaticiens qu'en 1989, l'ANSI, puis en 1990, l'ISO,
décidèrent de le normaliser, c'est-à-dire d'établir des règles
internationales et officielles pour ce langage. À l'heure actuelle, il
existe trois normes : la norme ANSI C89 ou ISO C90, la norme ISO C99
et la norme ISO C11.

\emph{{[}AT\&T{]}: American Telephone and Telegraph Company
}{[}ANSI{]}: American National Standards Institute *{[}ISO{]}:
International Organization for Standardization

\begin{infobox}
Si vous voulez en savoir plus sur
l'histoire du C, lisez donc
\MYhref{http://c.developpez.com/cours/historique-langage-c/}{ce
  tutoriel}.
\end{infobox}

\subsection{Pourquoi apprendre le C ?}
\label{pourquoi-apprendre-le-c}

C'est une très bonne question. :D Après tout, étant donné qu'il existe
énormément de langages différents, il est légitime de se demander
pourquoi choisir le C en particulier ? Il y a plusieurs raisons à
cela.

\begin{itemize}
\item Sa \textbf{popularité} : le C fait partie des langages de
  programmation les plus utilisés. Il possède une communauté très
  importante, de nombreux cours et beaucoup de documentations. Vous
  aurez donc toujours du monde pour vous aider. De plus, il existe un
  grand nombre de programmes et de bibliothèques développés en C.
\item Sa \textbf{rapidité} : le C est connu pour être un langage très
  rapide, ce qui en fait un langage de choix pour tout programme où la
  vitesse d'exécution est cruciale.
\item Sa \textbf{simplicité} : le C est un langage minimaliste pourvu
  de peu de concepts ce qui permet d'en faire le tour
  \emph{relativement} rapidement et d'éviter un niveau d'abstraction
  trop important.
\item Sa \textbf{légèreté} : le C est léger, ce qui le rend utile pour
  les programmes embarqués où la mémoire disponible est faible.
\item Sa \textbf{portabilité} : cela signifie qu'un programme
  développé en C peut être compilé pour fonctionner sur différentes
  machines sans devoir changer ledit code.
\end{itemize}

Ce ne sont que quelques raisons, mais elles sont à notre goût
suffisantes pour justifier l'apprentissage de ce langage. Bien
entendu, le C comporte aussi sa part de défauts. On peut citer la
tolérance aux comportements dangereux qui fait que le C demande de la
rigueur pour ne pas tomber dans certains « pièges », un nombre plus
restreint de concepts (c'est parfois un désavantage, car on est alors
obligé de recoder certains mécanismes qui existent nativement dans
d'autres langages), etc. D'ailleurs, si votre but est de développer
rapidement des programmes amusants, sachez que le C n'est pas adapté
pour cela et que nous vous conseillons, dans ce cas, de vous tourner
vers d'autres langages, comme par exemple le
\MYhref{https://zestedesavoir.com/tutoriels/799/apprendre-a-programmer-avec-python-3/}{Python}
ou le
\MYhref{https://zestedesavoir.com/tutoriels/634/une-introduction-a-ruby/}{Ruby}.

Le C possède aussi une caractéristique qui est à la fois un avantage et
un défaut : il s'agit d'un langage dit de « \textbf{bas niveau} ». Cela
signifie qu'il permet de programmer en étant « proche de sa machine »,
c'est-à-dire sans trop vous cacher son fonctionnement interne. Cette
propriété est à double tranchant : d'un côté elle rend l'apprentissage
plus difficile et augmente le risque d'erreurs ou de comportements
dangereux, mais de l'autre elle vous laisse une grande liberté d'action
et vous permet d'en apprendre plus sur le fonctionnement de votre
machine. Cette notion de « bas niveau » est d'ailleurs à opposer aux
langages dit de « \textbf{haut niveau} » qui permettent de programmer en
faisant abstraction d'un certain nombre de choses. Le développement est
rendu plus facile et plus rapide, mais en contrepartie, beaucoup de
mécanisme interne sont cachés et ne sont pas accessibles au programmeur.
Ces notions de haut et de bas niveau sont néanmoins à nuancer, car elles
dépendent du langage utilisé et du point de vue du programmeur (par
exemple, par rapport au langage machine, le C est un langage de haut
niveau).

Une petite note pour terminer : peut-être avez-vous entendu parler du
\textbf{C++} ? Il s'agit d'un langage de programmation qui a été inventé
dans les années 1980 par
\MYhref{https://en.wikipedia.org/wiki/Bjarne_Stroustrup}{Bjarne
Stroustrup}, un collègue de Dennis Ritchie, qui souhaitait rajouter des
éléments au C. Bien qu'il fût très proche du C lors de sa création, le
C++ est aujourd'hui un langage très différent du C et n'a pour ainsi
dire plus de rapport avec lui (si ce n'est une certaine proximité au
niveau d'une partie de sa syntaxe). Ceci est encore plus vrai en ce qui
concerne la manière de programmer et de raisonner qui sont
\emph{radicalement} différentes.

Ne croyez toutefois pas, comme peut le laisser penser leur nom ou leur
date de création, qu'il y a un langage meilleur que l'autre, ils sont
simplement \emph{différents}. Si d'ailleurs votre but est d'apprendre le
C++, nous vous encourageons à le faire. En effet, contrairement à ce qui
est souvent dit ou lu, \emph{il n'y a pas besoin de connaitre le C pour
apprendre le C++}.

\section{La norme}
\label{la-norme}

Comme précisé plus haut, le C est un langage qui a été normalisé à trois
reprises. Ces normes servent de référence à tous les programmeurs et les
aident chaque fois qu'ils ont un doute ou une question en rapport avec
le langage. Bien entendu, elle ne sont pas parfaites et ne répondent pas
à toutes les questions, mais elles restent \emph{la} référence pour tout
programmeur.

Ces normes sont également indispensables pour les compilateurs. En
effet, le respect de ces normes par les différents compilateurs permet
qu'il n'y ait pas de différences d'interprétation d'un même code.
Finalement, ces normes sont l'équivalent de nos règles d'orthographe, de
grammaire et de conjugaison. Imaginez si chacun écrivait ou conjuguait à
sa guise, ce serait un sacré bazar\ldots{}

Dans ce cours, nous avons décidé de nous reposer sur la norme ANSI C89
(ou ISO C90, c'est pareil). En effet, même s'il s'agit de la plus
ancienne, elle nous permettra néanmoins de développer avec n'importe
quel compilateur et sous n'importe quel système sans problèmes et sans
nous poser de questions sur la présence ou non de telle ou telle
fonctionnalité.

Rassurez-vous néanmoins : le fait de nous baser sur la norme C89 ne
signifie pas que vous allez découvrir une version obsolète du langage C.
En effet, d'une part, ce que vous allez voir tout au long de ce cours
est toujours valable au regard des normes plus récentes et, d'autre
part, les changements induits par les autres normes sont le plus souvent
mineurs et consistent pour ainsi dire tous en des \emph{ajouts} et non
en des modifications. De ce fait, il vous sera aisé, une fois ce cours
parcouru, de passer à une norme plus récente.

\begin{infobox} Pour les curieux, voici
    \MYhref{http://flash-gordon.me.uk/ansi.c.txt}{un lien} vers le
    brouillon de cette norme. Cela signifie qu'il ne s'agit pas de la
    version définitive et officielle, cependant il est largement
    suffisant pour notre niveau et, surtout, il est gratuit (la norme
    officielle coûtant \emph{très} cher :-° ). Notez que celui-ci est
    rédigé en anglais
\end{infobox}

\section{L’algorithmique}
\label{l-algorithmique}

L'algorithmique est liée à la programmation et constitue même une
branche à part des mathématiques.  Elle consiste à définir et établir
des \textbf{algorithmes}.

Un algorithme peut se définir comme étant une suite finie et
non-ambiguë d'opérations permettant de résoudre un problème. En clair,
il s'agit de calculs qui prennent plusieurs paramètres et fournissent
un résultat.  Les algorithmes ne sont pas limités à l'informatique,
ils existaient même avant son apparition ; prenez les recettes de
cuisine par exemple, ou des instructions de montage d'un meuble ou
d'un Lego, ce sont des algorithmes.

L'intérêt principal des algorithmes est qu'ils sont très utiles
lorsqu'ils sont en relation avec des ordinateurs. En effet, ces
derniers peuvent exécuter des milliards d'instructions à la seconde,
ce qui les rend bien plus rapides qu'un humain. Illustrons : imaginez
que vous deviez trier une liste de dix nombres dans l'ordre
croissant. C'est assez facile et faisable en quelques secondes. Et
pour plusieurs milliards de nombres ? C'est impossible pour un humain,
alors qu'un ordinateur le fera rapidement.

Ce qu'il faut retenir, c'est qu'un algorithme est une suite
d'opérations destinée à résoudre un problème donné. Nous aurons
l'occasion d'utiliser quelques algorithmes dans ce cours, mais nous ne
nous concentrerons pas dessus.

\fbox{\begin{minipage}{0.9\textwidth}Si vous voulez en savoir plus,
    lisez le tutoriel sur
    \MYhref{/tutoriels/621/algorithmique-pour-lapprenti-programmeur/}{l'algorithmique
      pour l'apprenti programmeur} en même temps que vous apprenez à
    programmer avec celui-ci.
  \end{minipage}}

\subsection{Le pseudo-code}
\label{le-pseudo-code}

Pour représenter un algorithme indépendamment de tout langage, on
utilise ce qu'on appelle un \textbf{pseudo-code}. Il s'agit de la
description des étapes de l'algorithme en langage naturel (dans notre
cas le français). Voici un exemple de pseudo-code.

\begin{C}
  Fonction max (x, y)
    
  Si x est supérieur à y Retourner x Sinon Retourner y

  Fin fonction
\end{C}

Dans ce cours, il y aura plusieurs exercices dans lesquels un
algorithme fourni devra être mis en œuvre, traduit en C. Si vous
voulez vous entrainer davantage tout en suivant ce cours, nous vous
conseillons \MYhref{http://www.france-ioi.org/}{France-IOI} qui permet
de mettre en application divers algorithmes dans plusieurs langages,
dont le C. Cela pourra être un excellent complément.

\hrulefill

Comme vous avez pu le constater, la programmation est un monde vaste,
très vaste, et assez complexe. Comme il existe une multitude de
langages de programmation, il faut se concentrer sur un seul d'entre
eux à la fois. Dans notre cas, il s'agit du C. Ce langage, et
retenez-le bien, est à la fois puissant et complexe. Rappelez-vous
bien qu'il vous faudra faire des efforts pour l'apprendre
correctement.

Si vous vous sentez prêts, alors rendez-vous dans le chapitre suivant,
qui vous montrera les outils utilisés par un programmeur C.